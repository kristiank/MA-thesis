\documentclass[12pt,a4paper]{article}

\usepackage[top=4cm, bottom=3cm, left=4cm, right=2.5cm]{geometry}
\usepackage{setspace}

\usepackage{float}
\usepackage{placeins} % defines \FloatBarrier

\usepackage{amsmath} % align ja gather

\usepackage{enumerate}

\usepackage{booktabs}

% polyglossia
\usepackage{polyglossia}
\usepackage{fontspec}
\usepackage{xunicode}
\usepackage{xltxtra}
\usepackage{url}
\usepackage{expex}

% \texttildelow to get a nice vertically centered tilde
\usepackage{textcomp}

% \mathfraktur jaoks
\usepackage{amsfonts}

% Use a Free/Libre font with Finnish–Hungarian-Cyrillic-UPA coverage
\setmainfont[Mapping=tex-text]{Linux Libertine O}
% set languages to use
\setmainlanguage{estonian}
\setotherlanguages{english}

\usepackage{csquotes}

% Polyglossias miskipärast ei tööta poolitamine
% \usepackage{polyglossia}
% \setdefaultlanguage{german}

% @todo: vormistada kirjanduse loetelu õigesti
% @todo: vormistada tekstis viited õigesti (nt lk kooloni järel jne)
\usepackage[backend=biber,style=authoryear]{biblatex}
\addbibresource{bibliography-mathesis.bib}

\usepackage{hyperref}
\urlstyle{}

% koodiplokkide esitamiseks
\usepackage{minted}

% Sõnastik
\usepackage[nopostdot,xindy]{glossaries}
\makeglossaries

% omad mallid
\newcommand{\vadja}[1]{\textit{#1}}
\newcommand{\msd}[1]{\textsc{#1}}


% sisukord
% Bakalaureusetöös on soovitatav piirduda kolmeastmelise hierarhiaga, magistritöös neljaastmelisega
\setcounter{tocdepth}{4}


% kasutatud mõisted ja lühendid

%% tüüpsõna on mingil põhjusel valitud sõna, millega tähistatakse üht sõnatüüpi (kusjuures tüüpsõna peab kuuluma samasse sõnatüüpi)
%% sõnatüüp on ühtviisi käituvate sõnade hulk (vt Viks)

% TODO lisada lähtekeelne termin, nagu (concatenation)
\newglossaryentry{muuttüüp}{
  name=Muuttüüp,
  description={on morfoloogilise klassifikatsiooni elementaar\-üksus. See on sõnaklass, mis erineb teistest sõnaklassidest mingite erijoonte poolest sõnade muutmisel.}
}
\newglossaryentry{sõnatüübimall}{
  name=Sõnatüübimall,
  description={on ekstraktmorfoloogiaga leitud tüüpsõna paradigma kirjeldus, mis koosneb iga muutvormi koostamismallidest ehk muutvormimallidest. Tüüpsõnamall on relatsioon tehnilise tüve ja kõigi selle paradigmasse kuuluvate muutvormide vahel.}
}
\newglossaryentry{muutvormimall}{
  name=Muutvormimall,
  description={kirjeldab üksiku muutvormi koostamisskeemi ja kannab selle grammatilised tunnused. On integraalne osa tüüpsõnamallist. Koostamisskeem koosneb muutujatest ja konstantidest, mille tähtkoostised lükitakse üks-teise järele. Muutujate tähtkoostised võivad olla mingil moel piiratud.}
}
\newglossaryentry{lemma}{
  name=Lemma,
  description={on suvaliselt valitud grammatiliste tunnuste komplekt, mida kasutatakse lekseemi viitamiseks.}
}
\newglossaryentry{tehniline-tüvi}{
  name=Tehniline tüvi,
  description={on tähtkoostiste järjend, millega saab tüüpsõnamalli muutvormide muutujad asendada elik väärtustada ja niiviisi koostada ühe konkreetse sõna kõik vormid.}
}
\newglossaryentry{mikrostruktuur}{
  name=Mikrostruktuur,
  description={on sõnastiku sõnaartikli sisemine struktuur.}
}
\newglossaryentry{konkatenatsioon}{
  name=Konkatenatsioon,
  description={ehk \ensuremath{\oplus} on tähtede ja tähtjärjendite lükkimine teine-teise järele, et moodustada uus tähtjärjend. Näiteks \textit{aa} \ensuremath{\oplus} \textit{be} moodustab \textit{aabe}.},
  symbol=ensuremath{\oplus}
}



\begin{document}


% @TODO: lisada magistritöö, ülikool ja juhendaja jne
% vaata siit https://tex.stackexchange.com/questions/184848/how-to-add-the-name-of-the-supervisor-in-a-thesis-field#184878
% @TODO: vormistada esileht eraldi
\title{Ekstraktmorfoloogia meetodiga tuletatud keeletehnoloogia vadja noomeni vormisõnastiku näitel}
\author{Kristian Kankainen}
% \supervisor{Heinike Heinsoo, Külli Prillop}
% \university{Tartu Ülikool}
% \department{Eesti ja üldkeeleteaduse õppetool}
\date{2019}
\maketitle


\newpage
\tableofcontents


% TODO kust pärineb järgmine tsitaat? "computational methodology for linguistic analysis is not the same thing as computational linguistics" ()


\newpage
\spacing{1.5}
\section{Sissejuhatus}

% %Käesolev töö ei pürgi looma lõplikku normatiivi, kuivõrd ta loob süsteemi, mis oskab vastata morfoloogilistele küsimustele. Aga loodud süsteemi peamine eesmärk on siiski võimaldada muuta ja jätkata tööd normatiivi arendamiseks ja mille ümber saaks keeleaktivistid ise koonduda, ilma et selleks oleks niivõrd vaja ei lingvistilist ega keeletehnoloogilist spetsialisti.

Magistritöö eesmärk on koostada vadja keele noomeni jaoks morfoloogilise
andmestiku ehk vormisõnastiku. Vormisõnastik täidab magistritöös kaht suuremat funktsiooni: 1)~on produktiivsele, tänapäeva keelekasutust peegeldav sõnaraamat ja 2)~selle sisu on teisendatav programmkoodi, mis integreerib vadja keele morfoloogiamooduli kahte keeletehnoloogilisse raamistikku.

Tööl on neli suuremat eesmärki ehk teesi:
\begin{enumerate}
\item näidata, et ekstraktmorfoloogia on intuitiivne viis luua arvutimorfoloogia
(võimaldab mitte-(arvutus-)lingvistil luua ja muuta andmestikku)
\item ekstraktmorfoloogia meetodiga loodud arvutimorfoloogia on võimalik integreerida automaatse teisendusega erinevatesse keeletehnoloogiatesse
\item ekstraktmorfoloogia meetodiga luuakse arhiveeritava ja püsivama väärtusega andmestik % kui seda on keeletehnoloogia
\item kõrvutada ja ühitada vadja noomeni ekstraktmorfoloogia leiud traditsioonilise
morfoloogiakäsitlusega
\item on laiendatav teise sõnaliigiga sõnade sisestamiseks
\item on laiendatav tõlkevastustega % TODO neid kahte ma väga ei "tõesta"
\end{enumerate}


Magistritöö eesmärk on luua vadja keelele morfoloogiline sõnastik sellisel moel, et keeletehnoloogilised komponendid, mh õigekirjakontrollija, on sellest automaatselt tuletatavad. Sõnastik oleks loodava vadja kirjakeele õigekeelsuse baasiks ja aitaks keeleõppes ja revitaliseerimises kaasa.

Morfoloogiline sõnastik on süsteem, mis sisaldab kõigi sõnaartiklite kõiki muutvorme. Töö andmestik piirdub vadja noomeniga. Töö andmestik hõlmab u~900~noomenit, mis on valdavas osas pärit Vad̕d̕a Sõnakopittõja sõnastikust (\cite{heinsoo_vadsonakopittoja_2015}) koos täiendustega teistest sõnaraamatutest.

Sõnade paradigmad, ehk käändvormide tüvemuutused, on saadud Tsvetkovi sõnaraamatust (\cite{laakso_vatjan_1989}) milledele on lisatud muutelõpud levinud vadja keele õpikust (\cite{_vadceeli:_2014}). Sõnade kirjapilti on ühtlustatud vadja loodava kirjakeele järgi (Heinsoo, isiklik kommunikatsioon).

Loodud sõnaraamatu kirjed on kõrvuti morfoloogiliste muutvormidega varustatud ka sõnatüübi\-tähisega. Sõnatüübid leitakse ekstraktmorfoloogia meetodi abil automaatselt ja sõnatüübistik uuendatakse iga kord sõnaraamatu kirjeid muudetakse.

Ekstrakt-morfoloogiline klassifikatsioon põhineb ainult sõnade muutvormide tähtkoostisel ja sõnatüübistik on maksimaalne selles suhtes, et igal tüvevokaalil ja astmevahelduslikul tähel on omaette sõnatüüp.

Kõrvuti eksisteeriv sõnatüübistik võimaldab arvutimorfoloogia automaatse loomise ja niiviisi on sõnaraamatusse uute kirjete lisamise juures süsteemil võimalik genereerida ja kasutajale kuvada sõna potentsiaalseid muutvorme. Samuti on võimalik sõnatüübistiku kirjelduse põhjal automaatselt luua programm\-koodi, mis täidab morfoloogia komponendi ülesannet ühes või mitmes programmeerimis\-keeles.

Ekstraktmorfoloogia meetod sobitub magistritöö ülesandele hästi mitmel põhjusel:
\begin{itemize}
\item meetod põhineb otse sõnavormidel ja ei vaja reegleid (vadja keelel puudub morfoloogiat piisavalt täpselt kirjeldav grammatika)
\item tekitatud sõnastik koos sõnatüübistikuga on võimalik automaatselt ümber kodeerida keeletehnoloogiasse (mh õigekirjakontrollija)
\item loodud sõnastik on iseenesest kasutajaliides, mis võimaldab vadja keele huvilistel endil oma sõnavara rikastada (ei vaja programmeerimis- ega keeleteaduslike oskusi)
\item sõnastikus esinevad muutvormide artikleid on võimalik varustada korpusesinemuste informatsiooniga ja õigekeelsuslike kommentaaridega
\end{itemize}


Magistritöö on struktureeritud järgmiselt:
%   2 Morfoloogilise sõnaraamatu teoreetilised lähtekohad
%        2.1 Vadja kirjakeel ja korpusplaneerimine
%        2.2 Arvutimorfoloogia eesmärk ja lingvistiline motiveeritus
%        2.3 Paradigmaatiline morfoloogia
%        2.4 Vadja kirjakeel ja normatiiv
%        2.5 Sõnavara
%        2.6 Ortograafia
%        2.7 Morfofonoloogia
%        2.8 Klassikaline paradigmaatiline morfoloogia
%        2.9 Morfeemi staatus ja definitsioon
%        2.10 Muuttüüp, tüüpsõna ja muutkond
%      3 Ekstraktmorfoloogia meetod
%        3.1 
%      4 Vadja morfoloogiliste tüüpsõnade analüüs
%        4.16 Ekstraktmorfoloogia üldistatud muuttüüpide algoritm
%        4.17 Põhivormid ja analoogiavormid
%      5 Programmkoodi tuletamine
%        5.1 Keskne kirjeldus Lexical Markup Framework vormingus
%        5.2 Grammatical Framework morfoloogiakomponent
%        5.3 Integreerimine Giella-taristuga
%      6 Arutelu
%      7 Kokkuvõte
%      8 Põhimõisted ja lühendid
%      9 Kirjandus
%      10 The use of Extract Morphology for Automatic Derivation of Language Technology for Votic
%      11 Lisad

% morfeemi staatus -- morfeem pole siin töös traditsioonilisel kujul kui väikseim tähenduslik vormiline üksus vaid hoopis mistahes-tähtkoostise-muutusena (vrd Baer)
% morfeemi puudumine töös toob kaasa ka (morfoloogiliste) reeglite puudumise. see on üsna ebatraditsiooniline, aga põhjendame siinkohal ajalooliselt siiski ära
% ekstraktmorfoloogia organiseeritus sarnaneb kõige lähedamini klassikalisele paradigmaatilisele morfoloogiakäsitlusele. Hockett mainib seda 'vanema ja väärikama teooriana' oma tänaseks klassikaks saanud artiklis kahe morf teooria üle. Ta räägib IP ja IA lähenemisest ja toob välja ajastu formaliseerimislembuse ning kuna IA oli juba formaliseeritud, oli tema meelest tookord aeg formaliseerida IP. Karlsson ütleb kokkuvõtvalt, "et sellest sai levinuim lähenemine". Tänapäeval on hakatud formaliseerima ka paradigmaatilist lähenemist, aga peamiselt kahes liinis, millest üks on rohkem reeglitel põhinev (nn uus) ja teine on vähem reeglitel põhinev (nn klassikaline). Arvutimorfoloogid on siiski hiljuti toonud välja olukorra, et paradigmaatilist lähenemist on võimalik modelleerida kasutades sama formalismi, millega IP morfoologiatki, nimelt lõplike automaatidega.
% reeglite lembusest


%% Magistritöö esimene eesmärk on luua H. Heinsoo Sõnakopittöjas esitatud sõnavarast morfoloogiline sõnastik, mis sisaldab sõnavara kõiki muutvorme. Selleks vajalik arvutimorfoloogiline kirjeldus ehitatakse sellisel moel, et see taandub tüüpsõnade muutvormitabelite esitamisele, mitte grammatiliste reeglite esitamisele. Niiviisi ehitatud teooriavaba(m) arvutimorfoloogiline kirjeldus võimaldab luua erinevaid keeletehnoloogiaid automaatselt programmkoodi tuletamise teel. Esitatakse kolme tehnoloogia automaatset tuletamist: 1)~ühe keeletehnoloogilise taristusse integreerimise kaudu õigekirjakontrollija, 2)~vadja keele arvutimorfoloogia moodul ühe loomulike keelte grammatikate koostamiseks mõeldud programmeerimiskeelele ja 3)~morfoloogia tehnoloogiaülene kirjeldus ühe rahvusvahelise standardi abil.

%% Kuna kõik tuletatud keeletehnoloogia edaspidine täiendamine ja täpsustamine käib ainult lekseemide muutvormitabelite täiendamise ja täpsustamise kaudu, peab esimese eesmärgi juurde lisama seda, et magistritöös loodud leksikograafiline süsteem võimaldab keeleaktivistide rühmal töötada oma sõnavara ja keeletehnoloogia kallal edaspidi ka ilma spetsialistist keeleteadlase ja keeletehnoloogi abil. Kas seda vadja keele puhul ka juhtub, jääb tuleviku näidata.

%% Magistritöö teine eesmärk on analüüsida leitud tüüpsõnad mitmel viisil: 1)~kirjeldada nende morfofonoloogiat keeleajalooliste arengute taustal, 2)~leida tüüpsõnade põhivormid ja analoogiavormid, 3)~esitada üks võimalik muuttüüpide süsteem ja võrrelda seda seni esitatutega ja viimalt 4)~analüüsida muuttüüpide produktiivsust.


%% ---


%% Magistritöö loob viisi ehitada arvutimorfoloogia puhtalt lekseemide sõnavormide esitamise teel ning teisendada ehitatud arvutimorfoloogilise mudeli automaatselt kahte keeletehnoloogilisse raamistikku.

%% Magistritöö kasutab loodud süsteemi selleks, et kirjeldada vadja keele normatiivsed morfoloogilised tüüpsõnad.

%% Tööd ajendab mõtteviis minimeerida tööd: loodud normatiivne morfoloogiline tüübistik on aluseks automaatselt tuletatud keeletehnoloogiale, kui normatiiv muutub, muutub ka keeletehnoloogia. Töö paneb leksikaalse ressursi esikohale ja kõik leitud sisulised vead õiendatakse otse ressursis, mitte keeletehnoloogilistes tarkvarades eraldi.




\newpage
\section{Vormisõnastiku koostamise teoreetilised lähtekohad}

% TODO mitte kasutada meie vorme verbides, mitte kasutada 'püüdleb'
Siin peatükis kirjeldatatakse vormisõnastiku koostamise ja vadja kirjakeelsete muutvormide rekonstrueerimise põhimõtteid. Andmete sisestamist nn Morfoloogia\-labori rakendusega kirjeldatakse peatükis~\ref{sec:vormisõnastiku-sisestamine} ja andmete kuju kirjeldatakse peatükis~\ref{sec:lmf}.

Vormisõnastiku definitsiooniks magistritöös on võetud Ülle Viksi järgi:

\spacing{1}
\begin{quote}
  ``Üks täielik vormisõnastik peaks esitama kõigi sõnade kõik muutevormid koos vastava grammatilise iseloomustusega. Ainult siis saab kasutaja sõnastikust ilma mingi vaevata ja täiesti kindlalt teada, milline on vajalik vorm antud sõnast või millise sõna millise vormiga on tegemist tundmatu sõnavormi puhul.'' (\cite[7]{viks_vaike_1992}).
\end{quote}
\spacing{1.5}



% TODO siia lisada Viksi definitsioon vormisõnastikust

%Kuna töö opereerib arvutilingvistika, deskriptiivse ja dokumentaalse lingvistika ääremail, peame selgitama töö teoreetilised lähtekohad. Siinsele kompendiumiks on ka põhimõisted seletatud pt~\ref{sec:põhimõisted} \nameref{sec:põhimõisted}.

%Töö püüdleb olla võimalikult teooriavaba, lastes vadja lekseemide sõnavormide tähtkoostised ise määrata nende paradigmade koostamis\-reeglid. (See on olla deduktiivne esialgse morfoloogia postuleerimises, vastandudes induktiivsele, s.o mingist grammatilisest kirjeldusest lähtudes.)

%Tööl on siiski teoreetilised lähtekohad, mis tulenevad ühelt poolt arvutimorfoloogia nõuetest ja teisalt klassikalisest paradigmaatilisest morfoloogiakäsitlusest. Järgmiselt püüan argumenteerida, et arvutimorfoloogia ei pea olema mingist lingvistisest teooriast ajendatud. Seejärel tutvustan tööle kõige lähedamini asetsevat morfoloogilist käsitlust.

% TODO EELMINE JUTT TÄIELIKULT ÜLE VAADATA!

\subsection{Vadja ühiskeel ja korpuse planeerimine}

% TODO lisada ühiskeele murdeline alus on Vaipooli (Li-Lu), sest see on mida "kõneldakse veel". Rohkem on uuritud ja kirjeldatud Kattila(?), mida Ernits on ka soovitanud ühiskeele aluseks valida.

Selles peatükis seletatakse vadja ühiskeeleks valitud aluseid ning korpuse planeerimise mõistet. Vadja keelele ei loodud 1930-ndateil Nõukogude Liidus ühiskeelt, nagu seda tehti näiteks karjala, vepsa ja isuri keele jaoks. Vadja kirjakeele loomise pürgimistest pärast Nõukogude Liidu lagunemist on kirjutanud lähemalt \cite{ernits_vadja_2006} ja vadjalaste ilma kirjakeeleta keele normi tunnetusest on rääkinud \cite{markus_concept_2013}.

% korpusplaneerimine on alamosa keeleplaneerimisest?
%Korpusplaneerimine on mis, miks seda tehakse ja kes seda teevad (kasuta Cooper 1996: 45).
% TODO lisada viited
Korpusplaneerimine on keele\-planeerimise üks osa ja koosneb (\cite{kloss_research_1970}) järgi kolmest osast:
\spacing{0}
\begin{enumerate}
\item kirjamine (ingl. \textit{graphization}) ehk kirjaviisi määramine
\item morfoloogia ühtlustamine ja standardiseerimine
\item sõnavara moderniseerimine ja rikastamine
\end{enumerate}
\spacing{1.5}

Neid osi ei pea vaatama järjestiku etappidena keele moderniseerimise poole, vaid (\cite{coulmas_language_1989}) on nimetanud seda pidevaks adapteerimise protsessiks.

% alustame KIRJAVIISI kohta, seejärel morfoloogia ja sõnavara
Käesolev magistritöö jääb morfoloogia ühtlustamise ja standardiseerimise alla, ent tahest-tahtmata on kasutatud kirjaviisiga tehtud valik, kus on järgitud Heinsoo õppematerjalide kirjaviisi. Valitud kirjaviis on siiski automaatselt teisendatav teise levinud, Konkova õppematerjalidele vastavaks, mis\-tõttu peaks magistri\-töös loodud materjal olema mõlema õppematerjali jaoks kasulik. Ainsad erinevused kahe kirja\-viisi vahel on ü~\textrightarrow~y ja č~\textrightarrow~c. % kas see peaks olema ainult sissejuhatuses või mis sellega viga on?

% klaviatuur on seotud kirjaviisiga
Soodustamaks Vadja keeles kirjutamist ja sedakaudu keele elavdamist, on magistritöö käigus ilmunud mõlemaid ortograafiaid toetav klaviatuuri\-paigutised arvutile ja nuti\-seadmeile (\cite{kankainen_annõmmõ_2019}). Vadja keele jaoks kasutatud erinevatest kirjaviisidest on kirjutanud pikemalt \cite{ernits_vadja_2010}.
% TODO lisada viide kankainen_annõmmõ_2019

% MORFOLOOGIA, seejärel sõnavara rikastamise kohta
Morfoloogia ühtlustamist iseloomustab siinse töö puhul sõnavormide rekonstrueerimine Jõgõperä murdeliste muutvormide põhjal Vaipooli murdealale iseäralike joontele vastavaks. Standardiseerimise suhtes ei võtta magistri\-töö süsteemset seisukohta, vaid on oma valikutes pigem kirjeldava iseloomuga. Jõgõperä ja Luuditsa murrete erinevustest on kirjutanud mh \cite{rozhanskiy_dialectal_2015}. Ühtlustamist kirjeldatakse lähemalt järgmises alajaotuses.

% SÕNAVARA moderniseerimine ja rikastamine 
Antud magistritöö ei pürgi vadja keele sõnavara moderniseerida ega rikastada -- hõlmatud sõnavara on varem juba avaldatud. Võimalikest vadja kirja\-keele sõnavara rikastamise eri viisidest on kirjutanud \cite{ernits_vadja_2010}. 
% TODO tõlgi artikli pealkiri ja kirjuta lahti siin
Vene laensõnade adapteerimisest Jõgõperä murdes ja selle problemaatikast on kirjutanud \cite{rozhanskiy_zaimstvovannyje_2009}.

% kas see jutt kuulub siia?
%\spacing{0}
%\begin{enumerate}
%\item lekseemide Jõgõperä-murdeliste muutvormide sisestamine ekstraktmorfoloogia %süsteemi, millest on saadud sõnatüübid
%\item saadud sõnatüüpide grupeerimine käändkondadesse
%\item käändkonniti sõnatüüpide ühtlustamine
%\end{enumerate}
%\spacing{1.5}

% TODO kuidas see ümberkirjutada ja selgeks teha?
Kuna magistritöö loob ühtse töökeskkonna, mille abil määratakse morfoloogia ainult sõnade muut\-vormide esitamise kaudu ja ei vaja programmeerimis\-oskusi, siis loodab siinne autor sellele, et edaspidine korpuse planeerimine võiks toimuda vadjalaste endi ja vadja keele huviliste eestvõttel -- muuda sõna muutvormi vormisõnastikus ja see muutub ka keele\-tehnoloogias.



\subsection{Sõnavara valik, paradigmade moodustamine ja ühtlustamine}
\label{sec:sõnavara-valik}

Sõnavara valik hõlmab Heinsoo õppematerjali sõnastikus esitatud nimi\-sõnu ja adjektiive (u~420 sõna). Sõnavara on veel laiendatud autori enda silma järgi Tsvetkovi sõnastikust ettesattunuga (u~460 sõna, mh~\vadja{asfal̕tti}, \vadja{bibli} ja \vadja{biblioteekkõ}). Kokku on vormi\-sõnastikusse koondatud 882 sõna\-artiklit.

Sõnaartiklitele on lisatud Tsvetkovi sõnaraamatus esitatud põhi\-käände\-vormid (\msd{sg nom}, \msd{sg gen}, \msd{sg par}, \msd{sg ill} ja  \msd{pl nom}, \msd{pl gen}, \msd{pl par}, harva ka \msd{pl ill}). Juhul kui Tsvetkovi sõna\-raamatus on esitatud mitut paralleelset põhikäändevormi, on eelistatud pikemaid vorme, nt \msd{sg ill} \vadja{asfal̕ttisõ} lühikese illatiivi asemel \vadja{asfal̕tti}. Niiviisi on morfoloogiliste sõnatüüpide kaardistamise protsessis välditud paralleelvorme, kuigi loodud lõpplahendus ehk arvutimorfoloogiline süsteem võimaldab sõna\-artiklitel esitada ka paralleelseid muuttüüpe. % TODO nagu Eesti arvutimorfoloogia traditsioonis on tehtud \cite[278]{viks_verbide_1976} ja SAMEST projektis aditiiv ja illatiiv lahku löödud
% TODO eelmisesse lisada ka pl part
% TODO kuigi see on avatud küsimus ja ilmselt muutub
% TODO ekstraktmorfoloogia meetod võimaldab paralleelvormide esitamise


Alljärgnevalt seletatakse põhikäändevormide ühtlustamise ja analoogiavormide koostamise põhimõtteid. % TODO ja mis veel seletatakse?
%Põhikäändevormidele on lisatud analoogiavormid \msd{sg ela} ... \msd{pl kom} Konkova õppematerjalides esitatut käändeid katmaks. Teised uurijad on ...


\subsubsection{Noomeni käänded ja ühtlustamine}



Käänete valiku ja muutelõppude puhul on järgitud Konkova õpikus esitatut (\cite[10]{konkova_vaddceeli_2014}), mis langeb kokku Heinsoo õppe\-materjalide põhimõtetega (\cite[88]{heinsoo_vadsonakopittoja_2015}). Käänded on näitlikustatud tabelis~\ref{tab:noomeni-käänded}.

\begin{table}[ht]
  \centering
  \begin{tabular}[t]{r l l l l l}
    % TODO vali paremad sõnad!
    % TODO vadja sõnad kursiivi!
     kääne & eespoolne & eespoolne & tagapoolne & tagapoolne & tagapoolne \\
    \hline
    \msd{sg nom} & pää & ärče & lafkõ & lammõz & ivuz \\
    \msd{sg gen} & pää & ärjä & lavga & lampa & ivusõ \\
    \msd{sg par} & pääte & ärčä & lafka & lammassõ & ivussõ \\
    \msd{sg ill} & pähhe & ärčäse & lafkasõ & lampasõ & ivussõsõ \\
    \msd{sg ine} & pääz & ärjez & lavgõz & lampaz & ivusõz \\
    \msd{sg ela} & pääss & ärjess & lavgõss & lampass & ivusõss \\
    \msd{sg all} & päälle & ärjelle & lavgõllõ & lampallõ & ivusõllõ \\
    \msd{sg ade} & pääll & ärjell & lavgõll & lampall & ivusõll \\
    \msd{sg abl} & päält & ärjelt & lavgõlt & lampalt & ivusõlt \\
    \msd{sg tra} & päässi & ärjessi & lavgõssi & lampassi & ivusõssi \\
    \msd{sg ter} & päässaa & ärjessaa & lavgõssaa & lampassaa & ivusõssaa \\
    \msd{sg com} & pääka & ärjeka & lavgõka & lampaka & ivusõka \\
    \msd{pl nom} & pääd & ärjed & lavgõd & lampad & ivusõd \\
    \msd{pl gen} & päije & ärčije & lafkojõ & lampajõ & ivussijõ \\
    \msd{pl par} & päit & ärčiit & lafkoit & lampait & ivussiit \\
    \msd{pl ill} & päise & ärčiise & lafkoisõ & lampaisõ & ivussiisõ \\
    \msd{pl ine} & päiz & ärčiiz & lafkoiz & lampaiz & ivussiiz \\
    \msd{pl ela} & päiss & ärčiiss & lafkoiss & lampaiss & ivussiiss \\
    \msd{pl all} & päille & ärčiille & lafkoillõ & lampaillõ & ivussiillõ \\
    \msd{pl ade} & päill & ärčiill & lafkoill & lampaill & ivussiill \\
    \msd{pl abl} & päilt & ärčiilt & lafkoilt & lampailt & ivussiilt \\
    \msd{pl tra} & päissi & ärčiissi & lafkoissi & lampaissi & ivussiissi \\
    \msd{pl ter} & päissaa & ärčiissaa & lafkoissaa & lampaissaa &  ivussiissaa \\
    \msd{pl com} & päika & ärčijka & lafkoika & lampaika & ivussijka \\
  \end{tabular}
  \caption{Noomeni käänded koos käändelõppudega ees- ja tagapoolse vokalismi kujul.}
  \label{tab:noomeni-käänded}
\end{table}

Hõlmatud käändeid on seega 24: ainsuse ja mitmuse nominatiiv, genitiiv, partitiiv, illatiiv, inessiiv, elatiiv, allatiiv, adessiiv, ablatiiv, translatiiv, terminatiiv ja komitatiiv.

% mis käänded on välja jäetud
Välja on jäetud essiivi, abessiivi, ekstsessiivi ja instruktiivi käänded, mida Ariste on pidanud produktiivsete käänetena \cite[17]{ariste_grammar_1968}. Markus ja Rošanski grammatika \cite{__2011} ei käsitle terminatiivi ja komitatiivi käänetena vaid käände ja järelasendi vahepealsetena, mille analüüsi nad on põhjendanud pikemalt \cite{markus_comitative_2014}.

% TODO vaata paragrahv üle
Alljärgnevalt seletatakse kuidas Tsvetkovi sõnastikust saadud põhi\-vormid on ühtlustatud ja mille põhjal analoogia\-käänded on moodustatud.



\paragraph*{Nominatiiv}
Tsvetkovi sõnaraamatus antud vormile on tüve\-lõpu\-vokaali puudumise korral tavaliselt lisatud lühike vokaal vastavalt vokalismile Heinsoo ja Konkova eeskuju järgi (\cite[88]{heinsoo_vadsonakopittoja_2015}, \cite[10]{konkova_vaddceeli_2014}).


\paragraph*{Genitiiv}
Tsvetkovi sõnaraamatus antud vormi lõpuvokaal on ühtlustatud vastavalt Heinsoo ja Konkova sõnastikes esitatule või rekonstrueeritud vastavalt sõna vokalismile.


\paragraph*{Partitiiv}
Tsvetkovi antud vormi lõpuvokaal on ühtlustatud Heinsoo ja Konkova sõnastikele vastavalt või rekonstrueeritud vastavalt sõna vokalismile. Mitmuse partitiivi puhul on eelistatud pikemat, muute\-lõpuga varianti puhta -\textit{i}-mitmusliku tüve asemel. Mõlemad vormid esinevad paralleelsete variantidena Heinsoo ja Konkova õppematerjalis (\cite[88]{heinsoo_vadsonakopittoja_2015}, \cite[10]{konkova_vaddceeli_2014}).


\paragraph*{Illatiiv}
Loodava kirjakeele ühtlustamise huvides on vormisõnastiku koostamisel eelistatud läbinähtava käändelõpuga vormi \textit{-se/-sõ} lühikese illatiivi asemel. Selline eelistus on siiski vastu\-olus olukorraga tänapäeva Luuditsa murdes, kus tavaliselt esineb lühike vorm ja pikk vorm esineb väga harva \cite[247]{markus_comitative_2014}. Heinsoo ja Konkova õppe\-materjalides on mõlemad vormid esitatud rööpselt (\cite[88]{heinsoo_vadsonakopittoja_2015}, \cite[10]{konkova_vaddceeli_2014}).


\paragraph*{Inessiiv}
On analoogiavorm ja selle muutevormid ainsuses on rekonstrueeritud \msd{pl nom} vormi alusel, kusjuures muutelõpp -\textit{d} on asendatud lõpuga -\textit{z}. Mitmuse vormid on rekonstrueeritud \msd{pl ill} põhjal.

% ``A characteristic feature of the Votic inessive is the fact that geminate stops -kk-, -pp-, -tt-, the geminate affricates -tts-, -ttš-, the geminate -ss-, and the consonant cluster -hs- always are in the strong grade before this case marker''
Ariste järgi on vadja keele inessiivile omaseks tunnuseks see, et geminaat\-klusiilid -\textit{kk}-, -\textit{pp}- ja -\textit{tt}-, geminaat\-afrikaadid -\textit{tts}- ja -\textit{ttš}-, geminaat -\textit{ss}- ning konsonant\-kluster -\textit{hs}- esinevad alati tugevas astmes (\cite[23]{ariste_grammar_1968}). Seda printsiipi järgides on \msd{sg ine} käände\-vormi vastavad tüved rekonstrueeritud tugeva\-astmelisteks.


\paragraph*{Elatiiv}
On analoogiavorm ja selle muutevormid ainsuses on rekonstrueeritud \msd{pl nom} vormi alusel, kusjuures muutelõpp -\textit{d} on asendatud lõpuga -\textit{ss}. Mitmuse vormid on rekonstrueeritud \msd{pl ill} põhjal.


\paragraph*{Allatiiv}
On analoogiavorm ja selle muutevormid ainsuses on rekonstrueeritud \msd{pl nom} vormi alusel, kusjuures muutelõpp -\textit{d} on asendatud lõpuga -\textit{lle} või -\textit{llõ} vastavalt sõna vokalismile. Mitmuse vormid on sarnaselt rekonstrueeritud \msd{pl ill} vormi põhjal.


\paragraph*{Adessiiv}
On analoogiavorm ja selle muutevormid ainsuses on rekonstrueeritud \msd{pl nom} vormi alusel, kusjuures muutelõpp -\textit{d} on asendatud lõpuga -\textit{ll}. Mitmuse vormid on rekonstrueeritud \msd{pl ill} põhjal.


\paragraph*{Ablatiiv}
On analoogiavorm ja selle muutevormid ainsuses on rekonstrueeritud \msd{pl nom} vormi alusel, kusjuures muutelõpp -\textit{d} on asendatud lõpuga -\textit{lt}. Mitmuse vormid on rekonstrueeritud \msd{pl ill} põhjal.


\paragraph*{Translatiiv}
On analoogiavorm ja selle muutevormid ainsuses on rekonstrueeritud \msd{pl nom} vormi alusel, kusjuures muutelõpp -\textit{d} on asendatud lõpuga -\textit{ssi}. Mitmuse vormid on rekonstrueeritud \msd{pl ill} põhjal.


\paragraph*{Terminatiiv}
On analoogiavorm ja selle muutevormid ainsuses on rekonstrueeritud \msd{sg ine} vormi alusel, kusjuures muutelõpp -\textit{z} on asendatud lõpuga -\textit{ssaa}. Mitmuse vormid on rekonstrueeritud \msd{pl ill} põhjal.

Ariste järgi võib terminatiiv põhineda illatiivi või allatiivi tüvele (\cite[34]{ariste_grammar_1968}). Markus ja Rožanski järgi põhineb see tänapäeva keeles genitiivi tüvele ja harva allatiivi tüvele (\cite[247]{markus_comitative_2014}).

Ülaltoodud märkuste järgi võib magistritöös valitud \msd{sg ine} tüve põhjal konstrueeritud muutevorm olla vale, kuigi valik ühtib Konkova ja Heinsoo õppe\-materjalides esitatuga, so tugeva\-astmelise geminaadiga.


\paragraph*{Komitatiiv}
On analoogiavorm ja selle muutevormid ainsuses on rekonstrueeritud \msd{pl nom} vormi alusel, kusjuures muutelõpp -\textit{d} on asendatud lõpuga -\textit{ka}. Mitmuse vormid on rekonstrueeritud \msd{pl ill} põhjal, kusjuures -\textit{i}-tüveliste sõnade puhul on moodustunud pikk i muudetud -ij- vastavalt Konkova esitatud vormile (-\textit{-ijka}) (\cite[10]{konkova_vaddceeli_2014}).






\subsection{Morfeemi staatus ja definitsioon}
\label{sec:morfeemi-staatus}


Magistritöö meetodis ei käsitleta morfeemi levinud lingvistilise seisukoha järgi kui \textit{väikseimat tähenduslikku üksust}, vaid järgib definitsiooni, et morfeem on lekseemi \textit{mistahes fonoloogilise kuju muutust, millega kaasneb tähenduslik muutus} (\cites[31]{beard_morpheme_1987}[49]{beard_lexeme-morpheme_1995}), %Morfeemipõhist suunda ajab nt \cite{stump_inflectional_2001}.
mis on sarnane klassikalisele paradigmaatilisele morfoloogia lähenemisele.

% TODO ÜLALPOOL LISADA VIITEID? ALLPOOL ON NII:
%Beardi teoorias ei ole morfeem grammatiliselt tähenduslik, vaid defineeritud kui mistahes muutusena lekseemi fonoloogilises kujus (\cite[31]{beard_morpheme_1987}). Seega on tema teoorias ainult lekseemid tähenduslikud märgid ning grammatilised afiksid (morfeemid) on seda vaid sattumuslikult (\cite[17]{beard_morpheme_1987}).



\subsection{Klassikaline paradigmaatiline morfoloogia}

%Nii Beard'i teoorias kui ka klassikalises paradigmaatilises lähenemises on ainult lekseemid tähenduslikud märgid ning grammatilised afiksid (morfeemid) on seda vaid sattumuslikult (\cite[17]{beard_morpheme_1987}, \cite[189]{matthews_morphology_1991}).

Matthews tõstab esile kaks paradidmaatilise morfoloogia käsitlust, klassikaline ja uus ehk strukturalistlik. Kahe käsitluse erinevuseks on see, et uus opereerib morfeemidel aga vana opereerib tervikutel sõna\-vormidel (\cite[196]{matthews_morphology_1991}).

% relate words as wholes 186
Klassikaline paradigmaatiline morfoloogia käsitleb sõna selle sõnavormide komplektina (\cite[186]{matthews_morphology_1991}).

Keele väikseimaks tähenduslikuks üksuseks on sõnavorm tervikuna. Sõnavormide grammatilised tähendused on klassikalise teooria järgi vaid sattumused (ingl. \textit{accidents}) (\cite[189]{matthews_morphology_1991}). % \cite[17]{beard_morpheme_1987}

Vanades ladina keele õpikutes ja grammatikates võidi esitada reegleid, mille abil ühe lekseemi kõik muutvormid moodustada. Reeglid opereerisid ainult sõnavormide tähtkoostisel. Näiteks võis reegel ühe sõnavormi lõputähti asendada muude tähtedega, et saada teine sõnavorm. Et asendatavatele tähtkoostistele ei omandatud mingit tähendust, näitlikustab ka see, et mõne reegli algvormiks võidi valida suvaline sõnavorm, mis oma tähtkoostise poolest kõige paremini sobis. (\cite[195-196]{matthews_morphology_1991}).

Magistritöös rakendatud ekstrakt\-morfoloogia meetod koosneb sarnastest sõna\-vormi koostamis\-reeglitest, nn muutvormi\-mallidest. Meetodit tutvustatakse pt~\ref{sec:ekstraktmorfoloogia-meetod}.

%Siiski on vadja keelel hulganisti lingvistilisi kirjeldusi, nagu grammatikaid (mh \cites{ahlqvist_wotisk_1856}{airila_vatjan_1934}{tsvetkov_vadja_2008}{ariste_grammar_1968}{__2011}), sõnaraamatuid (mh \cites{tsvetkov_vatjan_1995}{ariste_vadja_1943}{laakso_vatjan_1989}{raag_dictionary_1982}{pomberg_vadja_1991}{grunberg_vadja_2013}{heinsoo_vadsonakopittoja_2015}) ja ka etnograafilisi töid (mh \cites{kass_kasitoo-_1961}{malk_vadja_1977}).


%Käesolev töö ei pürgi looma lõplikku normatiivi, kuivõrd ta loob süsteemi, mis oskab vastata morfoloogilistele küsimustele. Aga loodud süsteemi peamine eesmärk on siiski võimaldada muuta ja jätkata tööd normatiivi arendamiseks ja mille ümber saaks keeleaktivistid ise koonduda, ilma et selleks oleks niivõrd vaja ei lingvistilist ega keeletehnoloogilist spetsialisti.

%Püüd luua vadja morfoloogiale normatiivne alus lihtsustab paljudele küsimustele vastusi leida, nt mis käändeid arvestada. Siiski on tööga loodud \textit{keele\-tehnoloogia tuletamise süsteem} avatud ka teistsugustele lähenemistele keeleainesele.




%% \subsection{Arvutimorfoloogia eesmärk ja lingvistiline motiveeritus}
%% 
%% % TODO leia allpool õige koht sellele: Kui lingvistika üldine eesmärk on leida ja kirjeldada keelenähtuste reeglipärasusi, siis jääb lahtiseks küsimus mis on arvuti\-lingvistika eesmärk -- kas see on formaliseerida lingvistika poolt leitud reeglipärasused või võib see mahutada ka nende reeglipärasuste leidmist? See töö lähtub arusaamast, et arvuti on abivahend lingvistile, mitte ei ole ainult lingvisti(ka) formalisatsioon.
%% % Seda tehakse kahel põhjusel -- esiteks puudub täielik morfoloogiakirjeldus vadja keelele. Teine põhjus on arvutiinsenerlik põhjus. Siinne töö maht ei luba täielikku morfoloogiakirjeldust, pealegi on keeled ajas muutuvad ja tuleb täiendusi teha. Seetõttu ei taha siinne töö esitada sellist formaalset morfoloogiakirjeldust, mida keegi teine ei saaks täiendada. Iga formalismi puhul kaasneb õpikõver, et sellest üldse aru saada, siinne töö esitab arvutimorfoloogia võimalikult formalismivabalt --- vormisõnastiku kujul --- milles sisalduva sõnavara on võimalik igal-ühel täiendada ja muuta ainult sellekaudu, et muuta konkreetse sõna sõnavormi.
%% % Tehnoloogiavaba(m) kirjeldus on standartne, ja keeletehnoloog, kes tahab lisada uue ... võib seda teha, kirjelduse formaat on kirjeldatud rahvusvahelise standardi dokumentatsioonis.
%% 
%% % Ehk alustada sellega: Kuna töö kasutab läbivalt sõna 'morfoloogia' natuke teistsuguses tähenduses kui see tavaliselt keeleteaduses kannab, tuleb selle tähendust kõigepealt lahti seletada. . Töö loobub morfeemist kui keele väikseim tähenduslik üksus ja seega ei tähenda siin töös morfoloogia morfeemilist morfoloogiat. Aronoff on pööranud tähelepanu tõigale, et 'morfoloogia' kannab teistsugust tähendust keeleteaduses kui ta seda kannab teistes teadusharudes. Enamik teadusharudes tähendab 'morfoloogia' umbes "võimalike vormide uurimine-kaardistamine". Keeleteaduses hoopis ...
%% 
%% %Aronoff on pööranud tähelepanu tõigale, et termin \emph{morfoloogia} on keeleteaduses alati erinenud selle tavatähendusest 'õpetus vormidest', mida termin kannab nt bioloogias ja geoloogias. Terminil on kitsam tähendus keele\-teaduses: 'grammatika see osa, mis tegeleb sõnavormide muutmisega ja sõna\-moodustamisega' ja ka filoloogias: 'keele grammatilise struktuuri üldised reeglid'. (\cite[1]{aronoff_morphology_1996})
%% % Aronoffi puänt on see, et miks kitsamalt ainult \emph{sõnade} morfoloogia, miks mitte ülejäänud keeleline vorm?
%% 
%% % Elektroonne vormisõnastik moodustab seega iga lekseemi jaoks relatsiooni ehk seose $(tsitaatvorm, {muutevormid koos vastava grammatilise iseloomustusega})$.
%% %Sellist vormisõnastikku võib moodustada erinevatel viisidel. Näiteks leksikaalse andme\-baasina, kus iga lekseemi puhul on nenditud kõik selle muutevormid koos vastava grammatilise iseloomustusega, või näiteks reeglite komplektina, mida rakendades saab koostada lekseemi muutvorme vastavalt nende grammatilistele iseloomustustele.
%% 
%% %Matemaatilises mõttes kujutab vormisõnastik vaid \textit{seost} muutevormide ja nende vastavate grammatiliste iseloomustuste vahel. 
%% 
%% % arvutimorfoloogia kui arbitraarne valik realiseerimaks vormisõnastiku sisu (kas info pakkimismehanism või lingvistiliselt motiveeritud)
%% %Arvutimorfoloogiad võivad seda seost (või vormisõnastiku funktsionaalsust) realiseerida arvutuslikult erinevatel viisidel ja ei pea olema lingvistilis-grammatiliselt motiveeritud. Kuna üks täielik vormisõnastik on mahult niivõrd suur (kui mitte lõpmatult suur), on selle mahu kompaktsem ja ülevaatlikum esitus peamiseks motivatsiooniks organiseerida selle koostamise reeglite abil, mis on ühel või teisel moel põhjendatud lingvistiliste-grammatiliste reeglipärasustega.
%% 
%% 
%% %\subsubsection{Morfoloogia formaalsete teooriate lingvistiline motiveeritus}
%% % pilk ajalukku (mida hiljem ümberlükata kui strukturalistlik-morfeemiline, aga mis on jätnud jälje ka arvutimorfoloogiate mõtemaailmale kui lingvistiliselt realistlikud)
%% %Eelmise sajandi keskpaiku jagas Charles Francis Hockett kõik seni Ameerikas sajandi algusest saadik ilmunud grammatikad kahe üldise mudeli järgi, IA (ingl. \textit{Item-and-Arrangement}, üksus ja distributsioon v järjestus v korraldus) ja IP (ingl. \textit{Item-and-Process}, üksus ja protsess ehk protsessi\-morfoloogia). Kõrvalmärkusena tõi ta välja ka kolmanda, ``vanema ja väärikama'' mudeli, WP (ingl. \textit{Word-and-Paradigm}, sõna ja paradigma), aga jättis selle oma käsitlusest välja (\cite[210]{hockett_two_1954}). Hockett võrdleb IA ja IP mudelite eeliseid ja argumenteerib, et IA toonane populaarsus seisneb eeskätt selles, et ajastu eelistab formaalseid mudeleid. Kuna IA-mudel oli juba formaliseeritud tahtis Hockett nüüd formaliseerida sellest vanema IP-mudeli (\cite[214]{hockett_two_1954}) ning sellest sai hiljem, Fred Karlssoni sõnade järgi, generatiivse lingvistika peamiseks mudeliks (\cite[126]{karlsson_uldkeeleteadus_2002}).
%% 
%% %IP-mudel põhineb (morfoloogilise) protsessi mõistel, millega ühest algvormiks valitud kujust (ingl. \textit{base}) luuakse teine vorm (\cite[210]{hockett_two_1954}). IA tekkis vastureaktsioonina IP protsessi\-mõiste suunalisusele -- enam ei tahetud tõsta esile üht vormi algsemaks teistest vormidest (\cite[211]{hockett_two_1954}). IA põhineb morfeemi mõistel, mida Hockett iseloomustab kui keele väikseimat grammatiliselt olulist üksust, ja selle distributsiooni määramisel (\cite[212]{hockett_two_1954}). Hocket nendib, et ka IA mudeli puhul tuleb siiski teha kohati suvalisi valikuid selle üle, mis kuulub morfeemi tasandile ja mis kuulub distributsiooni tasandile (\cite[212]{hockett_two_1954}).
%% 
%% % Stumpi neljamõõtmeline jaotus
%% Gregory Stump on arendanud Hocketti IP ja IA kaheks\-jagamise klassifikatsiooni edasi tänapäevaste morfoloogiliste teooriate põhjal. Nimetades IAd ümber leksikaalseks (ingl. \textit{lexical}) ja IPd inferentsiaalseks (ingl. \textit{inferential}) lisab ta klassifikatsioonile veel sisemise telje: inkrementaalsed (ingl. \textit{incremental}) ja realiseerivad (ingl \textit{realizational}) teooriad. (\cite{stump_inflectional_2001}, lk 1-2)
%% 
%% Inkrementaalsete teooriate järgi lisandub iga (olgu IA puhul leksikaalselt loetletud või IP puhul inferentsiaalse reegliga tuletatud) morfosüntaktilise tunnuse puhul sõnale ka selle vormiline eksponent (\cite[2]{stump_inflectional_2001}). Vormilised eksponendid on üks-üheses seoses grammatiliste tunnustega ja need väljenduvad ükshaaval elik inkrementaalselt.
%% 
%% Realiseerivate teooriate juures ei pea vormiline eksponent iga morfosüntaktilise tunnuse puhul eraldi ja koheselt väljenduma, vaid vormiline väljendus võib realiseeruda tunnuste suuremate komplektide puhul või üldse kui sõna kõik tunnused on teada (\cite[2]{stump_inflectional_2001}).
%% 
%% Realiseerivad teooriad võimaldavad niiviisi suurema paindlikkuse vormiliste väljendujate \textit{realiseerimisel}, loobudes vormiliste väljendujate üks-ühesest seosest morfosüntaktiliste tunnustega.
%% 
%% Stumpi jagab oma klassifikatsiooni järgi Lieberi morfoloogilise teooria leksikaalseks ja inkementaalseks. Halle ja Marantzi distributsioonilise morfoloogia teooria leksikaalseks ja realiseerivaks. Steele'i artikuleeritud morfoloogia teooria esindab inferentsiaalset ja inkrementaalset suunda. (\cite[2--3]{stump_inflectional_2001}).
%% 
%% Stumpi enda ja Matthewsi, Zwicky ning Andersoni teooriaid nimetab ta WP teooriateks, mis on inferentsiaalsed ja realiseerivad (\cite[3]{stump_inflectional_2001}).
%% % TODO siin oleks hea pöörata KLASSIKALISELE PARADIGMAATILISELE MORFOLE
%% 
%% % Finally, Word-and-Paradigm theories of inflection (e.g. those proposed
%% % by Matthews (), Zwicky (a), and Anderson ()) are of the
%% % inferential–realizational type. In inferential–realizational theories, an
%% % inflected word’s association with a particular set of morphosyntactic prop-
%% % erties licenses the application of rules determining the word’s inflectional
%% % form; likes, for example, arises by means of a rule appending -s to any verb
%% % stem associated with the properties ‘sg subject agreement’, ‘present tense’,
%% % and ‘indicative mood’.
%% 
%% Robert Beard on nimetanud ülaltoodud viimaste autorite arendatud realiseerivaid teooriaid eru-morfoloogiaks (ingl. \textit{'split' morphology}) (\cite[20]{beard_morpheme_1987}) ja pakkunud välja morfoloogia veel võimsama eraldamise, mis põhineb tema morfoloogia lahususe hüpoteesil (ingl. \textit{Separation Hypothesis}) (\cite{beard_lexeme-morpheme_1995}).
%% 
%% % ird-morfoloogia VS morfoloogia lahususe hüpotees
%% Morfoloogia lahususe hüpoteesil põhinevate teooriate ja realiseerivate (eru-)morfoloogia\-teooriate vahe on fundamentaalne ja lähtub nende käsitlusest süntaksi ja semantika vahekorrast. Kõige ilmekalt paistab nende vahe morfeemi definitsioonis, küsimuses kas morfeem on keele väikseim vormiline tähenduslik üksus või mitte.
%% 
%% Beardi teoorias ei ole morfeem grammatiliselt tähenduslik, vaid defineeritud kui mistahes muutusena lekseemi fonoloogilises kujus (\cite[31]{beard_morpheme_1987}). Seega on tema teoorias ainult lekseemid tähenduslikud märgid ning grammatilised afiksid (morfeemid) on seda vaid sattumuslikult (\cite[17]{beard_morpheme_1987}).
%% 
%% Käesolevas magistritöös rakendatud ekstraktmorfoloogia on oma organisatsiooni suhtes sõna ja paradigma mudel, aga selle käsitus morfeemist on lähedasem Beardi teooriale.
%% 
%% \subsubsection{Arvutimorfoloogiate lingvistiline motiveeritus}
%% 
%% % Karttust ja Koskenniemit tuua mängu sisse alles siis, kui on vaja näidata kuidas arvuti poolel on hakatud asju tõlgendama enda mõtteviisi järgi
%% Arvutilingvistikas on arvutimorfoloogiat üldiselt organiseeritud klassikalise morfeemi\-käsituse järgi. Seda ilmestab hästi 
%% % Karttuneni väljakutsed arvutimorfoloogias
%% Lauri Karttunen, kes nendib inimkeele mudeldamise puhul arvutimorfoloogias kaks väljakutset: 1)\nobreakspace morfotaktika ehk sõnast väiksemate üksuste kombineerumine ja 2)\nobreakspace morfoloogilised vaheldused ehk sõnast väiksemate üksuste kuju olenemine nende ümbritsevast kontekstist (\cite{karttunen_computing_2003}).
%% 
%% Mille mõlemad väljakutsed viitavad otseselt klassikalisele morfeemi\-käsitusele.
%% 
%% Karttuneni artikkel on vastus Stumpi teooriale ja ta näitlikustab selles kuidas Stumpi teooria on võimalik rakendada kasutades lõplike automaatide formalismi.
%% 
%% Karttunen toob välja olukorra, et arvuti\-morfoloogiad põhinevad arvutuslikel formalismidel, millega nad implementeerivad morfoloogiaid ja mitte ei põhine otse mingil lingvistilisel teoorial. Ta ütleb et morfoloogia\-uurija üllitiste peamine eesmärk on olla veenev, et tema teooria annab läbinägelikuma (ingl. \textit{insightful}) ja elegantsema kirjelduse kui teised teooriad ja formalismid (\cite[2]{karttunen_computing_2003}). Praktilised küsimused nagu sõnavaraline katvus, arvutus\-kiirus ja mälu\-maht ei ole relevantsed akadeemilisele morfoloogia\-uurijale (\cite[2]{karttunen_computing_2003}).
%% 
%% Seega võib öelda, et arvutimorfoloogia on laiem kui lingvistiline morfoloogia, kuna esimest ei piira mitte teooria, vaid arvutusliku meetodi võimsus. Karttunen tõestab artiklis, et Stumpi inferentsiaalne-realiseeriv teooria on taandatav lõplike automaatide formalismi arvutusvõimsusele.% (\cite{karttunen_computing_2003}). % TODO vaata kas viide tõesti hõlmab tervet paragrahvi
%% 
%% Sellest võib järeldada, et arvutilingvistikas on lingvistilise teooria roll pigem olla ajendiks kui tõetruuks postulaadiks, kuigi kindlasti on teooria ja selle implementatsioonilise praktika vahekord raskesti eraldatavad ja ajas muutuvad. Kuigi tendentsi tõetruuduse vähenemisele võib siiski täheldada tänapäeval ka Kimmo Koskenniemi töös, kus ta on hiljuti oma taandatud kahetasemelises morfoloogiamudelis püüdnud morfofoneemi mõiste juures loobuda selle tähendusliku külje lingvistilisest realismist, omastades seda puhtalt vormile:
%% \begin{quote}
%%   ``\textit{Morphophonemes} are represented just as the \textit{combinations of the corresponding letters} (or phonemes) which we can observe in the surface forms. On the one hand, such an interpretation of morphophonemes is crude, but on the other hand, it is a fact that anybody can observe.'' (\cite[157]{koskenniemi_informal_2013})
%% \end{quote}
%% 
%% \subsubsection{Sügavam epistomoloogiline põhjus: formaalse lingvistika lingvistiline motiveeritus}
%% % induktsioon ja deduktsioon
%% Sügavama epistemoloogilise põhjuse, miks arvutimorfoloogiaid on ajendanud pigem lingvistiline motivatsioon ja mitte arvutusteoreetilised võimalused, arvab siinkirjutaja leiduvat strukturaalse lingvistika formaliseerimisperioodi alguses, mis algas enne arvutusmasinate leiutamist (1940.--1960.-ndateil aastatel) ja ammu enne arvutite arvutus- ja mälumahtuvuse võimsuse plahvatuslikku suurenemist (1980.--2000.-ndail). % TODO lisada Karttuneni ajalooline ülevaade fst morfoloogia arengutest
%% 
%% % lingvistilise teooria formaliseerimine
%% Formaalseid teooriaid ja seega teooriate formaliseerimist peetakse teaduse lipulaevaks (\cite[2026]{auroux_history_2006}). Teooriate formaliseerimis\-protsessi jagab Pieter Seuren neljaks etapiks, kus esimene koosneb uuritava ainese tüüpide (ehk kategooriate) leidmisest ning nendele esitus\-kuju määramisest (\cite[2027]{auroux_history_2006}). (Teisisõnu tegeleb see \textit{type-token distinction}'i probleemiga). Teine etapp käib sageli käsi-käes esimese etapiga ja hõlmab tüüpide taksonoomia määramist, ehk selle määramist, mis andmed kuuluvad mis tüübi alla millal ja mis tingimustes (\cite[2027]{auroux_history_2006}). Kolmas etapp koosneb struktuuri määramisest tüüpide esinemisele, elik kuidas kategooriaid on võimalik omavahel kombineerida (\cite[2027 jj]{auroux_history_2006}) näiteks puu- või sõltuvus\-struktuuride abil. Neljas ja viimane etapp koosneb ühe ennustava ja kirjeldava väärtusega formaalse teooria ülesseadmisest algoritmina ehk sammsammulise tegevusjuhisena (\cite[2031]{auroux_history_2006}).
%% %final Stage 4, which consists in the setting up of a formal predictive and explanatory theory that has the precision of an algorithmic procedure.
%% 
%% % arvutimorfoloogia on arvutiprogramm mis on formaalne aparaat
%% Arvutimorfoloogia on arvutiprogramm (või mitme programmi komplekt), mis tahest-tahtmata hõlmab seelaadset formaalset sammsammulist tegevusjuhist.
%% 
%% % probleem asub 3 ja 4 etapi vahel -- kas teha deduktiivselt või induktiivselt?
%% Probleem, miks arvutimorfoloogiad juhinduvad lingvistilistest teooriatest ja mitte puht-arvutuslikest võimalustest asub formaliseerimis\-protsessi 3. ja 4. etapi vahel. Millisel viisil tuleb põhjendada struktuuri määravaid reegleid?
%% 
%% % zellig
%% % Generatiivse lingvistika suurkuju Noam Chomsky juhendaja Zellig Harris kirjutab
%% Zellig Harris (kes oli Noam Chomsky juhendaja) kirjeldab oma \textit{magnum opus} teoses grammatika formaliseerimise lähenemist, mis põhjendab strukturaalsete reeglite määramise ühe formaalse avastamis\-menetluse abil keeleainese korpus\-esinemustest. See on, formaalse teooria sammsammulised reeglid tuletatakse puhtalt struktuuride esinemistest korpusanalüüsi teel. Selline väga töömahukas grammatika loomise menetlusviis sai tema kaasaegsetelt kõva kriitikat olles nii ilmselgelt ebarealistlik ja ebapraktiline. Harris oli tundlik kriitikale ja mainib oma raamatu lõpus viisi, kuidas korpus\-esinemustest eraldi püstitatud reegleid saab hoopis vastupidises suunas \textit{testida} korpustekstide peal. See pani aluse generatiivsele grammatikale, mida arendas edasi tema kasvandik Noam Chomsky teoses \textit{Syntactic Structures} (\citeyear{chomsky_syntactic_1957}). (\cite[2031]{auroux_history_2006}).
%% 
%% % seletav tekst miks ma jauran
%% Eelnevaga olen ma tahtnud öelda seda, et arvutimorfoloogiate koostamis\-põhimõtted põhineda morfeemil ja morfotaktilistel reeglitel ja mitte puhtalt muutvormide nentimisel korpuse põhjal, on eeskätt ajalooliste traditsioonide järjepidevus. Käesolev töö ei järgi neid traditsioone.
%% 
%% See traditsioon on kristalliseerunud ka pealkirjas ``This volume grows out of a special session that we organized at the January 2009 Annual Meeting of the Linguistic Society of America entitled ``Computational Linguistics: Implementation of Analyses against Data''.'' (\cite{bender_computational_2010}).
%% 
%% % Matthews kirjeldab WP ``mudelit'' kui ... ja Karttuneni järgi kronoloogia Zwicky kaudu Stumpini, aga me ei peagi laskuma WP mudeli arvutusliku külge juurde -- see on väga lihtsal moel lahendatud ekstraktmorfoloogias. Ja Karttuneni konstateering, et Stumpi mudel on arvutuslikult samaväärne FSTga. Ja Roark ja Sproat argumenteerivad, et X ja Y on samuti taandatavad FST-le ja seega samaväärsed. Olen valmis oma argumentatsiooniga siin peatükis, et arvutimorfoloogiad võivad vormisõnastiku funktsionaalsuse realiseerida ilma lingvistilise motivatsioonita. Nagu Stump arvab, et tuleb eelistada ``A theory of inflectional morphology must be preferred to the extent that it minimizes any dependence on theoretical distinctions which are not empirically motivated.'' (\cite{stump_inflectional_2001} lk 9).
%% 



\subsection{Muuttüüp, tüüpsõna ja sõnatüüp}

Selles töös kasutatakse termineid \emph{muuttüüp}, \emph{tüüpsõna} ja \emph{sõnatüüp} sama tähendusega ja osutab lekseemi kõiki muutvorme. \emph{Tüüpsõna} on valitud lekseem, millega nimetatakse üht \emph{sõnatüüpi}. \emph{Sõnatüüp} on seega tüüpsõnast üldisem mõiste ja tähistab muutvormide koostamismalle, millega saab moodustada sellesse sõnatüüpi kuuluvate sõnade kõiki muutvorme.
%% 
%% % seda selle tõttu, sest töö põhined muuttüübil
%% % heal lapsel mitu nime, paar sõna 
%% 
%% Eesti traditsiooni järgi on muuttüüp tüüpsõnast üldisem. Kuidas siin töös terminoloogiliselt ümber käia, kas \textit{muuttüüp} või \textit{tüüpsõna\-mall}?
%% 
%% Muuttüübistik sõltub selle aluseks võetud klassifikatsioonist, ekstrakt\-morfoloogiat võiks vaadata kui lihtsalt üht väga formaalselt defineeritud muuttüübistikku.
%% 
%% Huldenil on omakorda üks väga formaalne viis, kuidas vähendada ekstrakt\-morfoloogiaga leitud muuttüüpide arvu. Kas see on hoopis muuttüübistik?
%% 
%% 





\newpage

\section{Vormisõnastiku sisestamine Morfoloogialabori rakenduse abil}
\label{sec:vormisõnastiku-sisestamine}

Morfoloogialabor\footnote{https://spraakbanken.gu.se/morfologilabbet/} on eraldiseisev veebi\-rakendus morfoloogiliste sõnaraamatute koostamiseks. See on loodud Språkbanken'is\footnote{Rootsi keelepank ehk keele\-ressursside keskus} prototüübina, et integreerida ekstraktmorfoloogia nende leksikaalsesse taristusse Karp.

Labor põhineb ekstraktmorfoloogial ja lihtsustab uute sõnade morfoloogilise informatsiooni lisamist sõnaraamatu koostamisel. Rakendus ennustab sisestatud sõnale sobivaid muuttüüpe ja moodustab muuttüübile vastavalt kõik muutvormid. Kasutaja peab kontrollima kas muutvormid on õiged või valima teine muuttüüp. Rakendus on ühendatud korpuspäringute süsteemiga KORP ja näitab muutvormi juures kasutajale selle esinemis\-arvu korpustes.

Magistritöö jooksul on koostatud ja kasutatud kaks vadja korpust: kirjakeelne korpus, mis koosneb Heinsoo õpematerjalide tekstidest ja murdeid kajastav korpus, mis koosneb vadja sõnaraamatu \cite{grunberg_vadja_2013} näitelausetest. Korpused on päritavad Eesti Keeleressursside Keskuse korpuspäringute süsteemist KORP\footnote{https://korp.keeleressursid.ee/?mode=vadja} ja avaldatud \cite{kankainen_keeleleek/votic-corpora:_2018}.

Morfoloogialaboriga salvestatud andmed ja nende kuju vastab suuresti pt~\ref{sec:lmf} kirjeldatule. Siinkirjutajal oli au viibida Språkbanken'is prototüübi ehitamise ajal (2018.~a kevadel) ja sai panustada andmete kuju standardiseerimisega.


Magistritöö vadja vormisõnastiku sisestamise töövoog on olnud:
\begin{enumerate}
\item sisestada lekseemi põhivorm
\item valida olemasolev tüüpsõna olemas\-olevate sõnatüüpide põhjal
\item sõnatüübi (ja tüüpsõna) puudumise juhul sisestada lekseemi kõik muutvormid
\end{enumerate}

Punkti (3) juhul loob rakendus ekstrakt\-morfoloogia abil uue sõnatüübi, mis on juba järgmise sisestatava lekseemi puhul süsteemi poolt ennustatav ja valitav.

Töö algusfaasis ja tühja andmestikuga alustamise tõttu ei ole loomulikult olnud võimalik valida paljudele sisestatud lekseemidele õiget sõnatüüpi ning on pidanud sisestama kõik muutvormid. Töö jooksul on sõnatüüpide arv kasvanud ja rakenduse abil on saanud valida õige sõnatüüp, mis on töö\-protsessi kiirendanud. Tihti on ka olnud abiks valida nt vokalismi suhtes vale sõnatüüp ja genereeritud muutvormides ainult korrigeerida vokalismi kajastavad osad.

Kuivõrd magistritöö jooksul on vadja sõnatüüpe ühtlustatud, on seda tehtud tagant-järgi valides juba salvestatud sõnadele uus sõnatüüp. Kuna seda on pidanud tegema kõikide sõnade puhul, mis kuuluvad uue, ühtlustatud sõnatüübi alla, on see olnud aega\-nõudev protsess. Ühtlustamise ajal tekkinud mõtteid, kuidas seda protsessi lihtsustada ja kiirendada, on edastatud Språkbanken'i arendajatele.


\newpage
\section{Ekstraktmorfoloogia meetod}
\label{sec:ekstraktmorfoloogia-meetod}

%Ekstraktmorfoloogia on juhendatud masinõppe meetod, mis (\cite[14]{forsberg_what_2016}) on välja pakkunud kui lihtsama ja loomulikuma viisina lingvistil määratleda arvutimorfoloogia, kui seda on juhinduda morfoloogilistest-grammatilistest kirjeldustest.
% Meetodit on lähemalt tutvustanud \cite{ahlberg_semi-supervised_2014} ja \cite{ahlberg_paradigm_2015}. % need viited tulevad ju allpool välja

%Meetod koosneb kahest komponendist: \emph{sõnatüübi eraldamise meetodist} ja tundmatule sõnale õige \emph{sõnatüübi ennustamise meetodist}.
% TODO kuhu? Magistritöös rakendatakse sõnatüübi eraldamise osa ja ennustamist ainult kaudselt

% ------

% mis on ekstmorf
Ekstraktmorfoloogia on juhendatud masinõppe meetod, mis koosneb kahest komponendist: \emph{sõnatüübi eraldamise meetodist} ja tundmatule sõnale õige \emph{sõnatüübi ennustamise meetodist}. Meetod on \textit{juhendatud}, sest sisestatud andmed, ehk muutvormi\-tabelid, peavad olema korrektsed.

% TODO mis on sõnatüüp? (tegelt juba pt 2.6)

Meetod on välja pakutud kui lihtsama ja loomulikuma viisina lingvistil määratleda arvutimorfoloogia, kuna see põhineb lekseemide sõnavormide esitamisel ja ei juhindu morfoloogiliste-grammatiliste reeglite kirjutamisest (\cite[14]{forsberg_what_2016}).

% mis on eraldamine
Sõnatüübi eraldamise (ekstraheerimise) korral eraldab meetod lekseemi muutvormide tabelist selle \glslink{tehniline-tüvi}{tehnilise tüve} ja \glslink{sõnatüübimall}{sõnatüübimalli}. Meetodi seda osa kirjeldatakse lähemalt pt-s~\ref{sec:ekstraktmorfoloogia-eraldamine}.
%Eraldatud sõna\-tüübi\-mallide põhjal tuletatakse magistritöös programmkoodi, mida kirjeldatakse lähemalt pt-s~\ref{sec:programmkoodi-tuletamine}.

% mis on ennustamine
Eraldatud sõnatüübimallide ja tehniliste tüvede põhjal on võimalik üldistada nende iseärasusi ja luua statistiline ennustus\-mudel. Ennustus\-mudeliga on võimalik määrata tundmatu sõnavormi kuuluvust ühe või teise sõnatüübi alla. Meetodi seda osa kirjeldatakse lähemalt pt-s~\ref{sec:ekstraktmorfoloogia-ennustamine}.

Magistritöös kasutatakse ekstraktmorfoloogia meetodit eesmärgiga, et luua vadja keelele keele\-tehnoloogiat. Selleks otstarbeks tuletatakse eraldatud sõnatüübi\-mallide põhjal programm\-koodi, mis teostab mallidest tuleneva sõnavormide analüüsimise ja sünteesimise võime. Programm\-koodi automaatset tuletamist kirjeldatakse pt-s~\ref{sec:programmkoodi-tuletamine}.


% See osa kirjeldab töös rakendatud ekstraktmorfoloogia meetodit. Töö kasutab ekstraktmorfoloogiat kaheks otstarbeks, esiteks vadja keele morfoloogiliste tüüpsõnade väljaselgitamiseks ja kirjeldamiseks ja teisalt programmkoodi automaatseks tuletamiseks saadud kirjelduse põhjal. Neid kahte rakendust kirjeldatatakse lähemalt vastavates peatükkides \textit{\nameref{sec:analüüs}} ja \textit{\nameref{sec:programmkoodi-tuletamine}}.




\subsection{Sõnatüübi eraldamine}
\label{sec:ekstraktmorfoloogia-eraldamine}
% võiks lihtsalt mainida, et opereerib tähtkoostisel
Sõnatüübi eraldamise mehhanism põhineb tehnilise tüve tuvastamisel. Tehniline tüvi moodustub nendest tähtedest, mis ilmnevad lekseemi igas muutvormis.

Veel ilma detailidesse takerdumata näitlikustatakse siinkohal lugejale meetodi sisendit ja väljundit. Sisendiks on ühe lekseemi muutvormitabel tervikuna (tabel~\ref{tab:sisendtabel-katto}). Väljundiks on meetodi poolt leitud tehniline tüvi ja sõna\-tüübi\-mall, mille põhjal on võimalik rekonstrueerida sisendi muutvormi\-tabel (tabel~\ref{tab:väljundtabel-katto}). Tabelitele viidatakse alljärgnevas tekstis mitmel korral.

\spacing{1.0}
\begin{table}[H] %[!htbp] % kuvab tabelit definiitselt enne neile järgnevat teksti
      \footnotesize
  \begin{minipage}[t]{.40\textwidth}
%    \centering
    \begin{tabular}[t]{l l}
      % TODO vaata sõnavormid üle vormisõnastikust
      muutvorm            & vormiinfo \\ \hline
      \textit{katto}      & \textsc{sg nom} \\
      \textit{katod}      & \textsc{pl nom} \\
      \textit{kato}       & \textsc{sg gen} \\
      \textit{kattojõ}    & \textsc{pl gen} \\
      \textit{kattoa}     & \textsc{sg part} \\
      \textit{kattoit}    & \textsc{pl part} \\
      \textit{kattose}    & \textsc{sg ill} \\
      \textit{kattoisõ}   & \textsc{pl ill} \\
      \textit{kattoz}     & \textsc{sg ine} \\
      \textit{kattoiz}    & \textsc{pl ine} \\
      \textit{katossõ}    & \textsc{sg ela} \\
      \textit{kattoissõ}  & \textsc{pl ela} \\
      \textit{katollõ}    & \textsc{sg all} \\
      \textit{kattoillõ}  & \textsc{pl all} \\
      \textit{katol}      & \textsc{sg ade} \\
      \textit{kattoil}    & \textsc{pl ade} \\
      \textit{katoltõ}    & \textsc{sg abl} \\
      \textit{kattoiltõ}  & \textsc{pl abl} \\
      \textit{katossi}    & \textsc{sg tran} \\
      \textit{kattoissi}  & \textsc{pl tran} \\
      \textit{kattossaa}  & \textsc{sg term} \\
      \textit{kattoissaa} & \textsc{pl term} \\
      \textit{katoka}     & \textsc{sg com} \\
      \textit{kattoika}   & \textsc{pl com} \\
    \end{tabular}
    \caption{Sisendi muutvormide tabel koos morfo\-loogilise informatsiooniga.}
    \label{tab:sisendtabel-katto}
  \end{minipage}
  \hfill
  \begin{minipage}[t]{.55\textwidth}
    \centering
    \begin{tabular}[t]{l l l}
      % TODO vaata sõnavormid üle vormisõnastikust
      tehniline tüvi                     & muutvormi\-mall           & vormiinfo \\
      \hline
      \underline{kat} t \underline{o}       & $x_1$ + t + $x_2$         & \textsc{sg nom} \\
      \underline{kat}   \underline{o} d     & $x_1$ + $x_2$ + d         & \textsc{pl nom} \\
      \underline{kat}   \underline{o}       & $x_1$ + $x_2$             & \textsc{sg gen} \\
      \underline{kat} t \underline{o} jõ    & $x_1$ + t + $x_2$ + jõ    & \textsc{pl gen} \\
      \underline{kat} t \underline{o} a     & $x_1$ + t + $x_2$ + a     & \textsc{sg part} \\
      \underline{kat} t \underline{o} ite   & $x_1$ + t + $x_2$ + it    & \textsc{pl part} \\
      \underline{kat} t \underline{o} sõ    & $x_1$ + t + $x_2$ + sõ    & \textsc{sg ill} \\
      \underline{kat} t \underline{o} isõ   & $x_1$ + t + $x_2$ + isõ   & \textsc{pl ill} \\
      \underline{kat} t \underline{o} z     & $x_1$ + t + $x_2$ + z     & \textsc{sg ine} \\
      \underline{kat} t \underline{o} iz    & $x_1$ + t + $x_2$ + iz    & \textsc{pl ine} \\
      \underline{kat}   \underline{o} sse   & $x_1$ + $x_2$ + ssõ       & \textsc{sg ela} \\
      \underline{kat} t \underline{o} issõ  & $x_1$ + t + $x_2$ + issõ  & \textsc{pl ela} \\
      \underline{kat}   \underline{o} llõ   & $x_1$ + $x_2$ + llõ       & \textsc{sg all} \\
      \underline{kat} t \underline{o} illõ  & $x_1$ + t + $x_2$ + illõ  & \textsc{pl all} \\
      \underline{kat}   \underline{o} l     & $x_1$ + $x_2$ + l         & \textsc{sg ade} \\
      \underline{kat} t \underline{o} il    & $x_1$ + t + $x_2$ + il    & \textsc{pl ade} \\
      \underline{kat}   \underline{o} ltõ   & $x_1$ + $x_2$ + ltõ       & \textsc{sg abl} \\
      \underline{kat} t \underline{o} iltõ  & $x_1$ + t + $x_2$ + iltõ  & \textsc{pl abl} \\
      \underline{kat}   \underline{o} ssi   & $x_1$ + $x_2$ + ssi       & \textsc{sg tran} \\
      \underline{kat} t \underline{o} issi  & $x_1$ + t + $x_2$ + issi  & \textsc{pl tran} \\
      \underline{kat} t \underline{o} ssaa  & $x_1$ + t + $x_2$ + ssaa  & \textsc{sg term} \\
      \underline{kat} t \underline{o} issaa & $x_1$ + t + $x_2$ + issaa & \textsc{pl term} \\
      \underline{kat}   \underline{o} ka    & $x_1$ + $x_2$ + ka        & \textsc{sg com} \\
      \underline{kat} t \underline{o} ika   & $x_1$ + t + $x_2$ + ika   & \textsc{pl com} \\
    \end{tabular}
    \caption{Väljundi tüüpsõnamall (kus\-juures muutujad $x_1 = $ \textit{kat} ja $x_2 = $ \textit{o} vastab sisendist leitud, allajoonitud tehnilisele tüvele).}
    \label{tab:väljundtabel-katto}
  \end{minipage}
\end{table}
\spacing{1.5}

% tähtkoostise põhjal
Ekstraktmorfoloogia meetod eraldab sisendtabelist sõna\-tüübi\-malli ja tehnilise tüve lekseemi muutvormide tähtkoostise põhjal, s.o sõnavormide tähtede põhjal. Meetod arvestab palatalisatsiooniga kuivõird vadja ortograafias seda märgitakse, rõhuga meetod ei arvesta, kuna on sisendis märkimata. Silpide arvuga meetod ei arvesta. % TODO kas saaks paremini see s.o seletus?

% tehniline tüvi
Tehniliseks tüveks loetakse sõna need tähtede jadad, mis esinevad (korduvad) üle kõigi selle muutvormide (allajoonitud tähed tabelis~\ref{tab:väljundtabel-katto}, veerg~1). Tehniline tüvi koosneb ühest või enamast osast (tähtede jadast), antud näite puhul kahest. Samasse sõnatüüpi kuuluvad lekseemid erinevad vaid oma tehnilise tüve poolest. % TODO kas morfeemi definitsiooniga võrdlemine ka kopeerida siia?
Sõnatüübi muutvorme eristab teine-teisest see tähtkoostis, mis ei kordu üle kõigi muutvormide (vrd morfeemi definitsioon pt~\ref{sec:morfeemi-staatus}).

Tehnilise tüve leidmine koosneb kahest etapist: 1)~pikima ühisosajada eraldamine ning 2)~ühisosajada ühene jaotamine muutvormide vahel tehniliseks tüveks. Sõnatüüp koosneb muutvormimallidest ja koostatakse tehnilise tüve jaotuse põhjal, kusjuures üldistatakse tehnilise tüve osad muutujateks ja sõnatüüp funktsiooniks.


\subsubsection{Tehnilise tüve eraldamine}

% tehniline tüvi == LCS
Tehniline tüvi on ekstraktmorfoloogias defineeritud kordse pikima ühisosajadana (\textit{Multiple Longest Common Subsequence}). %, mille tuvastamiseks rakendatakse lõplike automaatide formalismi (\cite{hulden_generalizing_2014}).
Näites (tabel~\ref{tab:sisendtabel-katto}) esinevad üle kõigi muutvormide järgmised tähed: \vadja{k, a, t} ja \vadja{o}. Aga kuna \textit{t}-sid esineb eri muutvormides rohkem kui üks, ei ole selge millele neist vastab pikima ühisosajada täht \textit{t}.  Näite puhul on võimalik kaks pikima ühisosajada jaotust: \textit{kat}~ja~\textit{o} või \textit{ka}~ja~\textit{to}.

Võimalike mitmesuste lahendamiseks kasutatakse ekstraktmorfoloogias ühestamise heuristikat. %, mida on lähemalt kirjeldanud \cite{hulden_generalizing_2014}.
Ühestamine eelistab mh pikemat esiosa (\cite[33]{hulden_generalizing_2014}), mistõttu valitakse näite puhul tehniliseks tüveks osad \textit{kat}~ja~\textit{o}.
% Üheseks tehniliseks tüveks mitme võimaliku variandi puhul valitakse 
% LCS ambigious
% Muutvormidest võib leiduda mitu erinevat ühisosajada ja nende vahel üheselt  tuleb  ja nendest ühe valimise


\subsubsection{Üldistamine sõnatüübi funktsiooniks}

Tehnilise tüve fikseerimise järel moodustatakse muutvormi\-mallid. Muutvormides asendatakse tehnilise tüve osad muutujatega (tabelis~\ref{tab:väljundtabel-katto}, veerg 2). Näiteks
\begin{align*}
  \textit{katto} &\rightarrow \underline{\textit{kat}} + \textit{t} + \underline{\textit{o}} \rightarrow x_1 + t + x_2 \nonumber \\
  \textit{katod} &\rightarrow \underline{\textit{kat}} + \underline{\textit{o}} + \textit{d} \rightarrow x_1 + x_2 + d \nonumber
\end{align*}
vastavalt \msd{sg nom} ja \msd{sg gen} vormidele.

On ilmne et, kui asendatakse muutvormimallides muutujad vastavate tehnilise tüve osadega, siis taas-moodustatakse muutvormid (\vadja{katto} ja \vadja{katod}). Muutujaid aga asendades teiste tähtedega, % moodustatakse muude (potentsiaalsete) lekseemide muutvormid,
nt \textit{čiut}~ja~\textit{o}, moodustatakse teise lekseemi, \textit{čiutto} käändetabel. Näiteks
\begin{align*}
  x_1 + t + x_2 &\rightarrow \underline{\textit{čiut}} + \textit{t} + \underline{\textit{o}} \rightarrow \textit{čiutto} \nonumber \\
  x_1 + x_2 + d &\rightarrow \underline{\textit{čiut}} + \underline{\textit{o}} + \textit{d} \rightarrow \textit{čiutod} \nonumber
\end{align*}
vastavalt \msd{sg nom} ja \msd{sg gen} muutvormi\-mallidele.

Järelikult kuuluvad mõlemad lekseemid sama sõnatüübi alla ja erinevad ainult oma tehnilise tüve poolest.

Lekseemi iseloomustab ekstraktmorfoloogias sõnatüüp ja tehniline tüvi. Sõnatüüp on funktsioon, mille muutuja(te)ks on tehniline tüvi. Sõnatüübi muutvorme eristab teine-teisest see tähtkoostis, mis ei kordu üle kõigi muutvormide (vrd morfeemi definitsioon pt~\ref{sec:morfeemi-staatus}).

Kuna see vajab spetsiifilist teadmist tehnilise tüve moodustamise kohta, ei ole tehnilisel tüvel opereeriv funktsioon sobilik praktiliseks inim-kasutamiseks. On võimalik moodustada kasutamist hõlbustav funktsioon, mille sisendiks on terve sõnavorm.

Tehniline tüvi ei sõltu definitsiooni järgi ühest konkreetsest muutvormist, vaid lekseemi tervest muutvormi\-tabelist (paradigmast). Seetõttu on ekstrakt\-morfoloogia lemma suhtes neutraalne. Lemma valik on teise\-järguline ja seda on võimalik hiljem vahetada. % TODO rohkem sellest kuskil?

Kui lemmaks on valitud kindel muutvorm (nt \msd{sg nom}), on võimalik koostada funktsioon, mille sisendiks on muutvorm tervikuna ja väljundiks on tehniline tüvi. Funktsiooni koostamise aluseks on lemmaks valitud vormi muutvormi\-mall, mille järgi sisend-sõnavormi tähtkoostis jaotatakse ühestamisheuristika põhjal (\cite[572]{ahlberg_semi-supervised_2014}). Niiviisi saab aheldada lemmavormi sisendiks võtva funktsioon kokku tehnilise tüve sisendiks võtva funktsiooniga ja moodustada kasutust hõlbustava funktsiooni, mille sisendiks on sõna lemmavorm ja väljundiks muutvormide tabel. Seda lähenemisviisi kasutatakse magistritöös Grammatical Framework programmkoodi tuletamise juures (pt~\ref{sec:gf}). % TODO aga Giella puhul? kas FST ei tee mitte täpselt seda sama? ... aga pole funktsioon?


%% ----- 
%% ülejäänud tähtkoostis iseloomustab muutvorme ja võrdlus morfeemi definitsiooniga
%%Lekseemi sõnatüüpi iseloomustab sellest eraldatud sõna\-tüübi\-mall, mis koostatakse tehnilise tüve ja muutvormide ülejäänud tähtkoostise põhjal.
%% LIIGUTA Tähtkoostis, mis ei esine üle kõigi muutvormide, loetakse muutvorme teine-teisest eristavaks muutvormi\-malliks (vrd morfeemi definitsioon pt~\ref{sec:morfeemi-staatus}).
%
%% Kuidas sõnatüübimallid koostatakse?
%Sõnatüübi\-mall koosneb muutvormimallidest. Iga muutvormi mall koostatakse eraldatud tehnilise tüve jaotamise
%
%Iga muutvormimall on omakorda moodustatud tehnilise tüve ja ülejäänud tähtkoostise jadadest sel moel, et tehnilise tüve osad on abstraheeritud muutujateks (tabeli~\ref{tab:väljundtabel-katto}, veerg 2).
%
%% TODO sõnatüübimalli võib vaadata funktsioonina
%Sõnatüüpi ja muutvormimalle võib vaadata funktsioonina, mille muutuja(te)ks on tehniline tüvi. Kui muut\-voormi\-mallides asendada muutujad eraldatud tehnilise tüvega, rekonstrueeritakse sõna muut\-vormid. %Kui muut\-vormi\-malli muutujad asendatakse tehnilise tüvega, siis taastatakse lekseemi muutvorm (nt       \underline{kat} t \underline{o} je    & $x_1$ + t + $x_2$ + je    & \textsc{pl gen}
%
%
%% sama sõnatüübi lekseemid erinevad vaid tüve poolest
%Sama sõnatüübi alla kuuluvad lekseemid erinevad vaid oma tehnilise tüve poolest. %Teised lekseemid, mis käituvad sama sõnatüübi järgi, erinevad ainult tehnilise tüve poolest.
%Näiteks \vadja{katto} (ee~katus) ja \vadja{čiutto} (ee~särk) tehnilised tüved on vastavavalt $x_1 = $ \textit{kat}, $x_2 = $ \textit{o} ja $x_1 = $ \textit{čiut}, $x_2 = $ \textit{o} ning moodustavad nt \msd{pl gen} muutvormimalli järgi sõnavormid.
%
%% TODO 'joonis' ei ole õige termin selleks
%\begin{figure}[H]
%  \centering
%  % tee graaf nii, et mall on jagatud mõlemaga: mall < üleval ja all sõnavormid
%
%  $ \mathfrak{katto}(x_1, x_2) = x_1 + \mathtt{t} + x_2 + \mathtt{j\~{o}} = \textup{kat} + \mathtt{t} + \textup{o} + \mathtt{j\~{o}} = \mathit{kattoj\~{o}} $
%  
%  $ \mathfrak{katto}(x_1, x_2) = x_1 + \mathtt{t} + x_2 + \mathtt{j\~{o}} = \textup{čiut} + \mathtt{t} + \textup{o} + \mathtt{j\~{o}} = \mathit{čiuttoj\~{o}} $
%  \caption{Lekseemide \vadja{katto} ja \vadja{čiutto} \msd{sg gen} vormide rekonstrueerimine muutvormimalli ja tehnilise tüvega.}
%  \label{fig:muutvormimalli-rakendamine}
%\end{figure}
%
%
%% meetod eraldab maksimaalseid tüüpe, nt vokalism 'tüttö'
%Kuivõrd sõna \vadja{tüttö} käitub morfoloogiliselt väga sarnaselt siin näitlikustatud sõnatüübiga, eraldatakse selleks siiski omaette sõnatüüp ekstrakt\-morfoloogias. Põhjus on vokalismis, mistõttu erinevad ees- ja tagapoolsed muutvormimallid oma täht\-koostise poolest ja seega ei saa kaks sõnatüüpi kokku langeda (\msd{pl gen} muut\-vormi\-malli puhul \vadja{je} ja \vadja{jõ}).
%
%% maksimaalne ja konkretiseeritud üksikhäälikuteni
%kuna sisendiks on võetud ortograafiline vormisõnastik, kajastab tähtkoostise põhjal eraldatud tähed ka ortograafiat -- nt palatalisatsiooni on, rõhku pole. % TODO
%Ekstraktmorfoloogia opereerib tähtkoostisel ja seetõttu on ka saadud (eraldatud) sõnatüübistik konkretiseeritud üksikhäälikuteni, mida \cite[282]{viks_verbide_1976} peab sõnade muuttüüpide seisukohalt täiesti tarbetuks. Sõnatüübistik sarnaneb E.~Muugi 1933. a VÕSis esitatud tüübistikuga, mis hõlmab 895 eesti keele noomeni\-tüüpi (\cite[282]{viks_verbide_1976}, \cite[282]{viks_muuttüübid_}). Viks ei näe põhjust neid pidada muuttüüpideks, vaid nimetab neid pigem struktuuritüüpideks \cite[282]{viks_verbide_1976}.
%% TODO välja arvatud see, et Muuk arvestab silpide arvu
%
%% info ei kadu -- tehniline tüvi + mall rekonstrueerib sisendi (infot ei kaotata, sobitub mälupõhisele)
%Kõik ekstraktmorfoloogiaga eraldatud informatsioon, s.o sõnatüübi mall ja tehniline tüvi, salvestatakse andmebaasi, mille kuju kirjeldatakse lähemalt pt-s~\ref{sec:lmf}. Kuna salvestatud informatsiooni põhjal on võimalik kõik sisend\-materjal taastada, on ekstrakt\-morfoloogia meetod n.n mitte-destruktiivne. % TODO või iseloomustada mälupõhisena?
%
%% TODO kuhu panna viiteid nt Huldeni fst algoritmile ja üldse lähteartiklitele?
%% kuhu panna jutt mallide loetavuse kohta?
%
%% into järgmisse peatükki
%Järgmine peatükk kirjeldab kuidas tehnilise tüve põhjal on võimalik luua statistiline mudel, mille abil saab määrata tundmatu sõnavormi kuuluvust ühe või teise sõnatüübi alla. % TODO pole stat mudel vaid statistikaga rehkendatud regulaaravaldis!
%
%
%% TODO tehniline tüvi on käändevormist sõltumata ja seega on ka lemmavorm vabalt valitav

\subsection{Sõnatüübi ennustamine}
\label{sec:ekstraktmorfoloogia-ennustamine}

% eraldamisest eraldiseisev
Sõnatüübi ennustamine on eraldiseisev protsess ja põhineb ainult eraldatud tehniliste tüvede muutujate tähtkoostise analüüsimisel. Ennustamine koosneb piirangute seadmisest tehnilise tüve muutujate kujule.

% kasutatakse ainult morfoloogialabori veebirakenduses, kuigi oleks võimalik keeletehnoloogiasse sisse kodeerida. seda ei tehta, sest esialgu on piirdutud õigekirjakontrollija GF sõnastiku loomisega. kitsendused sarnanevad GF raamistikus levinud Smart Paradigm funktsiooni mõistega ja oleks üks võimalik edasiarenduse suund.
Sõnatüübi ennustamismeetodit on magistritöös kasutatud vaid kaudselt, Morfoloogialabori veebiliidese kaudu vormisõnastiku koostamisel. Seda on tehtud kuna magistritöö eesmärgiks on olnud esmane integreerimine keeletehnoloogilistesse raamistikesse ja õigekirjakontrollija loomine, kus ennustamisvõimet ei ole otseselt tarvis. Ennustamismeetod sarnaneb \textit{Grammatical Framework}'i \textit{Smart Paradigm}-funktsiooni mõistega (\cite{detrez_smart_2012}) ja on magistritöö üks võimalik edasiarenduse suund. Forsberg ja Hulden (\cite{forsberg_deriving_2016}) on rakendanud meetodit tundmatu sõnavormi sõnatüübi ennustaja (ingl. \textit{morphological guesser}) ehitamiseks, mida Morfoloogialabori veebiliides kasutab uute sõnade sisestamisel neile pakutud sõnatüüpide järjestamiseks kasutajale.

% tehnilise tüve akumuleerimine
Vormisõnastikku salvestatud lekseemid kannavad informatsiooni tehnilise tüve ja sõnatüübimalli kohta. Mida rohkem lekseeme jagavad üht ja sama sõnatüübi malli, seda rohkem on informatsiooni selle sõnatüübi tehniliste tüvede kujude kohta.

% constrained by reflect morphophonological phenomena and orthographic conventions
Tehnilised tüved ei jaotu sõnatüüpide vahel arbitraarselt, vaid nende kuju peegeldab tihti morfo\-fonoloogilisi nähtuseid ja orto\-graafilisi tavasid. Seda informatsiooni on võimalik kasutada selleks, et seada sõna\-tüübi\-mallide muutujatele kitsendusi ja piirata selle ühtivate tehniliste tüvede hulka, ning tundmatu sõnavormi puhul ennustada selle kuulumist ühe või teise sõnatüüpi alla. (\cite[2579]{forsberg_deriving_2016}).

Forsberg ja Hulden kirjeldavad statistilist viisi, millega nad loovad tehnilise tüve kuju tähtkoostist kitsendava regulaarse grammatika. Statistik arvutab salvestatud andmete põhjal piirmäära, mida mitte ületades loetakse tehnilise tüve tähtkoostise algus- või lõpposa (või terviklikult) suletud klassiks. Suletud klassi puhul peab leitud tähtkoostis antud positsioonis esinema, et lubada tundmatu sõnavorm ühtida sõnatüübiga. (\cite[2580]{forsberg_deriving_2016}).

Meetodi näitlikustamiseks võib magistritöö vältel koostatud vormisõnastiku põhjal märkida seda, et kõik \textit{katto}-sõnatüübi alla kuuluvad lekseemid moodustaksid suletud klassi: tehnilise tüve esimene muutuja peab lõppema \textit{t}-tähega: \vadja{\textbf{jut}t\textbf{u}, \textbf{hat}t\textbf{u}, \textbf{kat}t\textbf{o}, \textbf{kut}t\textbf{su}, \textbf{laat}t\textbf{o}, \textbf{lant}t\textbf{u}, \textbf{pal̕t}t\textbf{o}, \textbf{port}t\textbf{u}, \textbf{Tart}t\textbf{u}, \textbf{čiut}t\textbf{o}}.


%%\subsection{Võrdlus Eesti leksikograafilise traditsiooniga}
% kuhu alla see paremini paigutada? ARVUTIMORFOLOOGIA EESMÄRGI ALLA?

%% Tavapärane on arvutimorfoloogia koostada reeglite kirjutamise kaudu, seejuures teha kõigepealt klassifikatsioon, mille järgi on leida kõik eri klassid elik üksused milledele rakenduvad samad reeglid.
%% Seda on teinud nt Viks eesti arvutimorfoloogia koostamiseks (mh \cite{viks_verbide_1976}) ja sarnase viisi metodoloogiat on püüdnud formaliseerida Koskenniemi (\cite{koskenniemi_informal_2013}).



\newpage
\section{Vadja morfoloogiliste sõnatüüpide analüüs}
\label{sec:analüüs}

See osa kirjeldab ekstraktmorfoloogiaga leitud vadja keele morfoloogilised sõnatüübid ja jaotab need käändkondadesse. Käändkondade süsteemiks on kasutatud \cite{ariste_grammar_1968}. Tsvetkovi sõnaraamatus esinevat variatsiooni on analoogia põhjal ühtlustatud kirjakeele lihtsama õppimise eesmärgil. Peatüki viimases osas analüüsitakse mil moel \cite{silfverberg_computational_2018} esitatud ekstraktmorfoloogia üldiste muuttüüpide algoritm kajastab käändkondi.

Ariste käändkonnad põhinevad muutustel, mis kajastuvad järgmistes käändeis: \msd{sg nom} ja \msd{pl nom}, \msd{sg gen} ja \msd{pl gen}, \msd{sg par} ja \msd{pl par}, \msd{sg ill} ja \msd{pl ill} ning \msd{sg ela} ja \msd{pl ela} \cite[42]{ariste_grammar_1968}.

%\subsection{Kirjakeele ühtlustamine}
Üle käändkonniti on rekonstrueeritud lõpukaolised vokaalid ja ühtlustatud peamiselt lõpuvokaali õ:a vaheldumine. Detailsemalt on ühtlustatud komponente kirjeldatud iga käändkonna juures.

% TODO tee endale selgeks mis asi on noomen ja kas komparatiivid kuuluvad selle alla
% TODO sõnavara sisaldab ka pronoomeneid, millel pole defektsed paradigmad
Noomeni sõnavarast on välja jäetud komparatiivsed vormid (mõlõpi, vanepi). 

% TODO
Alljärgnevates kirjeldustes on lisatud sõnatüübi muutvormimallid kujul, kus tehniline tüvi on allajoonitud. Selline esitus võimaldab lugejal paremini näha ekstrakt\-morfoloogilise muutvormimalli funktsioneerimist läbi.

\subsection{\RN{1} käändkond}

Esimesse käändkonda kuuluvad (\cite[40]{ariste_grammar_1968}) järgi ühetüvelised, ühesilbilised sõnad. Ekstraktmorfoloogia eraldab kõik sellesse käändkonda kuuluvad sõnatüübid, aga koondab ka mitmesilbilised liitsõnad, mille järelkomponent kuulub siia käändkonda (nt \vadja{pihlpuu} nagu \vadja{puu}, \vadja{bulipää} nagu \vadja{pää}).

\msd{sg par} vormidele on lisatud lõpuhäälik \textit{-õ} või \textit{-e} vastavalt sõna vokalismile.

Koostatud vormisõnastiku sõnade \msd{sg ill} vorme on esialgselt üsna kunstlikult ühtlustatud kirjakeele jaoks: \textit{pää:pähhe}, \textit{puu:puhhu} ja \textit{maa:mahha} ning diftongiga sõnade puhul \textit{või:võisõ}, \textit{täi:täise}.

Näide Tsvetkovi sõnaraamatus esinevast \msd{sg ill} vormi variatsioonist: \vadja{kuu}:\vadja{kuusõ}; \vadja{üü}:\vadja{ühese};  \vadja{vüü}:\vadja{vühe}\textasciitilde \vadja{vühese}\textasciitilde \vadja{vüüse}; \vadja{püü}:\vadja{pühe}\textasciitilde \vadja{pühese}.

Ekstraktmorfoloogia eraldatud sõnatüüpide paljus on ajendatud ühelt poolt vokalismist (muutelõpu erinevus \textit{või}:\textit{võisõ} ja \textit{täi}:\textit{täise}). Teisalt aga pika tüve\-vokaali lühenemise tõttu i-mitmuse vormides (\textit{pääd}:\textit{päije} ja \textit{tüüd}:\textit{tüije}), mistõttu jääb ühte sõnatüüpi -\textit{ä}-lised muutelõpud ja teise -\textit{ü}-lised.

Üks avatud küsimus ja tähelepanek on \msd{sg} ja \msd{pl} vormide kokkulangemine diftongiga sõnade puhul.

\subsubsection*{Ekstraktmorfoloogia sõnatüübid}

\spacing{1}

\vspace{1.8em}
\begin{minipage}{\textwidth}
\stepcounter{mallinumber}
\textbf{Tüüpsõnamall \arabic{mallinumber}\,\vadja{maa}}\\

\begin{sideways}
\begin{tabular}{l l}
muutvormimall & tunnused \\
\hline
\underline{ma}\,+\,a & \textsc{ sg nom } \\
\underline{ma}\,+\,a & \textsc{ sg gen } \\
\underline{ma}\,+\,atõ & \textsc{ sg par } \\
\underline{ma}\,+\,hha & \textsc{ sg ill } \\
\underline{ma}\,+\,az & \textsc{ sg ine } \\
\underline{ma}\,+\,ass & \textsc{ sg ela } \\
\underline{ma}\,+\,allõ & \textsc{ sg all } \\
\underline{ma}\,+\,all & \textsc{ sg ade } \\
\underline{ma}\,+\,alt & \textsc{ sg abl } \\
\underline{ma}\,+\,assi & \textsc{ sg tra } \\
\underline{ma}\,+\,assaa & \textsc{ sg ter } \\
\underline{ma}\,+\,aka & \textsc{ sg com } \\
\underline{ma}\,+\,ad & \textsc{ pl nom } \\
\underline{ma}\,+\,ijõ & \textsc{ pl gen } \\
\underline{ma}\,+\,it & \textsc{ pl par } \\
\underline{ma}\,+\,isõ & \textsc{ pl ill } \\
\underline{ma}\,+\,iz & \textsc{ pl ine } \\
\underline{ma}\,+\,iss & \textsc{ pl ela } \\
\underline{ma}\,+\,illõ & \textsc{ pl all } \\
\underline{ma}\,+\,ill & \textsc{ pl ade } \\
\underline{ma}\,+\,ilt & \textsc{ pl abl } \\
\underline{ma}\,+\,issi & \textsc{ pl tra } \\
\underline{ma}\,+\,issaa & \textsc{ pl ter } \\
\underline{ma}\,+\,ika & \textsc{ pl com } \\
\end{tabular}
\end{sideways}
\captionof{table}{Tüüpsõna \arabic{mallinumber}\,\textit{maa} ekstraheeritud muutvormimallid.}
\label{tab:tüüpsõnamall-maa}

\end{minipage}

 
\vspace{1em}
\noindent Tüüpsõna ei hõlma teisi lekseeme vormi\-sõnastikus.

Sõnatüübi\-mall kirjeldab tagapoolseid -\textit{aa}-tüvelisi sõnu.

\vspace{1.8em}
\begin{minipage}{\textwidth}
\stepcounter{mallinumber}
\textbf{Tüüpsõnamall \arabic{mallinumber}\,\vadja{puu}}\\

\begin{sideways}
\begin{tabular}{l l}
muutvormimall & tunnused \\
\hline
\underline{pu}\,+\,u & \textsc{ sg nom } \\
\underline{pu}\,+\,u & \textsc{ sg gen } \\
\underline{pu}\,+\,utõ & \textsc{ sg par } \\
\underline{pu}\,+\,hhu & \textsc{ sg ill } \\
\underline{pu}\,+\,uz & \textsc{ sg ine } \\
\underline{pu}\,+\,uss & \textsc{ sg ela } \\
\underline{pu}\,+\,ullõ & \textsc{ sg all } \\
\underline{pu}\,+\,ull & \textsc{ sg ade } \\
\underline{pu}\,+\,ult & \textsc{ sg abl } \\
\underline{pu}\,+\,ussi & \textsc{ sg tra } \\
\underline{pu}\,+\,ussaa & \textsc{ sg ter } \\
\underline{pu}\,+\,uka & \textsc{ sg com } \\
\underline{pu}\,+\,ud & \textsc{ pl nom } \\
\underline{pu}\,+\,ijõ & \textsc{ pl gen } \\
\underline{pu}\,+\,it & \textsc{ pl par } \\
\underline{pu}\,+\,isõ & \textsc{ pl ill } \\
\underline{pu}\,+\,iz & \textsc{ pl ine } \\
\underline{pu}\,+\,iss & \textsc{ pl ela } \\
\underline{pu}\,+\,illõ & \textsc{ pl all } \\
\underline{pu}\,+\,ill & \textsc{ pl ade } \\
\underline{pu}\,+\,ilt & \textsc{ pl abl } \\
\underline{pu}\,+\,issi & \textsc{ pl tra } \\
\underline{pu}\,+\,issaa & \textsc{ pl ter } \\
\underline{pu}\,+\,ika & \textsc{ pl com } \\
\end{tabular}
\end{sideways}
\captionof{table}{Tüüpsõna \arabic{mallinumber}\,\textit{puu} ekstraheeritud muutvormimallid.}
\label{tab:tüüpsõnamall-puu}

\end{minipage}

 
\vspace{1em}
\noindent Tüüpsõna hõlmab vormisõnastiku 5 lekseemi: \vadja{\underline{pu}u, \underline{ku}u, \underline{lu}u, \underline{su}u} ja \vadja{\underline{pihlpu}u}.

Sõnatüübi\-mall kirjeldab tagapoolseid -\textit{uu}-tüvelisi sõnu.
\paragraph*{\vadja{\underline{pi}i}}
\vadja{\underline{pi}i}, \vadja{\underline{pi}ite}, \vadja{\underline{pi}hhe}, \vadja{\underline{pi}iss}, \vadja{\underline{pi}id}, \vadja{\underline{pi}ije}, \vadja{\underline{pi}it}, \vadja{\underline{pi}ise}, \vadja{\underline{pi}iss}
 \\
Tüüpsõna ei hõlma teisi lekseeme vormi\-sõnastikus.

Sõnatüübi\-mall kirjeldab eespoolseid -\textit{ii}-tüvelisi sõnu.
\paragraph*{\vadja{\underline{pä}ä}}
\vadja{\underline{pä}ä}, \vadja{\underline{pä}äte}, \vadja{\underline{pä}hhe}, \vadja{\underline{pä}äss}, \vadja{\underline{pä}äd}, \vadja{\underline{pä}ije}, \vadja{\underline{pä}it}, \vadja{\underline{pä}ise}, \vadja{\underline{pä}iss}
 \\
Sõnatüüp hõlmab vormisõnastiku lekseeme: \vadja{pää, bulipää}.

Sõnatüübi\-mall kirjeldab eespoolseid -\textit{ää}-tüvelisi sõnu.
\paragraph*{\vadja{\underline{so}o}}
\vadja{\underline{so}o}, \vadja{\underline{so}otõ}, \vadja{\underline{so}hho}, \vadja{\underline{so}oss}, \vadja{\underline{so}od}, \vadja{\underline{so}ijõ}, \vadja{\underline{so}it}, \vadja{\underline{so}isõ}, \vadja{\underline{so}iss}
 \\
Sõnatüüp ei hõlma teisi lekseeme vormi\-sõnastikus.

Sõnatüübi\-mall kirjeldab tagapoolseid -\textit{oo}-tüvelisi sõnu.
\paragraph*{\vadja{\underline{te}e}}
\vadja{\underline{te}e}, \vadja{\underline{te}ete}, \vadja{\underline{te}hhe}, \vadja{\underline{te}ess}, \vadja{\underline{te}ed}, \vadja{\underline{te}ije}, \vadja{\underline{te}it}, \vadja{\underline{te}ise}, \vadja{\underline{te}iss}
 \\
Tüüpsõna ei hõlma teisi lekseeme vormi\-sõnastikus.

Sõnatüübi\-mall kirjeldab eespoolseid -\textit{ee}-tüvelisi sõnu.

\vspace{1.8em}
\begin{minipage}{\textwidth}
\stepcounter{mallinumber}
\textbf{Tüüpsõnamall \arabic{mallinumber}\,\vadja{tüü}}\\

\begin{sideways}
\begin{tabular}{l l}
muutvormimall & tunnused \\
\hline
\underline{tü}\,+\,ü & \textsc{ sg nom } \\
\underline{tü}\,+\,ü & \textsc{ sg gen } \\
\underline{tü}\,+\,üte & \textsc{ sg par } \\
\underline{tü}\,+\,hhe & \textsc{ sg ill } \\
\underline{tü}\,+\,üz & \textsc{ sg ine } \\
\underline{tü}\,+\,üss & \textsc{ sg ela } \\
\underline{tü}\,+\,ülle & \textsc{ sg all } \\
\underline{tü}\,+\,üll & \textsc{ sg ade } \\
\underline{tü}\,+\,ült & \textsc{ sg abl } \\
\underline{tü}\,+\,üssi & \textsc{ sg tra } \\
\underline{tü}\,+\,üssaa & \textsc{ sg ter } \\
\underline{tü}\,+\,üka & \textsc{ sg com } \\
\underline{tü}\,+\,üd & \textsc{ pl nom } \\
\underline{tü}\,+\,ije & \textsc{ pl gen } \\
\underline{tü}\,+\,it & \textsc{ pl par } \\
\underline{tü}\,+\,ise & \textsc{ pl ill } \\
\underline{tü}\,+\,iz & \textsc{ pl ine } \\
\underline{tü}\,+\,iss & \textsc{ pl ela } \\
\underline{tü}\,+\,ille & \textsc{ pl all } \\
\underline{tü}\,+\,ill & \textsc{ pl ade } \\
\underline{tü}\,+\,ilt & \textsc{ pl abl } \\
\underline{tü}\,+\,issi & \textsc{ pl tra } \\
\underline{tü}\,+\,issaa & \textsc{ pl ter } \\
\underline{tü}\,+\,ika & \textsc{ pl com } \\
\end{tabular}
\end{sideways}
\captionof{table}{Tüüpsõna \arabic{mallinumber}\,\textit{tüü} ekstraheeritud muutvormimallid.}
\label{tab:tüüpsõnamall-tüü}

\end{minipage}

 
\vspace{1em}
\noindent Tüüpsõna hõlmab vormisõnastiku 4 lekseemi: \vadja{\underline{tü}ü, \underline{vü}ü, \underline{ü}ü} ja \vadja{\underline{pü}ü}.

Sõnatüübi\-mall kirjeldab eespoolseid -\textit{üü}-tüvelisi sõnu.
\paragraph*{\vadja{\underline{täi}}}
\vadja{\underline{täi}}, \vadja{\underline{täi}te}, \vadja{\underline{täi}se}, \vadja{\underline{täi}ss}, \vadja{\underline{täi}d}, \vadja{\underline{täi}je}, \vadja{\underline{täi}t}, \vadja{\underline{täi}se}, \vadja{\underline{täi}ss}
 \\
Tüüpsõna ei hõlma teisi lekseeme vormi\-sõnastikus.

Sõnatüübi\-mall kirjeldab eespolseid diftongiga sõnu.
\paragraph*{\vadja{\underline{või}}}
\vadja{\underline{või}}, \vadja{\underline{või}tõ}, \vadja{\underline{või}sõ}, \vadja{\underline{või}ss}, \vadja{\underline{või}d}, \vadja{\underline{või}jõ}, \vadja{\underline{või}t}, \vadja{\underline{või}sõ}, \vadja{\underline{või}ss}
 \\
sõnatüüp ei hõlma teisi lekseeme

Sõnatüübi\-mall kirjeldab tagapolseid diftongiga sõnu.
\spacing{1.5}


\subsection{\RN{2} käändkond}

Teise käändkonda kuuluvad kahesilbilised sõnad, mille tüvevokaal on \vadja{-o}, \vadja{-u}, \vadja{-ü}, \vadja{-i} või \vadja{-õ} ning rohkem silpidega sõnad, mille tüvevokaal on \vadja{-o} \cite[42]{ariste_grammar_1968}.

Sellesse käändkonda kuuluvad paljud vene laensõnad. Kirja\-keele puhul puudub ülevaade vene laensõnade mugandamisest. Mugandamis\-strateegiatest ja problemaatikast vadja Jõgõperä murdes on kirjutanud \cite{rozhanskiy_zaimstvovannyje_2009}.


\subsubsection*{Ekstraktmorfoloogia sõnatüübid}
\spacing{1}

\vspace{1.8em}
\begin{minipage}{\textwidth}
\stepcounter{mallinumber}
\textbf{Tüüpsõnamall \arabic{mallinumber}\,\vadja{auči}}\\

\begin{sideways}
\begin{tabular}{l l}
muutvormimall & tunnused \\
\hline
\underline{au}\,+\,č\,+\,\underline{i} & \textsc{ sg nom } \\
\underline{au}\,+\,dž\,+\,\underline{i} & \textsc{ sg gen } \\
\underline{au}\,+\,č\,+\,\underline{i}\,+\,a & \textsc{ sg par } \\
\underline{au}\,+\,č\,+\,\underline{i}\,+\,sõ & \textsc{ sg ill } \\
\underline{au}\,+\,dž\,+\,\underline{i}\,+\,z & \textsc{ sg ine } \\
\underline{au}\,+\,dž\,+\,\underline{i}\,+\,ss & \textsc{ sg ela } \\
\underline{au}\,+\,dž\,+\,\underline{i}\,+\,llõ & \textsc{ sg all } \\
\underline{au}\,+\,dž\,+\,\underline{i}\,+\,ll & \textsc{ sg ade } \\
\underline{au}\,+\,dž\,+\,\underline{i}\,+\,lt & \textsc{ sg abl } \\
\underline{au}\,+\,dž\,+\,\underline{i}\,+\,ssi & \textsc{ sg tra } \\
\underline{au}\,+\,dž\,+\,\underline{i}\,+\,ssaa & \textsc{ sg ter } \\
\underline{au}\,+\,dž\,+\,\underline{i}\,+\,ka & \textsc{ sg com } \\
\underline{au}\,+\,dž\,+\,\underline{i}\,+\,d & \textsc{ pl nom } \\
\underline{au}\,+\,č\,+\,\underline{i}\,+\,jõ & \textsc{ pl gen } \\
\underline{au}\,+\,č\,+\,\underline{i}\,+\,it & \textsc{ pl par } \\
\underline{au}\,+\,č\,+\,\underline{i}\,+\,isõ & \textsc{ pl ill } \\
\underline{au}\,+\,č\,+\,\underline{i}\,+\,iz & \textsc{ pl ine } \\
\underline{au}\,+\,č\,+\,\underline{i}\,+\,iss & \textsc{ pl ela } \\
\underline{au}\,+\,č\,+\,\underline{i}\,+\,illõ & \textsc{ pl all } \\
\underline{au}\,+\,č\,+\,\underline{i}\,+\,ill & \textsc{ pl ade } \\
\underline{au}\,+\,č\,+\,\underline{i}\,+\,ilt & \textsc{ pl abl } \\
\underline{au}\,+\,č\,+\,\underline{i}\,+\,issi & \textsc{ pl tra } \\
\underline{au}\,+\,č\,+\,\underline{i}\,+\,issaa & \textsc{ pl ter } \\
\underline{au}\,+\,č\,+\,\underline{i}\,+\,jka & \textsc{ pl com } \\
\end{tabular}
\end{sideways}
\captionof{table}{Tüüpsõna \arabic{mallinumber}\,\textit{auči} ekstraheeritud muutvormimallid.}
\label{tab:tüüpsõnamall-auči}

\end{minipage}

 
\vspace{1em}
\noindent Tüüpsõna ei hõlma teisi lekseeme vormi\-sõnastikus.

Sõnatüübi\-mall kirjeldab tagapoolseid sõnu tüvemuutusega č:dž.
\paragraph*{\vadja{\underline{süüč}č\underline{i}}}
\vadja{\underline{süüč}\underline{i}}, \vadja{\underline{süüč}č\underline{i}ä}, \vadja{\underline{süüč}č\underline{i}se}, \vadja{\underline{süüč}\underline{i}ss}, \vadja{\underline{süüč}\underline{i}d}, \vadja{\underline{süüč}č\underline{i}je}, \vadja{\underline{süüč}č\underline{i}it}, \vadja{\underline{süüč}č\underline{i}ise}, \vadja{\underline{süüč}č\underline{i}iss}
 \\
Sõnatüüp ei hõlma teisi lekseeme vormi\-sõnastikus.

Sõnatüübi\-mall kirjeldab eespoolseid sõnu tüvemuutusega čč:č.

\vspace{1.8em}
\begin{minipage}{\textwidth}
\stepcounter{mallinumber}
\textbf{Tüüpsõnamall \arabic{mallinumber}\,\vadja{koffi}}\\

\begin{sideways}
\begin{tabular}{l l}
muutvormimall & tunnused \\
\hline
\underline{kof}\,+\,f\,+\,\underline{i} & \textsc{ sg nom } \\
\underline{kof}\,+\,\underline{i} & \textsc{ sg gen } \\
\underline{kof}\,+\,f\,+\,\underline{i}\,+\,a & \textsc{ sg par } \\
\underline{kof}\,+\,f\,+\,\underline{i}\,+\,sõ & \textsc{ sg ill } \\
\underline{kof}\,+\,\underline{i}\,+\,z & \textsc{ sg ine } \\
\underline{kof}\,+\,\underline{i}\,+\,ss & \textsc{ sg ela } \\
\underline{kof}\,+\,\underline{i}\,+\,llõ & \textsc{ sg all } \\
\underline{kof}\,+\,\underline{i}\,+\,ll & \textsc{ sg ade } \\
\underline{kof}\,+\,\underline{i}\,+\,lt & \textsc{ sg abl } \\
\underline{kof}\,+\,\underline{i}\,+\,ssi & \textsc{ sg tra } \\
\underline{kof}\,+\,\underline{i}\,+\,ssaa & \textsc{ sg ter } \\
\underline{kof}\,+\,\underline{i}\,+\,ka & \textsc{ sg com } \\
\underline{kof}\,+\,\underline{i}\,+\,d & \textsc{ pl nom } \\
\underline{kof}\,+\,f\,+\,\underline{i}\,+\,jõ & \textsc{ pl gen } \\
\underline{kof}\,+\,f\,+\,\underline{i}\,+\,it & \textsc{ pl par } \\
\underline{kof}\,+\,f\,+\,\underline{i}\,+\,isõ & \textsc{ pl ill } \\
\underline{kof}\,+\,f\,+\,\underline{i}\,+\,iz & \textsc{ pl ine } \\
\underline{kof}\,+\,f\,+\,\underline{i}\,+\,iss & \textsc{ pl ela } \\
\underline{kof}\,+\,f\,+\,\underline{i}\,+\,illõ & \textsc{ pl all } \\
\underline{kof}\,+\,f\,+\,\underline{i}\,+\,ill & \textsc{ pl ade } \\
\underline{kof}\,+\,f\,+\,\underline{i}\,+\,ilt & \textsc{ pl abl } \\
\underline{kof}\,+\,f\,+\,\underline{i}\,+\,issi & \textsc{ pl tra } \\
\underline{kof}\,+\,f\,+\,\underline{i}\,+\,issaa & \textsc{ pl ter } \\
\underline{kof}\,+\,f\,+\,\underline{i}\,+\,jka & \textsc{ pl com } \\
\end{tabular}
\end{sideways}
\captionof{table}{Tüüpsõna \arabic{mallinumber}\,\textit{koffi} ekstraheeritud muutvormimallid.}
\label{tab:tüüpsõnamall-koffi}

\end{minipage}

 
\vspace{1em}
\noindent Tüüpsõna ei hõlma teisi lekseeme vormi\-sõnastikus.

Sõnatüübi\-mall kirjeldab tagapoolseid sõnu tüvemuutusega ff:f.

\vspace{1.8em}
\begin{minipage}{\textwidth}
\stepcounter{mallinumber}
\textbf{Tüüpsõnamall \arabic{mallinumber}\,\vadja{suuto}}\\

\begin{sideways}
\begin{tabular}{l l}
muutvormimall & tunnused \\
\hline
\underline{suu}\,+\,t\,+\,\underline{o} & \textsc{ sg nom } \\
\underline{suu}\,+\,\underline{o} & \textsc{ sg gen } \\
\underline{suu}\,+\,t\,+\,\underline{o}\,+\,a & \textsc{ sg par } \\
\underline{suu}\,+\,t\,+\,\underline{o}\,+\,sõ & \textsc{ sg ill } \\
\underline{suu}\,+\,\underline{o}\,+\,z & \textsc{ sg ine } \\
\underline{suu}\,+\,\underline{o}\,+\,ss & \textsc{ sg ela } \\
\underline{suu}\,+\,\underline{o}\,+\,llõ & \textsc{ sg all } \\
\underline{suu}\,+\,\underline{o}\,+\,ll & \textsc{ sg ade } \\
\underline{suu}\,+\,\underline{o}\,+\,lt & \textsc{ sg abl } \\
\underline{suu}\,+\,\underline{o}\,+\,ssi & \textsc{ sg tra } \\
\underline{suu}\,+\,\underline{o}\,+\,ssaa & \textsc{ sg ter } \\
\underline{suu}\,+\,\underline{o}\,+\,ka & \textsc{ sg com } \\
\underline{suu}\,+\,\underline{o}\,+\,d & \textsc{ pl nom } \\
\underline{suu}\,+\,t\,+\,\underline{o}\,+\,jõ & \textsc{ pl gen } \\
\underline{suu}\,+\,t\,+\,\underline{o}\,+\,it & \textsc{ pl par } \\
\underline{suu}\,+\,t\,+\,\underline{o}\,+\,isõ & \textsc{ pl ill } \\
\underline{suu}\,+\,t\,+\,\underline{o}\,+\,iz & \textsc{ pl ine } \\
\underline{suu}\,+\,t\,+\,\underline{o}\,+\,iss & \textsc{ pl ela } \\
\underline{suu}\,+\,t\,+\,\underline{o}\,+\,illõ & \textsc{ pl all } \\
\underline{suu}\,+\,t\,+\,\underline{o}\,+\,ill & \textsc{ pl ade } \\
\underline{suu}\,+\,t\,+\,\underline{o}\,+\,ilt & \textsc{ pl abl } \\
\underline{suu}\,+\,t\,+\,\underline{o}\,+\,issi & \textsc{ pl tra } \\
\underline{suu}\,+\,t\,+\,\underline{o}\,+\,issaa & \textsc{ pl ter } \\
\underline{suu}\,+\,t\,+\,\underline{o}\,+\,ika & \textsc{ pl com } \\
\end{tabular}
\end{sideways}
\captionof{table}{Tüüpsõna \arabic{mallinumber}\,\textit{suuto} ekstraheeritud muutvormimallid.}
\label{tab:tüüpsõnamall-suuto}

\end{minipage}

 
\vspace{1em}
\noindent Tüüpsõna hõlmab vormisõnastiku 3 lekseemi: \vadja{\underline{suu}t\underline{o}, \underline{vaah}t\underline{o}} ja \vadja{\underline{leh}t\underline{o}}.

Sõnatüübi\-mall kirjeldab tagapoolseid sõnu tüvemuutusega t:∅. % TODO kuidas 0 ?
\paragraph*{\vadja{\underline{vah}t\underline{i}}}
\vadja{\underline{vah}\underline{i}}, \vadja{\underline{vah}t\underline{i}a}, \vadja{\underline{vah}t\underline{i}sõ}, \vadja{\underline{vah}\underline{i}ss}, \vadja{\underline{vah}\underline{i}d}, \vadja{\underline{vah}t\underline{i}jõ}, \vadja{\underline{vah}t\underline{i}it}, \vadja{\underline{vah}t\underline{i}isõ}, \vadja{\underline{vah}t\underline{i}iss}
 \\
Sõnatüüp ei hõlma teisi lekseeme vormi\-sõnastikus.

Sõnatüübi\-mall kirjeldab tagapoolseid sõnu tüvemuutusega t:∅ ja mille lõpuvokaal on \textit{i}. % TODO i tüvi?
\paragraph*{\vadja{\underline{al}k\underline{u}}}
\vadja{\underline{al}g\underline{u}}, \vadja{\underline{al}k\underline{u}a}, \vadja{\underline{al}k\underline{u}sõ}, \vadja{\underline{al}g\underline{u}ss}, \vadja{\underline{al}g\underline{u}d}, \vadja{\underline{al}k\underline{u}jõ}, \vadja{\underline{al}k\underline{u}it}, \vadja{\underline{al}k\underline{u}isõ}, \vadja{\underline{al}k\underline{u}iss}
 \\
Tüüpsõna hõlmab vormisõnastiku lekseeme \vadja{alku, lohko, pehko, plehku, touko, vihko, vinku, alko}.

Sõnatüübi\-mall kirjeldab tagapoolseid sõnu tüvemuutusega k:g, kusjuures tüvemuutus esineb konsont\-klustris, mistõttu gemineerumist ei toimu \msd{sg par} ja \msd{sg ill} tüvedes.
\paragraph*{\vadja{\underline{la}k\underline{o}}}
\vadja{\underline{la}g\underline{o}}, \vadja{\underline{la}kk\underline{o}a}, \vadja{\underline{la}kk\underline{o}sõ}, \vadja{\underline{la}g\underline{o}ss}, \vadja{\underline{la}g\underline{o}d}, \vadja{\underline{la}k\underline{o}jõ}, \vadja{\underline{la}k\underline{o}it}, \vadja{\underline{la}k\underline{o}isõ}, \vadja{\underline{la}k\underline{o}iss}
 \\
sõnatüüp hõlmab lekseeme \vadja{lako, luku, mako, maku, suku, vako, čako}

Sõnatüübi\-mall kirjeldab tagapoolseid sõnu tüvemuutusega k:g, kusjuures tüvi gemineerub \msd{sg par} ja \msd{sg ill} vormides.
\paragraph*{\vadja{\underline{läik}k\underline{i}}}
\vadja{\underline{läik}\underline{i}}, \vadja{\underline{läik}k\underline{i}ä}, \vadja{\underline{läik}k\underline{i}se}, \vadja{\underline{läik}\underline{i}ss}, \vadja{\underline{läik}\underline{i}d}, \vadja{\underline{läik}k\underline{i}je}, \vadja{\underline{läik}ke\underline{i}t}, \vadja{\underline{läik}ke\underline{i}se}, \vadja{\underline{läik}ke\underline{i}ss}
 \\
Sõnatüüp ei hõlma teisi lekseeme vormi\-sõnastikus.

Sõnatüübi\-mall kirjeldab eespoolseid sõnu tüvemuutusega kk:k ja mille lõpuvokaal on \textit{i}. % TODO i tüvi?

\vspace{1.8em}
\begin{minipage}{\textwidth}
\stepcounter{mallinumber}
\textbf{Tüüpsõnamall \arabic{mallinumber}\,\vadja{tükkü}}\\

\begin{sideways}
\begin{tabular}{l l}
muutvormimall & tunnused \\
\hline
\underline{tük}\,+\,k\,+\,\underline{ü} & \textsc{ sg nom } \\
\underline{tük}\,+\,\underline{ü} & \textsc{ sg gen } \\
\underline{tük}\,+\,k\,+\,\underline{ü}\,+\,ä & \textsc{ sg par } \\
\underline{tük}\,+\,k\,+\,\underline{ü}\,+\,se & \textsc{ sg ill } \\
\underline{tük}\,+\,k\,+\,\underline{ü}\,+\,z & \textsc{ sg ine } \\
\underline{tük}\,+\,\underline{ü}\,+\,ss & \textsc{ sg ela } \\
\underline{tük}\,+\,\underline{ü}\,+\,lle & \textsc{ sg all } \\
\underline{tük}\,+\,\underline{ü}\,+\,ll & \textsc{ sg ade } \\
\underline{tük}\,+\,\underline{ü}\,+\,lt & \textsc{ sg abl } \\
\underline{tük}\,+\,\underline{ü}\,+\,ssi & \textsc{ sg tra } \\
\underline{tük}\,+\,k\,+\,\underline{ü}\,+\,ssaa & \textsc{ sg ter } \\
\underline{tük}\,+\,\underline{ü}\,+\,ka & \textsc{ sg com } \\
\underline{tük}\,+\,\underline{ü}\,+\,d & \textsc{ pl nom } \\
\underline{tük}\,+\,k\,+\,\underline{ü}\,+\,je & \textsc{ pl gen } \\
\underline{tük}\,+\,k\,+\,\underline{ü}\,+\,it & \textsc{ pl par } \\
\underline{tük}\,+\,k\,+\,\underline{ü}\,+\,ise & \textsc{ pl ill } \\
\underline{tük}\,+\,k\,+\,\underline{ü}\,+\,iz & \textsc{ pl ine } \\
\underline{tük}\,+\,k\,+\,\underline{ü}\,+\,iss & \textsc{ pl ela } \\
\underline{tük}\,+\,k\,+\,\underline{ü}\,+\,ille & \textsc{ pl all } \\
\underline{tük}\,+\,k\,+\,\underline{ü}\,+\,ill & \textsc{ pl ade } \\
\underline{tük}\,+\,k\,+\,\underline{ü}\,+\,ilt & \textsc{ pl abl } \\
\underline{tük}\,+\,k\,+\,\underline{ü}\,+\,issi & \textsc{ pl tra } \\
\underline{tük}\,+\,k\,+\,\underline{ü}\,+\,issaa & \textsc{ pl ter } \\
\underline{tük}\,+\,k\,+\,\underline{ü}\,+\,ika & \textsc{ pl com } \\
\end{tabular}
\end{sideways}
\captionof{table}{Tüüpsõna \arabic{mallinumber}\,\textit{tükkü} ekstraheeritud muutvormimallid.}
\label{tab:tüüpsõnamall-tükkü}

\end{minipage}

 
\vspace{1em}
\noindent Tüüpsõna ei hõlma teisi lekseeme vormi\-sõnastikus.

Sõnatüübi\-mall kirjeldab eespoolseid sõnu tüvemuutusega kk:k.
\paragraph*{\vadja{\underline{vik}\underline{i}}}
\vadja{\underline{vik}\underline{i}}, \vadja{\underline{vik}k\underline{i}ä}, \vadja{\underline{vik}k\underline{i}se}, \vadja{\underline{vik}\underline{i}ss}, \vadja{\underline{vik}\underline{i}d}, \vadja{\underline{vik}k\underline{i}je}, \vadja{\underline{vik}k\underline{i}it}, \vadja{\underline{vik}k\underline{i}ise}, \vadja{\underline{vik}k\underline{i}iss}
 \\
sõnatüüp ei hõlma teisi lekseeme

Sõnatüübi\-mall kirjeldab eespoolseid sõnu tüvemuutusega kk:k ja mille lõpuvokaal on \textit{i}. % TODO i tüvi?
\paragraph*{\vadja{\underline{flak}k\underline{u}}}
\vadja{\underline{flak}\underline{u}}, \vadja{\underline{flak}k\underline{u}a}, \vadja{\underline{flak}k\underline{u}sõ}, \vadja{\underline{flak}\underline{u}ss}, \vadja{\underline{flak}\underline{u}d}, \vadja{\underline{flak}k\underline{u}jõ}, \vadja{\underline{flak}k\underline{u}it}, \vadja{\underline{flak}k\underline{u}isõ}, \vadja{\underline{flak}k\underline{u}iss}
 \\
Tüüpsõna hõlmab vormisõnastiku 26 lekseemi: \vadja{flakku, herkku, jõkilikko, kakko, kakku, kiikku, kolkku, kukko, kurkku, kuuzikko, lepikko, liivikko, luikko, lukku, lõõkku, majakko, musikko, mäčizikko, naizikko, oomnikko, pettelikko, rehtelkakku, seukko, võrkko, õzrikko} ja \vadja{čerikko}.

Sõnatüübi\-mall kirjeldab tagapoolseid sõnu tüvemuutusega kk:k.

\vspace{1.8em}
\begin{minipage}{\textwidth}
\stepcounter{mallinumber}
\textbf{Tüüpsõnamall \arabic{mallinumber}\,\vadja{galstukki}}\\

\begin{sideways}
\begin{tabular}{l l}
muutvormimall & tunnused \\
\hline
\underline{galstuk}\,+\,k\,+\,\underline{i} & \textsc{ sg nom } \\
\underline{galstuk}\,+\,\underline{i} & \textsc{ sg gen } \\
\underline{galstuk}\,+\,k\,+\,\underline{i}\,+\,a & \textsc{ sg par } \\
\underline{galstuk}\,+\,k\,+\,\underline{i}\,+\,sõ & \textsc{ sg ill } \\
\underline{galstuk}\,+\,k\,+\,\underline{i}\,+\,z & \textsc{ sg ine } \\
\underline{galstuk}\,+\,\underline{i}\,+\,ss & \textsc{ sg ela } \\
\underline{galstuk}\,+\,\underline{i}\,+\,llõ & \textsc{ sg all } \\
\underline{galstuk}\,+\,\underline{i}\,+\,ll & \textsc{ sg ade } \\
\underline{galstuk}\,+\,\underline{i}\,+\,lt & \textsc{ sg abl } \\
\underline{galstuk}\,+\,\underline{i}\,+\,ssi & \textsc{ sg tra } \\
\underline{galstuk}\,+\,k\,+\,\underline{i}\,+\,ssaa & \textsc{ sg ter } \\
\underline{galstuk}\,+\,\underline{i}\,+\,ka & \textsc{ sg com } \\
\underline{galstuk}\,+\,\underline{i}\,+\,d & \textsc{ pl nom } \\
\underline{galstuk}\,+\,k\,+\,\underline{i}\,+\,jõ & \textsc{ pl gen } \\
\underline{galstuk}\,+\,k\,+\,\underline{i}\,+\,it & \textsc{ pl par } \\
\underline{galstuk}\,+\,k\,+\,\underline{i}\,+\,isõ & \textsc{ pl ill } \\
\underline{galstuk}\,+\,k\,+\,\underline{i}\,+\,iz & \textsc{ pl ine } \\
\underline{galstuk}\,+\,k\,+\,\underline{i}\,+\,iss & \textsc{ pl ela } \\
\underline{galstuk}\,+\,k\,+\,\underline{i}\,+\,illõ & \textsc{ pl all } \\
\underline{galstuk}\,+\,k\,+\,\underline{i}\,+\,ill & \textsc{ pl ade } \\
\underline{galstuk}\,+\,k\,+\,\underline{i}\,+\,ilt & \textsc{ pl abl } \\
\underline{galstuk}\,+\,k\,+\,\underline{i}\,+\,issi & \textsc{ pl tra } \\
\underline{galstuk}\,+\,k\,+\,\underline{i}\,+\,issaa & \textsc{ pl ter } \\
\underline{galstuk}\,+\,k\,+\,\underline{i}\,+\,jka & \textsc{ pl com } \\
\end{tabular}
\end{sideways}
\captionof{table}{Tüüpsõna \arabic{mallinumber}\,\textit{galstukki} ekstraheeritud muutvormimallid.}
\label{tab:tüüpsõnamall-galstukki}

\end{minipage}

 
\vspace{1em}
\noindent Tüüpsõna hõlmab vormisõnastiku 7 lekseemi: \vadja{\underline{galstuk}k\underline{i}, \underline{kok}k\underline{i}, \underline{kolk}k\underline{i}, \underline{luuk}k\underline{i}, \underline{puk}k\underline{i}, \underline{vok}k\underline{i}} ja \vadja{\underline{fraak}k\underline{i}}.

Sõnatüübi\-mall kirjeldab tagapoolseid sõnu tüvemuutusega kk:k ja mille lõpuvokaal on \textit{i}. % TODO i tüvi?
\paragraph*{\vadja{\underline{põl}t\underline{o}}}
\vadja{\underline{põl}l\underline{o}}, \vadja{\underline{põl}t\underline{o}a}, \vadja{\underline{põl}t\underline{o}sõ}, \vadja{\underline{põl}l\underline{o}ss}, \vadja{\underline{põl}l\underline{o}d}, \vadja{\underline{põl}t\underline{o}jõ}, \vadja{\underline{põl}t\underline{o}it}, \vadja{\underline{põl}t\underline{o}isõ}, \vadja{\underline{põl}t\underline{o}iss}
 \\
Tüüpsõna hõlmab vormisõnastiku lekseeme: \vadja{põlto, mõlto}.

Sõnatüübi\-mall kirjeldab tagapoolseid sõnu tüvemuutusega lt:ll.

\vspace{1.8em}
\begin{minipage}{\textwidth}
\stepcounter{mallinumber}
\textbf{Tüüpsõnamall \arabic{mallinumber}\,\vadja{greebeni}}\\

\begin{sideways}
\begin{tabular}{l l}
muutvormimall & tunnused \\
\hline
\underline{greebeni} & \textsc{ sg nom } \\
\underline{greebeni} & \textsc{ sg gen } \\
\underline{greebeni}\,+\,ä & \textsc{ sg par } \\
\underline{greebeni}\,+\,se & \textsc{ sg ill } \\
\underline{greebeni}\,+\,z & \textsc{ sg ine } \\
\underline{greebeni}\,+\,ss & \textsc{ sg ela } \\
\underline{greebeni}\,+\,lle & \textsc{ sg all } \\
\underline{greebeni}\,+\,ll & \textsc{ sg ade } \\
\underline{greebeni}\,+\,lt & \textsc{ sg abl } \\
\underline{greebeni}\,+\,ssi & \textsc{ sg tra } \\
\underline{greebeni}\,+\,ssaa & \textsc{ sg ter } \\
\underline{greebeni}\,+\,ka & \textsc{ sg com } \\
\underline{greebeni}\,+\,d & \textsc{ pl nom } \\
\underline{greebeni}\,+\,je & \textsc{ pl gen } \\
\underline{greebeni}\,+\,it & \textsc{ pl par } \\
\underline{greebeni}\,+\,ise & \textsc{ pl ill } \\
\underline{greebeni}\,+\,iz & \textsc{ pl ine } \\
\underline{greebeni}\,+\,iss & \textsc{ pl ela } \\
\underline{greebeni}\,+\,ille & \textsc{ pl all } \\
\underline{greebeni}\,+\,ill & \textsc{ pl ade } \\
\underline{greebeni}\,+\,ilt & \textsc{ pl abl } \\
\underline{greebeni}\,+\,issi & \textsc{ pl tra } \\
\underline{greebeni}\,+\,issaa & \textsc{ pl ter } \\
\underline{greebeni}\,+\,jka & \textsc{ pl com } \\
\end{tabular}
\end{sideways}
\captionof{table}{Tüüpsõna \arabic{mallinumber}\,\textit{greebeni} ekstraheeritud muutvormimallid.}
\label{tab:tüüpsõnamall-greebeni}

\end{minipage}

 
\vspace{1em}
\noindent Tüüpsõna hõlmab vormisõnastiku 15 lekseemi: \vadja{\underline{greebeni}, \underline{Helsengi}, \underline{jevi}, \underline{kiikeri}, \underline{kiisseli}, \underline{meebeli}, \underline{nätel̕i}, \underline{Reeveli}, \underline{retsepti}, \underline{rööveli}, \underline{špeili}, \underline{väli}, \underline{vääri}, \underline{ängeli}} ja \vadja{\underline{bibli}}.

Sõnatüübi\-mall kirjeldab eespoolseid tüvemuutuseta sõnu ja mille lõpuvokaal on \textit{i}. % TODO i tüvi?
\paragraph*{\vadja{\underline{löülü}}}
\vadja{\underline{löülü}}, \vadja{\underline{löülü}ä}, \vadja{\underline{löülü}se}, \vadja{\underline{löülü}ss}, \vadja{\underline{löülü}d}, \vadja{\underline{löülü}je}, \vadja{\underline{löülü}it}, \vadja{\underline{löülü}ise}, \vadja{\underline{löülü}iss}
 \\
Tüüpsõna hõlmab vormisõnastiku lekseeme: \vadja{löülü, süčüzü, jürü}.

Sõnatüübi\-mall kirjeldab eespoolseid tüvemuutuseta sõnu.

\vspace{1.8em}
\begin{minipage}{\textwidth}
\stepcounter{mallinumber}
\textbf{Tüüpsõnamall \arabic{mallinumber}\,\vadja{airo}}\\

\begin{sideways}
\begin{tabular}{l l}
muutvormimall & tunnused \\
\hline
\underline{airo} & \textsc{ sg nom } \\
\underline{airo} & \textsc{ sg gen } \\
\underline{airo}\,+\,a & \textsc{ sg par } \\
\underline{airo}\,+\,sõ & \textsc{ sg ill } \\
\underline{airo}\,+\,z & \textsc{ sg ine } \\
\underline{airo}\,+\,ss & \textsc{ sg ela } \\
\underline{airo}\,+\,llõ & \textsc{ sg all } \\
\underline{airo}\,+\,ll & \textsc{ sg ade } \\
\underline{airo}\,+\,lt & \textsc{ sg abl } \\
\underline{airo}\,+\,ssi & \textsc{ sg tra } \\
\underline{airo}\,+\,ssaa & \textsc{ sg ter } \\
\underline{airo}\,+\,ka & \textsc{ sg com } \\
\underline{airo}\,+\,d & \textsc{ pl nom } \\
\underline{airo}\,+\,jõ & \textsc{ pl gen } \\
\underline{airo}\,+\,it & \textsc{ pl par } \\
\underline{airo}\,+\,isõ & \textsc{ pl ill } \\
\underline{airo}\,+\,iz & \textsc{ pl ine } \\
\underline{airo}\,+\,iss & \textsc{ pl ela } \\
\underline{airo}\,+\,illõ & \textsc{ pl all } \\
\underline{airo}\,+\,ill & \textsc{ pl ade } \\
\underline{airo}\,+\,ilt & \textsc{ pl abl } \\
\underline{airo}\,+\,issi & \textsc{ pl tra } \\
\underline{airo}\,+\,issaa & \textsc{ pl ter } \\
\underline{airo}\,+\,ika & \textsc{ pl com } \\
\end{tabular}
\end{sideways}
\captionof{table}{Tüüpsõna \arabic{mallinumber}\,\textit{airo} ekstraheeritud muutvormimallid.}
\label{tab:tüüpsõnamall-airo}

\end{minipage}

 
\vspace{1em}
\noindent Tüüpsõna hõlmab vormisõnastiku 42 lekseemi: \vadja{\underline{airo}, \underline{aju}, \underline{anõ}, \underline{čaaju}, \underline{čello}, \underline{elo}, \underline{haadu}, \underline{heeno}, \underline{hlaamu}, \underline{ilo}, \underline{javo}, \underline{jõulu}, \underline{kahu}, \underline{kalmo}, \underline{karu}, \underline{kehno}, \underline{kirstu}, \underline{koivu}, \underline{konno}, \underline{laulu}, \underline{lello}, \underline{morško}, \underline{muru}, \underline{nagru}, \underline{ohtõgo}, \underline{paju}, \underline{paksu}, \underline{pallo}, \underline{passibo}, \underline{pojo}, \underline{saadu}, \underline{savvu}, \underline{siivo}, \underline{škoulu}, \underline{talo}, \underline{varjo}, \underline{vello}, \underline{vilu}, \underline{viro}, \underline{vooro}, \underline{õhtõgo}} ja \vadja{\underline{ahjo}}.

Sõnatüübi\-mall kirjeldab tagapoolseid tüvemuutusea sõnu.

\vspace{1.8em}
\begin{minipage}{\textwidth}
\stepcounter{mallinumber}
\textbf{Tüüpsõnamall \arabic{mallinumber}\,\vadja{bagaži}}\\

\begin{sideways}
\begin{tabular}{l l}
muutvormimall & tunnused \\
\hline
\underline{bagaži} & \textsc{ sg nom } \\
\underline{bagaži} & \textsc{ sg gen } \\
\underline{bagaži}\,+\,a & \textsc{ sg par } \\
\underline{bagaži}\,+\,sõ & \textsc{ sg ill } \\
\underline{bagaži}\,+\,z & \textsc{ sg ine } \\
\underline{bagaži}\,+\,ss & \textsc{ sg ela } \\
\underline{bagaži}\,+\,llõ & \textsc{ sg all } \\
\underline{bagaži}\,+\,ll & \textsc{ sg ade } \\
\underline{bagaži}\,+\,lt & \textsc{ sg abl } \\
\underline{bagaži}\,+\,ssi & \textsc{ sg tra } \\
\underline{bagaži}\,+\,ssaa & \textsc{ sg ter } \\
\underline{bagaži}\,+\,ka & \textsc{ sg com } \\
\underline{bagaži}\,+\,d & \textsc{ pl nom } \\
\underline{bagaži}\,+\,jõ & \textsc{ pl gen } \\
\underline{bagaži}\,+\,it & \textsc{ pl par } \\
\underline{bagaži}\,+\,isõ & \textsc{ pl ill } \\
\underline{bagaži}\,+\,iz & \textsc{ pl ine } \\
\underline{bagaži}\,+\,iss & \textsc{ pl ela } \\
\underline{bagaži}\,+\,illõ & \textsc{ pl all } \\
\underline{bagaži}\,+\,ill & \textsc{ pl ade } \\
\underline{bagaži}\,+\,ilt & \textsc{ pl abl } \\
\underline{bagaži}\,+\,issi & \textsc{ pl tra } \\
\underline{bagaži}\,+\,issaa & \textsc{ pl ter } \\
\underline{bagaži}\,+\,jka & \textsc{ pl com } \\
\end{tabular}
\end{sideways}
\captionof{table}{Tüüpsõna \arabic{mallinumber}\,\textit{bagaži} ekstraheeritud muutvormimallid.}
\label{tab:tüüpsõnamall-bagaži}

\end{minipage}

 
\vspace{1em}
\noindent Tüüpsõna hõlmab vormisõnastiku 39 lekseemi: \vadja{\underline{bagaži}, \underline{balhoni}, \underline{baroni}, \underline{bil̕jardi}, \underline{bobuli}, \underline{bul̕joni}, \underline{d̕ivani}, \underline{dohtõri}, \underline{farfori}, \underline{flaneli}, \underline{gimnazi}, \underline{gitari}, \underline{glazi}, \underline{haili}, \underline{inspektori}, \underline{itkuri}, \underline{jaani}, \underline{kammõri}, \underline{kongressi}, \underline{kuhni}, \underline{lusti}, \underline{makarooni}, \underline{mal̕ari}, \underline{mandõri}, \underline{naapuri}, \underline{nojaabri}, \underline{nuumõri}, \underline{paperi}, \underline{plaastõri}, \underline{pošti}, \underline{stooli}, \underline{suukkuri}, \underline{taari}, \underline{tormi}, \underline{tunni}, \underline{vagzõli}, \underline{vari}, \underline{vinkuri}} ja \vadja{\underline{almõzi}}.

Sõnatüübi\-mall kirjeldab tagapoolseid tüvemuutuseta sõnu ja mille lõpuvokaal on \textit{i}. % TODO i tüvi?
\paragraph*{\vadja{\underline{poštaljon}}}
\vadja{\underline{poštaljon}i}, \vadja{\underline{poštaljon}ia}, \vadja{\underline{poštaljon}isõ}, \vadja{\underline{poštaljon}iss}, \vadja{\underline{poštaljon}id}, \vadja{\underline{poštaljon}ijõ}, \vadja{\underline{poštaljon}iit}, \vadja{\underline{poštaljon}iisõ}, \vadja{\underline{poštaljon}iiss}
 \\
sõnatüüp hõlmab lekseeme \vadja{poštaljon, parad}

Sõnatüübi\-mall kirjeldab tagapoolseid tüvemuutuseta sõnu.

\vspace{1.8em}
\begin{minipage}{\textwidth}
\stepcounter{mallinumber}
\textbf{Tüüpsõnamall \arabic{mallinumber}\,\vadja{sünti}}\\

\begin{sideways}
\begin{tabular}{l l}
muutvormimall & tunnused \\
\hline
\underline{sün}\,+\,t\,+\,\underline{i} & \textsc{ sg nom } \\
\underline{sün}\,+\,n\,+\,\underline{i} & \textsc{ sg gen } \\
\underline{sün}\,+\,t\,+\,\underline{i}\,+\,ä & \textsc{ sg par } \\
\underline{sün}\,+\,t\,+\,\underline{i}\,+\,se & \textsc{ sg ill } \\
\underline{sün}\,+\,n\,+\,\underline{i}\,+\,z & \textsc{ sg ine } \\
\underline{sün}\,+\,n\,+\,\underline{i}\,+\,ss & \textsc{ sg ela } \\
\underline{sün}\,+\,n\,+\,\underline{i}\,+\,lle & \textsc{ sg all } \\
\underline{sün}\,+\,n\,+\,\underline{i}\,+\,ll & \textsc{ sg ade } \\
\underline{sün}\,+\,n\,+\,\underline{i}\,+\,lt & \textsc{ sg abl } \\
\underline{sün}\,+\,n\,+\,\underline{i}\,+\,ssi & \textsc{ sg tra } \\
\underline{sün}\,+\,n\,+\,\underline{i}\,+\,ssaa & \textsc{ sg ter } \\
\underline{sün}\,+\,n\,+\,\underline{i}\,+\,ka & \textsc{ sg com } \\
\underline{sün}\,+\,n\,+\,\underline{i}\,+\,d & \textsc{ pl nom } \\
\underline{sün}\,+\,t\,+\,\underline{i}\,+\,je & \textsc{ pl gen } \\
\underline{sün}\,+\,t\,+\,\underline{i}\,+\,it & \textsc{ pl par } \\
\underline{sün}\,+\,t\,+\,\underline{i}\,+\,ise & \textsc{ pl ill } \\
\underline{sün}\,+\,t\,+\,\underline{i}\,+\,iz & \textsc{ pl ine } \\
\underline{sün}\,+\,t\,+\,\underline{i}\,+\,iss & \textsc{ pl ela } \\
\underline{sün}\,+\,t\,+\,\underline{i}\,+\,ille & \textsc{ pl all } \\
\underline{sün}\,+\,t\,+\,\underline{i}\,+\,ill & \textsc{ pl ade } \\
\underline{sün}\,+\,t\,+\,\underline{i}\,+\,ilt & \textsc{ pl abl } \\
\underline{sün}\,+\,t\,+\,\underline{i}\,+\,issi & \textsc{ pl tra } \\
\underline{sün}\,+\,t\,+\,\underline{i}\,+\,issaa & \textsc{ pl ter } \\
\underline{sün}\,+\,t\,+\,\underline{i}\,+\,jka & \textsc{ pl com } \\
\end{tabular}
\end{sideways}
\captionof{table}{Tüüpsõna \arabic{mallinumber}\,\textit{sünti} ekstraheeritud muutvormimallid.}
\label{tab:tüüpsõnamall-sünti}

\end{minipage}

 
\vspace{1em}
\noindent Tüüpsõna ei hõlma teisi lekseeme vormi\-sõnastikus.

Sõnatüübi\-mall kirjeldab eespoolseid sõnu tüvemuutusega nt:nn ja mille lõpuvokaal on \textit{i}.

\vspace{1.8em}
\begin{minipage}{\textwidth}
\stepcounter{mallinumber}
\textbf{Tüüpsõnamall \arabic{mallinumber}\,\vadja{lento}}\\

\begin{sideways}
\begin{tabular}{l l}
muutvormimall & tunnused \\
\hline
\underline{len}\,+\,t\,+\,\underline{o} & \textsc{ sg nom } \\
\underline{len}\,+\,n\,+\,\underline{o} & \textsc{ sg gen } \\
\underline{len}\,+\,t\,+\,\underline{o}\,+\,a & \textsc{ sg par } \\
\underline{len}\,+\,t\,+\,\underline{o}\,+\,sõ & \textsc{ sg ill } \\
\underline{len}\,+\,n\,+\,\underline{o}\,+\,z & \textsc{ sg ine } \\
\underline{len}\,+\,n\,+\,\underline{o}\,+\,ss & \textsc{ sg ela } \\
\underline{len}\,+\,n\,+\,\underline{o}\,+\,llõ & \textsc{ sg all } \\
\underline{len}\,+\,n\,+\,\underline{o}\,+\,ll & \textsc{ sg ade } \\
\underline{len}\,+\,n\,+\,\underline{o}\,+\,lt & \textsc{ sg abl } \\
\underline{len}\,+\,n\,+\,\underline{o}\,+\,ssi & \textsc{ sg tra } \\
\underline{len}\,+\,n\,+\,\underline{o}\,+\,ssaa & \textsc{ sg ter } \\
\underline{len}\,+\,n\,+\,\underline{o}\,+\,ka & \textsc{ sg com } \\
\underline{len}\,+\,n\,+\,\underline{o}\,+\,d & \textsc{ pl nom } \\
\underline{len}\,+\,t\,+\,\underline{o}\,+\,jõ & \textsc{ pl gen } \\
\underline{len}\,+\,t\,+\,\underline{o}\,+\,it & \textsc{ pl par } \\
\underline{len}\,+\,t\,+\,\underline{o}\,+\,isõ & \textsc{ pl ill } \\
\underline{len}\,+\,t\,+\,\underline{o}\,+\,iz & \textsc{ pl ine } \\
\underline{len}\,+\,t\,+\,\underline{o}\,+\,iss & \textsc{ pl ela } \\
\underline{len}\,+\,t\,+\,\underline{o}\,+\,illõ & \textsc{ pl all } \\
\underline{len}\,+\,t\,+\,\underline{o}\,+\,ill & \textsc{ pl ade } \\
\underline{len}\,+\,t\,+\,\underline{o}\,+\,ilt & \textsc{ pl abl } \\
\underline{len}\,+\,t\,+\,\underline{o}\,+\,issi & \textsc{ pl tra } \\
\underline{len}\,+\,t\,+\,\underline{o}\,+\,issaa & \textsc{ pl ter } \\
\underline{len}\,+\,t\,+\,\underline{o}\,+\,ika & \textsc{ pl com } \\
\end{tabular}
\end{sideways}
\captionof{table}{Tüüpsõna \arabic{mallinumber}\,\textit{lento} ekstraheeritud muutvormimallid.}
\label{tab:tüüpsõnamall-lento}

\end{minipage}

 
\vspace{1em}
\noindent Tüüpsõna hõlmab vormisõnastiku 4 lekseemi: \vadja{\underline{len}t\underline{o}, \underline{lin}t\underline{u}, \underline{rokkalin}t\underline{u}} ja \vadja{\underline{kan}t\underline{o}}.

Sõnatüübi\-mall kirjeldab tagapoolseid sõnu tüvemuutusega nt:nn.

\vspace{1.8em}
\begin{minipage}{\textwidth}
\stepcounter{mallinumber}
\textbf{Tüüpsõnamall \arabic{mallinumber}\,\vadja{vipu}}\\

\begin{sideways}
\begin{tabular}{l l}
muutvormimall & tunnused \\
\hline
\underline{vi}\,+\,p\,+\,\underline{u} & \textsc{ sg nom } \\
\underline{vi}\,+\,v\,+\,\underline{u} & \textsc{ sg gen } \\
\underline{vi}\,+\,pp\,+\,\underline{u}\,+\,a & \textsc{ sg par } \\
\underline{vi}\,+\,pp\,+\,\underline{u}\,+\,sõ & \textsc{ sg ill } \\
\underline{vi}\,+\,v\,+\,\underline{u}\,+\,z & \textsc{ sg ine } \\
\underline{vi}\,+\,v\,+\,\underline{u}\,+\,ss & \textsc{ sg ela } \\
\underline{vi}\,+\,v\,+\,\underline{u}\,+\,llõ & \textsc{ sg all } \\
\underline{vi}\,+\,v\,+\,\underline{u}\,+\,ll & \textsc{ sg ade } \\
\underline{vi}\,+\,v\,+\,\underline{u}\,+\,lt & \textsc{ sg abl } \\
\underline{vi}\,+\,v\,+\,\underline{u}\,+\,ssi & \textsc{ sg tra } \\
\underline{vi}\,+\,v\,+\,\underline{u}\,+\,ssaa & \textsc{ sg ter } \\
\underline{vi}\,+\,v\,+\,\underline{u}\,+\,ka & \textsc{ sg com } \\
\underline{vi}\,+\,v\,+\,\underline{u}\,+\,d & \textsc{ pl nom } \\
\underline{vi}\,+\,p\,+\,\underline{u}\,+\,jõ & \textsc{ pl gen } \\
\underline{vi}\,+\,p\,+\,\underline{u}\,+\,it & \textsc{ pl par } \\
\underline{vi}\,+\,p\,+\,\underline{u}\,+\,isõ & \textsc{ pl ill } \\
\underline{vi}\,+\,p\,+\,\underline{u}\,+\,iz & \textsc{ pl ine } \\
\underline{vi}\,+\,p\,+\,\underline{u}\,+\,iss & \textsc{ pl ela } \\
\underline{vi}\,+\,p\,+\,\underline{u}\,+\,illõ & \textsc{ pl all } \\
\underline{vi}\,+\,p\,+\,\underline{u}\,+\,ill & \textsc{ pl ade } \\
\underline{vi}\,+\,p\,+\,\underline{u}\,+\,ilt & \textsc{ pl abl } \\
\underline{vi}\,+\,p\,+\,\underline{u}\,+\,issi & \textsc{ pl tra } \\
\underline{vi}\,+\,p\,+\,\underline{u}\,+\,issaa & \textsc{ pl ter } \\
\underline{vi}\,+\,p\,+\,\underline{u}\,+\,ika & \textsc{ pl com } \\
\end{tabular}
\end{sideways}
\captionof{table}{Tüüpsõna \arabic{mallinumber}\,\textit{vipu} ekstraheeritud muutvormimallid.}
\label{tab:tüüpsõnamall-vipu}

\end{minipage}

 
\vspace{1em}
\noindent Tüüpsõna ei hõlma teisi lekseeme vormi\-sõnastikus.

Sõnatüübi\-mall kirjeldab tagapoolseid sõnu tüvemuutusega p:v.
\paragraph*{\vadja{\underline{hap}\underline{o}}}
\vadja{\underline{hap}\underline{o}}, \vadja{\underline{hap}p\underline{o}a}, \vadja{\underline{hap}p\underline{o}sõ}, \vadja{\underline{hap}\underline{o}ss}, \vadja{\underline{hap}\underline{o}d}, \vadja{\underline{hap}p\underline{o}jõ}, \vadja{\underline{hap}p\underline{o}it}, \vadja{\underline{hap}p\underline{o}isõ}, \vadja{\underline{hap}p\underline{o}iss}
 \\
Sõnatüüp ei hõlma teisi lekseeme vormi\-sõnastikus.

Sõnatüübi\-mall kirjeldab tagapoolseid sõnu tüvemuutusega p:pp.
\paragraph*{\vadja{\underline{vilp}p\underline{i}}}
\vadja{\underline{vilp}\underline{i}}, \vadja{\underline{vilp}p\underline{i}ä}, \vadja{\underline{vilp}p\underline{i}se}, \vadja{\underline{vilp}\underline{i}ss}, \vadja{\underline{vilp}\underline{i}d}, \vadja{\underline{vilp}p\underline{i}je}, \vadja{\underline{vilp}p\underline{i}it}, \vadja{\underline{vilp}p\underline{i}ise}, \vadja{\underline{vilp}p\underline{i}iss}
 \\
Tüüpsõna hõlmab vormisõnastiku lekseeme: \vadja{vilppi, šlääppi}.

Sõnatüübi\-mall kirjeldab eespoolseid sõnu tüvemuutusega pp:p ja mille lõpuvokaal on \textit{i}.% TODO i tüvi?
\paragraph*{\vadja{\underline{hüp}p\underline{ü}}}
\vadja{\underline{hüp}\underline{ü}}, \vadja{\underline{hüp}p\underline{ü}ä}, \vadja{\underline{hüp}p\underline{ü}se}, \vadja{\underline{hüp}\underline{ü}ss}, \vadja{\underline{hüp}\underline{ü}d}, \vadja{\underline{hüp}p\underline{ü}je}, \vadja{\underline{hüp}p\underline{ü}it}, \vadja{\underline{hüp}p\underline{ü}ise}, \vadja{\underline{hüp}p\underline{ü}iss}
 \\
Tüüpsõna ei hõlma teisi lekseeme vormi\-sõnastikus.

Sõnatüübi\-mall kirjeldab eespoolseid sõnu tüvemuutusega pp:p.

\vspace{1.8em}
\begin{minipage}{\textwidth}
\stepcounter{mallinumber}
\textbf{Tüüpsõnamall \arabic{mallinumber}\,\vadja{lippu}}\\

\begin{sideways}
\begin{tabular}{l l}
muutvormimall & tunnused \\
\hline
\underline{lip}\,+\,p\,+\,\underline{u} & \textsc{ sg nom } \\
\underline{lip}\,+\,\underline{u} & \textsc{ sg gen } \\
\underline{lip}\,+\,p\,+\,\underline{u}\,+\,a & \textsc{ sg par } \\
\underline{lip}\,+\,p\,+\,\underline{u}\,+\,sõ & \textsc{ sg ill } \\
\underline{lip}\,+\,p\,+\,\underline{u}\,+\,z & \textsc{ sg ine } \\
\underline{lip}\,+\,\underline{u}\,+\,ss & \textsc{ sg ela } \\
\underline{lip}\,+\,\underline{u}\,+\,llõ & \textsc{ sg all } \\
\underline{lip}\,+\,\underline{u}\,+\,ll & \textsc{ sg ade } \\
\underline{lip}\,+\,\underline{u}\,+\,lt & \textsc{ sg abl } \\
\underline{lip}\,+\,\underline{u}\,+\,ssi & \textsc{ sg tra } \\
\underline{lip}\,+\,p\,+\,\underline{u}\,+\,ssaa & \textsc{ sg ter } \\
\underline{lip}\,+\,\underline{u}\,+\,ka & \textsc{ sg com } \\
\underline{lip}\,+\,\underline{u}\,+\,d & \textsc{ pl nom } \\
\underline{lip}\,+\,p\,+\,\underline{u}\,+\,jõ & \textsc{ pl gen } \\
\underline{lip}\,+\,p\,+\,\underline{u}\,+\,it & \textsc{ pl par } \\
\underline{lip}\,+\,p\,+\,\underline{u}\,+\,isõ & \textsc{ pl ill } \\
\underline{lip}\,+\,p\,+\,\underline{u}\,+\,iz & \textsc{ pl ine } \\
\underline{lip}\,+\,p\,+\,\underline{u}\,+\,iss & \textsc{ pl ela } \\
\underline{lip}\,+\,p\,+\,\underline{u}\,+\,illõ & \textsc{ pl all } \\
\underline{lip}\,+\,p\,+\,\underline{u}\,+\,ill & \textsc{ pl ade } \\
\underline{lip}\,+\,p\,+\,\underline{u}\,+\,ilt & \textsc{ pl abl } \\
\underline{lip}\,+\,p\,+\,\underline{u}\,+\,issi & \textsc{ pl tra } \\
\underline{lip}\,+\,p\,+\,\underline{u}\,+\,issaa & \textsc{ pl ter } \\
\underline{lip}\,+\,p\,+\,\underline{u}\,+\,ika & \textsc{ pl com } \\
\end{tabular}
\end{sideways}
\captionof{table}{Tüüpsõna \arabic{mallinumber}\,\textit{lippu} ekstraheeritud muutvormimallid.}
\label{tab:tüüpsõnamall-lippu}

\end{minipage}

 
\vspace{1em}
\noindent Tüüpsõna hõlmab vormisõnastiku 4 lekseemi: \vadja{\underline{lip}p\underline{u}, \underline{lõp}p\underline{u}, \underline{puip}p\underline{u}} ja \vadja{\underline{kip}p\underline{u}}.

Sõnatüübi\-mall kirjeldab tagapoolseid sõnu tüvemuutusega pp:p.

\vspace{1.8em}
\begin{minipage}{\textwidth}
\stepcounter{mallinumber}
\textbf{Tüüpsõnamall \arabic{mallinumber}\,\vadja{lamppi}}\\

\begin{sideways}
\begin{tabular}{l l}
muutvormimall & tunnused \\
\hline
\underline{lamp}\,+\,p\,+\,\underline{i} & \textsc{ sg nom } \\
\underline{lamp}\,+\,\underline{i} & \textsc{ sg gen } \\
\underline{lamp}\,+\,p\,+\,\underline{i}\,+\,a & \textsc{ sg par } \\
\underline{lamp}\,+\,p\,+\,\underline{i}\,+\,sõ & \textsc{ sg ill } \\
\underline{lamp}\,+\,p\,+\,\underline{i}\,+\,z & \textsc{ sg ine } \\
\underline{lamp}\,+\,\underline{i}\,+\,ss & \textsc{ sg ela } \\
\underline{lamp}\,+\,\underline{i}\,+\,llõ & \textsc{ sg all } \\
\underline{lamp}\,+\,\underline{i}\,+\,ll & \textsc{ sg ade } \\
\underline{lamp}\,+\,\underline{i}\,+\,lt & \textsc{ sg abl } \\
\underline{lamp}\,+\,\underline{i}\,+\,ssi & \textsc{ sg tra } \\
\underline{lamp}\,+\,p\,+\,\underline{i}\,+\,ssaa & \textsc{ sg ter } \\
\underline{lamp}\,+\,\underline{i}\,+\,ka & \textsc{ sg com } \\
\underline{lamp}\,+\,\underline{i}\,+\,d & \textsc{ pl nom } \\
\underline{lamp}\,+\,p\,+\,\underline{i}\,+\,jõ & \textsc{ pl gen } \\
\underline{lamp}\,+\,p\,+\,\underline{i}\,+\,it & \textsc{ pl par } \\
\underline{lamp}\,+\,p\,+\,\underline{i}\,+\,isõ & \textsc{ pl ill } \\
\underline{lamp}\,+\,p\,+\,\underline{i}\,+\,iz & \textsc{ pl ine } \\
\underline{lamp}\,+\,p\,+\,\underline{i}\,+\,iss & \textsc{ pl ela } \\
\underline{lamp}\,+\,p\,+\,\underline{i}\,+\,illõ & \textsc{ pl all } \\
\underline{lamp}\,+\,p\,+\,\underline{i}\,+\,ill & \textsc{ pl ade } \\
\underline{lamp}\,+\,p\,+\,\underline{i}\,+\,ilt & \textsc{ pl abl } \\
\underline{lamp}\,+\,p\,+\,\underline{i}\,+\,issi & \textsc{ pl tra } \\
\underline{lamp}\,+\,p\,+\,\underline{i}\,+\,issaa & \textsc{ pl ter } \\
\underline{lamp}\,+\,p\,+\,\underline{i}\,+\,jka & \textsc{ pl com } \\
\end{tabular}
\end{sideways}
\captionof{table}{Tüüpsõna \arabic{mallinumber}\,\textit{lamppi} ekstraheeritud muutvormimallid.}
\label{tab:tüüpsõnamall-lamppi}

\end{minipage}

 
\vspace{1em}
\noindent Tüüpsõna hõlmab vormisõnastiku 5 lekseemi: \vadja{\underline{lamp}p\underline{i}, \underline{pap}p\underline{i}, \underline{sup}p\underline{i}, \underline{ukrop}p\underline{i}} ja \vadja{\underline{kaap}p\underline{i}}.

Sõnatüübi\-mall kirjeldab tagapoolseid sõnu tüvemuutusega pp:p ja mille lõpuvokaal on \textit{i}.% TODO i tüvi?

\vspace{1.8em}
\begin{minipage}{\textwidth}
\stepcounter{mallinumber}
\textbf{Tüüpsõnamall \arabic{mallinumber}\,\vadja{sese}}\\

\begin{sideways}
\begin{tabular}{l l}
muutvormimall & tunnused \\
\hline
\underline{se}\,+\,s\,+\,\underline{e} & \textsc{ sg nom } \\
\underline{se}\,+\,z\,+\,\underline{e} & \textsc{ sg gen } \\
\underline{se}\,+\,ss\,+\,\underline{e}\,+\,ä & \textsc{ sg par } \\
\underline{se}\,+\,ss\,+\,\underline{e}\,+\,se & \textsc{ sg ill } \\
\underline{se}\,+\,z\,+\,\underline{e}\,+\,z & \textsc{ sg ine } \\
\underline{se}\,+\,z\,+\,\underline{e}\,+\,ss & \textsc{ sg ela } \\
\underline{se}\,+\,z\,+\,\underline{e}\,+\,lle & \textsc{ sg all } \\
\underline{se}\,+\,z\,+\,\underline{e}\,+\,ll & \textsc{ sg ade } \\
\underline{se}\,+\,z\,+\,\underline{e}\,+\,lt & \textsc{ sg abl } \\
\underline{se}\,+\,z\,+\,\underline{e}\,+\,ssi & \textsc{ sg tra } \\
\underline{se}\,+\,z\,+\,\underline{e}\,+\,ssaa & \textsc{ sg ter } \\
\underline{se}\,+\,z\,+\,\underline{e}\,+\,ka & \textsc{ sg com } \\
\underline{se}\,+\,z\,+\,\underline{e}\,+\,d & \textsc{ pl nom } \\
\underline{se}\,+\,s\,+\,\underline{e}\,+\,je & \textsc{ pl gen } \\
\underline{se}\,+\,s\,+\,\underline{e}\,+\,it & \textsc{ pl par } \\
\underline{se}\,+\,s\,+\,\underline{e}\,+\,ise & \textsc{ pl ill } \\
\underline{se}\,+\,s\,+\,\underline{e}\,+\,iz & \textsc{ pl ine } \\
\underline{se}\,+\,s\,+\,\underline{e}\,+\,iss & \textsc{ pl ela } \\
\underline{se}\,+\,s\,+\,\underline{e}\,+\,ille & \textsc{ pl all } \\
\underline{se}\,+\,s\,+\,\underline{e}\,+\,ill & \textsc{ pl ade } \\
\underline{se}\,+\,s\,+\,\underline{e}\,+\,ilt & \textsc{ pl abl } \\
\underline{se}\,+\,s\,+\,\underline{e}\,+\,issi & \textsc{ pl tra } \\
\underline{se}\,+\,s\,+\,\underline{e}\,+\,issaa & \textsc{ pl ter } \\
\underline{se}\,+\,s\,+\,\underline{e}\,+\,ika & \textsc{ pl com } \\
\end{tabular}
\end{sideways}
\captionof{table}{Tüüpsõna \arabic{mallinumber}\,\textit{sese} ekstraheeritud muutvormimallid.}
\label{tab:tüüpsõnamall-sese}

\end{minipage}

 
\vspace{1em}
\noindent Tüüpsõna hõlmab vormisõnastiku 2 lekseemi: \vadja{\underline{se}s\underline{e}} ja \vadja{\underline{lä}s\underline{ü}}.

Sõnatüübi\-mall kirjeldab eespoolseid sõnu tüvemuutusega s:z.
\paragraph*{\vadja{\underline{si}s\underline{o}}}
\vadja{\underline{si}z\underline{o}}, \vadja{\underline{si}ss\underline{o}a}, \vadja{\underline{si}ss\underline{o}sõ}, \vadja{\underline{si}z\underline{o}ss}, \vadja{\underline{si}z\underline{o}d}, \vadja{\underline{si}s\underline{o}jõ}, \vadja{\underline{si}s\underline{o}it}, \vadja{\underline{si}s\underline{o}isõ}, \vadja{\underline{si}s\underline{o}iss}
 \\
Tüüpsõna hõlmab vormisõnastiku lekseeme \vadja{siso, nisu}.

Sõnatüübi\-mall kirjeldab tagapoolseid sõnu tüvemuutusega s:z.

\vspace{1.8em}
\begin{minipage}{\textwidth}
\stepcounter{mallinumber}
\textbf{Tüüpsõnamall \arabic{mallinumber}\,\vadja{mahsu}}\\

\begin{sideways}
\begin{tabular}{l l}
muutvormimall & tunnused \\
\hline
\underline{mah}\,+\,s\,+\,\underline{u} & \textsc{ sg nom } \\
\underline{mah}\,+\,z\,+\,\underline{u} & \textsc{ sg gen } \\
\underline{mah}\,+\,s\,+\,\underline{u}\,+\,a & \textsc{ sg par } \\
\underline{mah}\,+\,s\,+\,\underline{u}\,+\,sõ & \textsc{ sg ill } \\
\underline{mah}\,+\,z\,+\,\underline{u}\,+\,z & \textsc{ sg ine } \\
\underline{mah}\,+\,z\,+\,\underline{u}\,+\,ss & \textsc{ sg ela } \\
\underline{mah}\,+\,z\,+\,\underline{u}\,+\,llõ & \textsc{ sg all } \\
\underline{mah}\,+\,z\,+\,\underline{u}\,+\,ll & \textsc{ sg ade } \\
\underline{mah}\,+\,z\,+\,\underline{u}\,+\,lt & \textsc{ sg abl } \\
\underline{mah}\,+\,z\,+\,\underline{u}\,+\,ssi & \textsc{ sg tra } \\
\underline{mah}\,+\,s\,+\,\underline{u}\,+\,ssaa & \textsc{ sg ter } \\
\underline{mah}\,+\,z\,+\,\underline{u}\,+\,ka & \textsc{ sg com } \\
\underline{mah}\,+\,z\,+\,\underline{u}\,+\,d & \textsc{ pl nom } \\
\underline{mah}\,+\,s\,+\,\underline{u}\,+\,jõ & \textsc{ pl gen } \\
\underline{mah}\,+\,s\,+\,\underline{u}\,+\,it & \textsc{ pl par } \\
\underline{mah}\,+\,s\,+\,\underline{u}\,+\,isõ & \textsc{ pl ill } \\
\underline{mah}\,+\,s\,+\,\underline{u}\,+\,iz & \textsc{ pl ine } \\
\underline{mah}\,+\,s\,+\,\underline{u}\,+\,iss & \textsc{ pl ela } \\
\underline{mah}\,+\,s\,+\,\underline{u}\,+\,illõ & \textsc{ pl all } \\
\underline{mah}\,+\,s\,+\,\underline{u}\,+\,ill & \textsc{ pl ade } \\
\underline{mah}\,+\,s\,+\,\underline{u}\,+\,ilt & \textsc{ pl abl } \\
\underline{mah}\,+\,s\,+\,\underline{u}\,+\,issi & \textsc{ pl tra } \\
\underline{mah}\,+\,s\,+\,\underline{u}\,+\,issaa & \textsc{ pl ter } \\
\underline{mah}\,+\,s\,+\,\underline{u}\,+\,ika & \textsc{ pl com } \\
\end{tabular}
\end{sideways}
\captionof{table}{Tüüpsõna \arabic{mallinumber}\,\textit{mahsu} ekstraheeritud muutvormimallid.}
\label{tab:tüüpsõnamall-mahsu}

\end{minipage}

 
\vspace{1em}
\noindent Tüüpsõna hõlmab vormisõnastiku 2 lekseemi: \vadja{\underline{mah}s\underline{u}} ja \vadja{\underline{hai}s\underline{u}}.

Sõnatüübi\-mall kirjeldab tagapoolseid sõnu tüvemuutusega s:z, mille tüvi konsonant\-klustri tõttu ei gemineeru.

\vspace{1.8em}
\begin{minipage}{\textwidth}
\stepcounter{mallinumber}
\textbf{Tüüpsõnamall \arabic{mallinumber}\,\vadja{kursi}}\\

\begin{sideways}
\begin{tabular}{l l}
muutvormimall & tunnused \\
\hline
\underline{kur}\,+\,s\,+\,\underline{i} & \textsc{ sg nom } \\
\underline{kur}\,+\,z\,+\,\underline{i} & \textsc{ sg gen } \\
\underline{kur}\,+\,s\,+\,\underline{i}\,+\,a & \textsc{ sg par } \\
\underline{kur}\,+\,s\,+\,\underline{i}\,+\,sõ & \textsc{ sg ill } \\
\underline{kur}\,+\,z\,+\,\underline{i}\,+\,z & \textsc{ sg ine } \\
\underline{kur}\,+\,z\,+\,\underline{i}\,+\,ss & \textsc{ sg ela } \\
\underline{kur}\,+\,z\,+\,\underline{i}\,+\,llõ & \textsc{ sg all } \\
\underline{kur}\,+\,z\,+\,\underline{i}\,+\,ll & \textsc{ sg ade } \\
\underline{kur}\,+\,z\,+\,\underline{i}\,+\,lt & \textsc{ sg abl } \\
\underline{kur}\,+\,z\,+\,\underline{i}\,+\,ssi & \textsc{ sg tra } \\
\underline{kur}\,+\,s\,+\,\underline{i}\,+\,ssaa & \textsc{ sg ter } \\
\underline{kur}\,+\,z\,+\,\underline{i}\,+\,ka & \textsc{ sg com } \\
\underline{kur}\,+\,z\,+\,\underline{i}\,+\,d & \textsc{ pl nom } \\
\underline{kur}\,+\,s\,+\,\underline{i}\,+\,jõ & \textsc{ pl gen } \\
\underline{kur}\,+\,s\,+\,\underline{i}\,+\,it & \textsc{ pl par } \\
\underline{kur}\,+\,s\,+\,\underline{i}\,+\,isõ & \textsc{ pl ill } \\
\underline{kur}\,+\,s\,+\,\underline{i}\,+\,iz & \textsc{ pl ine } \\
\underline{kur}\,+\,s\,+\,\underline{i}\,+\,iss & \textsc{ pl ela } \\
\underline{kur}\,+\,s\,+\,\underline{i}\,+\,illõ & \textsc{ pl all } \\
\underline{kur}\,+\,s\,+\,\underline{i}\,+\,ill & \textsc{ pl ade } \\
\underline{kur}\,+\,s\,+\,\underline{i}\,+\,ilt & \textsc{ pl abl } \\
\underline{kur}\,+\,s\,+\,\underline{i}\,+\,issi & \textsc{ pl tra } \\
\underline{kur}\,+\,s\,+\,\underline{i}\,+\,issaa & \textsc{ pl ter } \\
\underline{kur}\,+\,s\,+\,\underline{i}\,+\,jka & \textsc{ pl com } \\
\end{tabular}
\end{sideways}
\captionof{table}{Tüüpsõna \arabic{mallinumber}\,\textit{kursi} ekstraheeritud muutvormimallid.}
\label{tab:tüüpsõnamall-kursi}

\end{minipage}

 
\vspace{1em}
\noindent Tüüpsõna ei hõlma teisi lekseeme vormi\-sõnastikus.

Sõnatüübi\-mall kirjeldab tagapoolseid sõnu tüvemuutusega s:z, mille tüvi konsonant\-klustri tõttu ei gemineeru ja mille lõpuvokaal on \textit{i}. % TODO i tüvi?

\vspace{1.8em}
\begin{minipage}{\textwidth}
\stepcounter{mallinumber}
\textbf{Tüüpsõnamall \arabic{mallinumber}\,\vadja{rusko}}\\

\begin{sideways}
\begin{tabular}{l l}
muutvormimall & tunnused \\
\hline
\underline{ru}\,+\,sk\,+\,\underline{o} & \textsc{ sg nom } \\
\underline{ru}\,+\,zg\,+\,\underline{o} & \textsc{ sg gen } \\
\underline{ru}\,+\,sk\,+\,\underline{o}\,+\,a & \textsc{ sg par } \\
\underline{ru}\,+\,sk\,+\,\underline{o}\,+\,sõ & \textsc{ sg ill } \\
\underline{ru}\,+\,zg\,+\,\underline{o}\,+\,z & \textsc{ sg ine } \\
\underline{ru}\,+\,zg\,+\,\underline{o}\,+\,ss & \textsc{ sg ela } \\
\underline{ru}\,+\,zg\,+\,\underline{o}\,+\,llõ & \textsc{ sg all } \\
\underline{ru}\,+\,zg\,+\,\underline{o}\,+\,ll & \textsc{ sg ade } \\
\underline{ru}\,+\,zg\,+\,\underline{o}\,+\,lt & \textsc{ sg abl } \\
\underline{ru}\,+\,zg\,+\,\underline{o}\,+\,ssi & \textsc{ sg tra } \\
\underline{ru}\,+\,zg\,+\,\underline{o}\,+\,ssaa & \textsc{ sg ter } \\
\underline{ru}\,+\,zg\,+\,\underline{o}\,+\,ka & \textsc{ sg com } \\
\underline{ru}\,+\,zg\,+\,\underline{o}\,+\,d & \textsc{ pl nom } \\
\underline{ru}\,+\,sk\,+\,\underline{o}\,+\,jõ & \textsc{ pl gen } \\
\underline{ru}\,+\,sk\,+\,\underline{o}\,+\,it & \textsc{ pl par } \\
\underline{ru}\,+\,sk\,+\,\underline{o}\,+\,isõ & \textsc{ pl ill } \\
\underline{ru}\,+\,sk\,+\,\underline{o}\,+\,iz & \textsc{ pl ine } \\
\underline{ru}\,+\,sk\,+\,\underline{o}\,+\,iss & \textsc{ pl ela } \\
\underline{ru}\,+\,sk\,+\,\underline{o}\,+\,illõ & \textsc{ pl all } \\
\underline{ru}\,+\,sk\,+\,\underline{o}\,+\,ill & \textsc{ pl ade } \\
\underline{ru}\,+\,sk\,+\,\underline{o}\,+\,ilt & \textsc{ pl abl } \\
\underline{ru}\,+\,sk\,+\,\underline{o}\,+\,issi & \textsc{ pl tra } \\
\underline{ru}\,+\,sk\,+\,\underline{o}\,+\,issaa & \textsc{ pl ter } \\
\underline{ru}\,+\,sk\,+\,\underline{o}\,+\,ika & \textsc{ pl com } \\
\end{tabular}
\end{sideways}
\captionof{table}{Tüüpsõna \arabic{mallinumber}\,\textit{rusko} ekstraheeritud muutvormimallid.}
\label{tab:tüüpsõnamall-rusko}

\end{minipage}

 
\vspace{1em}
\noindent Tüüpsõna hõlmab vormisõnastiku 4 lekseemi: \vadja{\underline{ru}sk\underline{o}, \underline{tui}sk\underline{u}, \underline{u}sk\underline{o}} ja \vadja{\underline{pää}sk\underline{o}}.

Sõnatüübi\-mall kirjeldab tagapoolseid sõnu tüvemuutusega sk:zg.

\vspace{1.8em}
\begin{minipage}{\textwidth}
\stepcounter{mallinumber}
\textbf{Tüüpsõnamall \arabic{mallinumber}\,\vadja{rissi}}\\

\begin{sideways}
\begin{tabular}{l l}
muutvormimall & tunnused \\
\hline
\underline{ris}\,+\,s\,+\,\underline{i} & \textsc{ sg nom } \\
\underline{ris}\,+\,\underline{i} & \textsc{ sg gen } \\
\underline{ris}\,+\,s\,+\,\underline{i}\,+\,ä & \textsc{ sg par } \\
\underline{ris}\,+\,s\,+\,\underline{i}\,+\,se & \textsc{ sg ill } \\
\underline{ris}\,+\,s\,+\,\underline{i}\,+\,z & \textsc{ sg ine } \\
\underline{ris}\,+\,\underline{i}\,+\,ss & \textsc{ sg ela } \\
\underline{ris}\,+\,\underline{i}\,+\,lle & \textsc{ sg all } \\
\underline{ris}\,+\,\underline{i}\,+\,ll & \textsc{ sg ade } \\
\underline{ris}\,+\,\underline{i}\,+\,lt & \textsc{ sg abl } \\
\underline{ris}\,+\,\underline{i}\,+\,ssi & \textsc{ sg tra } \\
\underline{ris}\,+\,s\,+\,\underline{i}\,+\,ssaa & \textsc{ sg ter } \\
\underline{ris}\,+\,\underline{i}\,+\,ka & \textsc{ sg com } \\
\underline{ris}\,+\,\underline{i}\,+\,d & \textsc{ pl nom } \\
\underline{ris}\,+\,s\,+\,\underline{i}\,+\,je & \textsc{ pl gen } \\
\underline{ris}\,+\,s\,+\,\underline{i}\,+\,it & \textsc{ pl par } \\
\underline{ris}\,+\,s\,+\,\underline{i}\,+\,ise & \textsc{ pl ill } \\
\underline{ris}\,+\,s\,+\,\underline{i}\,+\,iz & \textsc{ pl ine } \\
\underline{ris}\,+\,s\,+\,\underline{i}\,+\,iss & \textsc{ pl ela } \\
\underline{ris}\,+\,s\,+\,\underline{i}\,+\,ille & \textsc{ pl all } \\
\underline{ris}\,+\,s\,+\,\underline{i}\,+\,ill & \textsc{ pl ade } \\
\underline{ris}\,+\,s\,+\,\underline{i}\,+\,ilt & \textsc{ pl abl } \\
\underline{ris}\,+\,s\,+\,\underline{i}\,+\,issi & \textsc{ pl tra } \\
\underline{ris}\,+\,s\,+\,\underline{i}\,+\,issaa & \textsc{ pl ter } \\
\underline{ris}\,+\,s\,+\,\underline{i}\,+\,jka & \textsc{ pl com } \\
\end{tabular}
\end{sideways}
\captionof{table}{Tüüpsõna \arabic{mallinumber}\,\textit{rissi} ekstraheeritud muutvormimallid.}
\label{tab:tüüpsõnamall-rissi}

\end{minipage}

 
\vspace{1em}
\noindent Tüüpsõna ei hõlma teisi lekseeme vormi\-sõnastikus.

Sõnatüübi\-mall kirjeldab eespoolseid sõnu tüvemuutusega ss:s ja mille lõpuvokaal on \textit{i}. % TODO i tüvi?
\paragraph*{\vadja{\underline{pas}s\underline{i}}}
\vadja{\underline{pas}\underline{i}}, \vadja{\underline{pas}s\underline{i}a}, \vadja{\underline{pas}s\underline{i}sõ}, \vadja{\underline{pas}\underline{i}ss}, \vadja{\underline{pas}\underline{i}d}, \vadja{\underline{pas}s\underline{i}jõ}, \vadja{\underline{pas}s\underline{i}it}, \vadja{\underline{pas}s\underline{i}isõ}, \vadja{\underline{pas}s\underline{i}iss}
 \\
Tüüpsõna hõlmab vormisõnastiku lekseeme \vadja{passi, komissi}.

Sõnatüübi\-mall kirjeldab tagapoolseid sõnu tüvemuutusega ss:s ja mille lõpuvokaal on \textit{i}. % TODO i tüvi?
\paragraph*{\vadja{\underline{karjuš}š\underline{i}}}
\vadja{\underline{karjuš}\underline{i}}, \vadja{\underline{karjuš}š\underline{i}a}, \vadja{\underline{karjuš}š\underline{i}sõ}, \vadja{\underline{karjuš}\underline{i}ss}, \vadja{\underline{karjuš}\underline{i}d}, \vadja{\underline{karjuš}š\underline{i}jõ}, \vadja{\underline{karjuš}š\underline{i}it}, \vadja{\underline{karjuš}š\underline{i}isõ}, \vadja{\underline{karjuš}š\underline{i}iss}
 \\
sõnatüüp hõlmab lekseeme \vadja{karjušši, latõšši, potašši, fal̕šši}

Sõnatüübi\-mall kirjeldab tagapoolseid sõnu tüvemuutusega šš:š ja mille lõpuvokaal on \textit{i}. % TODO i tüvi?

\vspace{1.8em}
\begin{minipage}{\textwidth}
\stepcounter{mallinumber}
\textbf{Tüüpsõnamall \arabic{mallinumber}\,\vadja{täti}}\\

\begin{sideways}
\begin{tabular}{l l}
muutvormimall & tunnused \\
\hline
\underline{tä}\,+\,t\,+\,\underline{i} & \textsc{ sg nom } \\
\underline{tä}\,+\,d\,+\,\underline{i} & \textsc{ sg gen } \\
\underline{tä}\,+\,tt\,+\,\underline{i}\,+\,ä & \textsc{ sg par } \\
\underline{tä}\,+\,tt\,+\,\underline{i}\,+\,se & \textsc{ sg ill } \\
\underline{tä}\,+\,d\,+\,\underline{i}\,+\,z & \textsc{ sg ine } \\
\underline{tä}\,+\,d\,+\,\underline{i}\,+\,ss & \textsc{ sg ela } \\
\underline{tä}\,+\,d\,+\,\underline{i}\,+\,lle & \textsc{ sg all } \\
\underline{tä}\,+\,d\,+\,\underline{i}\,+\,ll & \textsc{ sg ade } \\
\underline{tä}\,+\,d\,+\,\underline{i}\,+\,lt & \textsc{ sg abl } \\
\underline{tä}\,+\,d\,+\,\underline{i}\,+\,ssi & \textsc{ sg tra } \\
\underline{tä}\,+\,d\,+\,\underline{i}\,+\,ssaa & \textsc{ sg ter } \\
\underline{tä}\,+\,d\,+\,\underline{i}\,+\,ka & \textsc{ sg com } \\
\underline{tä}\,+\,d\,+\,\underline{i}\,+\,d & \textsc{ pl nom } \\
\underline{tä}\,+\,t\,+\,\underline{i}\,+\,je & \textsc{ pl gen } \\
\underline{tä}\,+\,t\,+\,\underline{i}\,+\,it & \textsc{ pl par } \\
\underline{tä}\,+\,t\,+\,\underline{i}\,+\,ise & \textsc{ pl ill } \\
\underline{tä}\,+\,t\,+\,\underline{i}\,+\,iz & \textsc{ pl ine } \\
\underline{tä}\,+\,t\,+\,\underline{i}\,+\,iss & \textsc{ pl ela } \\
\underline{tä}\,+\,t\,+\,\underline{i}\,+\,ille & \textsc{ pl all } \\
\underline{tä}\,+\,t\,+\,\underline{i}\,+\,ill & \textsc{ pl ade } \\
\underline{tä}\,+\,t\,+\,\underline{i}\,+\,ilt & \textsc{ pl abl } \\
\underline{tä}\,+\,t\,+\,\underline{i}\,+\,issi & \textsc{ pl tra } \\
\underline{tä}\,+\,t\,+\,\underline{i}\,+\,issaa & \textsc{ pl ter } \\
\underline{tä}\,+\,t\,+\,\underline{i}\,+\,jka & \textsc{ pl com } \\
\end{tabular}
\end{sideways}
\captionof{table}{Tüüpsõna \arabic{mallinumber}\,\textit{täti} ekstraheeritud muutvormimallid.}
\label{tab:tüüpsõnamall-täti}

\end{minipage}

 
\vspace{1em}
\noindent Tüüpsõna ei hõlma teisi lekseeme vormi\-sõnastikus.

Sõnatüübi\-mall kirjeldab eespoolseid sõnu tüvemuutusega t:d ja mille lõpuvokaal on \textit{i}. % TODO i tüvi?
\paragraph*{\vadja{\underline{ko}tk\underline{o}}}
\vadja{\underline{ko}dg\underline{o}}, \vadja{\underline{ko}tk\underline{o}a}, \vadja{\underline{ko}tk\underline{o}sõ}, \vadja{\underline{ko}dg\underline{o}ss}, \vadja{\underline{ko}dg\underline{o}d}, \vadja{\underline{ko}tk\underline{o}jõ}, \vadja{\underline{ko}tk\underline{o}it}, \vadja{\underline{ko}tk\underline{o}isõ}, \vadja{\underline{ko}tk\underline{o}iss}
 \\
Tüüpsõna hõlmab vormisõnastiku 3 lekseemi: \vadja{kotko, laatko} ja \vadja{itku}.

Sõnatüübi\-mall kirjeldab tagapoolseid sõnu tüvemuutusega tk:dg.
\paragraph*{\vadja{\underline{kit}t\underline{si}}}
\vadja{\underline{kit}\underline{si}}, \vadja{\underline{kit}t\underline{si}ä}, \vadja{\underline{kit}t\underline{si}se}, \vadja{\underline{kit}\underline{si}ss}, \vadja{\underline{kit}\underline{si}d}, \vadja{\underline{kit}t\underline{si}je}, \vadja{\underline{kit}t\underline{si}it}, \vadja{\underline{kit}t\underline{si}ise}, \vadja{\underline{kit}t\underline{si}iss}
 \\
Tüüpsõna hõlmab vormisõnastiku 4 lekseemi: \vadja{kittsi, komferenttsi, pletti} ja \vadja{biletti}.

Sõnatüübi\-mall kirjeldab eespoolseid sõnu tüvemuutusega tt:t ja mille lõpuvokaal on \textit{i}. % TODO i tüvi?

\vspace{1.8em}
\begin{minipage}{\textwidth}
\stepcounter{mallinumber}
\textbf{Tüüpsõnamall \arabic{mallinumber}\,\vadja{rätte}}\\

\begin{sideways}
\begin{tabular}{l l}
muutvormimall & tunnused \\
\hline
\underline{rät}\,+\,t\,+\,\underline{e} & \textsc{ sg nom } \\
\underline{rät}\,+\,\underline{e} & \textsc{ sg gen } \\
\underline{rät}\,+\,t\,+\,\underline{e}\,+\,ä & \textsc{ sg par } \\
\underline{rät}\,+\,t\,+\,\underline{e}\,+\,se & \textsc{ sg ill } \\
\underline{rät}\,+\,t\,+\,\underline{e}\,+\,z & \textsc{ sg ine } \\
\underline{rät}\,+\,\underline{e}\,+\,ss & \textsc{ sg ela } \\
\underline{rät}\,+\,\underline{e}\,+\,lle & \textsc{ sg all } \\
\underline{rät}\,+\,\underline{e}\,+\,ll & \textsc{ sg ade } \\
\underline{rät}\,+\,\underline{e}\,+\,lt & \textsc{ sg abl } \\
\underline{rät}\,+\,\underline{e}\,+\,ssi & \textsc{ sg tra } \\
\underline{rät}\,+\,t\,+\,\underline{e}\,+\,ssaa & \textsc{ sg ter } \\
\underline{rät}\,+\,\underline{e}\,+\,ka & \textsc{ sg com } \\
\underline{rät}\,+\,\underline{e}\,+\,d & \textsc{ pl nom } \\
\underline{rät}\,+\,t\,+\,\underline{e}\,+\,je & \textsc{ pl gen } \\
\underline{rät}\,+\,t\,+\,\underline{e}\,+\,it & \textsc{ pl par } \\
\underline{rät}\,+\,t\,+\,\underline{e}\,+\,ise & \textsc{ pl ill } \\
\underline{rät}\,+\,t\,+\,\underline{e}\,+\,iz & \textsc{ pl ine } \\
\underline{rät}\,+\,t\,+\,\underline{e}\,+\,iss & \textsc{ pl ela } \\
\underline{rät}\,+\,t\,+\,\underline{e}\,+\,ille & \textsc{ pl all } \\
\underline{rät}\,+\,t\,+\,\underline{e}\,+\,ill & \textsc{ pl ade } \\
\underline{rät}\,+\,t\,+\,\underline{e}\,+\,ilt & \textsc{ pl abl } \\
\underline{rät}\,+\,t\,+\,\underline{e}\,+\,issi & \textsc{ pl tra } \\
\underline{rät}\,+\,t\,+\,\underline{e}\,+\,issaa & \textsc{ pl ter } \\
\underline{rät}\,+\,t\,+\,\underline{e}\,+\,ika & \textsc{ pl com } \\
\end{tabular}
\end{sideways}
\captionof{table}{Tüüpsõna \arabic{mallinumber}\,\textit{rätte} ekstraheeritud muutvormimallid.}
\label{tab:tüüpsõnamall-rätte}

\end{minipage}

 
\vspace{1em}
\noindent Tüüpsõna hõlmab vormisõnastiku 3 lekseemi: \vadja{\underline{rät}t\underline{e}, \underline{tüt}t\underline{ö}} ja \vadja{\underline{nenärät}t\underline{e}}.

Sõnatüübi\-mall kirjeldab eespoolseid sõnu tüvemuutusega tt:t.
\paragraph*{\vadja{\underline{hat}t\underline{u}}}
\vadja{\underline{hat}\underline{u}}, \vadja{\underline{hat}t\underline{u}a}, \vadja{\underline{hat}t\underline{u}sõ}, \vadja{\underline{hat}\underline{u}ss}, \vadja{\underline{hat}\underline{u}d}, \vadja{\underline{hat}t\underline{u}jõ}, \vadja{\underline{hat}t\underline{u}it}, \vadja{\underline{hat}t\underline{u}isõ}, \vadja{\underline{hat}t\underline{u}iss}
 \\
Tüüpsõna hõlmab vormisõnastiku lekseeme: \vadja{hattu, juttu, katto, kuttsu, laatto, lanttu, pal̕tto, porttu, Tarttu, čiutto}.

Sõnatüübi\-mall kirjeldab tagapoolseid sõnu tüvemuutusega tt:t.

\vspace{1.8em}
\begin{minipage}{\textwidth}
\stepcounter{mallinumber}
\textbf{Tüüpsõnamall \arabic{mallinumber}\,\vadja{dokumentti}}\\

\begin{sideways}
\begin{tabular}{l l}
muutvormimall & tunnused \\
\hline
\underline{dokument}\,+\,t\,+\,\underline{i} & \textsc{ sg nom } \\
\underline{dokument}\,+\,\underline{i} & \textsc{ sg gen } \\
\underline{dokument}\,+\,t\,+\,\underline{i}\,+\,a & \textsc{ sg par } \\
\underline{dokument}\,+\,t\,+\,\underline{i}\,+\,sõ & \textsc{ sg ill } \\
\underline{dokument}\,+\,t\,+\,\underline{i}\,+\,z & \textsc{ sg ine } \\
\underline{dokument}\,+\,\underline{i}\,+\,ss & \textsc{ sg ela } \\
\underline{dokument}\,+\,\underline{i}\,+\,llõ & \textsc{ sg all } \\
\underline{dokument}\,+\,\underline{i}\,+\,ll & \textsc{ sg ade } \\
\underline{dokument}\,+\,\underline{i}\,+\,lt & \textsc{ sg abl } \\
\underline{dokument}\,+\,\underline{i}\,+\,ssi & \textsc{ sg tra } \\
\underline{dokument}\,+\,t\,+\,\underline{i}\,+\,ssaa & \textsc{ sg ter } \\
\underline{dokument}\,+\,\underline{i}\,+\,ka & \textsc{ sg com } \\
\underline{dokument}\,+\,\underline{i}\,+\,d & \textsc{ pl nom } \\
\underline{dokument}\,+\,t\,+\,\underline{i}\,+\,jõ & \textsc{ pl gen } \\
\underline{dokument}\,+\,t\,+\,\underline{i}\,+\,it & \textsc{ pl par } \\
\underline{dokument}\,+\,t\,+\,\underline{i}\,+\,isõ & \textsc{ pl ill } \\
\underline{dokument}\,+\,t\,+\,\underline{i}\,+\,iz & \textsc{ pl ine } \\
\underline{dokument}\,+\,t\,+\,\underline{i}\,+\,iss & \textsc{ pl ela } \\
\underline{dokument}\,+\,t\,+\,\underline{i}\,+\,illõ & \textsc{ pl all } \\
\underline{dokument}\,+\,t\,+\,\underline{i}\,+\,ill & \textsc{ pl ade } \\
\underline{dokument}\,+\,t\,+\,\underline{i}\,+\,ilt & \textsc{ pl abl } \\
\underline{dokument}\,+\,t\,+\,\underline{i}\,+\,issi & \textsc{ pl tra } \\
\underline{dokument}\,+\,t\,+\,\underline{i}\,+\,issaa & \textsc{ pl ter } \\
\underline{dokument}\,+\,t\,+\,\underline{i}\,+\,jka & \textsc{ pl com } \\
\end{tabular}
\end{sideways}
\captionof{table}{Tüüpsõna \arabic{mallinumber}\,\textit{dokumentti} ekstraheeritud muutvormimallid.}
\label{tab:tüüpsõnamall-dokumentti}

\end{minipage}

 
\vspace{1em}
\noindent Tüüpsõna hõlmab vormisõnastiku 19 lekseemi: \vadja{\underline{dokument}t\underline{i}, \underline{fabrikant}t\underline{i}, \underline{Frant}t\underline{si}, \underline{fundament}t\underline{i}, \underline{kajut}t\underline{i}, \underline{kamet}t\underline{i}, \underline{kanfet}t\underline{i}, \underline{kat}t\underline{i}, \underline{komet}t\underline{i}, \underline{komfet}t\underline{i}, \underline{komnõt}t\underline{i}, \underline{laut}t\underline{i}, \underline{magnet}t\underline{i}, \underline{minut}t\underline{i}, \underline{muzõkant}t\underline{i}, \underline{protestant}t\underline{i}, \underline{protsent}t\underline{i}, \underline{Root}t\underline{si}} ja \vadja{\underline{bankrut}t\underline{i}}.

Sõnatüübi\-mall kirjeldab tagapoolseid sõnu tüvemuutusega tt:t ja mille lõpuvokaal on \textit{i}.  % TODO i tüvi?
\paragraph*{\vadja{\underline{komit̕et}}}
\vadja{\underline{komit̕et}i}, \vadja{\underline{komit̕et}tiä}, \vadja{\underline{komit̕et}ti}, \vadja{\underline{komit̕et}iss}, \vadja{\underline{komit̕et}id}, \vadja{\underline{komit̕et}tije}, \vadja{\underline{komit̕et}tiit}, \vadja{\underline{komit̕et}tiise}, \vadja{\underline{komit̕et}tiiss}
 \\
Sõnatüüp ei hõlma teisi lekseeme vormi\-sõnastikus.

Sõnatüübi\-mall kirjeldab tagapoolseid sõnu tüvemuutusega tt:t.
%
\vspace{1.8em}
\begin{minipage}{\textwidth}
\stepcounter{mallinumber}
\textbf{Tüüpsõnamall \arabic{mallinumber}\,\vadja{mato}}\\

\begin{sideways}
\begin{tabular}{l l}
muutvormimall & tunnused \\
\hline
\underline{ma}\,+\,t\,+\,\underline{o} & \textsc{ sg nom } \\
\underline{ma}\,+\,\underline{o} & \textsc{ sg gen } \\
\underline{ma}\,+\,tt\,+\,\underline{o}\,+\,a & \textsc{ sg par } \\
\underline{ma}\,+\,tt\,+\,\underline{o}\,+\,sõ & \textsc{ sg ill } \\
\underline{ma}\,+\,\underline{o}\,+\,z & \textsc{ sg ine } \\
\underline{ma}\,+\,\underline{o}\,+\,ss & \textsc{ sg ela } \\
\underline{ma}\,+\,\underline{o}\,+\,llõ & \textsc{ sg all } \\
\underline{ma}\,+\,\underline{o}\,+\,ll & \textsc{ sg ade } \\
\underline{ma}\,+\,\underline{o}\,+\,lt & \textsc{ sg abl } \\
\underline{ma}\,+\,\underline{o}\,+\,ssi & \textsc{ sg tra } \\
\underline{ma}\,+\,\underline{o}\,+\,ssaa & \textsc{ sg ter } \\
\underline{ma}\,+\,\underline{o}\,+\,ka & \textsc{ sg com } \\
\underline{ma}\,+\,\underline{o}\,+\,d & \textsc{ pl nom } \\
\underline{ma}\,+\,t\,+\,\underline{o}\,+\,jõ & \textsc{ pl gen } \\
\underline{ma}\,+\,t\,+\,\underline{o}\,+\,it & \textsc{ pl par } \\
\underline{ma}\,+\,t\,+\,\underline{o}\,+\,isõ & \textsc{ pl ill } \\
\underline{ma}\,+\,t\,+\,\underline{o}\,+\,iz & \textsc{ pl ine } \\
\underline{ma}\,+\,t\,+\,\underline{o}\,+\,iss & \textsc{ pl ela } \\
\underline{ma}\,+\,t\,+\,\underline{o}\,+\,illõ & \textsc{ pl all } \\
\underline{ma}\,+\,t\,+\,\underline{o}\,+\,ill & \textsc{ pl ade } \\
\underline{ma}\,+\,t\,+\,\underline{o}\,+\,ilt & \textsc{ pl abl } \\
\underline{ma}\,+\,t\,+\,\underline{o}\,+\,issi & \textsc{ pl tra } \\
\underline{ma}\,+\,t\,+\,\underline{o}\,+\,issaa & \textsc{ pl ter } \\
\underline{ma}\,+\,t\,+\,\underline{o}\,+\,ika & \textsc{ pl com } \\
\end{tabular}
\end{sideways}
\captionof{table}{Tüüpsõna \arabic{mallinumber}\,\textit{mato} ekstraheeritud muutvormimallid.}
\label{tab:tüüpsõnamall-mato}

\end{minipage}

 
\vspace{1em}
\noindent Tüüpsõna hõlmab vormisõnastiku 5 lekseemi: \vadja{\underline{ma}t\underline{o}, \underline{na}t\underline{o}, \underline{sa}t\underline{o}, \underline{ve}t\underline{o}} ja \vadja{\underline{ko}t\underline{o}}.

%Sõnatüübi\-mall kirjeldab tagapoolseid sõnu tüvemuutusega .
\paragraph*{\vadja{\underline{famil̕}i}}
\vadja{\underline{famil̕}a}, \vadja{\underline{famil̕}a}, \vadja{\underline{famil̕}asõ}, \vadja{\underline{famil̕}õss}, \vadja{\underline{famil̕}õd}, \vadja{\underline{famil̕}ojõ}, \vadja{\underline{famil̕}oit}, \vadja{\underline{famil̕}oisõ}, \vadja{\underline{famil̕}oiss}
 \\
Tüüpsõna hõlmab vormisõnastiku lekseeme \vadja{famil̕i} ja \vadja{bašn̕i}.

Sõnatüübi\-mall kirjeldab tagapoolseid tüvemuutuseta sõnu, mis eeldatavasti palatalisatsiooni tõttu käituvad eriliselt.
\spacing{1.5}


\subsection{\RN{3} käändkond}

Kolmandasse käändkonda kuuluvad kahesilbilised sõnad, mille tüvevokaal on \vadja{-a} ning rohkem silpidega sõnad, millel esineb esimeses silbis \vadja{-a-}, \vadja{-õ-} või \vadja{-i-} \cite[42]{ariste_grammar_1968}.

Avatuid küsimusi-tähelepanekuid:
\begin{itemize}
\item siin on sõnu millel on -u- 1. silbis, need peaksid käima hoopis \RN{5} alla
\item Tsvetkovil palju -õi-mitmusetüvega, need on ühtlustatud -oi-
\item paljud laensõnad kuuluvad siia alla, nende lõpuvokaalidega on häda
\end{itemize}

\subsubsection*{Ekstraktmorfoloogia sõnatüübid}
\spacing{1}
\paragraph*{\vadja{\underline{tuhat}tõ}}
\vadja{\underline{tuhat}ta}, \vadja{\underline{tuhat}ta}, \vadja{\underline{tuhat}tasõ}, \vadja{\underline{tuhat}õss}, \vadja{\underline{tuhat}õd}, \vadja{\underline{tuhat}tojõ}, \vadja{\underline{tuhat}toit}, \vadja{\underline{tuhat}toisõ}, \vadja{\underline{tuhat}toiss}
 \\
Tüüpsõna ei hõlma teisi lekseeme vormi\-sõnastikus.

\paragraph*{\vadja{\underline{la}fkõ}}
\vadja{\underline{la}vga}, \vadja{\underline{la}fka}, \vadja{\underline{la}fkasõ}, \vadja{\underline{la}vgõss}, \vadja{\underline{la}vgõd}, \vadja{\underline{la}fkojõ}, \vadja{\underline{la}fkoit}, \vadja{\underline{la}fkoisõ}, \vadja{\underline{la}fkoiss}
 \\
Sõnatüüp ei hõlma teisi lekseeme vormi\-sõnastikus.

\paragraph*{\vadja{\underline{mah}sõ}}
\vadja{\underline{mah}za}, \vadja{\underline{mah}sa}, \vadja{\underline{mah}sasõ}, \vadja{\underline{mah}zõss}, \vadja{\underline{mah}zõd}, \vadja{\underline{mah}sojõ}, \vadja{\underline{mah}soit}, \vadja{\underline{mah}soisõ}, \vadja{\underline{mah}soiss}
 \\
sõnatüüp ei hõlma teisi lekseeme


\vspace{1.8em}
\begin{minipage}{\textwidth}
\stepcounter{mallinumber}
\textbf{Tüüpsõnamall \arabic{mallinumber}\,\vadja{vihtõ}}\\

\begin{sideways}
\begin{tabular}{l l}
muutvormimall & tunnused \\
\hline
\underline{vih}\,+\,tõ & \textsc{ sg nom } \\
\underline{vih}\,+\,a & \textsc{ sg gen } \\
\underline{vih}\,+\,ta & \textsc{ sg par } \\
\underline{vih}\,+\,tasõ & \textsc{ sg ill } \\
\underline{vih}\,+\,õz & \textsc{ sg ine } \\
\underline{vih}\,+\,õss & \textsc{ sg ela } \\
\underline{vih}\,+\,õllõ & \textsc{ sg all } \\
\underline{vih}\,+\,õll & \textsc{ sg ade } \\
\underline{vih}\,+\,õlt & \textsc{ sg abl } \\
\underline{vih}\,+\,õssi & \textsc{ sg tra } \\
\underline{vih}\,+\,õssaa & \textsc{ sg ter } \\
\underline{vih}\,+\,õka & \textsc{ sg com } \\
\underline{vih}\,+\,õd & \textsc{ pl nom } \\
\underline{vih}\,+\,tojõ & \textsc{ pl gen } \\
\underline{vih}\,+\,toit & \textsc{ pl par } \\
\underline{vih}\,+\,toisõ & \textsc{ pl ill } \\
\underline{vih}\,+\,toiz & \textsc{ pl ine } \\
\underline{vih}\,+\,toiss & \textsc{ pl ela } \\
\underline{vih}\,+\,toillõ & \textsc{ pl all } \\
\underline{vih}\,+\,toill & \textsc{ pl ade } \\
\underline{vih}\,+\,toilt & \textsc{ pl abl } \\
\underline{vih}\,+\,toissi & \textsc{ pl tra } \\
\underline{vih}\,+\,toissaa & \textsc{ pl ter } \\
\underline{vih}\,+\,toika & \textsc{ pl com } \\
\end{tabular}
\end{sideways}
\captionof{table}{Tüüpsõna \arabic{mallinumber}\,\textit{vihtõ} ekstraheeritud muutvormimallid.}
\label{tab:tüüpsõnamall-vihtõ}

\end{minipage}

 
\vspace{1em}
\noindent Tüüpsõna ei hõlma teisi lekseeme vormi\-sõnastikus.

\paragraph*{\vadja{\underline{a}itõ}}
\vadja{\underline{a}jja}, \vadja{\underline{a}ita}, \vadja{\underline{a}itasõ}, \vadja{\underline{a}ijõss}, \vadja{\underline{a}ijõd}, \vadja{\underline{a}itojõ}, \vadja{\underline{a}itoit}, \vadja{\underline{a}itoisõ}, \vadja{\underline{a}itoiss}
 \\
sõnatüüp ei hõlma teisi lekseeme


\vspace{1.8em}
\begin{minipage}{\textwidth}
\stepcounter{mallinumber}
\textbf{Tüüpsõnamall \arabic{mallinumber}\,\vadja{aikõ}}\\

\begin{sideways}
\begin{tabular}{l l}
muutvormimall & tunnused \\
\hline
\underline{ai}\,+\,kõ & \textsc{ sg nom } \\
\underline{ai}\,+\,ga & \textsc{ sg gen } \\
\underline{ai}\,+\,ka & \textsc{ sg par } \\
\underline{ai}\,+\,kasõ & \textsc{ sg ill } \\
\underline{ai}\,+\,gõz & \textsc{ sg ine } \\
\underline{ai}\,+\,gõss & \textsc{ sg ela } \\
\underline{ai}\,+\,gõllõ & \textsc{ sg all } \\
\underline{ai}\,+\,gõll & \textsc{ sg ade } \\
\underline{ai}\,+\,gõlt & \textsc{ sg abl } \\
\underline{ai}\,+\,gõssi & \textsc{ sg tra } \\
\underline{ai}\,+\,gõssaa & \textsc{ sg ter } \\
\underline{ai}\,+\,gõka & \textsc{ sg com } \\
\underline{ai}\,+\,gõd & \textsc{ pl nom } \\
\underline{ai}\,+\,kojõ & \textsc{ pl gen } \\
\underline{ai}\,+\,koit & \textsc{ pl par } \\
\underline{ai}\,+\,koisõ & \textsc{ pl ill } \\
\underline{ai}\,+\,koiz & \textsc{ pl ine } \\
\underline{ai}\,+\,koiss & \textsc{ pl ela } \\
\underline{ai}\,+\,koillõ & \textsc{ pl all } \\
\underline{ai}\,+\,koill & \textsc{ pl ade } \\
\underline{ai}\,+\,koilt & \textsc{ pl abl } \\
\underline{ai}\,+\,koissi & \textsc{ pl tra } \\
\underline{ai}\,+\,koissaa & \textsc{ pl ter } \\
\underline{ai}\,+\,koika & \textsc{ pl com } \\
\end{tabular}
\end{sideways}
\captionof{table}{Tüüpsõna \arabic{mallinumber}\,\textit{aikõ} ekstraheeritud muutvormimallid.}
\label{tab:tüüpsõnamall-aikõ}

\end{minipage}

 
\vspace{1em}
\noindent Tüüpsõna hõlmab vormisõnastiku 10 lekseemi: \vadja{\underline{ai}kõ, \underline{jal}kõ, \underline{lii}kõ, \underline{lõn}kõ, \underline{nah}kõ, \underline{rah}kõ, \underline{vil}kõ, \underline{vin}kõ, \underline{võl}kõ} ja \vadja{\underline{aastai}kõ}.


\vspace{1.8em}
\begin{minipage}{\textwidth}
\stepcounter{mallinumber}
\textbf{Tüüpsõnamall \arabic{mallinumber}\,\vadja{sika}}\\

\begin{sideways}
\begin{tabular}{l l}
muutvormimall & tunnused \\
\hline
\underline{si}\,+\,ka & \textsc{ sg nom } \\
\underline{si}\,+\,ga & \textsc{ sg gen } \\
\underline{si}\,+\,kka & \textsc{ sg par } \\
\underline{si}\,+\,kkasõ & \textsc{ sg ill } \\
\underline{si}\,+\,gaz & \textsc{ sg ine } \\
\underline{si}\,+\,gass & \textsc{ sg ela } \\
\underline{si}\,+\,gallõ & \textsc{ sg all } \\
\underline{si}\,+\,gall & \textsc{ sg ade } \\
\underline{si}\,+\,galt & \textsc{ sg abl } \\
\underline{si}\,+\,gassi & \textsc{ sg tra } \\
\underline{si}\,+\,gassaa & \textsc{ sg ter } \\
\underline{si}\,+\,gaka & \textsc{ sg com } \\
\underline{si}\,+\,gad & \textsc{ pl nom } \\
\underline{si}\,+\,kojõ & \textsc{ pl gen } \\
\underline{si}\,+\,koit & \textsc{ pl par } \\
\underline{si}\,+\,koisõ & \textsc{ pl ill } \\
\underline{si}\,+\,koiz & \textsc{ pl ine } \\
\underline{si}\,+\,koiss & \textsc{ pl ela } \\
\underline{si}\,+\,koillõ & \textsc{ pl all } \\
\underline{si}\,+\,koill & \textsc{ pl ade } \\
\underline{si}\,+\,koilt & \textsc{ pl abl } \\
\underline{si}\,+\,koissi & \textsc{ pl tra } \\
\underline{si}\,+\,koissaa & \textsc{ pl ter } \\
\underline{si}\,+\,koika & \textsc{ pl com } \\
\end{tabular}
\end{sideways}
\captionof{table}{Tüüpsõna \arabic{mallinumber}\,\textit{sika} ekstraheeritud muutvormimallid.}
\label{tab:tüüpsõnamall-sika}

\end{minipage}

 
\vspace{1em}
\noindent Tüüpsõna ei hõlma teisi lekseeme vormi\-sõnastikus.


\vspace{1.8em}
\begin{minipage}{\textwidth}
\stepcounter{mallinumber}
\textbf{Tüüpsõnamall \arabic{mallinumber}\,\vadja{borovikkõ}}\\

\begin{sideways}
\begin{tabular}{l l}
muutvormimall & tunnused \\
\hline
\underline{borovik}\,+\,kõ & \textsc{ sg nom } \\
\underline{borovik}\,+\,a & \textsc{ sg gen } \\
\underline{borovik}\,+\,ka & \textsc{ sg par } \\
\underline{borovik}\,+\,kasõ & \textsc{ sg ill } \\
\underline{borovik}\,+\,kõz & \textsc{ sg ine } \\
\underline{borovik}\,+\,õss & \textsc{ sg ela } \\
\underline{borovik}\,+\,õllõ & \textsc{ sg all } \\
\underline{borovik}\,+\,õll & \textsc{ sg ade } \\
\underline{borovik}\,+\,õlt & \textsc{ sg abl } \\
\underline{borovik}\,+\,õssi & \textsc{ sg tra } \\
\underline{borovik}\,+\,kõssaa & \textsc{ sg ter } \\
\underline{borovik}\,+\,õka & \textsc{ sg com } \\
\underline{borovik}\,+\,õd & \textsc{ pl nom } \\
\underline{borovik}\,+\,kojõ & \textsc{ pl gen } \\
\underline{borovik}\,+\,koit & \textsc{ pl par } \\
\underline{borovik}\,+\,koisõ & \textsc{ pl ill } \\
\underline{borovik}\,+\,koiz & \textsc{ pl ine } \\
\underline{borovik}\,+\,koiss & \textsc{ pl ela } \\
\underline{borovik}\,+\,koillõ & \textsc{ pl all } \\
\underline{borovik}\,+\,koill & \textsc{ pl ade } \\
\underline{borovik}\,+\,koilt & \textsc{ pl abl } \\
\underline{borovik}\,+\,koissi & \textsc{ pl tra } \\
\underline{borovik}\,+\,koissaa & \textsc{ pl ter } \\
\underline{borovik}\,+\,koika & \textsc{ pl com } \\
\end{tabular}
\end{sideways}
\captionof{table}{Tüüpsõna \arabic{mallinumber}\,\textit{borovikkõ} ekstraheeritud muutvormimallid.}
\label{tab:tüüpsõnamall-borovikkõ}

\end{minipage}

 
\vspace{1em}
\noindent Tüüpsõna hõlmab vormisõnastiku 41 lekseemi: \vadja{\underline{borovik}kõ, \underline{domovik}kõ, \underline{durak}kõ, \underline{fartuk}kõ, \underline{fiizik}kõ, \underline{fookusnik}kõ, \underline{frištik}kõ, \underline{gribanik}kõ, \underline{harak}kõ, \underline{itik}kõ, \underline{joožik}kõ, \underline{kaamenšik}kõ, \underline{kabak}kõ, \underline{kamal̕ik}kõ, \underline{katol̕ik}kõ, \underline{kelk}kõ, \underline{koomik}kõ, \underline{kopek}kõ, \underline{latik}kõ, \underline{luzik}kõ, \underline{luuk}kõ, \underline{mark}kõ, \underline{muuzik}kõ, \underline{mõiznik}kõ, \underline{noorik}kõ, \underline{nuužnik}kõ, \underline{obak}kõ, \underline{paik}kõ, \underline{palk}kõ, \underline{pinžak}kõ, \underline{podark}kõ, \underline{poštimark}kõ, \underline{rank}kõ, \underline{rohosirk}kõ, \underline{tark}kõ, \underline{tik}kõ, \underline{tubak}kõ, \underline{urok}kõ, \underline{vak}kõ, \underline{vunuk}kõ} ja \vadja{\underline{bašmuk}kõ}.

\paragraph*{\vadja{\underline{sil}tõ}}
\vadja{\underline{sil}la}, \vadja{\underline{sil}ta}, \vadja{\underline{sil}tasõ}, \vadja{\underline{sil}lõss}, \vadja{\underline{sil}lõd}, \vadja{\underline{sil}tojõ}, \vadja{\underline{sil}toit}, \vadja{\underline{sil}toisõ}, \vadja{\underline{sil}toiss}
 \\
sõnatüüp ei hõlma teisi lekseeme


\vspace{1.8em}
\begin{minipage}{\textwidth}
\stepcounter{mallinumber}
\textbf{Tüüpsõnamall \arabic{mallinumber}\,\vadja{paha}}\\

\begin{sideways}
\begin{tabular}{l l}
muutvormimall & tunnused \\
\hline
\underline{pah}\,+\,a & \textsc{ sg nom } \\
\underline{pah}\,+\,a & \textsc{ sg gen } \\
\underline{pah}\,+\,ha & \textsc{ sg par } \\
\underline{pah}\,+\,hasõ & \textsc{ sg ill } \\
\underline{pah}\,+\,õz & \textsc{ sg ine } \\
\underline{pah}\,+\,õss & \textsc{ sg ela } \\
\underline{pah}\,+\,õllõ & \textsc{ sg all } \\
\underline{pah}\,+\,õll & \textsc{ sg ade } \\
\underline{pah}\,+\,õlt & \textsc{ sg abl } \\
\underline{pah}\,+\,õssi & \textsc{ sg tra } \\
\underline{pah}\,+\,õssaa & \textsc{ sg ter } \\
\underline{pah}\,+\,õka & \textsc{ sg com } \\
\underline{pah}\,+\,õd & \textsc{ pl nom } \\
\underline{pah}\,+\,hojõ & \textsc{ pl gen } \\
\underline{pah}\,+\,hoit & \textsc{ pl par } \\
\underline{pah}\,+\,hoisõ & \textsc{ pl ill } \\
\underline{pah}\,+\,hoiz & \textsc{ pl ine } \\
\underline{pah}\,+\,hoiss & \textsc{ pl ela } \\
\underline{pah}\,+\,hoillõ & \textsc{ pl all } \\
\underline{pah}\,+\,hoill & \textsc{ pl ade } \\
\underline{pah}\,+\,hoilt & \textsc{ pl abl } \\
\underline{pah}\,+\,hoissi & \textsc{ pl tra } \\
\underline{pah}\,+\,hoissaa & \textsc{ pl ter } \\
\underline{pah}\,+\,hoika & \textsc{ pl com } \\
\end{tabular}
\end{sideways}
\captionof{table}{Tüüpsõna \arabic{mallinumber}\,\textit{paha} ekstraheeritud muutvormimallid.}
\label{tab:tüüpsõnamall-paha}

\end{minipage}

 
\vspace{1em}
\noindent Tüüpsõna ei hõlma teisi lekseeme vormi\-sõnastikus.

\paragraph*{\vadja{\underline{az̕z̕}õ}}
\vadja{\underline{az̕z̕}a}, \vadja{\underline{az̕z̕}a}, \vadja{\underline{az̕z̕}asõ}, \vadja{\underline{az̕z̕}õss}, \vadja{\underline{az̕z̕}õd}, \vadja{\underline{az̕z̕}ojõ}, \vadja{\underline{az̕z̕}oit}, \vadja{\underline{az̕z̕}oisõ}, \vadja{\underline{az̕z̕}oiss}
 \\
sõnatüüp hõlmab lekseeme \vadja{az̕z̕õ, bad̕d̕õ, bahvõlõ, bl̕aahõ, bobrõ, borkkanõ, bruudõ, čirjavõ, čirjõ, d̕eelõ, dobrõ, filmõ, glaizõ, grammõ, gribavihmõ, iivõ, jumalõ, jurmõ, kabjõ, kaglõ, kagrõ, kajagõ, kambõlõ, kanavõ, karjõ, kassõ, katagõ, kavalõ, kvartirõ, lad̕d̕õ, ladvõ, lahjõ, lahnõ, lainõ, laivõ, liivõ, linnõ, l̕istõ, maailmõ, maamõ, mahlõ, mannõ, marjõ, matalõ, metlõ, muragõ, mussõmarjõ, naglõ, n̕egrõ, niglõ, ostanofkõ, paglõ, progonõ, pudrõ, pupuškõ, rauhõ, saappõgõ, sarjõ, saunõ, siglõ, sisavõ, sl̕ifkõ, summõ, surmõ, suukkurliivõ, sõbrõ, šuubõ, ženihõ, taičinõ, trubõ, vihmõ, vikahtõ, villõ, õravõ, õzrõ, akkunõ}


\vspace{1.8em}
\begin{minipage}{\textwidth}
\stepcounter{mallinumber}
\textbf{Tüüpsõnamall \arabic{mallinumber}\,\vadja{bank}}\\

\begin{sideways}
\begin{tabular}{l l}
muutvormimall & tunnused \\
\hline
\underline{bank} & \textsc{ sg nom } \\
\underline{bank}\,+\,a & \textsc{ sg gen } \\
\underline{bank}\,+\,a & \textsc{ sg par } \\
\underline{bank}\,+\,asõ & \textsc{ sg ill } \\
\underline{bank}\,+\,õz & \textsc{ sg ine } \\
\underline{bank}\,+\,õss & \textsc{ sg ela } \\
\underline{bank}\,+\,õllõ & \textsc{ sg all } \\
\underline{bank}\,+\,õll & \textsc{ sg ade } \\
\underline{bank}\,+\,õlt & \textsc{ sg abl } \\
\underline{bank}\,+\,õssi & \textsc{ sg tra } \\
\underline{bank}\,+\,õssaa & \textsc{ sg ter } \\
\underline{bank}\,+\,õka & \textsc{ sg com } \\
\underline{bank}\,+\,õd & \textsc{ pl nom } \\
\underline{bank}\,+\,ojõ & \textsc{ pl gen } \\
\underline{bank}\,+\,oit & \textsc{ pl par } \\
\underline{bank}\,+\,oisõ & \textsc{ pl ill } \\
\underline{bank}\,+\,oiz & \textsc{ pl ine } \\
\underline{bank}\,+\,oiss & \textsc{ pl ela } \\
\underline{bank}\,+\,oillõ & \textsc{ pl all } \\
\underline{bank}\,+\,oill & \textsc{ pl ade } \\
\underline{bank}\,+\,oilt & \textsc{ pl abl } \\
\underline{bank}\,+\,oissi & \textsc{ pl tra } \\
\underline{bank}\,+\,oissaa & \textsc{ pl ter } \\
\underline{bank}\,+\,oika & \textsc{ pl com } \\
\end{tabular}
\end{sideways}
\captionof{table}{Tüüpsõna \arabic{mallinumber}\,\textit{bank} ekstraheeritud muutvormimallid.}
\label{tab:tüüpsõnamall-bank}

\end{minipage}

 
\vspace{1em}
\noindent Tüüpsõna hõlmab vormisõnastiku 34 lekseemi: \vadja{\underline{bank}, \underline{bl̕uud}, \underline{bl̕uudõčk}, \underline{boran}, \underline{fartõl}, \underline{fialk}, \underline{figur}, \underline{fortočk}, \underline{frikad̕el̕k}, \underline{golod}, \underline{greettsin}, \underline{gupk}, \underline{invaliid}, \underline{kaban}, \underline{kamal}, \underline{kamin}, \underline{kanal}, \underline{kipun}, \underline{kluub}, \underline{kohin}, \underline{l̕ihoratk}, \underline{mašin}, \underline{mašinist}, \underline{muudõr}, \underline{omõn}, \underline{pagan}, \underline{pen̕sioner}, \underline{sammõl}, \underline{zanavesk}, \underline{žurnalist}, \underline{tarelk}, \underline{vaahtõr}, \underline{viks}} ja \vadja{\underline{ahvõn}}.

\paragraph*{\vadja{\underline{baldõhin}a}}
\vadja{\underline{baldõhin}a}, \vadja{\underline{baldõhin}a}, \vadja{\underline{baldõhin}asõ}, \vadja{\underline{baldõhin}ass}, \vadja{\underline{baldõhin}ad}, \vadja{\underline{baldõhin}ojõ}, \vadja{\underline{baldõhin}oit}, \vadja{\underline{baldõhin}oisõ}, \vadja{\underline{baldõhin}oiss}
 \\
Tüüpsõna hõlmab vormisõnastiku lekseeme \vadja{baldõhina, barabana, fotokartočka, grana, griba, kala, kana, liha, lina, litra, maja, raha, suma, sõna, tara, telefona, televizora, tila, vana} ja \vadja{astia}.

\paragraph*{\vadja{\underline{pin}tõ}}
\vadja{\underline{pin}na}, \vadja{\underline{pin}ta}, \vadja{\underline{pin}tasõ}, \vadja{\underline{pin}nõss}, \vadja{\underline{pin}nõd}, \vadja{\underline{pin}tojõ}, \vadja{\underline{pin}toit}, \vadja{\underline{pin}toisõ}, \vadja{\underline{pin}toiss}
 \\
Tüüpsõna hõlmab vormisõnastiku lekseeme \vadja{pintõ, rantõ, rintõ, kantõ}.

\paragraph*{\vadja{\underline{ra}pa}}
\vadja{\underline{ra}va}, \vadja{\underline{ra}ppa}, \vadja{\underline{ra}ppasõ}, \vadja{\underline{ra}vass}, \vadja{\underline{ra}vad}, \vadja{\underline{ra}pojõ}, \vadja{\underline{ra}poit}, \vadja{\underline{ra}poisõ}, \vadja{\underline{ra}poiss}
 \\
Tüüpsõna hõlmab vormisõnastiku 3 lekseemi: \vadja{rapa, sõpa} ja \vadja{napa}.

\\
\vspace{1.8em}
\begin{minipage}{\textwidth}
\stepcounter{mallinumber}
\textbf{Tüüpsõnamall \arabic{mallinumber}\,\vadja{aapõ}}\\

\begin{sideways}
\begin{tabular}{l l}
muutvormimall & tunnused \\
\hline
\underline{aa}\,+\,põ & \textsc{ sg nom } \\
\underline{aa}\,+\,va & \textsc{ sg gen } \\
\underline{aa}\,+\,pa & \textsc{ sg par } \\
\underline{aa}\,+\,pasõ & \textsc{ sg ill } \\
\underline{aa}\,+\,võz & \textsc{ sg ine } \\
\underline{aa}\,+\,võss & \textsc{ sg ela } \\
\underline{aa}\,+\,võllõ & \textsc{ sg all } \\
\underline{aa}\,+\,võll & \textsc{ sg ade } \\
\underline{aa}\,+\,võlt & \textsc{ sg abl } \\
\underline{aa}\,+\,võssi & \textsc{ sg tra } \\
\underline{aa}\,+\,võssaa & \textsc{ sg ter } \\
\underline{aa}\,+\,võka & \textsc{ sg com } \\
\underline{aa}\,+\,võd & \textsc{ pl nom } \\
\underline{aa}\,+\,pojõ & \textsc{ pl gen } \\
\underline{aa}\,+\,poit & \textsc{ pl par } \\
\underline{aa}\,+\,poisõ & \textsc{ pl ill } \\
\underline{aa}\,+\,poiz & \textsc{ pl ine } \\
\underline{aa}\,+\,poiss & \textsc{ pl ela } \\
\underline{aa}\,+\,poillõ & \textsc{ pl all } \\
\underline{aa}\,+\,poill & \textsc{ pl ade } \\
\underline{aa}\,+\,poilt & \textsc{ pl abl } \\
\underline{aa}\,+\,poissi & \textsc{ pl tra } \\
\underline{aa}\,+\,poissaa & \textsc{ pl ter } \\
\underline{aa}\,+\,poika & \textsc{ pl com } \\
\end{tabular}
\end{sideways}
\captionof{table}{Tüüpsõna \arabic{mallinumber}\,\textit{aapõ} ekstraheeritud muutvormimallid.}
\label{tab:tüüpsõnamall-aapõ}

\end{minipage}

 
\vspace{1em}
\noindent Tüüpsõna ei hõlma teisi lekseeme vormi\-sõnastikus.

\paragraph*{\vadja{\underline{liip}põ}}
\vadja{\underline{liip}a}, \vadja{\underline{liip}pa}, \vadja{\underline{liip}pasõ}, \vadja{\underline{liip}õss}, \vadja{\underline{liip}õd}, \vadja{\underline{liip}pojõ}, \vadja{\underline{liip}poit}, \vadja{\underline{liip}poisõ}, \vadja{\underline{liip}poiss}
 \\
Tüüpsõna hõlmab vormisõnastiku 2 lekseemi: \vadja{liippõ} ja \vadja{kauppõ}.

\paragraph*{\vadja{\underline{jär}č\underline{ü}}}
\vadja{\underline{jär}j\underline{ü}}, \vadja{\underline{jär}č\underline{ü}ä}, \vadja{\underline{jär}č\underline{ü}se}, \vadja{\underline{jär}j\underline{ü}ss}, \vadja{\underline{jär}j\underline{ü}d}, \vadja{\underline{jär}č\underline{ü}je}, \vadja{\underline{jär}č\underline{ü}it}, \vadja{\underline{jär}č\underline{ü}ise}, \vadja{\underline{jär}č\underline{ü}iss}
 \\
sõnatüüp ei hõlma teisi lekseeme


\vspace{1.8em}
\begin{minipage}{\textwidth}
\stepcounter{mallinumber}
\textbf{Tüüpsõnamall \arabic{mallinumber}\,\vadja{partõ}}\\

\begin{sideways}
\begin{tabular}{l l}
muutvormimall & tunnused \\
\hline
\underline{par}\,+\,tõ & \textsc{ sg nom } \\
\underline{par}\,+\,ra & \textsc{ sg gen } \\
\underline{par}\,+\,ta & \textsc{ sg par } \\
\underline{par}\,+\,tasõ & \textsc{ sg ill } \\
\underline{par}\,+\,rõz & \textsc{ sg ine } \\
\underline{par}\,+\,rõss & \textsc{ sg ela } \\
\underline{par}\,+\,rõllõ & \textsc{ sg all } \\
\underline{par}\,+\,rõll & \textsc{ sg ade } \\
\underline{par}\,+\,rõlt & \textsc{ sg abl } \\
\underline{par}\,+\,rõssi & \textsc{ sg tra } \\
\underline{par}\,+\,rõssaa & \textsc{ sg ter } \\
\underline{par}\,+\,rõka & \textsc{ sg com } \\
\underline{par}\,+\,rõd & \textsc{ pl nom } \\
\underline{par}\,+\,tojõ & \textsc{ pl gen } \\
\underline{par}\,+\,toit & \textsc{ pl par } \\
\underline{par}\,+\,toisõ & \textsc{ pl ill } \\
\underline{par}\,+\,toiz & \textsc{ pl ine } \\
\underline{par}\,+\,toiss & \textsc{ pl ela } \\
\underline{par}\,+\,toillõ & \textsc{ pl all } \\
\underline{par}\,+\,toill & \textsc{ pl ade } \\
\underline{par}\,+\,toilt & \textsc{ pl abl } \\
\underline{par}\,+\,toissi & \textsc{ pl tra } \\
\underline{par}\,+\,toissaa & \textsc{ pl ter } \\
\underline{par}\,+\,toika & \textsc{ pl com } \\
\end{tabular}
\end{sideways}
\captionof{table}{Tüüpsõna \arabic{mallinumber}\,\textit{partõ} ekstraheeritud muutvormimallid.}
\label{tab:tüüpsõnamall-partõ}

\end{minipage}

 
\vspace{1em}
\noindent Tüüpsõna hõlmab vormisõnastiku 2 lekseemi: \vadja{\underline{par}tõ} ja \vadja{\underline{kõr}tõ}.


\vspace{1.8em}
\begin{minipage}{\textwidth}
\stepcounter{mallinumber}
\textbf{Tüüpsõnamall \arabic{mallinumber}\,\vadja{kraaskõ}}\\

\begin{sideways}
\begin{tabular}{l l}
muutvormimall & tunnused \\
\hline
\underline{kraa}\,+\,skõ & \textsc{ sg nom } \\
\underline{kraa}\,+\,zga & \textsc{ sg gen } \\
\underline{kraa}\,+\,ska & \textsc{ sg par } \\
\underline{kraa}\,+\,skasõ & \textsc{ sg ill } \\
\underline{kraa}\,+\,zgõz & \textsc{ sg ine } \\
\underline{kraa}\,+\,zgõss & \textsc{ sg ela } \\
\underline{kraa}\,+\,zgõllõ & \textsc{ sg all } \\
\underline{kraa}\,+\,zgõll & \textsc{ sg ade } \\
\underline{kraa}\,+\,zgõlt & \textsc{ sg abl } \\
\underline{kraa}\,+\,zgõssi & \textsc{ sg tra } \\
\underline{kraa}\,+\,zgõssaa & \textsc{ sg ter } \\
\underline{kraa}\,+\,zgõka & \textsc{ sg com } \\
\underline{kraa}\,+\,zgõd & \textsc{ pl nom } \\
\underline{kraa}\,+\,skojõ & \textsc{ pl gen } \\
\underline{kraa}\,+\,skoit & \textsc{ pl par } \\
\underline{kraa}\,+\,skoisõ & \textsc{ pl ill } \\
\underline{kraa}\,+\,skoiz & \textsc{ pl ine } \\
\underline{kraa}\,+\,skoiss & \textsc{ pl ela } \\
\underline{kraa}\,+\,skoillõ & \textsc{ pl all } \\
\underline{kraa}\,+\,skoill & \textsc{ pl ade } \\
\underline{kraa}\,+\,skoilt & \textsc{ pl abl } \\
\underline{kraa}\,+\,skoissi & \textsc{ pl tra } \\
\underline{kraa}\,+\,skoissaa & \textsc{ pl ter } \\
\underline{kraa}\,+\,skoika & \textsc{ pl com } \\
\end{tabular}
\end{sideways}
\captionof{table}{Tüüpsõna \arabic{mallinumber}\,\textit{kraaskõ} ekstraheeritud muutvormimallid.}
\label{tab:tüüpsõnamall-kraaskõ}

\end{minipage}

 
\vspace{1em}
\noindent Tüüpsõna hõlmab vormisõnastiku 6 lekseemi: \vadja{\underline{kraa}skõ, \underline{lai}skõ, \underline{nagriskaa}skõ, \underline{ni}skõ, \underline{pa}skõ} ja \vadja{\underline{kaa}skõ}.


\vspace{1.8em}
\begin{minipage}{\textwidth}
\stepcounter{mallinumber}
\textbf{Tüüpsõnamall \arabic{mallinumber}\,\vadja{klaassõ}}\\

\begin{sideways}
\begin{tabular}{l l}
muutvormimall & tunnused \\
\hline
\underline{klaas}\,+\,sõ & \textsc{ sg nom } \\
\underline{klaas}\,+\,a & \textsc{ sg gen } \\
\underline{klaas}\,+\,sa & \textsc{ sg par } \\
\underline{klaas}\,+\,sasõ & \textsc{ sg ill } \\
\underline{klaas}\,+\,sõz & \textsc{ sg ine } \\
\underline{klaas}\,+\,õss & \textsc{ sg ela } \\
\underline{klaas}\,+\,õllõ & \textsc{ sg all } \\
\underline{klaas}\,+\,õll & \textsc{ sg ade } \\
\underline{klaas}\,+\,õlt & \textsc{ sg abl } \\
\underline{klaas}\,+\,õssi & \textsc{ sg tra } \\
\underline{klaas}\,+\,sõssaa & \textsc{ sg ter } \\
\underline{klaas}\,+\,õka & \textsc{ sg com } \\
\underline{klaas}\,+\,õd & \textsc{ pl nom } \\
\underline{klaas}\,+\,sojõ & \textsc{ pl gen } \\
\underline{klaas}\,+\,soit & \textsc{ pl par } \\
\underline{klaas}\,+\,soisõ & \textsc{ pl ill } \\
\underline{klaas}\,+\,soiz & \textsc{ pl ine } \\
\underline{klaas}\,+\,soiss & \textsc{ pl ela } \\
\underline{klaas}\,+\,soillõ & \textsc{ pl all } \\
\underline{klaas}\,+\,soill & \textsc{ pl ade } \\
\underline{klaas}\,+\,soilt & \textsc{ pl abl } \\
\underline{klaas}\,+\,soissi & \textsc{ pl tra } \\
\underline{klaas}\,+\,soissaa & \textsc{ pl ter } \\
\underline{klaas}\,+\,soika & \textsc{ pl com } \\
\end{tabular}
\end{sideways}
\captionof{table}{Tüüpsõna \arabic{mallinumber}\,\textit{klaassõ} ekstraheeritud muutvormimallid.}
\label{tab:tüüpsõnamall-klaassõ}

\end{minipage}

 
\vspace{1em}
\noindent Tüüpsõna hõlmab vormisõnastiku 2 lekseemi: \vadja{\underline{klaas}sõ} ja \vadja{\underline{bruus}sõ}.

\paragraph*{\vadja{\underline{podu}škõ}}
\vadja{\underline{podu}žga}, \vadja{\underline{podu}ška}, \vadja{\underline{podu}škasõ}, \vadja{\underline{podu}žgõss}, \vadja{\underline{podu}žgõd}, \vadja{\underline{podu}škojõ}, \vadja{\underline{podu}škoit}, \vadja{\underline{podu}škoisõ}, \vadja{\underline{podu}škoiss}
 \\
Tüüpsõna ei hõlma teisi lekseeme vormi\-sõnastikus.


\vspace{1.8em}
\begin{minipage}{\textwidth}
\stepcounter{mallinumber}
\textbf{Tüüpsõnamall \arabic{mallinumber}\,\vadja{dovariššõ}}\\

\begin{sideways}
\begin{tabular}{l l}
muutvormimall & tunnused \\
\hline
\underline{dovariš}\,+\,šõ & \textsc{ sg nom } \\
\underline{dovariš}\,+\,a & \textsc{ sg gen } \\
\underline{dovariš}\,+\,ša & \textsc{ sg par } \\
\underline{dovariš}\,+\,šasõ & \textsc{ sg ill } \\
\underline{dovariš}\,+\,az & \textsc{ sg ine } \\
\underline{dovariš}\,+\,ass & \textsc{ sg ela } \\
\underline{dovariš}\,+\,allõ & \textsc{ sg all } \\
\underline{dovariš}\,+\,all & \textsc{ sg ade } \\
\underline{dovariš}\,+\,alt & \textsc{ sg abl } \\
\underline{dovariš}\,+\,assi & \textsc{ sg tra } \\
\underline{dovariš}\,+\,assaa & \textsc{ sg ter } \\
\underline{dovariš}\,+\,aka & \textsc{ sg com } \\
\underline{dovariš}\,+\,ad & \textsc{ pl nom } \\
\underline{dovariš}\,+\,šojõ & \textsc{ pl gen } \\
\underline{dovariš}\,+\,šoit & \textsc{ pl par } \\
\underline{dovariš}\,+\,šoisõ & \textsc{ pl ill } \\
\underline{dovariš}\,+\,šoiz & \textsc{ pl ine } \\
\underline{dovariš}\,+\,šoiss & \textsc{ pl ela } \\
\underline{dovariš}\,+\,šoillõ & \textsc{ pl all } \\
\underline{dovariš}\,+\,šoill & \textsc{ pl ade } \\
\underline{dovariš}\,+\,šoilt & \textsc{ pl abl } \\
\underline{dovariš}\,+\,šoissi & \textsc{ pl tra } \\
\underline{dovariš}\,+\,šoissaa & \textsc{ pl ter } \\
\underline{dovariš}\,+\,šoika & \textsc{ pl com } \\
\end{tabular}
\end{sideways}
\captionof{table}{Tüüpsõna \arabic{mallinumber}\,\textit{dovariššõ} ekstraheeritud muutvormimallid.}
\label{tab:tüüpsõnamall-dovariššõ}

\end{minipage}

 
\vspace{1em}
\noindent Tüüpsõna ei hõlma teisi lekseeme vormi\-sõnastikus.


\vspace{1.8em}
\begin{minipage}{\textwidth}
\stepcounter{mallinumber}
\textbf{Tüüpsõnamall \arabic{mallinumber}\,\vadja{sata}}\\

\begin{sideways}
\begin{tabular}{l l}
muutvormimall & tunnused \\
\hline
\underline{sa}\,+\,ta & \textsc{ sg nom } \\
\underline{sa}\,+\,a & \textsc{ sg gen } \\
\underline{sa}\,+\,tta & \textsc{ sg par } \\
\underline{sa}\,+\,ttasõ & \textsc{ sg ill } \\
\underline{sa}\,+\,az & \textsc{ sg ine } \\
\underline{sa}\,+\,ass & \textsc{ sg ela } \\
\underline{sa}\,+\,allõ & \textsc{ sg all } \\
\underline{sa}\,+\,all & \textsc{ sg ade } \\
\underline{sa}\,+\,alt & \textsc{ sg abl } \\
\underline{sa}\,+\,assi & \textsc{ sg tra } \\
\underline{sa}\,+\,assaa & \textsc{ sg ter } \\
\underline{sa}\,+\,aka & \textsc{ sg com } \\
\underline{sa}\,+\,ad & \textsc{ pl nom } \\
\underline{sa}\,+\,tojõ & \textsc{ pl gen } \\
\underline{sa}\,+\,toit & \textsc{ pl par } \\
\underline{sa}\,+\,toisõ & \textsc{ pl ill } \\
\underline{sa}\,+\,toiz & \textsc{ pl ine } \\
\underline{sa}\,+\,toiss & \textsc{ pl ela } \\
\underline{sa}\,+\,toillõ & \textsc{ pl all } \\
\underline{sa}\,+\,toill & \textsc{ pl ade } \\
\underline{sa}\,+\,toilt & \textsc{ pl abl } \\
\underline{sa}\,+\,toissi & \textsc{ pl tra } \\
\underline{sa}\,+\,toissaa & \textsc{ pl ter } \\
\underline{sa}\,+\,toika & \textsc{ pl com } \\
\end{tabular}
\end{sideways}
\captionof{table}{Tüüpsõna \arabic{mallinumber}\,\textit{sata} ekstraheeritud muutvormimallid.}
\label{tab:tüüpsõnamall-sata}

\end{minipage}

 
\vspace{1em}
\noindent Tüüpsõna hõlmab vormisõnastiku 3 lekseemi: \vadja{\underline{sa}ta, \underline{sõ}ta} ja \vadja{\underline{mu}ta}.


\vspace{1.8em}
\begin{minipage}{\textwidth}
\stepcounter{mallinumber}
\textbf{Tüüpsõnamall \arabic{mallinumber}\,\vadja{l̕iitkõ}}\\

\begin{sideways}
\begin{tabular}{l l}
muutvormimall & tunnused \\
\hline
\underline{l̕ii}\,+\,tkõ & \textsc{ sg nom } \\
\underline{l̕ii}\,+\,dga & \textsc{ sg gen } \\
\underline{l̕ii}\,+\,tka & \textsc{ sg par } \\
\underline{l̕ii}\,+\,tkasõ & \textsc{ sg ill } \\
\underline{l̕ii}\,+\,dgõz & \textsc{ sg ine } \\
\underline{l̕ii}\,+\,dgõss & \textsc{ sg ela } \\
\underline{l̕ii}\,+\,dgõllõ & \textsc{ sg all } \\
\underline{l̕ii}\,+\,dgõll & \textsc{ sg ade } \\
\underline{l̕ii}\,+\,dgõlt & \textsc{ sg abl } \\
\underline{l̕ii}\,+\,dgõssi & \textsc{ sg tra } \\
\underline{l̕ii}\,+\,dgõssaa & \textsc{ sg ter } \\
\underline{l̕ii}\,+\,dgõka & \textsc{ sg com } \\
\underline{l̕ii}\,+\,dgõd & \textsc{ pl nom } \\
\underline{l̕ii}\,+\,tkojõ & \textsc{ pl gen } \\
\underline{l̕ii}\,+\,tkoit & \textsc{ pl par } \\
\underline{l̕ii}\,+\,tkoisõ & \textsc{ pl ill } \\
\underline{l̕ii}\,+\,tkoiz & \textsc{ pl ine } \\
\underline{l̕ii}\,+\,tkoiss & \textsc{ pl ela } \\
\underline{l̕ii}\,+\,tkoillõ & \textsc{ pl all } \\
\underline{l̕ii}\,+\,tkoill & \textsc{ pl ade } \\
\underline{l̕ii}\,+\,tkoilt & \textsc{ pl abl } \\
\underline{l̕ii}\,+\,tkoissi & \textsc{ pl tra } \\
\underline{l̕ii}\,+\,tkoissaa & \textsc{ pl ter } \\
\underline{l̕ii}\,+\,tkoika & \textsc{ pl com } \\
\end{tabular}
\end{sideways}
\captionof{table}{Tüüpsõna \arabic{mallinumber}\,\textit{l̕iitkõ} ekstraheeritud muutvormimallid.}
\label{tab:tüüpsõnamall-l̕iitkõ}

\end{minipage}

 
\vspace{1em}
\noindent Tüüpsõna hõlmab vormisõnastiku 2 lekseemi: \vadja{\underline{l̕ii}tkõ} ja \vadja{\underline{bu}tkõ}.


\vspace{1.8em}
\begin{minipage}{\textwidth}
\stepcounter{mallinumber}
\textbf{Tüüpsõnamall \arabic{mallinumber}\,\vadja{kand̕idaattõ}}\\

\begin{sideways}
\begin{tabular}{l l}
muutvormimall & tunnused \\
\hline
\underline{kand̕idaat}\,+\,tõ & \textsc{ sg nom } \\
\underline{kand̕idaat}\,+\,a & \textsc{ sg gen } \\
\underline{kand̕idaat}\,+\,ta & \textsc{ sg par } \\
\underline{kand̕idaat}\,+\,tasõ & \textsc{ sg ill } \\
\underline{kand̕idaat}\,+\,tõz & \textsc{ sg ine } \\
\underline{kand̕idaat}\,+\,õss & \textsc{ sg ela } \\
\underline{kand̕idaat}\,+\,õllõ & \textsc{ sg all } \\
\underline{kand̕idaat}\,+\,õll & \textsc{ sg ade } \\
\underline{kand̕idaat}\,+\,õlt & \textsc{ sg abl } \\
\underline{kand̕idaat}\,+\,õssi & \textsc{ sg tra } \\
\underline{kand̕idaat}\,+\,tõssaa & \textsc{ sg ter } \\
\underline{kand̕idaat}\,+\,õka & \textsc{ sg com } \\
\underline{kand̕idaat}\,+\,õd & \textsc{ pl nom } \\
\underline{kand̕idaat}\,+\,tojõ & \textsc{ pl gen } \\
\underline{kand̕idaat}\,+\,toit & \textsc{ pl par } \\
\underline{kand̕idaat}\,+\,toisõ & \textsc{ pl ill } \\
\underline{kand̕idaat}\,+\,toiz & \textsc{ pl ine } \\
\underline{kand̕idaat}\,+\,toiss & \textsc{ pl ela } \\
\underline{kand̕idaat}\,+\,toillõ & \textsc{ pl all } \\
\underline{kand̕idaat}\,+\,toill & \textsc{ pl ade } \\
\underline{kand̕idaat}\,+\,toilt & \textsc{ pl abl } \\
\underline{kand̕idaat}\,+\,toissi & \textsc{ pl tra } \\
\underline{kand̕idaat}\,+\,toissaa & \textsc{ pl ter } \\
\underline{kand̕idaat}\,+\,toika & \textsc{ pl com } \\
\end{tabular}
\end{sideways}
\captionof{table}{Tüüpsõna \arabic{mallinumber}\,\textit{kand̕idaattõ} ekstraheeritud muutvormimallid.}
\label{tab:tüüpsõnamall-kand̕idaattõ}

\end{minipage}

 
\vspace{1em}
\noindent Tüüpsõna hõlmab vormisõnastiku 7 lekseemi: \vadja{\underline{kand̕idaat}tõ, \underline{laut}tõ, \underline{pliit}tõ, \underline{riit}tõ, \underline{žiivõt}tõ, \underline{taat}tõ} ja \vadja{\underline{gaazapliit}tõ}.


\vspace{1.8em}
\begin{minipage}{\textwidth}
\stepcounter{mallinumber}
\textbf{Tüüpsõnamall \arabic{mallinumber}\,\vadja{inostranttsõ}}\\

\begin{sideways}
\begin{tabular}{l l}
muutvormimall & tunnused \\
\hline
\underline{inostrant}\,+\,t\,+\,\underline{s}\,+\,õ & \textsc{ sg nom } \\
\underline{inostrant}\,+\,\underline{s}\,+\,a & \textsc{ sg gen } \\
\underline{inostrant}\,+\,t\,+\,\underline{s}\,+\,a & \textsc{ sg par } \\
\underline{inostrant}\,+\,t\,+\,\underline{s}\,+\,asõ & \textsc{ sg ill } \\
\underline{inostrant}\,+\,t\,+\,\underline{s}\,+\,õz & \textsc{ sg ine } \\
\underline{inostrant}\,+\,\underline{s}\,+\,õss & \textsc{ sg ela } \\
\underline{inostrant}\,+\,\underline{s}\,+\,õllõ & \textsc{ sg all } \\
\underline{inostrant}\,+\,\underline{s}\,+\,õll & \textsc{ sg ade } \\
\underline{inostrant}\,+\,\underline{s}\,+\,õlt & \textsc{ sg abl } \\
\underline{inostrant}\,+\,\underline{s}\,+\,õssi & \textsc{ sg tra } \\
\underline{inostrant}\,+\,t\,+\,\underline{s}\,+\,õssaa & \textsc{ sg ter } \\
\underline{inostrant}\,+\,\underline{s}\,+\,õka & \textsc{ sg com } \\
\underline{inostrant}\,+\,\underline{s}\,+\,õd & \textsc{ pl nom } \\
\underline{inostrant}\,+\,t\,+\,\underline{s}\,+\,ojõ & \textsc{ pl gen } \\
\underline{inostrant}\,+\,t\,+\,\underline{s}\,+\,oit & \textsc{ pl par } \\
\underline{inostrant}\,+\,t\,+\,\underline{s}\,+\,oisõ & \textsc{ pl ill } \\
\underline{inostrant}\,+\,t\,+\,\underline{s}\,+\,oiz & \textsc{ pl ine } \\
\underline{inostrant}\,+\,t\,+\,\underline{s}\,+\,oiss & \textsc{ pl ela } \\
\underline{inostrant}\,+\,t\,+\,\underline{s}\,+\,oillõ & \textsc{ pl all } \\
\underline{inostrant}\,+\,t\,+\,\underline{s}\,+\,oill & \textsc{ pl ade } \\
\underline{inostrant}\,+\,t\,+\,\underline{s}\,+\,oilt & \textsc{ pl abl } \\
\underline{inostrant}\,+\,t\,+\,\underline{s}\,+\,oissi & \textsc{ pl tra } \\
\underline{inostrant}\,+\,t\,+\,\underline{s}\,+\,oissaa & \textsc{ pl ter } \\
\underline{inostrant}\,+\,t\,+\,\underline{s}\,+\,oika & \textsc{ pl com } \\
\end{tabular}
\end{sideways}
\captionof{table}{Tüüpsõna \arabic{mallinumber}\,\textit{inostranttsõ} ekstraheeritud muutvormimallid.}
\label{tab:tüüpsõnamall-inostranttsõ}

\end{minipage}

 
\vspace{1em}
\noindent Tüüpsõna hõlmab vormisõnastiku 6 lekseemi: \vadja{\underline{inostrant}t\underline{s}õ, \underline{liit}t\underline{s}õ, \underline{tablit}t\underline{s}õ, \underline{vat}t\underline{s}õ, \underline{õt}t\underline{s}õ} ja \vadja{\underline{bol̕nit}t\underline{s}õ}.

\paragraph*{\vadja{\underline{jõ}\underline{u}tu}}
\vadja{\underline{jõ}vv\underline{u}}, \vadja{\underline{jõ}\underline{u}tua}, \vadja{\underline{jõ}\underline{u}tusõ}, \vadja{\underline{jõ}vv\underline{u}ss}, \vadja{\underline{jõ}vv\underline{u}d}, \vadja{\underline{jõ}\underline{u}tujõ}, \vadja{\underline{jõ}\underline{u}tuit}, \vadja{\underline{jõ}\underline{u}tuisõ}, \vadja{\underline{jõ}\underline{u}tuiss}
 \\
Sõnatüüp ei hõlma teisi lekseeme vormi\-sõnastikus.

\paragraph*{\vadja{\underline{po}utõ}}
\vadja{\underline{po}vva}, \vadja{\underline{po}uta}, \vadja{\underline{po}utasõ}, \vadja{\underline{po}vvõss}, \vadja{\underline{po}vvõd}, \vadja{\underline{po}utojõ}, \vadja{\underline{po}utoit}, \vadja{\underline{po}utoisõ}, \vadja{\underline{po}utoiss}
 \\
sõnatüüp hõlmab lekseeme \vadja{poutõ, lautõ}

\paragraph*{\vadja{\underline{štan}ad}}
\vadja{\underline{štan}ojõ}, \vadja{\underline{štan}oit}, \vadja{\underline{štan}oisõ}, \vadja{\underline{štan}ass}, \vadja{\underline{štan}ad}, \vadja{\underline{štan}ojõ}, \vadja{\underline{štan}oit}, \vadja{\underline{štan}oisõ}, \vadja{\underline{štan}oiss}
 \\
Sõnatüüp ei hõlma teisi lekseeme vormi\-sõnastikus.

\spacing{1.5}


\subsection{\RN{4} käändkond}

Neljandasse käändkonda kuuluvad mitmed sõnad, mis on ainsuses eespoolse vokalismiga, ent mitmuses on tagapoolsed \cite[43]{ariste_grammar_1968}. Selliseid sõnu Heinsoo loodavas kirjakeeles ei esine (isiklik kommunikatsioon). % TODO: küsi ja kuidas viidata?


\subsection{\RN{5} käändkond}

Viiendasse käändkonda kuuluvad kahesilbilised sõnad, mille tüvevokaal on \vadja{-a} ja millel esineb esimeses silbis \vadja{-o-}, \vadja{-u-} või \vadja{-õ-}. Kattumise kohta \RN{3} käändkonna sõnadega, mille esimene silp sisaldab \vadja{-õ-}, mainib Ariste, et enamik neist kuulub siia. \cite[44]{ariste_grammar_1968}.

Avatuid küsimusi-tähelepanekuid:
\begin{itemize}
\item 5 käändkonna liikmed Aristel -õi- on suuresti muudetud -ii-
\item 'mussõ' leiti mitu pl 'mussii' VKSi näitelausete hulgast
\end{itemize}

\subsubsection*{Ekstraktmorfoloogia sõnatüübid}
\spacing{1}

\vspace{1.8em}
\begin{minipage}{\textwidth}
\stepcounter{mallinumber}
\textbf{Tüüpsõnamall \arabic{mallinumber}\,\vadja{kuha}}\\

\begin{sideways}
\begin{tabular}{l l}
muutvormimall & tunnused \\
\hline
\underline{kuh}\,+\,a & \textsc{ sg nom } \\
\underline{kuh}\,+\,a & \textsc{ sg gen } \\
\underline{kuh}\,+\,ha & \textsc{ sg par } \\
\underline{kuh}\,+\,hasõ & \textsc{ sg ill } \\
\underline{kuh}\,+\,õz & \textsc{ sg ine } \\
\underline{kuh}\,+\,õss & \textsc{ sg ela } \\
\underline{kuh}\,+\,õllõ & \textsc{ sg all } \\
\underline{kuh}\,+\,õll & \textsc{ sg ade } \\
\underline{kuh}\,+\,õlt & \textsc{ sg abl } \\
\underline{kuh}\,+\,õssi & \textsc{ sg tra } \\
\underline{kuh}\,+\,õssaa & \textsc{ sg ter } \\
\underline{kuh}\,+\,õka & \textsc{ sg com } \\
\underline{kuh}\,+\,õd & \textsc{ pl nom } \\
\underline{kuh}\,+\,hijõ & \textsc{ pl gen } \\
\underline{kuh}\,+\,hiit & \textsc{ pl par } \\
\underline{kuh}\,+\,hiisõ & \textsc{ pl ill } \\
\underline{kuh}\,+\,hiiz & \textsc{ pl ine } \\
\underline{kuh}\,+\,hiiss & \textsc{ pl ela } \\
\underline{kuh}\,+\,hiillõ & \textsc{ pl all } \\
\underline{kuh}\,+\,hiill & \textsc{ pl ade } \\
\underline{kuh}\,+\,hiilt & \textsc{ pl abl } \\
\underline{kuh}\,+\,hiissi & \textsc{ pl tra } \\
\underline{kuh}\,+\,hiissaa & \textsc{ pl ter } \\
\underline{kuh}\,+\,hijka & \textsc{ pl com } \\
\end{tabular}
\end{sideways}
\captionof{table}{Tüüpsõna \arabic{mallinumber}\,\textit{kuha} ekstraheeritud muutvormimallid.}
\label{tab:tüüpsõnamall-kuha}

\end{minipage}

 
\vspace{1em}
\noindent Tüüpsõna ei hõlma teisi lekseeme vormi\-sõnastikus.

\paragraph*{\vadja{\underline{poi}kõ}}
\vadja{\underline{poi}ga}, \vadja{\underline{poi}ka}, \vadja{\underline{poi}kasõ}, \vadja{\underline{poi}gõss}, \vadja{\underline{poi}gõd}, \vadja{\underline{poi}kijõ}, \vadja{\underline{poi}kiit}, \vadja{\underline{poi}kiisõ}, \vadja{\underline{poi}kiiss}
 \\
Tüüpsõna hõlmab vormisõnastiku lekseeme \vadja{poikõ, rookõ, lõukõ}.


\vspace{1.8em}
\begin{minipage}{\textwidth}
\stepcounter{mallinumber}
\textbf{Tüüpsõnamall \arabic{mallinumber}\,\vadja{biblioteekkõ}}\\

\begin{sideways}
\begin{tabular}{l l}
muutvormimall & tunnused \\
\hline
\underline{biblioteek}\,+\,kõ & \textsc{ sg nom } \\
\underline{biblioteek}\,+\,a & \textsc{ sg gen } \\
\underline{biblioteek}\,+\,ka & \textsc{ sg par } \\
\underline{biblioteek}\,+\,kasõ & \textsc{ sg ill } \\
\underline{biblioteek}\,+\,kõz & \textsc{ sg ine } \\
\underline{biblioteek}\,+\,õss & \textsc{ sg ela } \\
\underline{biblioteek}\,+\,õllõ & \textsc{ sg all } \\
\underline{biblioteek}\,+\,õll & \textsc{ sg ade } \\
\underline{biblioteek}\,+\,õlt & \textsc{ sg abl } \\
\underline{biblioteek}\,+\,õssi & \textsc{ sg tra } \\
\underline{biblioteek}\,+\,kõssaa & \textsc{ sg ter } \\
\underline{biblioteek}\,+\,õka & \textsc{ sg com } \\
\underline{biblioteek}\,+\,õd & \textsc{ pl nom } \\
\underline{biblioteek}\,+\,kijõ & \textsc{ pl gen } \\
\underline{biblioteek}\,+\,kiit & \textsc{ pl par } \\
\underline{biblioteek}\,+\,kiisõ & \textsc{ pl ill } \\
\underline{biblioteek}\,+\,kiiz & \textsc{ pl ine } \\
\underline{biblioteek}\,+\,kiiss & \textsc{ pl ela } \\
\underline{biblioteek}\,+\,kiillõ & \textsc{ pl all } \\
\underline{biblioteek}\,+\,kiill & \textsc{ pl ade } \\
\underline{biblioteek}\,+\,kiilt & \textsc{ pl abl } \\
\underline{biblioteek}\,+\,kiissi & \textsc{ pl tra } \\
\underline{biblioteek}\,+\,kiissaa & \textsc{ pl ter } \\
\underline{biblioteek}\,+\,kijka & \textsc{ pl com } \\
\end{tabular}
\end{sideways}
\captionof{table}{Tüüpsõna \arabic{mallinumber}\,\textit{biblioteekkõ} ekstraheeritud muutvormimallid.}
\label{tab:tüüpsõnamall-biblioteekkõ}

\end{minipage}

 
\vspace{1em}
\noindent Tüüpsõna hõlmab vormisõnastiku 10 lekseemi: \vadja{\underline{biblioteek}kõ, \underline{hoik}kõ, \underline{ikolook}kõ, \underline{jaanikuk}kõ, \underline{kolk}kõ, \underline{kon̕jõk}kõ, \underline{kuk}kõ, \underline{rok}kõ, \underline{suk}kõ} ja \vadja{\underline{bambuk}kõ}.

\paragraph*{\vadja{\underline{kõik}}}
\vadja{\underline{kõik}a}, \vadja{\underline{kõik}ka}, \vadja{\underline{kõik}kasõ}, \vadja{\underline{kõik}õss}, \vadja{\underline{kõik}õd}, \vadja{\underline{kõik}kijõ}, \vadja{\underline{kõik}kiit}, \vadja{\underline{kõik}kiisõ}, \vadja{\underline{kõik}kiiss}
 \\
Tüüpsõna ei hõlma teisi lekseeme vormi\-sõnastikus.


\vspace{1.8em}
\begin{minipage}{\textwidth}
\stepcounter{mallinumber}
\textbf{Tüüpsõnamall \arabic{mallinumber}\,\vadja{kultõ}}\\

\begin{sideways}
\begin{tabular}{l l}
muutvormimall & tunnused \\
\hline
\underline{kul}\,+\,tõ & \textsc{ sg nom } \\
\underline{kul}\,+\,la & \textsc{ sg gen } \\
\underline{kul}\,+\,ta & \textsc{ sg par } \\
\underline{kul}\,+\,tasõ & \textsc{ sg ill } \\
\underline{kul}\,+\,lõz & \textsc{ sg ine } \\
\underline{kul}\,+\,lõss & \textsc{ sg ela } \\
\underline{kul}\,+\,lõllõ & \textsc{ sg all } \\
\underline{kul}\,+\,lõll & \textsc{ sg ade } \\
\underline{kul}\,+\,lõlt & \textsc{ sg abl } \\
\underline{kul}\,+\,lõssi & \textsc{ sg tra } \\
\underline{kul}\,+\,lõssaa & \textsc{ sg ter } \\
\underline{kul}\,+\,lõka & \textsc{ sg com } \\
\underline{kul}\,+\,lõd & \textsc{ pl nom } \\
\underline{kul}\,+\,tõjõ & \textsc{ pl gen } \\
\underline{kul}\,+\,tõit & \textsc{ pl par } \\
\underline{kul}\,+\,tõisõ & \textsc{ pl ill } \\
\underline{kul}\,+\,tõiz & \textsc{ pl ine } \\
\underline{kul}\,+\,tõiss & \textsc{ pl ela } \\
\underline{kul}\,+\,tõillõ & \textsc{ pl all } \\
\underline{kul}\,+\,tõill & \textsc{ pl ade } \\
\underline{kul}\,+\,tõilt & \textsc{ pl abl } \\
\underline{kul}\,+\,tõissi & \textsc{ pl tra } \\
\underline{kul}\,+\,tõissaa & \textsc{ pl ter } \\
\underline{kul}\,+\,tõika & \textsc{ pl com } \\
\end{tabular}
\end{sideways}
\captionof{table}{Tüüpsõna \arabic{mallinumber}\,\textit{kultõ} ekstraheeritud muutvormimallid.}
\label{tab:tüüpsõnamall-kultõ}

\end{minipage}

 
\vspace{1em}
\noindent Tüüpsõna ei hõlma teisi lekseeme vormi\-sõnastikus.

\paragraph*{\vadja{\underline{vim}põ}}
\vadja{\underline{vim}ma}, \vadja{\underline{vim}pa}, \vadja{\underline{vim}pasõ}, \vadja{\underline{vim}mõss}, \vadja{\underline{vim}mõd}, \vadja{\underline{vim}pijõ}, \vadja{\underline{vim}piit}, \vadja{\underline{vim}piisõ}, \vadja{\underline{vim}piiss}
 \\
Sõnatüüp hõlmab vormisõnastiku lekseeme: \vadja{vimpõ, kumpõ}.

\paragraph*{\vadja{\underline{kompjuter}a}}
\vadja{\underline{kompjuter}a}, \vadja{\underline{kompjuter}a}, \vadja{\underline{kompjuter}asõ}, \vadja{\underline{kompjuter}ass}, \vadja{\underline{kompjuter}ad}, \vadja{\underline{kompjuter}ijõ}, \vadja{\underline{kompjuter}iit}, \vadja{\underline{kompjuter}iisõ}, \vadja{\underline{kompjuter}iiss}
 \\
Sõnatüüp hõlmab vormisõnastiku lekseeme: \vadja{kompjutera, kuja, loba, õmpõja, õpõttõja, ižora}.

\paragraph*{\vadja{\underline{maamun}a}}
\vadja{\underline{maamun}a}, \vadja{\underline{maamun}a}, \vadja{\underline{maamun}asõ}, \vadja{\underline{maamun}ass}, \vadja{\underline{maamun}ad}, \vadja{\underline{maamun}õjõ}, \vadja{\underline{maamun}õit}, \vadja{\underline{maamun}õisõ}, \vadja{\underline{maamun}õiss}
 \\
Sõnatüüp hõlmab vormisõnastiku lekseeme: \vadja{maamuna, muna, kal̕indora}.

\paragraph*{\vadja{\underline{mokom}}}
\vadja{\underline{mokom}a}, \vadja{\underline{mokom}a}, \vadja{\underline{mokom}asõ}, \vadja{\underline{mokom}õss}, \vadja{\underline{mokom}õd}, \vadja{\underline{mokom}ijõ}, \vadja{\underline{mokom}iit}, \vadja{\underline{mokom}iisõ}, \vadja{\underline{mokom}iiss}
 \\
Tüüpsõna ei hõlma teisi lekseeme vormi\-sõnastikus.


\vspace{1.8em}
\begin{minipage}{\textwidth}
\stepcounter{mallinumber}
\textbf{Tüüpsõnamall \arabic{mallinumber}\,\vadja{bukvõ}}\\

\begin{sideways}
\begin{tabular}{l l}
muutvormimall & tunnused \\
\hline
\underline{bukv}\,+\,õ & \textsc{ sg nom } \\
\underline{bukv}\,+\,a & \textsc{ sg gen } \\
\underline{bukv}\,+\,a & \textsc{ sg par } \\
\underline{bukv}\,+\,asõ & \textsc{ sg ill } \\
\underline{bukv}\,+\,õz & \textsc{ sg ine } \\
\underline{bukv}\,+\,õss & \textsc{ sg ela } \\
\underline{bukv}\,+\,õllõ & \textsc{ sg all } \\
\underline{bukv}\,+\,õll & \textsc{ sg ade } \\
\underline{bukv}\,+\,õlt & \textsc{ sg abl } \\
\underline{bukv}\,+\,õssi & \textsc{ sg tra } \\
\underline{bukv}\,+\,õssaa & \textsc{ sg ter } \\
\underline{bukv}\,+\,õka & \textsc{ sg com } \\
\underline{bukv}\,+\,õd & \textsc{ pl nom } \\
\underline{bukv}\,+\,ijõ & \textsc{ pl gen } \\
\underline{bukv}\,+\,iit & \textsc{ pl par } \\
\underline{bukv}\,+\,iisõ & \textsc{ pl ill } \\
\underline{bukv}\,+\,iiz & \textsc{ pl ine } \\
\underline{bukv}\,+\,iiss & \textsc{ pl ela } \\
\underline{bukv}\,+\,iillõ & \textsc{ pl all } \\
\underline{bukv}\,+\,iill & \textsc{ pl ade } \\
\underline{bukv}\,+\,iilt & \textsc{ pl abl } \\
\underline{bukv}\,+\,iissi & \textsc{ pl tra } \\
\underline{bukv}\,+\,iissaa & \textsc{ pl ter } \\
\underline{bukv}\,+\,ijka & \textsc{ pl com } \\
\end{tabular}
\end{sideways}
\captionof{table}{Tüüpsõna \arabic{mallinumber}\,\textit{bukvõ} ekstraheeritud muutvormimallid.}
\label{tab:tüüpsõnamall-bukvõ}

\end{minipage}

 
\vspace{1em}
\noindent Tüüpsõna hõlmab vormisõnastiku 28 lekseemi: \vadja{\underline{bukv}õ, \underline{duum}õ, \underline{form}õ, \underline{ilm}õ, \underline{koir}õ, \underline{konn}õ, \underline{kuhj}õ, \underline{kuiv}õ, \underline{kumm}õ, \underline{kuuluv}õ, \underline{kõrv}õ, \underline{mood}õ, \underline{muudr}õ, \underline{mõiz}õ, \underline{obraaz}õ, \underline{post}õ, \underline{programm}õ, \underline{rooj}õ, \underline{sool}õ, \underline{sveež}õ, \underline{trubačist}õ, \underline{tuim}õ, \underline{tuttav}õ, \underline{velosiped}õ, \underline{vohm}õ, \underline{vool}õ, \underline{võim}õ} ja \vadja{\underline{bomb}õ}.


\vspace{1.8em}
\begin{minipage}{\textwidth}
\stepcounter{mallinumber}
\textbf{Tüüpsõnamall \arabic{mallinumber}\,\vadja{propkõ}}\\

\begin{sideways}
\begin{tabular}{l l}
muutvormimall & tunnused \\
\hline
\underline{pro}\,+\,pkõ & \textsc{ sg nom } \\
\underline{pro}\,+\,bga & \textsc{ sg gen } \\
\underline{pro}\,+\,pka & \textsc{ sg par } \\
\underline{pro}\,+\,pkasõ & \textsc{ sg ill } \\
\underline{pro}\,+\,bgõz & \textsc{ sg ine } \\
\underline{pro}\,+\,bgõss & \textsc{ sg ela } \\
\underline{pro}\,+\,bgõllõ & \textsc{ sg all } \\
\underline{pro}\,+\,bgõll & \textsc{ sg ade } \\
\underline{pro}\,+\,bgõlt & \textsc{ sg abl } \\
\underline{pro}\,+\,bgõssi & \textsc{ sg tra } \\
\underline{pro}\,+\,bgõssaa & \textsc{ sg ter } \\
\underline{pro}\,+\,bgõka & \textsc{ sg com } \\
\underline{pro}\,+\,bgõd & \textsc{ pl nom } \\
\underline{pro}\,+\,pkijõ & \textsc{ pl gen } \\
\underline{pro}\,+\,pkiit & \textsc{ pl par } \\
\underline{pro}\,+\,pkiisõ & \textsc{ pl ill } \\
\underline{pro}\,+\,pkiiz & \textsc{ pl ine } \\
\underline{pro}\,+\,pkiiss & \textsc{ pl ela } \\
\underline{pro}\,+\,pkiillõ & \textsc{ pl all } \\
\underline{pro}\,+\,pkiill & \textsc{ pl ade } \\
\underline{pro}\,+\,pkiilt & \textsc{ pl abl } \\
\underline{pro}\,+\,pkiissi & \textsc{ pl tra } \\
\underline{pro}\,+\,pkiissaa & \textsc{ pl ter } \\
\underline{pro}\,+\,pkijka & \textsc{ pl com } \\
\end{tabular}
\end{sideways}
\captionof{table}{Tüüpsõna \arabic{mallinumber}\,\textit{propkõ} ekstraheeritud muutvormimallid.}
\label{tab:tüüpsõnamall-propkõ}

\end{minipage}

 
\vspace{1em}
\noindent Tüüpsõna hõlmab vormisõnastiku 3 lekseemi: \vadja{\underline{pro}pkõ, \underline{sko}pkõ} ja \vadja{\underline{ju}pkõ}.


\vspace{1.8em}
\begin{minipage}{\textwidth}
\stepcounter{mallinumber}
\textbf{Tüüpsõnamall \arabic{mallinumber}\,\vadja{rooppõ}}\\

\begin{sideways}
\begin{tabular}{l l}
muutvormimall & tunnused \\
\hline
\underline{roop}\,+\,põ & \textsc{ sg nom } \\
\underline{roop}\,+\,a & \textsc{ sg gen } \\
\underline{roop}\,+\,pa & \textsc{ sg par } \\
\underline{roop}\,+\,pasõ & \textsc{ sg ill } \\
\underline{roop}\,+\,põz & \textsc{ sg ine } \\
\underline{roop}\,+\,õss & \textsc{ sg ela } \\
\underline{roop}\,+\,õllõ & \textsc{ sg all } \\
\underline{roop}\,+\,õll & \textsc{ sg ade } \\
\underline{roop}\,+\,õlt & \textsc{ sg abl } \\
\underline{roop}\,+\,õssi & \textsc{ sg tra } \\
\underline{roop}\,+\,põssaa & \textsc{ sg ter } \\
\underline{roop}\,+\,õka & \textsc{ sg com } \\
\underline{roop}\,+\,õd & \textsc{ pl nom } \\
\underline{roop}\,+\,põjõ & \textsc{ pl gen } \\
\underline{roop}\,+\,õit & \textsc{ pl par } \\
\underline{roop}\,+\,õisõ & \textsc{ pl ill } \\
\underline{roop}\,+\,õiz & \textsc{ pl ine } \\
\underline{roop}\,+\,õiss & \textsc{ pl ela } \\
\underline{roop}\,+\,õillõ & \textsc{ pl all } \\
\underline{roop}\,+\,õill & \textsc{ pl ade } \\
\underline{roop}\,+\,õilt & \textsc{ pl abl } \\
\underline{roop}\,+\,õissi & \textsc{ pl tra } \\
\underline{roop}\,+\,õissaa & \textsc{ pl ter } \\
\underline{roop}\,+\,õika & \textsc{ pl com } \\
\end{tabular}
\end{sideways}
\captionof{table}{Tüüpsõna \arabic{mallinumber}\,\textit{rooppõ} ekstraheeritud muutvormimallid.}
\label{tab:tüüpsõnamall-rooppõ}

\end{minipage}

 
\vspace{1em}
\noindent Tüüpsõna ei hõlma teisi lekseeme vormi\-sõnastikus.

\paragraph*{\vadja{\underline{kur}p}}
\vadja{\underline{kur}va}, \vadja{\underline{kur}pa}, \vadja{\underline{kur}pasõ}, \vadja{\underline{kur}võss}, \vadja{\underline{kur}võd}, \vadja{\underline{kur}pijõ}, \vadja{\underline{kur}piit}, \vadja{\underline{kur}piisõ}, \vadja{\underline{kur}piiss}
 \\
Sõnatüüp ei hõlma teisi lekseeme vormi\-sõnastikus.

\paragraph*{\vadja{\underline{u}sa}}
\vadja{\underline{u}za}, \vadja{\underline{u}ssa}, \vadja{\underline{u}ssasõ}, \vadja{\underline{u}zass}, \vadja{\underline{u}zad}, \vadja{\underline{u}sijõ}, \vadja{\underline{u}siit}, \vadja{\underline{u}siisõ}, \vadja{\underline{u}siiss}
 \\
Tüüpsõna ei hõlma teisi lekseeme vormi\-sõnastikus.


\vspace{1.8em}
\begin{minipage}{\textwidth}
\stepcounter{mallinumber}
\textbf{Tüüpsõnamall \arabic{mallinumber}\,\vadja{õhsõ}}\\

\begin{sideways}
\begin{tabular}{l l}
muutvormimall & tunnused \\
\hline
\underline{õh}\,+\,sõ & \textsc{ sg nom } \\
\underline{õh}\,+\,za & \textsc{ sg gen } \\
\underline{õh}\,+\,sa & \textsc{ sg par } \\
\underline{õh}\,+\,sasõ & \textsc{ sg ill } \\
\underline{õh}\,+\,zõz & \textsc{ sg ine } \\
\underline{õh}\,+\,zõss & \textsc{ sg ela } \\
\underline{õh}\,+\,zõllõ & \textsc{ sg all } \\
\underline{õh}\,+\,zõll & \textsc{ sg ade } \\
\underline{õh}\,+\,zõlt & \textsc{ sg abl } \\
\underline{õh}\,+\,zõssi & \textsc{ sg tra } \\
\underline{õh}\,+\,sõssaa & \textsc{ sg ter } \\
\underline{õh}\,+\,zõka & \textsc{ sg com } \\
\underline{õh}\,+\,zõd & \textsc{ pl nom } \\
\underline{õh}\,+\,sijõ & \textsc{ pl gen } \\
\underline{õh}\,+\,siit & \textsc{ pl par } \\
\underline{õh}\,+\,siisõ & \textsc{ pl ill } \\
\underline{õh}\,+\,siiz & \textsc{ pl ine } \\
\underline{õh}\,+\,siiss & \textsc{ pl ela } \\
\underline{õh}\,+\,siillõ & \textsc{ pl all } \\
\underline{õh}\,+\,siill & \textsc{ pl ade } \\
\underline{õh}\,+\,siilt & \textsc{ pl abl } \\
\underline{õh}\,+\,siissi & \textsc{ pl tra } \\
\underline{õh}\,+\,siissaa & \textsc{ pl ter } \\
\underline{õh}\,+\,sijka & \textsc{ pl com } \\
\end{tabular}
\end{sideways}
\captionof{table}{Tüüpsõna \arabic{mallinumber}\,\textit{õhsõ} ekstraheeritud muutvormimallid.}
\label{tab:tüüpsõnamall-õhsõ}

\end{minipage}

 
\vspace{1em}
\noindent Tüüpsõna hõlmab vormisõnastiku 2 lekseemi: \vadja{\underline{õh}sõ} ja \vadja{\underline{sor}sõ}.

\paragraph*{\vadja{\underline{lui}skõ}}
\vadja{\underline{lui}zga}, \vadja{\underline{lui}ska}, \vadja{\underline{lui}skasõ}, \vadja{\underline{lui}zgõss}, \vadja{\underline{lui}zgõd}, \vadja{\underline{lui}skijõ}, \vadja{\underline{lui}skiit}, \vadja{\underline{lui}skiisõ}, \vadja{\underline{lui}skiiss}
 \\
Sõnatüüp ei hõlma teisi lekseeme vormi\-sõnastikus.

\paragraph*{\vadja{\underline{mus}sõ}}
\vadja{\underline{mus}a}, \vadja{\underline{mus}sa}, \vadja{\underline{mus}sasõ}, \vadja{\underline{mus}ass}, \vadja{\underline{mus}ad}, \vadja{\underline{mus}sõjõ}, \vadja{\underline{mus}sõit}, \vadja{\underline{mus}sõisõ}, \vadja{\underline{mus}sõiss}
 \\
Tüüpsõna hõlmab vormisõnastiku 2 lekseemi: \vadja{mussõ} ja \vadja{kapussõ}.


\vspace{1.8em}
\begin{minipage}{\textwidth}
\stepcounter{mallinumber}
\textbf{Tüüpsõnamall \arabic{mallinumber}\,\vadja{moškõ}}\\

\begin{sideways}
\begin{tabular}{l l}
muutvormimall & tunnused \\
\hline
\underline{mo}\,+\,škõ & \textsc{ sg nom } \\
\underline{mo}\,+\,žga & \textsc{ sg gen } \\
\underline{mo}\,+\,ška & \textsc{ sg par } \\
\underline{mo}\,+\,škasõ & \textsc{ sg ill } \\
\underline{mo}\,+\,žgõz & \textsc{ sg ine } \\
\underline{mo}\,+\,žgõss & \textsc{ sg ela } \\
\underline{mo}\,+\,žgõllõ & \textsc{ sg all } \\
\underline{mo}\,+\,žgõll & \textsc{ sg ade } \\
\underline{mo}\,+\,žgõlt & \textsc{ sg abl } \\
\underline{mo}\,+\,žgõssi & \textsc{ sg tra } \\
\underline{mo}\,+\,žgõssaa & \textsc{ sg ter } \\
\underline{mo}\,+\,žgõka & \textsc{ sg com } \\
\underline{mo}\,+\,žgõd & \textsc{ pl nom } \\
\underline{mo}\,+\,škijõ & \textsc{ pl gen } \\
\underline{mo}\,+\,škiit & \textsc{ pl par } \\
\underline{mo}\,+\,škiisõ & \textsc{ pl ill } \\
\underline{mo}\,+\,škiiz & \textsc{ pl ine } \\
\underline{mo}\,+\,škiiss & \textsc{ pl ela } \\
\underline{mo}\,+\,škiillõ & \textsc{ pl all } \\
\underline{mo}\,+\,škiill & \textsc{ pl ade } \\
\underline{mo}\,+\,škiilt & \textsc{ pl abl } \\
\underline{mo}\,+\,škiissi & \textsc{ pl tra } \\
\underline{mo}\,+\,škiissaa & \textsc{ pl ter } \\
\underline{mo}\,+\,škijka & \textsc{ pl com } \\
\end{tabular}
\end{sideways}
\captionof{table}{Tüüpsõna \arabic{mallinumber}\,\textit{moškõ} ekstraheeritud muutvormimallid.}
\label{tab:tüüpsõnamall-moškõ}

\end{minipage}

 
\vspace{1em}
\noindent Tüüpsõna hõlmab vormisõnastiku 2 lekseemi: \vadja{\underline{mo}škõ} ja \vadja{\underline{krõ}škõ}.


\vspace{1.8em}
\begin{minipage}{\textwidth}
\stepcounter{mallinumber}
\textbf{Tüüpsõnamall \arabic{mallinumber}\,\vadja{lootõ}}\\

\begin{sideways}
\begin{tabular}{l l}
muutvormimall & tunnused \\
\hline
\underline{loo}\,+\,tõ & \textsc{ sg nom } \\
\underline{loo}\,+\,vva & \textsc{ sg gen } \\
\underline{loo}\,+\,ta & \textsc{ sg par } \\
\underline{loo}\,+\,tasõ & \textsc{ sg ill } \\
\underline{loo}\,+\,vvõz & \textsc{ sg ine } \\
\underline{loo}\,+\,vvõss & \textsc{ sg ela } \\
\underline{loo}\,+\,vvõllõ & \textsc{ sg all } \\
\underline{loo}\,+\,vvõll & \textsc{ sg ade } \\
\underline{loo}\,+\,vvõlt & \textsc{ sg abl } \\
\underline{loo}\,+\,vvõssi & \textsc{ sg tra } \\
\underline{loo}\,+\,vvõssaa & \textsc{ sg ter } \\
\underline{loo}\,+\,vvõka & \textsc{ sg com } \\
\underline{loo}\,+\,vvõd & \textsc{ pl nom } \\
\underline{loo}\,+\,tijõ & \textsc{ pl gen } \\
\underline{loo}\,+\,tiit & \textsc{ pl par } \\
\underline{loo}\,+\,tiisõ & \textsc{ pl ill } \\
\underline{loo}\,+\,tiiz & \textsc{ pl ine } \\
\underline{loo}\,+\,tiiss & \textsc{ pl ela } \\
\underline{loo}\,+\,tiillõ & \textsc{ pl all } \\
\underline{loo}\,+\,tiill & \textsc{ pl ade } \\
\underline{loo}\,+\,tiilt & \textsc{ pl abl } \\
\underline{loo}\,+\,tiissi & \textsc{ pl tra } \\
\underline{loo}\,+\,tiissaa & \textsc{ pl ter } \\
\underline{loo}\,+\,tijka & \textsc{ pl com } \\
\end{tabular}
\end{sideways}
\captionof{table}{Tüüpsõna \arabic{mallinumber}\,\textit{lootõ} ekstraheeritud muutvormimallid.}
\label{tab:tüüpsõnamall-lootõ}

\end{minipage}

 
\vspace{1em}
\noindent Tüüpsõna ei hõlma teisi lekseeme vormi\-sõnastikus.


\vspace{1.8em}
\begin{minipage}{\textwidth}
\stepcounter{mallinumber}
\textbf{Tüüpsõnamall \arabic{mallinumber}\,\vadja{biskvittõ}}\\

\begin{sideways}
\begin{tabular}{l l}
muutvormimall & tunnused \\
\hline
\underline{biskvit}\,+\,tõ & \textsc{ sg nom } \\
\underline{biskvit}\,+\,a & \textsc{ sg gen } \\
\underline{biskvit}\,+\,ta & \textsc{ sg par } \\
\underline{biskvit}\,+\,tasõ & \textsc{ sg ill } \\
\underline{biskvit}\,+\,tõz & \textsc{ sg ine } \\
\underline{biskvit}\,+\,õss & \textsc{ sg ela } \\
\underline{biskvit}\,+\,õllõ & \textsc{ sg all } \\
\underline{biskvit}\,+\,õll & \textsc{ sg ade } \\
\underline{biskvit}\,+\,õlt & \textsc{ sg abl } \\
\underline{biskvit}\,+\,õssi & \textsc{ sg tra } \\
\underline{biskvit}\,+\,tõssaa & \textsc{ sg ter } \\
\underline{biskvit}\,+\,õka & \textsc{ sg com } \\
\underline{biskvit}\,+\,õd & \textsc{ pl nom } \\
\underline{biskvit}\,+\,tajõ & \textsc{ pl gen } \\
\underline{biskvit}\,+\,tiit & \textsc{ pl par } \\
\underline{biskvit}\,+\,tiisõ & \textsc{ pl ill } \\
\underline{biskvit}\,+\,tiiz & \textsc{ pl ine } \\
\underline{biskvit}\,+\,tiiss & \textsc{ pl ela } \\
\underline{biskvit}\,+\,tiillõ & \textsc{ pl all } \\
\underline{biskvit}\,+\,tiill & \textsc{ pl ade } \\
\underline{biskvit}\,+\,tiilt & \textsc{ pl abl } \\
\underline{biskvit}\,+\,tiissi & \textsc{ pl tra } \\
\underline{biskvit}\,+\,tiissaa & \textsc{ pl ter } \\
\underline{biskvit}\,+\,tijka & \textsc{ pl com } \\
\end{tabular}
\end{sideways}
\captionof{table}{Tüüpsõna \arabic{mallinumber}\,\textit{biskvittõ} ekstraheeritud muutvormimallid.}
\label{tab:tüüpsõnamall-biskvittõ}

\end{minipage}

 
\vspace{1em}
\noindent Tüüpsõna ei hõlma teisi lekseeme vormi\-sõnastikus.

\spacing{1.5}
  
\subsection{\RN{6} käändkond}

Kuuendasse käändkonda kuuluvad Ariste sõnul need sõnad, mis lõppevad \vadja{-õa/-eä/-iä}. Ta toob eraldi välja Jõgõperä murde erinevused üheainsa näitesõnaga. \cite[44]{ariste_grammar_1968}. Vadja kirjakeeles püütakse järgida ... TODO kirjutada.

Avatuid küsimusi-tähelepanekuid:
\begin{itemize}
\item käändkonna liikmete pluurali tüved on ühtlustatud -- kas jätta nii või taastada Tsvetkovi variatiivsus?
\end{itemize}

\subsubsection*{Ekstraktmorfoloogia sõnatüübid}
\spacing{1}

\vspace{1.8em}
\begin{minipage}{\textwidth}
\stepcounter{mallinumber}
\textbf{Tüüpsõnamall \arabic{mallinumber}\,\vadja{kerkiä}}\\

\begin{sideways}
\begin{tabular}{l l}
muutvormimall & tunnused \\
\hline
\underline{kerki}\,+\,ä & \textsc{ sg nom } \\
\underline{kerki}\,+\,ä & \textsc{ sg gen } \\
\underline{kerki}\,+\,ätä & \textsc{ sg par } \\
\underline{kerki}\,+\,äse & \textsc{ sg ill } \\
\underline{kerki}\,+\,äz & \textsc{ sg ine } \\
\underline{kerki}\,+\,äss & \textsc{ sg ela } \\
\underline{kerki}\,+\,älle & \textsc{ sg all } \\
\underline{kerki}\,+\,äll & \textsc{ sg ade } \\
\underline{kerki}\,+\,ält & \textsc{ sg abl } \\
\underline{kerki}\,+\,ässi & \textsc{ sg tra } \\
\underline{kerki}\,+\,ässaa & \textsc{ sg ter } \\
\underline{kerki}\,+\,äka & \textsc{ sg com } \\
\underline{kerki}\,+\,äd & \textsc{ pl nom } \\
\underline{kerki}\,+\,je & \textsc{ pl gen } \\
\underline{kerki}\,+\,it & \textsc{ pl par } \\
\underline{kerki}\,+\,ise & \textsc{ pl ill } \\
\underline{kerki}\,+\,iz & \textsc{ pl ine } \\
\underline{kerki}\,+\,iss & \textsc{ pl ela } \\
\underline{kerki}\,+\,ille & \textsc{ pl all } \\
\underline{kerki}\,+\,ill & \textsc{ pl ade } \\
\underline{kerki}\,+\,ilt & \textsc{ pl abl } \\
\underline{kerki}\,+\,issi & \textsc{ pl tra } \\
\underline{kerki}\,+\,issaa & \textsc{ pl ter } \\
\underline{kerki}\,+\,jka & \textsc{ pl com } \\
\end{tabular}
\end{sideways}
\captionof{table}{Tüüpsõna \arabic{mallinumber}\,\textit{kerkiä} ekstraheeritud muutvormimallid.}
\label{tab:tüüpsõnamall-kerkiä}

\end{minipage}

 
\vspace{1em}
\noindent Tüüpsõna hõlmab vormisõnastiku 5 lekseemi: \vadja{\underline{kerki}ä, \underline{peh̕mi}ä, \underline{pimmi}ä, \underline{siiti}ä} ja \vadja{\underline{jämi}ä}.

\paragraph*{\vadja{\underline{terv}\underline{e}}}
\vadja{\underline{terv}\underline{e}}, \vadja{\underline{terv}\underline{e}ttä}, \vadja{\underline{terv}\underline{e}se}, \vadja{\underline{terv}\underline{e}ss}, \vadja{\underline{terv}\underline{e}d}, \vadja{\underline{terv}ij\underline{e}}, \vadja{\underline{terv}\underline{e}it}, \vadja{\underline{terv}\underline{e}ise}, \vadja{\underline{terv}\underline{e}iss}
 \\
Sõnatüüp ei hõlma teisi lekseeme vormi\-sõnastikus.

\paragraph*{\vadja{\underline{kank}a}}
\vadja{\underline{kank}a}, \vadja{\underline{kank}atõ}, \vadja{\underline{kank}asõ}, \vadja{\underline{kank}ass}, \vadja{\underline{kank}ad}, \vadja{\underline{kank}õjõ}, \vadja{\underline{kank}õit}, \vadja{\underline{kank}õisõ}, \vadja{\underline{kank}õiss}
 \\
sõnatüüp hõlmab lekseeme \vadja{kanka, kõrka, maikka, makka, ruska, valka, õika, harma}

\spacing{1.5}

\subsection{\RN{7} käändkond}

Seitsmendasse käändkonda kuuluvad kahesilbilised sõnad, mille \msd{sg nom} lõpp on \vadja{-i}, ent mille tüvevokaal on \vadja{-e/-õ} \cite[45]{ariste_grammar_1968}.

Avatuid küsimusi-tähelepanekuid:
\begin{itemize}
\item 7 käändkonna kohta TODO kirjuta et isuri mõju tõttu on -i:-e:-iä levinud, aga normeerime nagu Aristel ja Tsvetkovil ka paralleelina tihti
\item eespoolsed on i:e:eä ja tagapoolsed on i:õ:õa
\item väci:väe aga mida teha lahti:lahe? -- VKSis esineb Lu lahõ
\end{itemize}


\subsubsection*{Ekstraktmorfoloogia sõnatüübid}
\spacing{1}

\vspace{1.8em}
\begin{minipage}{\textwidth}
\stepcounter{mallinumber}
\textbf{Tüüpsõnamall \arabic{mallinumber}\,\vadja{väči}}\\

\begin{sideways}
\begin{tabular}{l l}
muutvormimall & tunnused \\
\hline
\underline{vä}\,+\,či & \textsc{ sg nom } \\
\underline{vä}\,+\,e & \textsc{ sg gen } \\
\underline{vä}\,+\,ččeä & \textsc{ sg par } \\
\underline{vä}\,+\,ččese & \textsc{ sg ill } \\
\underline{vä}\,+\,ez & \textsc{ sg ine } \\
\underline{vä}\,+\,ess & \textsc{ sg ela } \\
\underline{vä}\,+\,elle & \textsc{ sg all } \\
\underline{vä}\,+\,ell & \textsc{ sg ade } \\
\underline{vä}\,+\,elt & \textsc{ sg abl } \\
\underline{vä}\,+\,essi & \textsc{ sg tra } \\
\underline{vä}\,+\,essaa & \textsc{ sg ter } \\
\underline{vä}\,+\,eka & \textsc{ sg com } \\
\underline{vä}\,+\,ed & \textsc{ pl nom } \\
\underline{vä}\,+\,ččije & \textsc{ pl gen } \\
\underline{vä}\,+\,ččiit & \textsc{ pl par } \\
\underline{vä}\,+\,ččiise & \textsc{ pl ill } \\
\underline{vä}\,+\,ččiiz & \textsc{ pl ine } \\
\underline{vä}\,+\,ččiiss & \textsc{ pl ela } \\
\underline{vä}\,+\,ččiille & \textsc{ pl all } \\
\underline{vä}\,+\,ččiill & \textsc{ pl ade } \\
\underline{vä}\,+\,ččiilt & \textsc{ pl abl } \\
\underline{vä}\,+\,ččiissi & \textsc{ pl tra } \\
\underline{vä}\,+\,ččiissaa & \textsc{ pl ter } \\
\underline{vä}\,+\,ččijka & \textsc{ pl com } \\
\end{tabular}
\end{sideways}
\captionof{table}{Tüüpsõna \arabic{mallinumber}\,\textit{väči} ekstraheeritud muutvormimallid.}
\label{tab:tüüpsõnamall-väči}

\end{minipage}

 
\vspace{1em}
\noindent Tüüpsõna hõlmab vormisõnastiku 2 lekseemi: \vadja{\underline{vä}či} ja \vadja{\underline{mä}či}.

\paragraph*{\vadja{\underline{lah}ti}}
\vadja{\underline{lah}õ}, \vadja{\underline{lah}tõa}, \vadja{\underline{lah}tõsõ}, \vadja{\underline{lah}õss}, \vadja{\underline{lah}õd}, \vadja{\underline{lah}tijõ}, \vadja{\underline{lah}tiit}, \vadja{\underline{lah}tiisõ}, \vadja{\underline{lah}tiiss}
 \\
Sõnatüüp ei hõlma teisi lekseeme vormi\-sõnastikus.

\paragraph*{\vadja{\underline{la}ki}}
\vadja{\underline{la}gõ}, \vadja{\underline{la}kkõa}, \vadja{\underline{la}kkõsõ}, \vadja{\underline{la}gõss}, \vadja{\underline{la}gõd}, \vadja{\underline{la}kijõ}, \vadja{\underline{la}kiit}, \vadja{\underline{la}kiisõ}, \vadja{\underline{la}kiiss}
 \\
Sõnatüüp hõlmab vormisõnastiku lekseeme: \vadja{laki, nõki, jõki}.


\vspace{1.8em}
\begin{minipage}{\textwidth}
\stepcounter{mallinumber}
\textbf{Tüüpsõnamall \arabic{mallinumber}\,\vadja{kurki}}\\

\begin{sideways}
\begin{tabular}{l l}
muutvormimall & tunnused \\
\hline
\underline{kur}\,+\,ki & \textsc{ sg nom } \\
\underline{kur}\,+\,gõ & \textsc{ sg gen } \\
\underline{kur}\,+\,kõa & \textsc{ sg par } \\
\underline{kur}\,+\,kõsõ & \textsc{ sg ill } \\
\underline{kur}\,+\,gõz & \textsc{ sg ine } \\
\underline{kur}\,+\,gõss & \textsc{ sg ela } \\
\underline{kur}\,+\,gõllõ & \textsc{ sg all } \\
\underline{kur}\,+\,gõll & \textsc{ sg ade } \\
\underline{kur}\,+\,gõlt & \textsc{ sg abl } \\
\underline{kur}\,+\,gõssi & \textsc{ sg tra } \\
\underline{kur}\,+\,gõssaa & \textsc{ sg ter } \\
\underline{kur}\,+\,gõka & \textsc{ sg com } \\
\underline{kur}\,+\,gõd & \textsc{ pl nom } \\
\underline{kur}\,+\,kijõ & \textsc{ pl gen } \\
\underline{kur}\,+\,kiit & \textsc{ pl par } \\
\underline{kur}\,+\,kiisõ & \textsc{ pl ill } \\
\underline{kur}\,+\,kiiz & \textsc{ pl ine } \\
\underline{kur}\,+\,kiiss & \textsc{ pl ela } \\
\underline{kur}\,+\,kiillõ & \textsc{ pl all } \\
\underline{kur}\,+\,kiill & \textsc{ pl ade } \\
\underline{kur}\,+\,kiilt & \textsc{ pl abl } \\
\underline{kur}\,+\,kiissi & \textsc{ pl tra } \\
\underline{kur}\,+\,kiissaa & \textsc{ pl ter } \\
\underline{kur}\,+\,kijka & \textsc{ pl com } \\
\end{tabular}
\end{sideways}
\captionof{table}{Tüüpsõna \arabic{mallinumber}\,\textit{kurki} ekstraheeritud muutvormimallid.}
\label{tab:tüüpsõnamall-kurki}

\end{minipage}

 
\vspace{1em}
\noindent Tüüpsõna hõlmab vormisõnastiku 3 lekseemi: \vadja{\underline{kur}ki, \underline{õn}ki} ja \vadja{\underline{kan}ki}.

\paragraph*{\vadja{\underline{põ}ski}}
\vadja{\underline{põ}zgõ}, \vadja{\underline{põ}skõa}, \vadja{\underline{põ}skõsõ}, \vadja{\underline{põ}zgõss}, \vadja{\underline{põ}zgõd}, \vadja{\underline{põ}skijõ}, \vadja{\underline{põ}skiit}, \vadja{\underline{põ}skiisõ}, \vadja{\underline{põ}skiiss}
 \\
Tüüpsõna ei hõlma teisi lekseeme vormi\-sõnastikus.

\paragraph*{\vadja{\underline{irv}i}}
\vadja{\underline{irv}e}, \vadja{\underline{irv}eä}, \vadja{\underline{irv}ese}, \vadja{\underline{irv}ess}, \vadja{\underline{irv}ed}, \vadja{\underline{irv}ije}, \vadja{\underline{irv}iit}, \vadja{\underline{irv}iise}, \vadja{\underline{irv}iiss}
 \\
Tüüpsõna hõlmab vormisõnastiku lekseeme \vadja{irvi, järvi, leemi, nimi, pilvi} ja \vadja{čivi}.

\paragraph*{\vadja{\underline{sarv}i}}
\vadja{\underline{sarv}õ}, \vadja{\underline{sarv}õa}, \vadja{\underline{sarv}õsõ}, \vadja{\underline{sarv}õss}, \vadja{\underline{sarv}õd}, \vadja{\underline{sarv}ijõ}, \vadja{\underline{sarv}iit}, \vadja{\underline{sarv}iisõ}, \vadja{\underline{sarv}iiss}
 \\
Tüüpsõna hõlmab vormisõnastiku lekseeme: \vadja{sarvi, savi, Soomi, sõrmi, taimi, talvi, tammi, õnni, õvvi, põlvi}.


\vspace{1.8em}
\begin{minipage}{\textwidth}
\stepcounter{mallinumber}
\textbf{Tüüpsõnamall \arabic{mallinumber}\,\vadja{enči}}\\

\begin{sideways}
\begin{tabular}{l l}
muutvormimall & tunnused \\
\hline
\underline{en}\,+\,či & \textsc{ sg nom } \\
\underline{en}\,+\,̕n̕e & \textsc{ sg gen } \\
\underline{en}\,+\,čeä & \textsc{ sg par } \\
\underline{en}\,+\,čese & \textsc{ sg ill } \\
\underline{en}\,+\,̕n̕ez & \textsc{ sg ine } \\
\underline{en}\,+\,̕n̕ess & \textsc{ sg ela } \\
\underline{en}\,+\,̕n̕elle & \textsc{ sg all } \\
\underline{en}\,+\,̕n̕ell & \textsc{ sg ade } \\
\underline{en}\,+\,̕n̕elt & \textsc{ sg abl } \\
\underline{en}\,+\,̕n̕essi & \textsc{ sg tra } \\
\underline{en}\,+\,̕n̕essaa & \textsc{ sg ter } \\
\underline{en}\,+\,̕n̕eka & \textsc{ sg com } \\
\underline{en}\,+\,̕n̕ed & \textsc{ pl nom } \\
\underline{en}\,+\,čije & \textsc{ pl gen } \\
\underline{en}\,+\,čiit & \textsc{ pl par } \\
\underline{en}\,+\,čiise & \textsc{ pl ill } \\
\underline{en}\,+\,čiiz & \textsc{ pl ine } \\
\underline{en}\,+\,čiiss & \textsc{ pl ela } \\
\underline{en}\,+\,čiille & \textsc{ pl all } \\
\underline{en}\,+\,čiill & \textsc{ pl ade } \\
\underline{en}\,+\,čiilt & \textsc{ pl abl } \\
\underline{en}\,+\,čiissi & \textsc{ pl tra } \\
\underline{en}\,+\,čiissaa & \textsc{ pl ter } \\
\underline{en}\,+\,čijka & \textsc{ pl com } \\
\end{tabular}
\end{sideways}
\captionof{table}{Tüüpsõna \arabic{mallinumber}\,\textit{enči} ekstraheeritud muutvormimallid.}
\label{tab:tüüpsõnamall-enči}

\end{minipage}

 
\vspace{1em}
\noindent Tüüpsõna ei hõlma teisi lekseeme vormi\-sõnastikus.

\paragraph*{\vadja{\underline{sii}pi}}
\vadja{\underline{sii}ve}, \vadja{\underline{sii}peä}, \vadja{\underline{sii}pese}, \vadja{\underline{sii}vess}, \vadja{\underline{sii}ved}, \vadja{\underline{sii}pije}, \vadja{\underline{sii}piit}, \vadja{\underline{sii}piise}, \vadja{\underline{sii}piiss}
 \\
sõnatüüp ei hõlma teisi lekseeme

\paragraph*{\vadja{\underline{kuu}si}}
\vadja{\underline{kuu}zõ}, \vadja{\underline{kuu}ssõ}, \vadja{\underline{kuu}ssõsõ}, \vadja{\underline{kuu}zõss}, \vadja{\underline{kuu}zõd}, \vadja{\underline{kuu}sijõ}, \vadja{\underline{kuu}siit}, \vadja{\underline{kuu}siisõ}, \vadja{\underline{kuu}siiss}
 \\
Sõnatüüp ei hõlma teisi lekseeme vormi\-sõnastikus.

\paragraph*{\vadja{\underline{ta}uti}}
\vadja{\underline{ta}vvõ}, \vadja{\underline{ta}utõa}, \vadja{\underline{ta}utõsõ}, \vadja{\underline{ta}vvõss}, \vadja{\underline{ta}vvõd}, \vadja{\underline{ta}utijõ}, \vadja{\underline{ta}utiit}, \vadja{\underline{ta}utiisõ}, \vadja{\underline{ta}utiiss}
 \\
sõnatüüp ei hõlma teisi lekseeme

\spacing{1.5}


\subsection{\RN{8} käändkond}

Kaheksandasse käändkonda kuuluvad \vadja{-ä}-tüvelised sõnad \cite[46]{ariste_grammar_1968}.

Avatuid küsimusi-tähelepanekuid:
\begin{itemize}
\item 8 käändkond on väga variatiivne tüvevokaali suhtes (eined, leived, čenned aga sepäd,
\item eine (Heinsoo, Konkova ning Rozhanskiy ja Markus) aga einä (VKS)
\item läikkiv on ühtlustatud läikkive
\item Tsvetkovil paljud geminatsioonid puudu (õjja)
\item tegija-liides on eespoolsete sõnade puhul ühtlustatud -jä:-jä:-jä, mitte -je:-jä:-jä, VKSis esineb rohkem -jä Lu/Li/J märgenditega (Konkoval eespoolseid sõnu ei esine)
\item kuigi Tsvetkovil on häälduspäraselt ülesmärgitud 'õmpõlia' ja 'müüjä', on need läbivalt ühtlustatud (lisatud -j- nii \msd{sg nom} kui ka \msd{pl} vormidele)
\item tegija-liides on tagapoolsete sõnade puhul ühtlustatud -ja:-ja:-ja (kuigi Konkoval esineb -jõ:-ja:-ja)
\end{itemize}


\subsubsection*{Ekstraktmorfoloogia sõnatüübid}
\spacing{1}

\vspace{1.8em}
\begin{minipage}{\textwidth}
\stepcounter{mallinumber}
\textbf{Tüüpsõnamall \arabic{mallinumber}\,\vadja{ičä}}\\

\begin{sideways}
\begin{tabular}{l l}
muutvormimall & tunnused \\
\hline
\underline{i}\,+\,čä & \textsc{ sg nom } \\
\underline{i}\,+\,ä & \textsc{ sg gen } \\
\underline{i}\,+\,ččä & \textsc{ sg par } \\
\underline{i}\,+\,ččäse & \textsc{ sg ill } \\
\underline{i}\,+\,äz & \textsc{ sg ine } \\
\underline{i}\,+\,äss & \textsc{ sg ela } \\
\underline{i}\,+\,älle & \textsc{ sg all } \\
\underline{i}\,+\,äll & \textsc{ sg ade } \\
\underline{i}\,+\,ält & \textsc{ sg abl } \\
\underline{i}\,+\,ässi & \textsc{ sg tra } \\
\underline{i}\,+\,ässaa & \textsc{ sg ter } \\
\underline{i}\,+\,äka & \textsc{ sg com } \\
\underline{i}\,+\,äd & \textsc{ pl nom } \\
\underline{i}\,+\,čije & \textsc{ pl gen } \\
\underline{i}\,+\,čiit & \textsc{ pl par } \\
\underline{i}\,+\,čiise & \textsc{ pl ill } \\
\underline{i}\,+\,čiiz & \textsc{ pl ine } \\
\underline{i}\,+\,čiiss & \textsc{ pl ela } \\
\underline{i}\,+\,čiille & \textsc{ pl all } \\
\underline{i}\,+\,čiill & \textsc{ pl ade } \\
\underline{i}\,+\,čiilt & \textsc{ pl abl } \\
\underline{i}\,+\,čiissi & \textsc{ pl tra } \\
\underline{i}\,+\,čiissaa & \textsc{ pl ter } \\
\underline{i}\,+\,čijka & \textsc{ pl com } \\
\end{tabular}
\end{sideways}
\captionof{table}{Tüüpsõna \arabic{mallinumber}\,\textit{ičä} ekstraheeritud muutvormimallid.}
\label{tab:tüüpsõnamall-ičä}

\end{minipage}

 
\vspace{1em}
\noindent Tüüpsõna ei hõlma teisi lekseeme vormi\-sõnastikus.

\paragraph*{\vadja{\underline{sel}če}}
\vadja{\underline{sel}lä}, \vadja{\underline{sel}čä}, \vadja{\underline{sel}čäse}, \vadja{\underline{sel}less}, \vadja{\underline{sel}led}, \vadja{\underline{sel}čije}, \vadja{\underline{sel}čiit}, \vadja{\underline{sel}čiise}, \vadja{\underline{sel}čiiss}
 \\
sõnatüüp ei hõlma teisi lekseeme


\vspace{1.8em}
\begin{minipage}{\textwidth}
\stepcounter{mallinumber}
\textbf{Tüüpsõnamall \arabic{mallinumber}\,\vadja{eine}}\\

\begin{sideways}
\begin{tabular}{l l}
muutvormimall & tunnused \\
\hline
\underline{ein}\,+\,e & \textsc{ sg nom } \\
\underline{ein}\,+\,ä & \textsc{ sg gen } \\
\underline{ein}\,+\,ä & \textsc{ sg par } \\
\underline{ein}\,+\,äse & \textsc{ sg ill } \\
\underline{ein}\,+\,ez & \textsc{ sg ine } \\
\underline{ein}\,+\,ess & \textsc{ sg ela } \\
\underline{ein}\,+\,elle & \textsc{ sg all } \\
\underline{ein}\,+\,ell & \textsc{ sg ade } \\
\underline{ein}\,+\,elt & \textsc{ sg abl } \\
\underline{ein}\,+\,essi & \textsc{ sg tra } \\
\underline{ein}\,+\,essaa & \textsc{ sg ter } \\
\underline{ein}\,+\,eka & \textsc{ sg com } \\
\underline{ein}\,+\,ed & \textsc{ pl nom } \\
\underline{ein}\,+\,ije & \textsc{ pl gen } \\
\underline{ein}\,+\,iit & \textsc{ pl par } \\
\underline{ein}\,+\,iise & \textsc{ pl ill } \\
\underline{ein}\,+\,iiz & \textsc{ pl ine } \\
\underline{ein}\,+\,iiss & \textsc{ pl ela } \\
\underline{ein}\,+\,iille & \textsc{ pl all } \\
\underline{ein}\,+\,iill & \textsc{ pl ade } \\
\underline{ein}\,+\,iilt & \textsc{ pl abl } \\
\underline{ein}\,+\,iissi & \textsc{ pl tra } \\
\underline{ein}\,+\,iissaa & \textsc{ pl ter } \\
\underline{ein}\,+\,ijka & \textsc{ pl com } \\
\end{tabular}
\end{sideways}
\captionof{table}{Tüüpsõna \arabic{mallinumber}\,\textit{eine} ekstraheeritud muutvormimallid.}
\label{tab:tüüpsõnamall-eine}

\end{minipage}

 
\vspace{1em}
\noindent Tüüpsõna hõlmab vormisõnastiku 20 lekseemi: \vadja{\underline{ein}e, \underline{esimespäiv}e, \underline{fökl}e, \underline{irviein}e, \underline{lehm}e, \underline{läikkiv}e, \underline{läsiv}e, \underline{nätil̕päiv}e, \underline{petäj}e, \underline{piim}e, \underline{pominpäiv}e, \underline{pädr}e, \underline{päiv}e, \underline{rehtel}e, \underline{sein}e, \underline{silm}e, \underline{tühj}e, \underline{äjj}e, \underline{ämm}e} ja \vadja{\underline{čülm}e}.

\paragraph*{\vadja{\underline{läkin}e}}
\vadja{\underline{läkin}ä}, \vadja{\underline{läkin}ä}, \vadja{\underline{läkin}äse}, \vadja{\underline{läkin}ess}, \vadja{\underline{läkin}ed}, \vadja{\underline{läkin}eje}, \vadja{\underline{läkin}eit}, \vadja{\underline{läkin}eise}, \vadja{\underline{läkin}eiss}
 \\
sõnatüüp hõlmab lekseeme \vadja{läkine, dääde}

\paragraph*{\vadja{\underline{rissim}ä}}
\vadja{\underline{rissim}ä}, \vadja{\underline{rissim}ä}, \vadja{\underline{rissim}äse}, \vadja{\underline{rissim}äss}, \vadja{\underline{rissim}äd}, \vadja{\underline{rissim}ije}, \vadja{\underline{rissim}it}, \vadja{\underline{rissim}ise}, \vadja{\underline{rissim}iss}
 \\
Tüüpsõna hõlmab vormisõnastiku 2 lekseemi: \vadja{rissimä} ja \vadja{emä}.


\vspace{1.8em}
\begin{minipage}{\textwidth}
\stepcounter{mallinumber}
\textbf{Tüüpsõnamall \arabic{mallinumber}\,\vadja{pähčen}}\\

\begin{sideways}
\begin{tabular}{l l}
muutvormimall & tunnused \\
\hline
\underline{pähčen} & \textsc{ sg nom } \\
\underline{pähčen}\,+\,ä & \textsc{ sg gen } \\
\underline{pähčen}\,+\,ä & \textsc{ sg par } \\
\underline{pähčen}\,+\,äse & \textsc{ sg ill } \\
\underline{pähčen}\,+\,ez & \textsc{ sg ine } \\
\underline{pähčen}\,+\,ess & \textsc{ sg ela } \\
\underline{pähčen}\,+\,elle & \textsc{ sg all } \\
\underline{pähčen}\,+\,ell & \textsc{ sg ade } \\
\underline{pähčen}\,+\,elt & \textsc{ sg abl } \\
\underline{pähčen}\,+\,essi & \textsc{ sg tra } \\
\underline{pähčen}\,+\,essaa & \textsc{ sg ter } \\
\underline{pähčen}\,+\,eka & \textsc{ sg com } \\
\underline{pähčen}\,+\,ed & \textsc{ pl nom } \\
\underline{pähčen}\,+\,ije & \textsc{ pl gen } \\
\underline{pähčen}\,+\,iit & \textsc{ pl par } \\
\underline{pähčen}\,+\,iise & \textsc{ pl ill } \\
\underline{pähčen}\,+\,iiz & \textsc{ pl ine } \\
\underline{pähčen}\,+\,iiss & \textsc{ pl ela } \\
\underline{pähčen}\,+\,iille & \textsc{ pl all } \\
\underline{pähčen}\,+\,iill & \textsc{ pl ade } \\
\underline{pähčen}\,+\,iilt & \textsc{ pl abl } \\
\underline{pähčen}\,+\,iissi & \textsc{ pl tra } \\
\underline{pähčen}\,+\,iissaa & \textsc{ pl ter } \\
\underline{pähčen}\,+\,ijka & \textsc{ pl com } \\
\end{tabular}
\end{sideways}
\captionof{table}{Tüüpsõna \arabic{mallinumber}\,\textit{pähčen} ekstraheeritud muutvormimallid.}
\label{tab:tüüpsõnamall-pähčen}

\end{minipage}

 
\vspace{1em}
\noindent Tüüpsõna hõlmab vormisõnastiku 2 lekseemi: \vadja{\underline{pähčen}} ja \vadja{\underline{ičäv}}.


\vspace{1.8em}
\begin{minipage}{\textwidth}
\stepcounter{mallinumber}
\textbf{Tüüpsõnamall \arabic{mallinumber}\,\vadja{räpäle}}\\

\begin{sideways}
\begin{tabular}{l l}
muutvormimall & tunnused \\
\hline
\underline{räpäl}\,+\,e & \textsc{ sg nom } \\
\underline{räpäl}\,+\,ä & \textsc{ sg gen } \\
\underline{räpäl}\,+\,ä & \textsc{ sg par } \\
\underline{räpäl}\,+\,äse & \textsc{ sg ill } \\
\underline{räpäl}\,+\,ez & \textsc{ sg ine } \\
\underline{räpäl}\,+\,ess & \textsc{ sg ela } \\
\underline{räpäl}\,+\,elle & \textsc{ sg all } \\
\underline{räpäl}\,+\,ell & \textsc{ sg ade } \\
\underline{räpäl}\,+\,elt & \textsc{ sg abl } \\
\underline{räpäl}\,+\,essi & \textsc{ sg tra } \\
\underline{räpäl}\,+\,essaa & \textsc{ sg ter } \\
\underline{räpäl}\,+\,eka & \textsc{ sg com } \\
\underline{räpäl}\,+\,ed & \textsc{ pl nom } \\
\underline{räpäl}\,+\,öje & \textsc{ pl gen } \\
\underline{räpäl}\,+\,öit & \textsc{ pl par } \\
\underline{räpäl}\,+\,öise & \textsc{ pl ill } \\
\underline{räpäl}\,+\,öiz & \textsc{ pl ine } \\
\underline{räpäl}\,+\,öiss & \textsc{ pl ela } \\
\underline{räpäl}\,+\,öille & \textsc{ pl all } \\
\underline{räpäl}\,+\,öill & \textsc{ pl ade } \\
\underline{räpäl}\,+\,öilt & \textsc{ pl abl } \\
\underline{räpäl}\,+\,öissi & \textsc{ pl tra } \\
\underline{räpäl}\,+\,öissaa & \textsc{ pl ter } \\
\underline{räpäl}\,+\,öika & \textsc{ pl com } \\
\end{tabular}
\end{sideways}
\captionof{table}{Tüüpsõna \arabic{mallinumber}\,\textit{räpäle} ekstraheeritud muutvormimallid.}
\label{tab:tüüpsõnamall-räpäle}

\end{minipage}

 
\vspace{1em}
\noindent Tüüpsõna ei hõlma teisi lekseeme vormi\-sõnastikus.

\paragraph*{\vadja{\underline{läsij}ä}}
\vadja{\underline{läsij}ä}, \vadja{\underline{läsij}ä}, \vadja{\underline{läsij}äse}, \vadja{\underline{läsij}ess}, \vadja{\underline{läsij}ed}, \vadja{\underline{läsij}ije}, \vadja{\underline{läsij}iit}, \vadja{\underline{läsij}iise}, \vadja{\underline{läsij}iiss}
 \\
sõnatüüp hõlmab lekseeme \vadja{läsijä, müüjä, tečejä, köühä}


\vspace{1.8em}
\begin{minipage}{\textwidth}
\stepcounter{mallinumber}
\textbf{Tüüpsõnamall \arabic{mallinumber}\,\vadja{kitai}}\\

\begin{sideways}
\begin{tabular}{l l}
muutvormimall & tunnused \\
\hline
\underline{kita}\,+\,i & \textsc{ sg nom } \\
\underline{kita}\,+\,ja & \textsc{ sg gen } \\
\underline{kita}\,+\,ja & \textsc{ sg par } \\
\underline{kita}\,+\,jasõ & \textsc{ sg ill } \\
\underline{kita}\,+\,jaz & \textsc{ sg ine } \\
\underline{kita}\,+\,jass & \textsc{ sg ela } \\
\underline{kita}\,+\,jallõ & \textsc{ sg all } \\
\underline{kita}\,+\,jall & \textsc{ sg ade } \\
\underline{kita}\,+\,jalt & \textsc{ sg abl } \\
\underline{kita}\,+\,jassi & \textsc{ sg tra } \\
\underline{kita}\,+\,jassaa & \textsc{ sg ter } \\
\underline{kita}\,+\,jaka & \textsc{ sg com } \\
\underline{kita}\,+\,jad & \textsc{ pl nom } \\
\underline{kita}\,+\,jojõ & \textsc{ pl gen } \\
\underline{kita}\,+\,joit & \textsc{ pl par } \\
\underline{kita}\,+\,joisõ & \textsc{ pl ill } \\
\underline{kita}\,+\,joiz & \textsc{ pl ine } \\
\underline{kita}\,+\,joiss & \textsc{ pl ela } \\
\underline{kita}\,+\,joillõ & \textsc{ pl all } \\
\underline{kita}\,+\,joill & \textsc{ pl ade } \\
\underline{kita}\,+\,joilt & \textsc{ pl abl } \\
\underline{kita}\,+\,joissi & \textsc{ pl tra } \\
\underline{kita}\,+\,joissaa & \textsc{ pl ter } \\
\underline{kita}\,+\,joika & \textsc{ pl com } \\
\end{tabular}
\end{sideways}
\captionof{table}{Tüüpsõna \arabic{mallinumber}\,\textit{kitai} ekstraheeritud muutvormimallid.}
\label{tab:tüüpsõnamall-kitai}

\end{minipage}

 
\vspace{1em}
\noindent Tüüpsõna ei hõlma teisi lekseeme vormi\-sõnastikus.


\vspace{1.8em}
\begin{minipage}{\textwidth}
\stepcounter{mallinumber}
\textbf{Tüüpsõnamall \arabic{mallinumber}\,\vadja{slona}}\\

\begin{sideways}
\begin{tabular}{l l}
muutvormimall & tunnused \\
\hline
\underline{slon}\,+\,a & \textsc{ sg nom } \\
\underline{slon}\,+\,a & \textsc{ sg gen } \\
\underline{slon}\,+\,a & \textsc{ sg par } \\
\underline{slon}\,+\,asõ & \textsc{ sg ill } \\
\underline{slon}\,+\,az & \textsc{ sg ine } \\
\underline{slon}\,+\,ass & \textsc{ sg ela } \\
\underline{slon}\,+\,allõ & \textsc{ sg all } \\
\underline{slon}\,+\,all & \textsc{ sg ade } \\
\underline{slon}\,+\,alt & \textsc{ sg abl } \\
\underline{slon}\,+\,assi & \textsc{ sg tra } \\
\underline{slon}\,+\,assaa & \textsc{ sg ter } \\
\underline{slon}\,+\,aka & \textsc{ sg com } \\
\underline{slon}\,+\,ad & \textsc{ pl nom } \\
\underline{slon}\,+\,ijõ & \textsc{ pl gen } \\
\underline{slon}\,+\,õit & \textsc{ pl par } \\
\underline{slon}\,+\,õisõ & \textsc{ pl ill } \\
\underline{slon}\,+\,õiz & \textsc{ pl ine } \\
\underline{slon}\,+\,õiss & \textsc{ pl ela } \\
\underline{slon}\,+\,õillõ & \textsc{ pl all } \\
\underline{slon}\,+\,õill & \textsc{ pl ade } \\
\underline{slon}\,+\,õilt & \textsc{ pl abl } \\
\underline{slon}\,+\,õissi & \textsc{ pl tra } \\
\underline{slon}\,+\,õissaa & \textsc{ pl ter } \\
\underline{slon}\,+\,õika & \textsc{ pl com } \\
\end{tabular}
\end{sideways}
\captionof{table}{Tüüpsõna \arabic{mallinumber}\,\textit{slona} ekstraheeritud muutvormimallid.}
\label{tab:tüüpsõnamall-slona}

\end{minipage}

 
\vspace{1em}
\noindent Tüüpsõna ei hõlma teisi lekseeme vormi\-sõnastikus.

\paragraph*{\vadja{\underline{õmpõlij}a}}
\vadja{\underline{õmpõlij}a}, \vadja{\underline{õmpõlij}a}, \vadja{\underline{õmpõlij}asõ}, \vadja{\underline{õmpõlij}ass}, \vadja{\underline{õmpõlij}ad}, \vadja{\underline{õmpõlij}ijõ}, \vadja{\underline{õmpõlij}ait}, \vadja{\underline{õmpõlij}aisõ}, \vadja{\underline{õmpõlij}aiss}
 \\
Tüüpsõna ei hõlma teisi lekseeme vormi\-sõnastikus.

\paragraph*{\vadja{\underline{kompjuter}a}}
\vadja{\underline{kompjuter}a}, \vadja{\underline{kompjuter}a}, \vadja{\underline{kompjuter}asõ}, \vadja{\underline{kompjuter}ass}, \vadja{\underline{kompjuter}ad}, \vadja{\underline{kompjuter}ijõ}, \vadja{\underline{kompjuter}iit}, \vadja{\underline{kompjuter}iisõ}, \vadja{\underline{kompjuter}iiss}
 \\
Sõnatüüp hõlmab vormisõnastiku lekseeme: \vadja{kompjutera, kuja, loba, õmpõja, õpõttõja, ižora}.

\paragraph*{\vadja{\underline{čen}če}}
\vadja{\underline{čen}nä}, \vadja{\underline{čen}čä}, \vadja{\underline{čen}čäse}, \vadja{\underline{čen}ness}, \vadja{\underline{čen}ned}, \vadja{\underline{čen}čije}, \vadja{\underline{čen}čiit}, \vadja{\underline{čen}čiise}, \vadja{\underline{čen}čiiss}
 \\
Sõnatüüp ei hõlma teisi lekseeme vormi\-sõnastikus.


\vspace{1.8em}
\begin{minipage}{\textwidth}
\stepcounter{mallinumber}
\textbf{Tüüpsõnamall \arabic{mallinumber}\,\vadja{tünke}}\\

\begin{sideways}
\begin{tabular}{l l}
muutvormimall & tunnused \\
\hline
\underline{tün}\,+\,ke & \textsc{ sg nom } \\
\underline{tün}\,+\,gä & \textsc{ sg gen } \\
\underline{tün}\,+\,kä & \textsc{ sg par } \\
\underline{tün}\,+\,käse & \textsc{ sg ill } \\
\underline{tün}\,+\,gez & \textsc{ sg ine } \\
\underline{tün}\,+\,gess & \textsc{ sg ela } \\
\underline{tün}\,+\,gelle & \textsc{ sg all } \\
\underline{tün}\,+\,gell & \textsc{ sg ade } \\
\underline{tün}\,+\,gelt & \textsc{ sg abl } \\
\underline{tün}\,+\,gessi & \textsc{ sg tra } \\
\underline{tün}\,+\,gessaa & \textsc{ sg ter } \\
\underline{tün}\,+\,geka & \textsc{ sg com } \\
\underline{tün}\,+\,ged & \textsc{ pl nom } \\
\underline{tün}\,+\,kije & \textsc{ pl gen } \\
\underline{tün}\,+\,kiit & \textsc{ pl par } \\
\underline{tün}\,+\,kiise & \textsc{ pl ill } \\
\underline{tün}\,+\,kiiz & \textsc{ pl ine } \\
\underline{tün}\,+\,kiiss & \textsc{ pl ela } \\
\underline{tün}\,+\,kiille & \textsc{ pl all } \\
\underline{tün}\,+\,kiill & \textsc{ pl ade } \\
\underline{tün}\,+\,kiilt & \textsc{ pl abl } \\
\underline{tün}\,+\,kiissi & \textsc{ pl tra } \\
\underline{tün}\,+\,kiissaa & \textsc{ pl ter } \\
\underline{tün}\,+\,kijka & \textsc{ pl com } \\
\end{tabular}
\end{sideways}
\captionof{table}{Tüüpsõna \arabic{mallinumber}\,\textit{tünke} ekstraheeritud muutvormimallid.}
\label{tab:tüüpsõnamall-tünke}

\end{minipage}

 
\vspace{1em}
\noindent Tüüpsõna ei hõlma teisi lekseeme vormi\-sõnastikus.


\vspace{1.8em}
\begin{minipage}{\textwidth}
\stepcounter{mallinumber}
\textbf{Tüüpsõnamall \arabic{mallinumber}\,\vadja{änte}}\\

\begin{sideways}
\begin{tabular}{l l}
muutvormimall & tunnused \\
\hline
\underline{än}\,+\,te & \textsc{ sg nom } \\
\underline{än}\,+\,nä & \textsc{ sg gen } \\
\underline{än}\,+\,tä & \textsc{ sg par } \\
\underline{än}\,+\,täse & \textsc{ sg ill } \\
\underline{än}\,+\,nez & \textsc{ sg ine } \\
\underline{än}\,+\,ness & \textsc{ sg ela } \\
\underline{än}\,+\,nelle & \textsc{ sg all } \\
\underline{än}\,+\,nell & \textsc{ sg ade } \\
\underline{än}\,+\,nelt & \textsc{ sg abl } \\
\underline{än}\,+\,nessi & \textsc{ sg tra } \\
\underline{än}\,+\,nessaa & \textsc{ sg ter } \\
\underline{än}\,+\,neka & \textsc{ sg com } \\
\underline{än}\,+\,ned & \textsc{ pl nom } \\
\underline{än}\,+\,tije & \textsc{ pl gen } \\
\underline{än}\,+\,tiit & \textsc{ pl par } \\
\underline{än}\,+\,tiise & \textsc{ pl ill } \\
\underline{än}\,+\,tiiz & \textsc{ pl ine } \\
\underline{än}\,+\,tiiss & \textsc{ pl ela } \\
\underline{än}\,+\,tiille & \textsc{ pl all } \\
\underline{än}\,+\,tiill & \textsc{ pl ade } \\
\underline{än}\,+\,tiilt & \textsc{ pl abl } \\
\underline{än}\,+\,tiissi & \textsc{ pl tra } \\
\underline{än}\,+\,tiissaa & \textsc{ pl ter } \\
\underline{än}\,+\,tijka & \textsc{ pl com } \\
\end{tabular}
\end{sideways}
\captionof{table}{Tüüpsõna \arabic{mallinumber}\,\textit{änte} ekstraheeritud muutvormimallid.}
\label{tab:tüüpsõnamall-änte}

\end{minipage}

 
\vspace{1em}
\noindent Tüüpsõna ei hõlma teisi lekseeme vormi\-sõnastikus.

\paragraph*{\vadja{\underline{lei}pe}}
\vadja{\underline{lei}vä}, \vadja{\underline{lei}pä}, \vadja{\underline{lei}päse}, \vadja{\underline{lei}vess}, \vadja{\underline{lei}ved}, \vadja{\underline{lei}pije}, \vadja{\underline{lei}piit}, \vadja{\underline{lei}piise}, \vadja{\underline{lei}piiss}
 \\
Sõnatüüp ei hõlma teisi lekseeme vormi\-sõnastikus.

\paragraph*{\vadja{\underline{sep}pe}}
\vadja{\underline{sep}ä}, \vadja{\underline{sep}pä}, \vadja{\underline{sep}päse}, \vadja{\underline{sep}äss}, \vadja{\underline{sep}äd}, \vadja{\underline{sep}pije}, \vadja{\underline{sep}piit}, \vadja{\underline{sep}piise}, \vadja{\underline{sep}piiss}
 \\
sõnatüüp hõlmab lekseeme \vadja{seppe, leppe}

\paragraph*{\vadja{\underline{är}če}}
\vadja{\underline{är}jä}, \vadja{\underline{är}čä}, \vadja{\underline{är}čäse}, \vadja{\underline{är}jess}, \vadja{\underline{är}jed}, \vadja{\underline{är}čije}, \vadja{\underline{är}čiit}, \vadja{\underline{är}čiise}, \vadja{\underline{är}čiiss}
 \\
Tüüpsõna hõlmab vormisõnastiku lekseeme \vadja{ärče, märče}.

\paragraph*{\vadja{\underline{pe}sä}}
\vadja{\underline{pe}zä}, \vadja{\underline{pe}ssä}, \vadja{\underline{pe}ssäse}, \vadja{\underline{pe}zäss}, \vadja{\underline{pe}zäd}, \vadja{\underline{pe}sije}, \vadja{\underline{pe}siit}, \vadja{\underline{pe}siise}, \vadja{\underline{pe}siiss}
 \\
Tüüpsõna hõlmab vormisõnastiku lekseeme: \vadja{pesä, rissisä, česä}.


\vspace{1.8em}
\begin{minipage}{\textwidth}
\stepcounter{mallinumber}
\textbf{Tüüpsõnamall \arabic{mallinumber}\,\vadja{lisä}}\\

\begin{sideways}
\begin{tabular}{l l}
muutvormimall & tunnused \\
\hline
\underline{li}\,+\,s\,+\,\underline{ä} & \textsc{ sg nom } \\
\underline{li}\,+\,z\,+\,\underline{ä} & \textsc{ sg gen } \\
\underline{li}\,+\,ss\,+\,\underline{ä} & \textsc{ sg par } \\
\underline{li}\,+\,ss\,+\,\underline{ä}\,+\,se & \textsc{ sg ill } \\
\underline{li}\,+\,z\,+\,\underline{ä}\,+\,z & \textsc{ sg ine } \\
\underline{li}\,+\,z\,+\,\underline{ä}\,+\,ss & \textsc{ sg ela } \\
\underline{li}\,+\,z\,+\,\underline{ä}\,+\,lle & \textsc{ sg all } \\
\underline{li}\,+\,z\,+\,\underline{ä}\,+\,ll & \textsc{ sg ade } \\
\underline{li}\,+\,z\,+\,\underline{ä}\,+\,lt & \textsc{ sg abl } \\
\underline{li}\,+\,z\,+\,\underline{ä}\,+\,ssi & \textsc{ sg tra } \\
\underline{li}\,+\,z\,+\,\underline{ä}\,+\,ssaa & \textsc{ sg ter } \\
\underline{li}\,+\,z\,+\,\underline{ä}\,+\,ka & \textsc{ sg com } \\
\underline{li}\,+\,z\,+\,\underline{ä}\,+\,d & \textsc{ pl nom } \\
\underline{li}\,+\,ss\,+\,\underline{ä}\,+\,ije & \textsc{ pl gen } \\
\underline{li}\,+\,ss\,+\,\underline{ä}\,+\,it & \textsc{ pl par } \\
\underline{li}\,+\,ss\,+\,\underline{ä}\,+\,ise & \textsc{ pl ill } \\
\underline{li}\,+\,ss\,+\,\underline{ä}\,+\,iz & \textsc{ pl ine } \\
\underline{li}\,+\,ss\,+\,\underline{ä}\,+\,iss & \textsc{ pl ela } \\
\underline{li}\,+\,ss\,+\,\underline{ä}\,+\,ille & \textsc{ pl all } \\
\underline{li}\,+\,ss\,+\,\underline{ä}\,+\,ill & \textsc{ pl ade } \\
\underline{li}\,+\,ss\,+\,\underline{ä}\,+\,ilt & \textsc{ pl abl } \\
\underline{li}\,+\,ss\,+\,\underline{ä}\,+\,issi & \textsc{ pl tra } \\
\underline{li}\,+\,ss\,+\,\underline{ä}\,+\,issaa & \textsc{ pl ter } \\
\underline{li}\,+\,ss\,+\,\underline{ä}\,+\,ika & \textsc{ pl com } \\
\end{tabular}
\end{sideways}
\captionof{table}{Tüüpsõna \arabic{mallinumber}\,\textit{lisä} ekstraheeritud muutvormimallid.}
\label{tab:tüüpsõnamall-lisä}

\end{minipage}

 
\vspace{1em}
\noindent Tüüpsõna ei hõlma teisi lekseeme vormi\-sõnastikus.


\vspace{1.8em}
\begin{minipage}{\textwidth}
\stepcounter{mallinumber}
\textbf{Tüüpsõnamall \arabic{mallinumber}\,\vadja{mätä}}\\

\begin{sideways}
\begin{tabular}{l l}
muutvormimall & tunnused \\
\hline
\underline{mä}\,+\,tä & \textsc{ sg nom } \\
\underline{mä}\,+\,ä & \textsc{ sg gen } \\
\underline{mä}\,+\,ttä & \textsc{ sg par } \\
\underline{mä}\,+\,ttäse & \textsc{ sg ill } \\
\underline{mä}\,+\,äz & \textsc{ sg ine } \\
\underline{mä}\,+\,äss & \textsc{ sg ela } \\
\underline{mä}\,+\,älle & \textsc{ sg all } \\
\underline{mä}\,+\,äll & \textsc{ sg ade } \\
\underline{mä}\,+\,ält & \textsc{ sg abl } \\
\underline{mä}\,+\,ässi & \textsc{ sg tra } \\
\underline{mä}\,+\,ässaa & \textsc{ sg ter } \\
\underline{mä}\,+\,äka & \textsc{ sg com } \\
\underline{mä}\,+\,äd & \textsc{ pl nom } \\
\underline{mä}\,+\,ttije & \textsc{ pl gen } \\
\underline{mä}\,+\,ttiit & \textsc{ pl par } \\
\underline{mä}\,+\,ttiise & \textsc{ pl ill } \\
\underline{mä}\,+\,ttiiz & \textsc{ pl ine } \\
\underline{mä}\,+\,ttiiss & \textsc{ pl ela } \\
\underline{mä}\,+\,ttiille & \textsc{ pl all } \\
\underline{mä}\,+\,ttiill & \textsc{ pl ade } \\
\underline{mä}\,+\,ttiilt & \textsc{ pl abl } \\
\underline{mä}\,+\,ttiissi & \textsc{ pl tra } \\
\underline{mä}\,+\,ttiissaa & \textsc{ pl ter } \\
\underline{mä}\,+\,ttijka & \textsc{ pl com } \\
\end{tabular}
\end{sideways}
\captionof{table}{Tüüpsõna \arabic{mallinumber}\,\textit{mätä} ekstraheeritud muutvormimallid.}
\label{tab:tüüpsõnamall-mätä}

\end{minipage}

 
\vspace{1em}
\noindent Tüüpsõna ei hõlma teisi lekseeme vormi\-sõnastikus.

\paragraph*{\vadja{\underline{met}t\underline{s}e}}
\vadja{\underline{met}\underline{s}ä}, \vadja{\underline{met}t\underline{s}ä}, \vadja{\underline{met}t\underline{s}äse}, \vadja{\underline{met}\underline{s}äss}, \vadja{\underline{met}\underline{s}äd}, \vadja{\underline{met}t\underline{s}ije}, \vadja{\underline{met}t\underline{s}iit}, \vadja{\underline{met}t\underline{s}iise}, \vadja{\underline{met}t\underline{s}iiss}
 \\
Sõnatüüp ei hõlma teisi lekseeme vormi\-sõnastikus.


\vspace{1.8em}
\begin{minipage}{\textwidth}
\stepcounter{mallinumber}
\textbf{Tüüpsõnamall \arabic{mallinumber}\,\vadja{nenä}}\\

\begin{sideways}
\begin{tabular}{l l}
muutvormimall & tunnused \\
\hline
\underline{nen}\,+\,ä & \textsc{ sg nom } \\
\underline{nen}\,+\,ä & \textsc{ sg gen } \\
\underline{nen}\,+\,nä & \textsc{ sg par } \\
\underline{nen}\,+\,äse & \textsc{ sg ill } \\
\underline{nen}\,+\,äz & \textsc{ sg ine } \\
\underline{nen}\,+\,äss & \textsc{ sg ela } \\
\underline{nen}\,+\,älle & \textsc{ sg all } \\
\underline{nen}\,+\,äll & \textsc{ sg ade } \\
\underline{nen}\,+\,ält & \textsc{ sg abl } \\
\underline{nen}\,+\,ässi & \textsc{ sg tra } \\
\underline{nen}\,+\,ässaa & \textsc{ sg ter } \\
\underline{nen}\,+\,äka & \textsc{ sg com } \\
\underline{nen}\,+\,äd & \textsc{ pl nom } \\
\underline{nen}\,+\,ije & \textsc{ pl gen } \\
\underline{nen}\,+\,iit & \textsc{ pl par } \\
\underline{nen}\,+\,iise & \textsc{ pl ill } \\
\underline{nen}\,+\,iiz & \textsc{ pl ine } \\
\underline{nen}\,+\,iiss & \textsc{ pl ela } \\
\underline{nen}\,+\,iille & \textsc{ pl all } \\
\underline{nen}\,+\,iill & \textsc{ pl ade } \\
\underline{nen}\,+\,iilt & \textsc{ pl abl } \\
\underline{nen}\,+\,iissi & \textsc{ pl tra } \\
\underline{nen}\,+\,iissaa & \textsc{ pl ter } \\
\underline{nen}\,+\,ijka & \textsc{ pl com } \\
\end{tabular}
\end{sideways}
\captionof{table}{Tüüpsõna \arabic{mallinumber}\,\textit{nenä} ekstraheeritud muutvormimallid.}
\label{tab:tüüpsõnamall-nenä}

\end{minipage}

 
\vspace{1em}
\noindent Tüüpsõna ei hõlma teisi lekseeme vormi\-sõnastikus.


\vspace{1.8em}
\begin{minipage}{\textwidth}
\stepcounter{mallinumber}
\textbf{Tüüpsõnamall \arabic{mallinumber}\,\vadja{čülä}}\\

\begin{sideways}
\begin{tabular}{l l}
muutvormimall & tunnused \\
\hline
\underline{čül}\,+\,ä & \textsc{ sg nom } \\
\underline{čül}\,+\,ä & \textsc{ sg gen } \\
\underline{čül}\,+\,lä & \textsc{ sg par } \\
\underline{čül}\,+\,äse & \textsc{ sg ill } \\
\underline{čül}\,+\,äz & \textsc{ sg ine } \\
\underline{čül}\,+\,äss & \textsc{ sg ela } \\
\underline{čül}\,+\,älle & \textsc{ sg all } \\
\underline{čül}\,+\,äll & \textsc{ sg ade } \\
\underline{čül}\,+\,ält & \textsc{ sg abl } \\
\underline{čül}\,+\,ässi & \textsc{ sg tra } \\
\underline{čül}\,+\,ässaa & \textsc{ sg ter } \\
\underline{čül}\,+\,äka & \textsc{ sg com } \\
\underline{čül}\,+\,äd & \textsc{ pl nom } \\
\underline{čül}\,+\,ije & \textsc{ pl gen } \\
\underline{čül}\,+\,iit & \textsc{ pl par } \\
\underline{čül}\,+\,iise & \textsc{ pl ill } \\
\underline{čül}\,+\,iiz & \textsc{ pl ine } \\
\underline{čül}\,+\,iiss & \textsc{ pl ela } \\
\underline{čül}\,+\,iille & \textsc{ pl all } \\
\underline{čül}\,+\,iill & \textsc{ pl ade } \\
\underline{čül}\,+\,iilt & \textsc{ pl abl } \\
\underline{čül}\,+\,iissi & \textsc{ pl tra } \\
\underline{čül}\,+\,iissaa & \textsc{ pl ter } \\
\underline{čül}\,+\,ijka & \textsc{ pl com } \\
\end{tabular}
\end{sideways}
\captionof{table}{Tüüpsõna \arabic{mallinumber}\,\textit{čülä} ekstraheeritud muutvormimallid.}
\label{tab:tüüpsõnamall-čülä}

\end{minipage}

 
\vspace{1em}
\noindent Tüüpsõna ei hõlma teisi lekseeme vormi\-sõnastikus.

\paragraph*{\vadja{\underline{püh}ä}}
\vadja{\underline{püh}ä}, \vadja{\underline{püh}hä}, \vadja{\underline{püh}häse}, \vadja{\underline{püh}äss}, \vadja{\underline{püh}äd}, \vadja{\underline{püh}hije}, \vadja{\underline{püh}hiit}, \vadja{\underline{püh}hiise}, \vadja{\underline{püh}hiiss}
 \\
sõnatüüp ei hõlma teisi lekseeme


\vspace{1.8em}
\begin{minipage}{\textwidth}
\stepcounter{mallinumber}
\textbf{Tüüpsõnamall \arabic{mallinumber}\,\vadja{üvä}}\\

\begin{sideways}
\begin{tabular}{l l}
muutvormimall & tunnused \\
\hline
\underline{üv}\,+\,ä & \textsc{ sg nom } \\
\underline{üv}\,+\,ä & \textsc{ sg gen } \\
\underline{üv}\,+\,vä & \textsc{ sg par } \\
\underline{üv}\,+\,väse & \textsc{ sg ill } \\
\underline{üv}\,+\,äz & \textsc{ sg ine } \\
\underline{üv}\,+\,äss & \textsc{ sg ela } \\
\underline{üv}\,+\,älle & \textsc{ sg all } \\
\underline{üv}\,+\,äll & \textsc{ sg ade } \\
\underline{üv}\,+\,ält & \textsc{ sg abl } \\
\underline{üv}\,+\,ässi & \textsc{ sg tra } \\
\underline{üv}\,+\,ässaa & \textsc{ sg ter } \\
\underline{üv}\,+\,äka & \textsc{ sg com } \\
\underline{üv}\,+\,äd & \textsc{ pl nom } \\
\underline{üv}\,+\,vije & \textsc{ pl gen } \\
\underline{üv}\,+\,viit & \textsc{ pl par } \\
\underline{üv}\,+\,viise & \textsc{ pl ill } \\
\underline{üv}\,+\,viiz & \textsc{ pl ine } \\
\underline{üv}\,+\,viiss & \textsc{ pl ela } \\
\underline{üv}\,+\,viille & \textsc{ pl all } \\
\underline{üv}\,+\,viill & \textsc{ pl ade } \\
\underline{üv}\,+\,viilt & \textsc{ pl abl } \\
\underline{üv}\,+\,viissi & \textsc{ pl tra } \\
\underline{üv}\,+\,viissaa & \textsc{ pl ter } \\
\underline{üv}\,+\,vijka & \textsc{ pl com } \\
\end{tabular}
\end{sideways}
\captionof{table}{Tüüpsõna \arabic{mallinumber}\,\textit{üvä} ekstraheeritud muutvormimallid.}
\label{tab:tüüpsõnamall-üvä}

\end{minipage}

 
\vspace{1em}
\noindent Tüüpsõna ei hõlma teisi lekseeme vormi\-sõnastikus.


\vspace{1.8em}
\begin{minipage}{\textwidth}
\stepcounter{mallinumber}
\textbf{Tüüpsõnamall \arabic{mallinumber}\,\vadja{õja}}\\

\begin{sideways}
\begin{tabular}{l l}
muutvormimall & tunnused \\
\hline
\underline{õj}\,+\,a & \textsc{ sg nom } \\
\underline{õj}\,+\,a & \textsc{ sg gen } \\
\underline{õj}\,+\,ja & \textsc{ sg par } \\
\underline{õj}\,+\,asõ & \textsc{ sg ill } \\
\underline{õj}\,+\,az & \textsc{ sg ine } \\
\underline{õj}\,+\,ass & \textsc{ sg ela } \\
\underline{õj}\,+\,allõ & \textsc{ sg all } \\
\underline{õj}\,+\,all & \textsc{ sg ade } \\
\underline{õj}\,+\,alt & \textsc{ sg abl } \\
\underline{õj}\,+\,assi & \textsc{ sg tra } \\
\underline{õj}\,+\,assaa & \textsc{ sg ter } \\
\underline{õj}\,+\,aka & \textsc{ sg com } \\
\underline{õj}\,+\,ad & \textsc{ pl nom } \\
\underline{õj}\,+\,ijõ & \textsc{ pl gen } \\
\underline{õj}\,+\,it & \textsc{ pl par } \\
\underline{õj}\,+\,isõ & \textsc{ pl ill } \\
\underline{õj}\,+\,iz & \textsc{ pl ine } \\
\underline{õj}\,+\,iss & \textsc{ pl ela } \\
\underline{õj}\,+\,illõ & \textsc{ pl all } \\
\underline{õj}\,+\,ill & \textsc{ pl ade } \\
\underline{õj}\,+\,ilt & \textsc{ pl abl } \\
\underline{õj}\,+\,issi & \textsc{ pl tra } \\
\underline{õj}\,+\,issaa & \textsc{ pl ter } \\
\underline{õj}\,+\,ika & \textsc{ pl com } \\
\end{tabular}
\end{sideways}
\captionof{table}{Tüüpsõna \arabic{mallinumber}\,\textit{õja} ekstraheeritud muutvormimallid.}
\label{tab:tüüpsõnamall-õja}

\end{minipage}

 
\vspace{1em}
\noindent Tüüpsõna ei hõlma teisi lekseeme vormi\-sõnastikus.


\vspace{1.8em}
\begin{minipage}{\textwidth}
\stepcounter{mallinumber}
\textbf{Tüüpsõnamall \arabic{mallinumber}\,\vadja{õma}}\\

\begin{sideways}
\begin{tabular}{l l}
muutvormimall & tunnused \\
\hline
\underline{õm}\,+\,a & \textsc{ sg nom } \\
\underline{õm}\,+\,a & \textsc{ sg gen } \\
\underline{õm}\,+\,ma & \textsc{ sg par } \\
\underline{õm}\,+\,masõ & \textsc{ sg ill } \\
\underline{õm}\,+\,õz & \textsc{ sg ine } \\
\underline{õm}\,+\,õss & \textsc{ sg ela } \\
\underline{õm}\,+\,õllõ & \textsc{ sg all } \\
\underline{õm}\,+\,õll & \textsc{ sg ade } \\
\underline{õm}\,+\,õlt & \textsc{ sg abl } \\
\underline{õm}\,+\,õssi & \textsc{ sg tra } \\
\underline{õm}\,+\,õssaa & \textsc{ sg ter } \\
\underline{õm}\,+\,õka & \textsc{ sg com } \\
\underline{õm}\,+\,õd & \textsc{ pl nom } \\
\underline{õm}\,+\,mijõ & \textsc{ pl gen } \\
\underline{õm}\,+\,miit & \textsc{ pl par } \\
\underline{õm}\,+\,miisõ & \textsc{ pl ill } \\
\underline{õm}\,+\,miiz & \textsc{ pl ine } \\
\underline{õm}\,+\,miiss & \textsc{ pl ela } \\
\underline{õm}\,+\,miillõ & \textsc{ pl all } \\
\underline{õm}\,+\,miill & \textsc{ pl ade } \\
\underline{õm}\,+\,miilt & \textsc{ pl abl } \\
\underline{õm}\,+\,miissi & \textsc{ pl tra } \\
\underline{õm}\,+\,miissaa & \textsc{ pl ter } \\
\underline{õm}\,+\,mijka & \textsc{ pl com } \\
\end{tabular}
\end{sideways}
\captionof{table}{Tüüpsõna \arabic{mallinumber}\,\textit{õma} ekstraheeritud muutvormimallid.}
\label{tab:tüüpsõnamall-õma}

\end{minipage}

 
\vspace{1em}
\noindent Tüüpsõna ei hõlma teisi lekseeme vormi\-sõnastikus.

\paragraph*{\vadja{\underline{kõv}a}}
\vadja{\underline{kõv}a}, \vadja{\underline{kõv}at}, \vadja{\underline{kõv}asõ}, \vadja{\underline{kõv}ass}, \vadja{\underline{kõv}ad}, \vadja{\underline{kõv}ijõ}, \vadja{\underline{kõv}iit}, \vadja{\underline{kõv}iisõ}, \vadja{\underline{kõv}iiss}
 \\
Tüüpsõna ei hõlma teisi lekseeme vormi\-sõnastikus.

\spacing{1.5}



\subsection{\RN{9} käändkond}
See käändkond on spetsiifiline Kattila murdele ja seda ei käsitleta siin töös.


\subsection{\RN{10} käändkond}

Kümnendasse käändkonda koondub suur osa kahetüvelisi sõnu, mille \msd{sg nom} lõpp on \vadja{-i}, ent mille tüvevokaal on \vadja{-e/-õ}. Ariste märgib, et \msd{sp par} on mitu erinevat realisatsiooni, kuigi nende moodustamis\-viis põhimõtteliselt järgib sama malli. \cite[47]{ariste_grammar_1968}.

Avatuid küsimusi-tähelepanekuid:
\begin{itemize}
\item 10 käändkond Ariste sõnul on sg par väga variatiivne
\item ühtlustatud on -i:-õ:-tõ lõpuvokaalid
\item kuigi voosi:voovvõ hääldub vuuvvõ on see märgitud voovvõ
\end{itemize}

\subsubsection*{Ekstraktmorfoloogia sõnatüübid}
\spacing{1}

\vspace{1.8em}
\begin{minipage}{\textwidth}
\stepcounter{mallinumber}
\textbf{Tüüpsõnamall \arabic{mallinumber}\,\vadja{lumi}}\\

\begin{sideways}
\begin{tabular}{l l}
muutvormimall & tunnused \\
\hline
\underline{lu}\,+\,mi & \textsc{ sg nom } \\
\underline{lu}\,+\,mõ & \textsc{ sg gen } \\
\underline{lu}\,+\,ntõ & \textsc{ sg par } \\
\underline{lu}\,+\,mõsõ & \textsc{ sg ill } \\
\underline{lu}\,+\,mõz & \textsc{ sg ine } \\
\underline{lu}\,+\,mõss & \textsc{ sg ela } \\
\underline{lu}\,+\,mõllõ & \textsc{ sg all } \\
\underline{lu}\,+\,mõll & \textsc{ sg ade } \\
\underline{lu}\,+\,mõlt & \textsc{ sg abl } \\
\underline{lu}\,+\,mõssi & \textsc{ sg tra } \\
\underline{lu}\,+\,mõssaa & \textsc{ sg ter } \\
\underline{lu}\,+\,mõka & \textsc{ sg com } \\
\underline{lu}\,+\,mõd & \textsc{ pl nom } \\
\underline{lu}\,+\,mijõ & \textsc{ pl gen } \\
\underline{lu}\,+\,miit & \textsc{ pl par } \\
\underline{lu}\,+\,miisõ & \textsc{ pl ill } \\
\underline{lu}\,+\,miiz & \textsc{ pl ine } \\
\underline{lu}\,+\,miiss & \textsc{ pl ela } \\
\underline{lu}\,+\,miillõ & \textsc{ pl all } \\
\underline{lu}\,+\,miill & \textsc{ pl ade } \\
\underline{lu}\,+\,miilt & \textsc{ pl abl } \\
\underline{lu}\,+\,miissi & \textsc{ pl tra } \\
\underline{lu}\,+\,miissaa & \textsc{ pl ter } \\
\underline{lu}\,+\,mijka & \textsc{ pl com } \\
\end{tabular}
\end{sideways}
\captionof{table}{Tüüpsõna \arabic{mallinumber}\,\textit{lumi} ekstraheeritud muutvormimallid.}
\label{tab:tüüpsõnamall-lumi}

\end{minipage}

 
\vspace{1em}
\noindent Tüüpsõna ei hõlma teisi lekseeme vormi\-sõnastikus.

\paragraph*{\vadja{\underline{iir}i}}
\vadja{\underline{iir}e}, \vadja{\underline{iir}te}, \vadja{\underline{iir}ese}, \vadja{\underline{iir}ess}, \vadja{\underline{iir}ed}, \vadja{\underline{iir}ije}, \vadja{\underline{iir}iit}, \vadja{\underline{iir}iise}, \vadja{\underline{iir}iiss}
 \\
Tüüpsõna hõlmab vormisõnastiku 9 lekseemi: \vadja{iiri, meeli, meri, peeni, süli, veri, ääni, ääri} ja \vadja{čeeli}.


\vspace{1.8em}
\begin{minipage}{\textwidth}
\stepcounter{mallinumber}
\textbf{Tüüpsõnamall \arabic{mallinumber}\,\vadja{juuri}}\\

\begin{sideways}
\begin{tabular}{l l}
muutvormimall & tunnused \\
\hline
\underline{juur}\,+\,i & \textsc{ sg nom } \\
\underline{juur}\,+\,õ & \textsc{ sg gen } \\
\underline{juur}\,+\,tõ & \textsc{ sg par } \\
\underline{juur}\,+\,õsõ & \textsc{ sg ill } \\
\underline{juur}\,+\,õz & \textsc{ sg ine } \\
\underline{juur}\,+\,õss & \textsc{ sg ela } \\
\underline{juur}\,+\,õllõ & \textsc{ sg all } \\
\underline{juur}\,+\,õll & \textsc{ sg ade } \\
\underline{juur}\,+\,õlt & \textsc{ sg abl } \\
\underline{juur}\,+\,õssi & \textsc{ sg tra } \\
\underline{juur}\,+\,õssaa & \textsc{ sg ter } \\
\underline{juur}\,+\,õka & \textsc{ sg com } \\
\underline{juur}\,+\,õd & \textsc{ pl nom } \\
\underline{juur}\,+\,ijõ & \textsc{ pl gen } \\
\underline{juur}\,+\,iit & \textsc{ pl par } \\
\underline{juur}\,+\,iisõ & \textsc{ pl ill } \\
\underline{juur}\,+\,iiz & \textsc{ pl ine } \\
\underline{juur}\,+\,iiss & \textsc{ pl ela } \\
\underline{juur}\,+\,iillõ & \textsc{ pl all } \\
\underline{juur}\,+\,iill & \textsc{ pl ade } \\
\underline{juur}\,+\,iilt & \textsc{ pl abl } \\
\underline{juur}\,+\,iissi & \textsc{ pl tra } \\
\underline{juur}\,+\,iissaa & \textsc{ pl ter } \\
\underline{juur}\,+\,ijka & \textsc{ pl com } \\
\end{tabular}
\end{sideways}
\captionof{table}{Tüüpsõna \arabic{mallinumber}\,\textit{juuri} ekstraheeritud muutvormimallid.}
\label{tab:tüüpsõnamall-juuri}

\end{minipage}

 
\vspace{1em}
\noindent Tüüpsõna hõlmab vormisõnastiku 13 lekseemi: \vadja{\underline{juur}i, \underline{kaan}i, \underline{koor}i, \underline{lõh}i, \underline{noor}i, \underline{ool}i, \underline{pool}i, \underline{sool}i, \underline{suur}i, \underline{tul}i, \underline{tuul}i, \underline{un}i} ja \vadja{\underline{hool}i}.

\paragraph*{\vadja{\underline{u}hsi}}
\vadja{\underline{u}hzõ}, \vadja{\underline{u}ssõ}, \vadja{\underline{u}hsõsõ}, \vadja{\underline{u}hzõss}, \vadja{\underline{u}hzõd}, \vadja{\underline{u}hsijõ}, \vadja{\underline{u}hsiit}, \vadja{\underline{u}hsiisõ}, \vadja{\underline{u}hsiiss}
 \\
Tüüpsõna hõlmab vormisõnastiku 2 lekseemi: \vadja{uhsi} ja \vadja{lahsi}.

\paragraph*{\vadja{\underline{ku}si}}
\vadja{\underline{ku}zõ}, \vadja{\underline{ku}ssõ}, \vadja{\underline{ku}ssõ}, \vadja{\underline{ku}zõss}, \vadja{\underline{ku}zõd}, \vadja{\underline{ku}ssijõ}, \vadja{\underline{ku}ssiit}, \vadja{\underline{ku}ssiisõ}, \vadja{\underline{ku}ssiiss}
 \\
sõnatüüp ei hõlma teisi lekseeme

\paragraph*{\vadja{\underline{uu}si}}
\vadja{\underline{uu}vvõ}, \vadja{\underline{uu}ttõ}, \vadja{\underline{uu}vvõsõ}, \vadja{\underline{uu}vvõss}, \vadja{\underline{uu}vvõd}, \vadja{\underline{uu}sijõ}, \vadja{\underline{uu}siit}, \vadja{\underline{uu}siisõ}, \vadja{\underline{uu}siiss}
 \\
sõnatüüp hõlmab lekseeme \vadja{uusi, voosi, kuusi}

\paragraph*{\vadja{\underline{üh}s}}
\vadja{\underline{üh}e}, \vadja{\underline{üh}te}, \vadja{\underline{üh}tese}, \vadja{\underline{üh}ess}, \vadja{\underline{üh}ed}, \vadja{\underline{üh}sije}, \vadja{\underline{üh}siit}, \vadja{\underline{üh}siise}, \vadja{\underline{üh}siiss}
 \\
Sõnatüüp ei hõlma teisi lekseeme vormi\-sõnastikus.

\paragraph*{\vadja{\underline{ül}či}}
\vadja{\underline{ül}le}, \vadja{\underline{ül}čiä}, \vadja{\underline{ül}čese}, \vadja{\underline{ül}less}, \vadja{\underline{ül}led}, \vadja{\underline{ül}čije}, \vadja{\underline{ül}čiit}, \vadja{\underline{ül}čiise}, \vadja{\underline{ül}čiiss}
 \\
sõnatüüp hõlmab lekseeme \vadja{ülči, jälči}

Sõnatüübi\-mall kirjeldab tagapoolseid sõnu tüvemuutusega .
\paragraph*{\vadja{\underline{kah}s}}
\vadja{\underline{kah}õ}, \vadja{\underline{kah}tõ}, \vadja{\underline{kah}tõ}, \vadja{\underline{kah}õss}, \vadja{\underline{kah}õd}, \vadja{\underline{kah}sijõ}, \vadja{\underline{kah}siit}, \vadja{\underline{kah}siisõ}, \vadja{\underline{kah}siiss}
 \\
Tüüpsõna ei hõlma teisi lekseeme vormi\-sõnastikus.


\vspace{1.8em}
\begin{minipage}{\textwidth}
\stepcounter{mallinumber}
\textbf{Tüüpsõnamall \arabic{mallinumber}\,\vadja{varsi}}\\

\begin{sideways}
\begin{tabular}{l l}
muutvormimall & tunnused \\
\hline
\underline{var}\,+\,si & \textsc{ sg nom } \\
\underline{var}\,+\,rõ & \textsc{ sg gen } \\
\underline{var}\,+\,ttõ & \textsc{ sg par } \\
\underline{var}\,+\,tõsõ & \textsc{ sg ill } \\
\underline{var}\,+\,rõz & \textsc{ sg ine } \\
\underline{var}\,+\,rõss & \textsc{ sg ela } \\
\underline{var}\,+\,rõllõ & \textsc{ sg all } \\
\underline{var}\,+\,rõll & \textsc{ sg ade } \\
\underline{var}\,+\,rõlt & \textsc{ sg abl } \\
\underline{var}\,+\,rõssi & \textsc{ sg tra } \\
\underline{var}\,+\,rõssaa & \textsc{ sg ter } \\
\underline{var}\,+\,rõka & \textsc{ sg com } \\
\underline{var}\,+\,rõd & \textsc{ pl nom } \\
\underline{var}\,+\,sijõ & \textsc{ pl gen } \\
\underline{var}\,+\,siit & \textsc{ pl par } \\
\underline{var}\,+\,siisõ & \textsc{ pl ill } \\
\underline{var}\,+\,siiz & \textsc{ pl ine } \\
\underline{var}\,+\,siiss & \textsc{ pl ela } \\
\underline{var}\,+\,siillõ & \textsc{ pl all } \\
\underline{var}\,+\,siill & \textsc{ pl ade } \\
\underline{var}\,+\,siilt & \textsc{ pl abl } \\
\underline{var}\,+\,siissi & \textsc{ pl tra } \\
\underline{var}\,+\,siissaa & \textsc{ pl ter } \\
\underline{var}\,+\,sijka & \textsc{ pl com } \\
\end{tabular}
\end{sideways}
\captionof{table}{Tüüpsõna \arabic{mallinumber}\,\textit{varsi} ekstraheeritud muutvormimallid.}
\label{tab:tüüpsõnamall-varsi}

\end{minipage}

 
\vspace{1em}
\noindent Tüüpsõna ei hõlma teisi lekseeme vormi\-sõnastikus.


\vspace{1.8em}
\begin{minipage}{\textwidth}
\stepcounter{mallinumber}
\textbf{Tüüpsõnamall \arabic{mallinumber}\,\vadja{mesi}}\\

\begin{sideways}
\begin{tabular}{l l}
muutvormimall & tunnused \\
\hline
\underline{me}\,+\,si & \textsc{ sg nom } \\
\underline{me}\,+\,e & \textsc{ sg gen } \\
\underline{me}\,+\,tte & \textsc{ sg par } \\
\underline{me}\,+\,ttese & \textsc{ sg ill } \\
\underline{me}\,+\,ez & \textsc{ sg ine } \\
\underline{me}\,+\,ess & \textsc{ sg ela } \\
\underline{me}\,+\,elle & \textsc{ sg all } \\
\underline{me}\,+\,ell & \textsc{ sg ade } \\
\underline{me}\,+\,elt & \textsc{ sg abl } \\
\underline{me}\,+\,essi & \textsc{ sg tra } \\
\underline{me}\,+\,essaa & \textsc{ sg ter } \\
\underline{me}\,+\,eka & \textsc{ sg com } \\
\underline{me}\,+\,ed & \textsc{ pl nom } \\
\underline{me}\,+\,sije & \textsc{ pl gen } \\
\underline{me}\,+\,siit & \textsc{ pl par } \\
\underline{me}\,+\,siise & \textsc{ pl ill } \\
\underline{me}\,+\,siiz & \textsc{ pl ine } \\
\underline{me}\,+\,siiss & \textsc{ pl ela } \\
\underline{me}\,+\,siille & \textsc{ pl all } \\
\underline{me}\,+\,siill & \textsc{ pl ade } \\
\underline{me}\,+\,siilt & \textsc{ pl abl } \\
\underline{me}\,+\,siissi & \textsc{ pl tra } \\
\underline{me}\,+\,siissaa & \textsc{ pl ter } \\
\underline{me}\,+\,sijka & \textsc{ pl com } \\
\end{tabular}
\end{sideways}
\captionof{table}{Tüüpsõna \arabic{mallinumber}\,\textit{mesi} ekstraheeritud muutvormimallid.}
\label{tab:tüüpsõnamall-mesi}

\end{minipage}

 
\vspace{1em}
\noindent Tüüpsõna hõlmab vormisõnastiku 4 lekseemi: \vadja{\underline{me}si, \underline{sü}si, \underline{ve}si} ja \vadja{\underline{čä}si}.

\paragraph*{\vadja{\underline{su}si}}
\vadja{\underline{su}õ}, \vadja{\underline{su}ttõ}, \vadja{\underline{su}ttõsõ}, \vadja{\underline{su}õss}, \vadja{\underline{su}õd}, \vadja{\underline{su}sijõ}, \vadja{\underline{su}siit}, \vadja{\underline{su}siisõ}, \vadja{\underline{su}siiss}
 \\
sõnatüüp ei hõlma teisi lekseeme

\spacing{1.5}


\subsection{\RN{11} käändkond}

Üheteistkümnendasse käändkonda kuuluvad need sõnad, mille \msd{sg nom} lõpp on \vadja{-Z}, ent mille vokaaltüvi on \vadja{-s-} \cite[48]{ariste_grammar_1968}.

Avatuid küsimusi-tähelepanekuid:
\begin{itemize}
\item \vadja{-Z}-lõpu sandhi nähtus on kõigi liikmete puhul ühtlustatud \vadja{-z} lõpulisteks
\item kas seda peab mainima, et Jõgõperä murdes on -s-, Kattila murdes on -hs- ja teistes murretes on -ss-
\end{itemize}

\subsubsection*{Ekstraktmorfoloogia sõnatüübid}
\spacing{1}
\paragraph*{\vadja{\underline{rihenneü}z}}
\vadja{\underline{rihenneü}se}, \vadja{\underline{rihenneü}sse}, \vadja{\underline{rihenneü}sesse}, \vadja{\underline{rihenneü}sess}, \vadja{\underline{rihenneü}sed}, \vadja{\underline{rihenneü}sije}, \vadja{\underline{rihenneü}siit}, \vadja{\underline{rihenneü}siise}, \vadja{\underline{rihenneü}siiss}
 \\
Tüüpsõna ei hõlma teisi lekseeme vormi\-sõnastikus.


\vspace{1.8em}
\begin{minipage}{\textwidth}
\stepcounter{mallinumber}
\textbf{Tüüpsõnamall \arabic{mallinumber}\,\vadja{makuz}}\\

\begin{sideways}
\begin{tabular}{l l}
muutvormimall & tunnused \\
\hline
\underline{maku}\,+\,z & \textsc{ sg nom } \\
\underline{maku}\,+\,sõ & \textsc{ sg gen } \\
\underline{maku}\,+\,ssõ & \textsc{ sg par } \\
\underline{maku}\,+\,zõsõ & \textsc{ sg ill } \\
\underline{maku}\,+\,zõz & \textsc{ sg ine } \\
\underline{maku}\,+\,zõss & \textsc{ sg ela } \\
\underline{maku}\,+\,zõllõ & \textsc{ sg all } \\
\underline{maku}\,+\,zõll & \textsc{ sg ade } \\
\underline{maku}\,+\,zõlt & \textsc{ sg abl } \\
\underline{maku}\,+\,zõssi & \textsc{ sg tra } \\
\underline{maku}\,+\,zõssaa & \textsc{ sg ter } \\
\underline{maku}\,+\,zõka & \textsc{ sg com } \\
\underline{maku}\,+\,zõd & \textsc{ pl nom } \\
\underline{maku}\,+\,sijõ & \textsc{ pl gen } \\
\underline{maku}\,+\,siit & \textsc{ pl par } \\
\underline{maku}\,+\,siisõ & \textsc{ pl ill } \\
\underline{maku}\,+\,siiz & \textsc{ pl ine } \\
\underline{maku}\,+\,siiss & \textsc{ pl ela } \\
\underline{maku}\,+\,siillõ & \textsc{ pl all } \\
\underline{maku}\,+\,siill & \textsc{ pl ade } \\
\underline{maku}\,+\,siilt & \textsc{ pl abl } \\
\underline{maku}\,+\,siissi & \textsc{ pl tra } \\
\underline{maku}\,+\,siissaa & \textsc{ pl ter } \\
\underline{maku}\,+\,sijka & \textsc{ pl com } \\
\end{tabular}
\end{sideways}
\captionof{table}{Tüüpsõna \arabic{mallinumber}\,\textit{makuz} ekstraheeritud muutvormimallid.}
\label{tab:tüüpsõnamall-makuz}

\end{minipage}

 
\vspace{1em}
\noindent Tüüpsõna hõlmab vormisõnastiku 4 lekseemi: \vadja{\underline{maku}z, \underline{nagri}z, \underline{paganu}z} ja \vadja{\underline{kolau}z}.

\spacing{1.5}


\subsection{\RN{12} käändkond}

Kaheteistkümnes käändkond koondab need sõnad, mille \msd{sg nom} lõpp on \vadja{-n/-ne/-nõ}, ent mille vokaaltüves on \vadja{-se-:-ze-/-sö-:-zö-} sõltuvalt astmevaheldusele \cite[49]{ariste_grammar_1968}.

Avatuid küsimusi-tähelepanekuid:
\begin{itemize}
\item pl tüvi ühtlustatud -s- igal pool TODO üle vaadata s:z vaheldus pluuralis, kas see on s kui 1. silp on pikk v kinnine? (Tsvetkovil pole reeglipäraselt vaid variatsiooniline)
\item kas pl gen peaks vahelduma -z- (iloin)? või -s- (keskolin)?
\item talviisijõ,talviiziit
\item õpõin on väga erandlik sõna
\end{itemize}

\subsubsection*{Ekstraktmorfoloogia sõnatüübid}
\spacing{1}
\paragraph*{\vadja{\underline{õ}p\underline{õ}in}}
\vadja{\underline{õ}p\underline{õ}izõ}, \vadja{\underline{õ}v\underline{õ}issõ}, \vadja{\underline{õ}p\underline{õ}zõsõ}, \vadja{\underline{õ}p\underline{õ}izõss}, \vadja{\underline{õ}p\underline{õ}izõd}, \vadja{\underline{õ}p\underline{õ}izijõ}, \vadja{\underline{õ}p\underline{õ}iziit}, \vadja{\underline{õ}p\underline{õ}iziisõ}, \vadja{\underline{õ}p\underline{õ}iziiss}
 \\
Tüüpsõna ei hõlma teisi lekseeme vormi\-sõnastikus.

\paragraph*{\vadja{\underline{enti}n}}
\vadja{\underline{enti}ze}, \vadja{\underline{enti}sse}, \vadja{\underline{enti}zese}, \vadja{\underline{enti}zess}, \vadja{\underline{enti}zed}, \vadja{\underline{enti}sije}, \vadja{\underline{enti}siit}, \vadja{\underline{enti}siise}, \vadja{\underline{enti}siiss}
 \\
Sõnatüüp hõlmab vormisõnastiku lekseeme: \vadja{entin, esimein, inimin, jäin, miltäin, rehelin, sinin, venäläin, eglin}.


\vspace{1.8em}
\begin{minipage}{\textwidth}
\stepcounter{mallinumber}
\textbf{Tüüpsõnamall \arabic{mallinumber}\,\vadja{süsiin}}\\

\begin{sideways}
\begin{tabular}{l l}
muutvormimall & tunnused \\
\hline
\underline{süsi}\,+\,in & \textsc{ sg nom } \\
\underline{süsi}\,+\,ze & \textsc{ sg gen } \\
\underline{süsi}\,+\,isse & \textsc{ sg par } \\
\underline{süsi}\,+\,zese & \textsc{ sg ill } \\
\underline{süsi}\,+\,zez & \textsc{ sg ine } \\
\underline{süsi}\,+\,zess & \textsc{ sg ela } \\
\underline{süsi}\,+\,zelle & \textsc{ sg all } \\
\underline{süsi}\,+\,zell & \textsc{ sg ade } \\
\underline{süsi}\,+\,zelt & \textsc{ sg abl } \\
\underline{süsi}\,+\,zessi & \textsc{ sg tra } \\
\underline{süsi}\,+\,zessaa & \textsc{ sg ter } \\
\underline{süsi}\,+\,zeka & \textsc{ sg com } \\
\underline{süsi}\,+\,zed & \textsc{ pl nom } \\
\underline{süsi}\,+\,sije & \textsc{ pl gen } \\
\underline{süsi}\,+\,siit & \textsc{ pl par } \\
\underline{süsi}\,+\,siise & \textsc{ pl ill } \\
\underline{süsi}\,+\,siiz & \textsc{ pl ine } \\
\underline{süsi}\,+\,siiss & \textsc{ pl ela } \\
\underline{süsi}\,+\,siille & \textsc{ pl all } \\
\underline{süsi}\,+\,siill & \textsc{ pl ade } \\
\underline{süsi}\,+\,siilt & \textsc{ pl abl } \\
\underline{süsi}\,+\,siissi & \textsc{ pl tra } \\
\underline{süsi}\,+\,siissaa & \textsc{ pl ter } \\
\underline{süsi}\,+\,sijka & \textsc{ pl com } \\
\end{tabular}
\end{sideways}
\captionof{table}{Tüüpsõna \arabic{mallinumber}\,\textit{süsiin} ekstraheeritud muutvormimallid.}
\label{tab:tüüpsõnamall-süsiin}

\end{minipage}

 
\vspace{1em}
\noindent Tüüpsõna ei hõlma teisi lekseeme vormi\-sõnastikus.


\vspace{1.8em}
\begin{minipage}{\textwidth}
\stepcounter{mallinumber}
\textbf{Tüüpsõnamall \arabic{mallinumber}\,\vadja{õnnõliin}}\\

\begin{sideways}
\begin{tabular}{l l}
muutvormimall & tunnused \\
\hline
\underline{õnnõli}\,+\,in & \textsc{ sg nom } \\
\underline{õnnõli}\,+\,zõ & \textsc{ sg gen } \\
\underline{õnnõli}\,+\,issõ & \textsc{ sg par } \\
\underline{õnnõli}\,+\,zõsõ & \textsc{ sg ill } \\
\underline{õnnõli}\,+\,zõz & \textsc{ sg ine } \\
\underline{õnnõli}\,+\,zõss & \textsc{ sg ela } \\
\underline{õnnõli}\,+\,zõllõ & \textsc{ sg all } \\
\underline{õnnõli}\,+\,zõll & \textsc{ sg ade } \\
\underline{õnnõli}\,+\,zõlt & \textsc{ sg abl } \\
\underline{õnnõli}\,+\,zõssi & \textsc{ sg tra } \\
\underline{õnnõli}\,+\,zõssaa & \textsc{ sg ter } \\
\underline{õnnõli}\,+\,zõka & \textsc{ sg com } \\
\underline{õnnõli}\,+\,zõd & \textsc{ pl nom } \\
\underline{õnnõli}\,+\,sijõ & \textsc{ pl gen } \\
\underline{õnnõli}\,+\,siit & \textsc{ pl par } \\
\underline{õnnõli}\,+\,siisõ & \textsc{ pl ill } \\
\underline{õnnõli}\,+\,siiz & \textsc{ pl ine } \\
\underline{õnnõli}\,+\,siiss & \textsc{ pl ela } \\
\underline{õnnõli}\,+\,siillõ & \textsc{ pl all } \\
\underline{õnnõli}\,+\,siill & \textsc{ pl ade } \\
\underline{õnnõli}\,+\,siilt & \textsc{ pl abl } \\
\underline{õnnõli}\,+\,siissi & \textsc{ pl tra } \\
\underline{õnnõli}\,+\,siissaa & \textsc{ pl ter } \\
\underline{õnnõli}\,+\,sijka & \textsc{ pl com } \\
\end{tabular}
\end{sideways}
\captionof{table}{Tüüpsõna \arabic{mallinumber}\,\textit{õnnõliin} ekstraheeritud muutvormimallid.}
\label{tab:tüüpsõnamall-õnnõliin}

\end{minipage}

 
\vspace{1em}
\noindent Tüüpsõna ei hõlma teisi lekseeme vormi\-sõnastikus.

\paragraph*{\vadja{\underline{čimolai}n}}
\vadja{\underline{čimolai}zõ}, \vadja{\underline{čimolai}ssõ}, \vadja{\underline{čimolai}zõsõ}, \vadja{\underline{čimolai}zõss}, \vadja{\underline{čimolai}zõd}, \vadja{\underline{čimolai}sijõ}, \vadja{\underline{čimolai}siit}, \vadja{\underline{čimolai}siisõ}, \vadja{\underline{čimolai}siiss}
 \\
Tüüpsõna hõlmab vormisõnastiku lekseeme \vadja{čimolain, greekklain, hatukkõin, iirikkõin, il̕l̕õkkõin, iloin, jõkain, kehnokkõin, keskolin, kõikõllain, kõrvõlin, leivekkõin, luin, lättilain, magnettiin, main, mokomõin, mustõlain, nain, partõin, perennain, prikukkõin, puin, roottsilain, ruskolain, saunlain, soomõlain, sopuin, sukulain, talviin, tarttulain, tõin, ukrainalain, virolain, õhtõgoin, ühellain, audžikkõin}.


\vspace{1.8em}
\begin{minipage}{\textwidth}
\stepcounter{mallinumber}
\textbf{Tüüpsõnamall \arabic{mallinumber}\,\vadja{koivuin}}\\

\begin{sideways}
\begin{tabular}{l l}
muutvormimall & tunnused \\
\hline
\underline{koivui}\,+\,n & \textsc{ sg nom } \\
\underline{koivui}\,+\,zõ & \textsc{ sg gen } \\
\underline{koivui}\,+\,ssõ & \textsc{ sg par } \\
\underline{koivui}\,+\,zõsõ & \textsc{ sg ill } \\
\underline{koivui}\,+\,zõz & \textsc{ sg ine } \\
\underline{koivui}\,+\,zõss & \textsc{ sg ela } \\
\underline{koivui}\,+\,zõllõ & \textsc{ sg all } \\
\underline{koivui}\,+\,zõll & \textsc{ sg ade } \\
\underline{koivui}\,+\,zõlt & \textsc{ sg abl } \\
\underline{koivui}\,+\,zõssi & \textsc{ sg tra } \\
\underline{koivui}\,+\,zõssaa & \textsc{ sg ter } \\
\underline{koivui}\,+\,zõka & \textsc{ sg com } \\
\underline{koivui}\,+\,zõd & \textsc{ pl nom } \\
\underline{koivui}\,+\,zijõ & \textsc{ pl gen } \\
\underline{koivui}\,+\,ziit & \textsc{ pl par } \\
\underline{koivui}\,+\,ziisõ & \textsc{ pl ill } \\
\underline{koivui}\,+\,ziiz & \textsc{ pl ine } \\
\underline{koivui}\,+\,ziiss & \textsc{ pl ela } \\
\underline{koivui}\,+\,ziillõ & \textsc{ pl all } \\
\underline{koivui}\,+\,ziill & \textsc{ pl ade } \\
\underline{koivui}\,+\,ziilt & \textsc{ pl abl } \\
\underline{koivui}\,+\,ziissi & \textsc{ pl tra } \\
\underline{koivui}\,+\,ziissaa & \textsc{ pl ter } \\
\underline{koivui}\,+\,zijka & \textsc{ pl com } \\
\end{tabular}
\end{sideways}
\captionof{table}{Tüüpsõna \arabic{mallinumber}\,\textit{koivuin} ekstraheeritud muutvormimallid.}
\label{tab:tüüpsõnamall-koivuin}

\end{minipage}

 
\vspace{1em}
\noindent Tüüpsõna hõlmab vormisõnastiku 8 lekseemi: \vadja{\underline{koivui}n, \underline{kultõi}n, \underline{kõltõi}n, \underline{pakkõi}n, \underline{rohoi}n, \underline{uuti}n, \underline{voosi}n} ja \vadja{\underline{kalttõi}n}.

\spacing{1.5}


\subsection{\RN{13} käändkond}

Kolmeteistkümnendasse käändkonda kuuluvad need sõnad, mis lõpevad pika vokaaliga \msd{sg nom}. Lisaks kuuluvad siia mõned sõnad, mis lõppevad diftongiga \msd{sg nom}. \cite[49]{ariste_grammar_1968}.

Avatuid küsimusi-tähelepanekuid:
\begin{itemize}
\item Aristel pole \vadja{seemen} vaid on seemee:seemenee:seemeetä
\item Tsvetkovil pole süä:süäme vaid on süä:süä:süttä/süät
\item Konkoval on võttim:võttimõ:võttima (Tsvetkovil on näitelauses võti)
\end{itemize}

Veel kuuluvad siia käändkonda ordinaalid kolmest edasi \cite[50]{ariste_grammar_1968}.
Numeraalide puhul on järgitud Rozhanskiy ja Markuse välja toodud:
\begin{itemize}
\item \msd{sg nom} lõpp on \vadja{-iz}
\end{itemize}


\subsubsection*{Ekstraktmorfoloogia sõnatüübid}
\spacing{1}

\vspace{1.8em}
\begin{minipage}{\textwidth}
\stepcounter{mallinumber}
\textbf{Tüüpsõnamall \arabic{mallinumber}\,\vadja{čümme}}\\

\begin{sideways}
\begin{tabular}{l l}
muutvormimall & tunnused \\
\hline
\underline{čümme} & \textsc{ sg nom } \\
\underline{čümme}\,+\,ne & \textsc{ sg gen } \\
\underline{čümme}\,+\,nä & \textsc{ sg par } \\
\underline{čümme}\,+\,nese & \textsc{ sg ill } \\
\underline{čümme}\,+\,z & \textsc{ sg ine } \\
\underline{čümme}\,+\,ss & \textsc{ sg ela } \\
\underline{čümme}\,+\,lle & \textsc{ sg all } \\
\underline{čümme}\,+\,ll & \textsc{ sg ade } \\
\underline{čümme}\,+\,lt & \textsc{ sg abl } \\
\underline{čümme}\,+\,ssi & \textsc{ sg tra } \\
\underline{čümme}\,+\,ssaa & \textsc{ sg ter } \\
\underline{čümme}\,+\,ka & \textsc{ sg com } \\
\underline{čümme}\,+\,d & \textsc{ pl nom } \\
\underline{čümme}\,+\,nije & \textsc{ pl gen } \\
\underline{čümme}\,+\,niit & \textsc{ pl par } \\
\underline{čümme}\,+\,niise & \textsc{ pl ill } \\
\underline{čümme}\,+\,niiz & \textsc{ pl ine } \\
\underline{čümme}\,+\,niiss & \textsc{ pl ela } \\
\underline{čümme}\,+\,niille & \textsc{ pl all } \\
\underline{čümme}\,+\,niill & \textsc{ pl ade } \\
\underline{čümme}\,+\,niilt & \textsc{ pl abl } \\
\underline{čümme}\,+\,niissi & \textsc{ pl tra } \\
\underline{čümme}\,+\,niissaa & \textsc{ pl ter } \\
\underline{čümme}\,+\,nijka & \textsc{ pl com } \\
\end{tabular}
\end{sideways}
\captionof{table}{Tüüpsõna \arabic{mallinumber}\,\textit{čümme} ekstraheeritud muutvormimallid.}
\label{tab:tüüpsõnamall-čümme}

\end{minipage}

 
\vspace{1em}
\noindent Tüüpsõna ei hõlma teisi lekseeme vormi\-sõnastikus.


\vspace{1.8em}
\begin{minipage}{\textwidth}
\stepcounter{mallinumber}
\textbf{Tüüpsõnamall \arabic{mallinumber}\,\vadja{čümmenäz}}\\

\begin{sideways}
\begin{tabular}{l l}
muutvormimall & tunnused \\
\hline
\underline{čümmen}\,+\,äz & \textsc{ sg nom } \\
\underline{čümmen}\,+\,ettemä & \textsc{ sg gen } \\
\underline{čümmen}\,+\,eii & \textsc{ sg par } \\
\underline{čümmen}\,+\,ettemäse & \textsc{ sg ill } \\
\underline{čümmen}\,+\,ettemäz & \textsc{ sg ine } \\
\underline{čümmen}\,+\,ettemäss & \textsc{ sg ela } \\
\underline{čümmen}\,+\,ettemälle & \textsc{ sg all } \\
\underline{čümmen}\,+\,ettemäll & \textsc{ sg ade } \\
\underline{čümmen}\,+\,ettemält & \textsc{ sg abl } \\
\underline{čümmen}\,+\,ettemässi & \textsc{ sg tra } \\
\underline{čümmen}\,+\,ettemässaa & \textsc{ sg ter } \\
\underline{čümmen}\,+\,ettemäka & \textsc{ sg com } \\
\underline{čümmen}\,+\,ettemäd & \textsc{ pl nom } \\
\underline{čümmen}\,+\,ettemije & \textsc{ pl gen } \\
\underline{čümmen}\,+\,ettemiit & \textsc{ pl par } \\
\underline{čümmen}\,+\,ettemiise & \textsc{ pl ill } \\
\underline{čümmen}\,+\,ettemiiz & \textsc{ pl ine } \\
\underline{čümmen}\,+\,ettemiiss & \textsc{ pl ela } \\
\underline{čümmen}\,+\,ettemiille & \textsc{ pl all } \\
\underline{čümmen}\,+\,ettemiill & \textsc{ pl ade } \\
\underline{čümmen}\,+\,ettemiilt & \textsc{ pl abl } \\
\underline{čümmen}\,+\,ettemiissi & \textsc{ pl tra } \\
\underline{čümmen}\,+\,ettemiissaa & \textsc{ pl ter } \\
\underline{čümmen}\,+\,ettemijka & \textsc{ pl com } \\
\end{tabular}
\end{sideways}
\captionof{table}{Tüüpsõna \arabic{mallinumber}\,\textit{čümmenäz} ekstraheeritud muutvormimallid.}
\label{tab:tüüpsõnamall-čümmenäz}

\end{minipage}

 
\vspace{1em}
\noindent Tüüpsõna ei hõlma teisi lekseeme vormi\-sõnastikus.


\vspace{1.8em}
\begin{minipage}{\textwidth}
\stepcounter{mallinumber}
\textbf{Tüüpsõnamall \arabic{mallinumber}\,\vadja{nel̕l̕äz}}\\

\begin{sideways}
\begin{tabular}{l l}
muutvormimall & tunnused \\
\hline
\underline{nel̕l̕}\,+\,äz & \textsc{ sg nom } \\
\underline{nel̕l̕}\,+\,ettemä & \textsc{ sg gen } \\
\underline{nel̕l̕}\,+\,että & \textsc{ sg par } \\
\underline{nel̕l̕}\,+\,ettemäse & \textsc{ sg ill } \\
\underline{nel̕l̕}\,+\,etteemäz & \textsc{ sg ine } \\
\underline{nel̕l̕}\,+\,etteemäss & \textsc{ sg ela } \\
\underline{nel̕l̕}\,+\,etteemälle & \textsc{ sg all } \\
\underline{nel̕l̕}\,+\,etteemäll & \textsc{ sg ade } \\
\underline{nel̕l̕}\,+\,etteemält & \textsc{ sg abl } \\
\underline{nel̕l̕}\,+\,etteemässi & \textsc{ sg tra } \\
\underline{nel̕l̕}\,+\,etteemässaa & \textsc{ sg ter } \\
\underline{nel̕l̕}\,+\,etteemäka & \textsc{ sg com } \\
\underline{nel̕l̕}\,+\,etteemäd & \textsc{ pl nom } \\
\underline{nel̕l̕}\,+\,ettemije & \textsc{ pl gen } \\
\underline{nel̕l̕}\,+\,ettemiit & \textsc{ pl par } \\
\underline{nel̕l̕}\,+\,ettemiise & \textsc{ pl ill } \\
\underline{nel̕l̕}\,+\,ettemiiz & \textsc{ pl ine } \\
\underline{nel̕l̕}\,+\,ettemiiss & \textsc{ pl ela } \\
\underline{nel̕l̕}\,+\,ettemiille & \textsc{ pl all } \\
\underline{nel̕l̕}\,+\,ettemiill & \textsc{ pl ade } \\
\underline{nel̕l̕}\,+\,ettemiilt & \textsc{ pl abl } \\
\underline{nel̕l̕}\,+\,ettemiissi & \textsc{ pl tra } \\
\underline{nel̕l̕}\,+\,ettemiissaa & \textsc{ pl ter } \\
\underline{nel̕l̕}\,+\,ettemijka & \textsc{ pl com } \\
\end{tabular}
\end{sideways}
\captionof{table}{Tüüpsõna \arabic{mallinumber}\,\textit{nel̕l̕äz} ekstraheeritud muutvormimallid.}
\label{tab:tüüpsõnamall-nel̕l̕äz}

\end{minipage}

 
\vspace{1em}
\noindent Tüüpsõna ei hõlma teisi lekseeme vormi\-sõnastikus.


\vspace{1.8em}
\begin{minipage}{\textwidth}
\stepcounter{mallinumber}
\textbf{Tüüpsõnamall \arabic{mallinumber}\,\vadja{seemen}}\\

\begin{sideways}
\begin{tabular}{l l}
muutvormimall & tunnused \\
\hline
\underline{seem}\,+\,e\,+\,\underline{n} & \textsc{ sg nom } \\
\underline{seem}\,+\,\underline{n}\,+\,e & \textsc{ sg gen } \\
\underline{seem}\,+\,e\,+\,\underline{n}\,+\,t & \textsc{ sg par } \\
\underline{seem}\,+\,\underline{n}\,+\,ese & \textsc{ sg ill } \\
\underline{seem}\,+\,\underline{n}\,+\,ez & \textsc{ sg ine } \\
\underline{seem}\,+\,\underline{n}\,+\,ess & \textsc{ sg ela } \\
\underline{seem}\,+\,\underline{n}\,+\,elle & \textsc{ sg all } \\
\underline{seem}\,+\,\underline{n}\,+\,ell & \textsc{ sg ade } \\
\underline{seem}\,+\,\underline{n}\,+\,elt & \textsc{ sg abl } \\
\underline{seem}\,+\,\underline{n}\,+\,essi & \textsc{ sg tra } \\
\underline{seem}\,+\,\underline{n}\,+\,essaa & \textsc{ sg ter } \\
\underline{seem}\,+\,\underline{n}\,+\,eka & \textsc{ sg com } \\
\underline{seem}\,+\,\underline{n}\,+\,ed & \textsc{ pl nom } \\
\underline{seem}\,+\,\underline{n}\,+\,ije & \textsc{ pl gen } \\
\underline{seem}\,+\,\underline{n}\,+\,iit & \textsc{ pl par } \\
\underline{seem}\,+\,\underline{n}\,+\,iise & \textsc{ pl ill } \\
\underline{seem}\,+\,\underline{n}\,+\,iiz & \textsc{ pl ine } \\
\underline{seem}\,+\,\underline{n}\,+\,iiss & \textsc{ pl ela } \\
\underline{seem}\,+\,\underline{n}\,+\,iille & \textsc{ pl all } \\
\underline{seem}\,+\,\underline{n}\,+\,iill & \textsc{ pl ade } \\
\underline{seem}\,+\,\underline{n}\,+\,iilt & \textsc{ pl abl } \\
\underline{seem}\,+\,\underline{n}\,+\,iissi & \textsc{ pl tra } \\
\underline{seem}\,+\,\underline{n}\,+\,iissaa & \textsc{ pl ter } \\
\underline{seem}\,+\,\underline{n}\,+\,ijka & \textsc{ pl com } \\
\end{tabular}
\end{sideways}
\captionof{table}{Tüüpsõna \arabic{mallinumber}\,\textit{seemen} ekstraheeritud muutvormimallid.}
\label{tab:tüüpsõnamall-seemen}

\end{minipage}

 
\vspace{1em}
\noindent Tüüpsõna ei hõlma teisi lekseeme vormi\-sõnastikus.


\vspace{1.8em}
\begin{minipage}{\textwidth}
\stepcounter{mallinumber}
\textbf{Tüüpsõnamall \arabic{mallinumber}\,\vadja{süä}}\\

\begin{sideways}
\begin{tabular}{l l}
muutvormimall & tunnused \\
\hline
\underline{sü}\,+\,ä & \textsc{ sg nom } \\
\underline{sü}\,+\,ä & \textsc{ sg gen } \\
\underline{sü}\,+\,ttä & \textsc{ sg par } \\
\underline{sü}\,+\,ttäse & \textsc{ sg ill } \\
\underline{sü}\,+\,äz & \textsc{ sg ine } \\
\underline{sü}\,+\,äss & \textsc{ sg ela } \\
\underline{sü}\,+\,älle & \textsc{ sg all } \\
\underline{sü}\,+\,äll & \textsc{ sg ade } \\
\underline{sü}\,+\,ält & \textsc{ sg abl } \\
\underline{sü}\,+\,ässi & \textsc{ sg tra } \\
\underline{sü}\,+\,ässaa & \textsc{ sg ter } \\
\underline{sü}\,+\,äka & \textsc{ sg com } \\
\underline{sü}\,+\,äd & \textsc{ pl nom } \\
\underline{sü}\,+\,ttije & \textsc{ pl gen } \\
\underline{sü}\,+\,ttiit & \textsc{ pl par } \\
\underline{sü}\,+\,ttiise & \textsc{ pl ill } \\
\underline{sü}\,+\,ttiiz & \textsc{ pl ine } \\
\underline{sü}\,+\,ttiiss & \textsc{ pl ela } \\
\underline{sü}\,+\,ttiille & \textsc{ pl all } \\
\underline{sü}\,+\,ttiill & \textsc{ pl ade } \\
\underline{sü}\,+\,ttiilt & \textsc{ pl abl } \\
\underline{sü}\,+\,ttiissi & \textsc{ pl tra } \\
\underline{sü}\,+\,ttiissaa & \textsc{ pl ter } \\
\underline{sü}\,+\,ttijka & \textsc{ pl com } \\
\end{tabular}
\end{sideways}
\captionof{table}{Tüüpsõna \arabic{mallinumber}\,\textit{süä} ekstraheeritud muutvormimallid.}
\label{tab:tüüpsõnamall-süä}

\end{minipage}

 
\vspace{1em}
\noindent Tüüpsõna ei hõlma teisi lekseeme vormi\-sõnastikus.


\vspace{1.8em}
\begin{minipage}{\textwidth}
\stepcounter{mallinumber}
\textbf{Tüüpsõnamall \arabic{mallinumber}\,\vadja{tütär}}\\

\begin{sideways}
\begin{tabular}{l l}
muutvormimall & tunnused \\
\hline
\underline{tüt}\,+\,\underline{är} & \textsc{ sg nom } \\
\underline{tüt}\,+\,t\,+\,\underline{är}\,+\,e & \textsc{ sg gen } \\
\underline{tüt}\,+\,\underline{är}\,+\,te & \textsc{ sg par } \\
\underline{tüt}\,+\,t\,+\,\underline{är}\,+\,ese & \textsc{ sg ill } \\
\underline{tüt}\,+\,t\,+\,\underline{är}\,+\,ez & \textsc{ sg ine } \\
\underline{tüt}\,+\,t\,+\,\underline{är}\,+\,ess & \textsc{ sg ela } \\
\underline{tüt}\,+\,t\,+\,\underline{är}\,+\,elle & \textsc{ sg all } \\
\underline{tüt}\,+\,t\,+\,\underline{är}\,+\,ell & \textsc{ sg ade } \\
\underline{tüt}\,+\,t\,+\,\underline{är}\,+\,elt & \textsc{ sg abl } \\
\underline{tüt}\,+\,t\,+\,\underline{är}\,+\,essi & \textsc{ sg tra } \\
\underline{tüt}\,+\,t\,+\,\underline{är}\,+\,essaa & \textsc{ sg ter } \\
\underline{tüt}\,+\,t\,+\,\underline{är}\,+\,eka & \textsc{ sg com } \\
\underline{tüt}\,+\,t\,+\,\underline{är}\,+\,ed & \textsc{ pl nom } \\
\underline{tüt}\,+\,t\,+\,\underline{är}\,+\,ije & \textsc{ pl gen } \\
\underline{tüt}\,+\,t\,+\,\underline{är}\,+\,iit & \textsc{ pl par } \\
\underline{tüt}\,+\,t\,+\,\underline{är}\,+\,iise & \textsc{ pl ill } \\
\underline{tüt}\,+\,t\,+\,\underline{är}\,+\,iiz & \textsc{ pl ine } \\
\underline{tüt}\,+\,t\,+\,\underline{är}\,+\,iiss & \textsc{ pl ela } \\
\underline{tüt}\,+\,t\,+\,\underline{är}\,+\,iille & \textsc{ pl all } \\
\underline{tüt}\,+\,t\,+\,\underline{är}\,+\,iill & \textsc{ pl ade } \\
\underline{tüt}\,+\,t\,+\,\underline{är}\,+\,iilt & \textsc{ pl abl } \\
\underline{tüt}\,+\,t\,+\,\underline{är}\,+\,iissi & \textsc{ pl tra } \\
\underline{tüt}\,+\,t\,+\,\underline{är}\,+\,iissaa & \textsc{ pl ter } \\
\underline{tüt}\,+\,t\,+\,\underline{är}\,+\,ijka & \textsc{ pl com } \\
\end{tabular}
\end{sideways}
\captionof{table}{Tüüpsõna \arabic{mallinumber}\,\textit{tütär} ekstraheeritud muutvormimallid.}
\label{tab:tüüpsõnamall-tütär}

\end{minipage}

 
\vspace{1em}
\noindent Tüüpsõna ei hõlma teisi lekseeme vormi\-sõnastikus.

\paragraph*{\vadja{\underline{õnnõt}\underline{o}}}
\vadja{\underline{õnnõt}t\underline{o}ma}, \vadja{\underline{õnnõt}\underline{o}ta}, \vadja{\underline{õnnõt}t\underline{o}masõ}, \vadja{\underline{õnnõt}t\underline{o}mass}, \vadja{\underline{õnnõt}t\underline{o}mad}, \vadja{\underline{õnnõt}t\underline{o}mijõ}, \vadja{\underline{õnnõt}t\underline{o}miit}, \vadja{\underline{õnnõt}t\underline{o}miisõ}, \vadja{\underline{õnnõt}t\underline{o}miiss}
 \\
Tüüpsõna hõlmab vormisõnastiku lekseeme: \vadja{õnnõto, hoolito}.

\paragraph*{\vadja{\underline{kahõs}a}}
\vadja{\underline{kahõs}sõmõ}, \vadja{\underline{kahõs}sõma}, \vadja{\underline{kahõs}sõmasõ}, \vadja{\underline{kahõs}sõmass}, \vadja{\underline{kahõs}sõmad}, \vadja{\underline{kahõs}sõmijõ}, \vadja{\underline{kahõs}sõmiit}, \vadja{\underline{kahõs}sõmiisõ}, \vadja{\underline{kahõs}sõmiiss}
 \\
sõnatüüp ei hõlma teisi lekseeme


\vspace{1.8em}
\begin{minipage}{\textwidth}
\stepcounter{mallinumber}
\textbf{Tüüpsõnamall \arabic{mallinumber}\,\vadja{kuuvvaiz}}\\

\begin{sideways}
\begin{tabular}{l l}
muutvormimall & tunnused \\
\hline
\underline{kuuvv}\,+\,aiz & \textsc{ sg nom } \\
\underline{kuuvv}\,+\,õttõma & \textsc{ sg gen } \\
\underline{kuuvv}\,+\,õt & \textsc{ sg par } \\
\underline{kuuvv}\,+\,õttõmasõ & \textsc{ sg ill } \\
\underline{kuuvv}\,+\,õttõmaz & \textsc{ sg ine } \\
\underline{kuuvv}\,+\,õttõmass & \textsc{ sg ela } \\
\underline{kuuvv}\,+\,õttõmallõ & \textsc{ sg all } \\
\underline{kuuvv}\,+\,õttõmall & \textsc{ sg ade } \\
\underline{kuuvv}\,+\,õttõmalt & \textsc{ sg abl } \\
\underline{kuuvv}\,+\,õttõmassi & \textsc{ sg tra } \\
\underline{kuuvv}\,+\,õttõmassaa & \textsc{ sg ter } \\
\underline{kuuvv}\,+\,õttõmaka & \textsc{ sg com } \\
\underline{kuuvv}\,+\,õttõmad & \textsc{ pl nom } \\
\underline{kuuvv}\,+\,õttõmijõ & \textsc{ pl gen } \\
\underline{kuuvv}\,+\,õttõmiit & \textsc{ pl par } \\
\underline{kuuvv}\,+\,õttõmiisõ & \textsc{ pl ill } \\
\underline{kuuvv}\,+\,õttõmiiz & \textsc{ pl ine } \\
\underline{kuuvv}\,+\,õttõmiiss & \textsc{ pl ela } \\
\underline{kuuvv}\,+\,õttõmiillõ & \textsc{ pl all } \\
\underline{kuuvv}\,+\,õttõmiill & \textsc{ pl ade } \\
\underline{kuuvv}\,+\,õttõmiilt & \textsc{ pl abl } \\
\underline{kuuvv}\,+\,õttõmiissi & \textsc{ pl tra } \\
\underline{kuuvv}\,+\,õttõmiissaa & \textsc{ pl ter } \\
\underline{kuuvv}\,+\,õttõmijka & \textsc{ pl com } \\
\end{tabular}
\end{sideways}
\captionof{table}{Tüüpsõna \arabic{mallinumber}\,\textit{kuuvvaiz} ekstraheeritud muutvormimallid.}
\label{tab:tüüpsõnamall-kuuvvaiz}

\end{minipage}

 
\vspace{1em}
\noindent Tüüpsõna ei hõlma teisi lekseeme vormi\-sõnastikus.

\paragraph*{\vadja{\underline{kõlm}az}}
\vadja{\underline{kõlm}õttõma}, \vadja{\underline{kõlm}aissõ}, \vadja{\underline{kõlm}õttõmasõ}, \vadja{\underline{kõlm}õttõmass}, \vadja{\underline{kõlm}õttõmad}, \vadja{\underline{kõlm}õttõmijõ}, \vadja{\underline{kõlm}õttõmiit}, \vadja{\underline{kõlm}õttõmiisõ}, \vadja{\underline{kõlm}õttõmiiss}
 \\
Sõnatüüp ei hõlma teisi lekseeme vormi\-sõnastikus.

\spacing{1.5}


\subsection{\RN{14} käändkond}

Neljateistkümnenda käändkonna sõnad lõpevad \vadja{-aZ/-äZ, -iZ} või \vadja{-e/-õ} \cite[50]{ariste_grammar_1968}.

Avatuid küsimusi-tähelepanekuid:
\begin{itemize}
\item \vadja{-Z}-lõpu sandhi nähtus on kõigi liikmete puhul ühtlustatud \vadja{-z} lõpulisteks
\item plurale tantum 'ivusõd' kustutatud sest 'ivuz' olemas
\item Tsvetkovi antud paralleelvariantidest on valitud vaid üks (korpuse, analoogsete sõnade ülekaalu ning Heinsoo ja Konkova põhjal)
\item valitud 'lähe' tugevaastmeline sg tüvi, sest VKSis esineb ühes Li näitelauses
\item -kõz-liides muudetud eespoolseks vastavate sõnade juures
\end{itemize}

\subsubsection*{Ekstraktmorfoloogia sõnatüübid}
\spacing{1}
\paragraph*{\vadja{\underline{õ}g\underline{a}z}}
\vadja{\underline{õ}kk\underline{a}}, \vadja{\underline{õ}g\underline{a}ssõ}, \vadja{\underline{õ}kk\underline{a}sõ}, \vadja{\underline{õ}kk\underline{a}ss}, \vadja{\underline{õ}kk\underline{a}d}, \vadja{\underline{õ}kk\underline{a}jõ}, \vadja{\underline{õ}kk\underline{a}it}, \vadja{\underline{õ}kk\underline{a}isõ}, \vadja{\underline{õ}kk\underline{a}iss}
 \\
Sõnatüüp ei hõlma teisi lekseeme vormi\-sõnastikus.

\paragraph*{\vadja{\underline{puh}\underline{a}z}}
\vadja{\underline{puh}t\underline{a}}, \vadja{\underline{puh}\underline{a}ssõ}, \vadja{\underline{puh}t\underline{a}sõ}, \vadja{\underline{puh}t\underline{a}ss}, \vadja{\underline{puh}t\underline{a}d}, \vadja{\underline{puh}t\underline{a}jõ}, \vadja{\underline{puh}t\underline{a}it}, \vadja{\underline{puh}t\underline{a}isõ}, \vadja{\underline{puh}t\underline{a}iss}
 \\
sõnatüüp hõlmab lekseeme \vadja{puhaz, ahaz}


\vspace{1.8em}
\begin{minipage}{\textwidth}
\stepcounter{mallinumber}
\textbf{Tüüpsõnamall \arabic{mallinumber}\,\vadja{lähe}}\\

\begin{sideways}
\begin{tabular}{l l}
muutvormimall & tunnused \\
\hline
\underline{läh}\,+\,\underline{e} & \textsc{ sg nom } \\
\underline{läh}\,+\,t\,+\,\underline{e} & \textsc{ sg gen } \\
\underline{läh}\,+\,\underline{e}\,+\,tt & \textsc{ sg par } \\
\underline{läh}\,+\,t\,+\,\underline{e}\,+\,se & \textsc{ sg ill } \\
\underline{läh}\,+\,t\,+\,\underline{e}\,+\,z & \textsc{ sg ine } \\
\underline{läh}\,+\,t\,+\,\underline{e}\,+\,ss & \textsc{ sg ela } \\
\underline{läh}\,+\,t\,+\,\underline{e}\,+\,lle & \textsc{ sg all } \\
\underline{läh}\,+\,t\,+\,\underline{e}\,+\,ll & \textsc{ sg ade } \\
\underline{läh}\,+\,t\,+\,\underline{e}\,+\,lt & \textsc{ sg abl } \\
\underline{läh}\,+\,t\,+\,\underline{e}\,+\,ssi & \textsc{ sg tra } \\
\underline{läh}\,+\,t\,+\,\underline{e}\,+\,ssaa & \textsc{ sg ter } \\
\underline{läh}\,+\,t\,+\,\underline{e}\,+\,ka & \textsc{ sg com } \\
\underline{läh}\,+\,t\,+\,\underline{e}\,+\,d & \textsc{ pl nom } \\
\underline{läh}\,+\,t\,+\,\underline{e}\,+\,je & \textsc{ pl gen } \\
\underline{läh}\,+\,t\,+\,\underline{e}\,+\,it & \textsc{ pl par } \\
\underline{läh}\,+\,t\,+\,\underline{e}\,+\,ise & \textsc{ pl ill } \\
\underline{läh}\,+\,t\,+\,\underline{e}\,+\,iz & \textsc{ pl ine } \\
\underline{läh}\,+\,t\,+\,\underline{e}\,+\,iss & \textsc{ pl ela } \\
\underline{läh}\,+\,t\,+\,\underline{e}\,+\,ille & \textsc{ pl all } \\
\underline{läh}\,+\,t\,+\,\underline{e}\,+\,ill & \textsc{ pl ade } \\
\underline{läh}\,+\,t\,+\,\underline{e}\,+\,ilt & \textsc{ pl abl } \\
\underline{läh}\,+\,t\,+\,\underline{e}\,+\,issi & \textsc{ pl tra } \\
\underline{läh}\,+\,t\,+\,\underline{e}\,+\,issaa & \textsc{ pl ter } \\
\underline{läh}\,+\,t\,+\,\underline{e}\,+\,ika & \textsc{ pl com } \\
\end{tabular}
\end{sideways}
\captionof{table}{Tüüpsõna \arabic{mallinumber}\,\textit{lähe} ekstraheeritud muutvormimallid.}
\label{tab:tüüpsõnamall-lähe}

\end{minipage}

 
\vspace{1em}
\noindent Tüüpsõna ei hõlma teisi lekseeme vormi\-sõnastikus.

\paragraph*{\vadja{\underline{rü}iz}}
\vadja{\underline{rü}čče}, \vadja{\underline{rü}isse}, \vadja{\underline{rü}ččese}, \vadja{\underline{rü}ččess}, \vadja{\underline{rü}ččed}, \vadja{\underline{rü}ččije}, \vadja{\underline{rü}ččiit}, \vadja{\underline{rü}ččiise}, \vadja{\underline{rü}ččiiss}
 \\
Sõnatüüp ei hõlma teisi lekseeme vormi\-sõnastikus.

\paragraph*{\vadja{\underline{rak}\underline{õ}}}
\vadja{\underline{rak}k\underline{õ}}, \vadja{\underline{rak}\underline{õ}ttõ}, \vadja{\underline{rak}k\underline{õ}sõ}, \vadja{\underline{rak}k\underline{õ}ss}, \vadja{\underline{rak}k\underline{õ}d}, \vadja{\underline{rak}k\underline{õ}jõ}, \vadja{\underline{rak}k\underline{õ}it}, \vadja{\underline{rak}k\underline{õ}isõ}, \vadja{\underline{rak}k\underline{õ}iss}
 \\
sõnatüüp ei hõlma teisi lekseeme

\paragraph*{\vadja{\underline{rik}\underline{a}z}}
\vadja{\underline{rik}k\underline{a}}, \vadja{\underline{rik}\underline{a}ssõ}, \vadja{\underline{rik}k\underline{a}sõ}, \vadja{\underline{rik}k\underline{a}ss}, \vadja{\underline{rik}k\underline{a}d}, \vadja{\underline{rik}k\underline{a}jõ}, \vadja{\underline{rik}k\underline{a}it}, \vadja{\underline{rik}k\underline{a}isõ}, \vadja{\underline{rik}k\underline{a}iss}
 \\
Sõnatüüp ei hõlma teisi lekseeme vormi\-sõnastikus.


\vspace{1.8em}
\begin{minipage}{\textwidth}
\stepcounter{mallinumber}
\textbf{Tüüpsõnamall \arabic{mallinumber}\,\vadja{bul̕bukõz}}\\

\begin{sideways}
\begin{tabular}{l l}
muutvormimall & tunnused \\
\hline
\underline{bul̕buk}\,+\,õz & \textsc{ sg nom } \\
\underline{bul̕buk}\,+\,ka & \textsc{ sg gen } \\
\underline{bul̕buk}\,+\,assõ & \textsc{ sg par } \\
\underline{bul̕buk}\,+\,kasõ & \textsc{ sg ill } \\
\underline{bul̕buk}\,+\,kaz & \textsc{ sg ine } \\
\underline{bul̕buk}\,+\,kass & \textsc{ sg ela } \\
\underline{bul̕buk}\,+\,kallõ & \textsc{ sg all } \\
\underline{bul̕buk}\,+\,kall & \textsc{ sg ade } \\
\underline{bul̕buk}\,+\,kalt & \textsc{ sg abl } \\
\underline{bul̕buk}\,+\,kassi & \textsc{ sg tra } \\
\underline{bul̕buk}\,+\,kassaa & \textsc{ sg ter } \\
\underline{bul̕buk}\,+\,kaka & \textsc{ sg com } \\
\underline{bul̕buk}\,+\,kad & \textsc{ pl nom } \\
\underline{bul̕buk}\,+\,kajõ & \textsc{ pl gen } \\
\underline{bul̕buk}\,+\,kait & \textsc{ pl par } \\
\underline{bul̕buk}\,+\,kaisõ & \textsc{ pl ill } \\
\underline{bul̕buk}\,+\,kaiz & \textsc{ pl ine } \\
\underline{bul̕buk}\,+\,kaiss & \textsc{ pl ela } \\
\underline{bul̕buk}\,+\,kaillõ & \textsc{ pl all } \\
\underline{bul̕buk}\,+\,kaill & \textsc{ pl ade } \\
\underline{bul̕buk}\,+\,kailt & \textsc{ pl abl } \\
\underline{bul̕buk}\,+\,kaissi & \textsc{ pl tra } \\
\underline{bul̕buk}\,+\,kaissaa & \textsc{ pl ter } \\
\underline{bul̕buk}\,+\,kaika & \textsc{ pl com } \\
\end{tabular}
\end{sideways}
\captionof{table}{Tüüpsõna \arabic{mallinumber}\,\textit{bul̕bukõz} ekstraheeritud muutvormimallid.}
\label{tab:tüüpsõnamall-bul̕bukõz}

\end{minipage}

 
\vspace{1em}
\noindent Tüüpsõna hõlmab vormisõnastiku 8 lekseemi: \vadja{\underline{bul̕buk}õz, \underline{čirk}õz, \underline{liivõk}õz, \underline{mansik}õz, \underline{musik}õz, \underline{nenäk}õz, \underline{õnnõk}õz} ja \vadja{\underline{baabuk}õz}.


\vspace{1.8em}
\begin{minipage}{\textwidth}
\stepcounter{mallinumber}
\textbf{Tüüpsõnamall \arabic{mallinumber}\,\vadja{kalliz}}\\

\begin{sideways}
\begin{tabular}{l l}
muutvormimall & tunnused \\
\hline
\underline{kall}\,+\,iz & \textsc{ sg nom } \\
\underline{kall}\,+\,i & \textsc{ sg gen } \\
\underline{kall}\,+\,issõ & \textsc{ sg par } \\
\underline{kall}\,+\,isõ & \textsc{ sg ill } \\
\underline{kall}\,+\,iz & \textsc{ sg ine } \\
\underline{kall}\,+\,iss & \textsc{ sg ela } \\
\underline{kall}\,+\,illõ & \textsc{ sg all } \\
\underline{kall}\,+\,ill & \textsc{ sg ade } \\
\underline{kall}\,+\,ilt & \textsc{ sg abl } \\
\underline{kall}\,+\,issi & \textsc{ sg tra } \\
\underline{kall}\,+\,issaa & \textsc{ sg ter } \\
\underline{kall}\,+\,ika & \textsc{ sg com } \\
\underline{kall}\,+\,id & \textsc{ pl nom } \\
\underline{kall}\,+\,ejõ & \textsc{ pl gen } \\
\underline{kall}\,+\,eit & \textsc{ pl par } \\
\underline{kall}\,+\,eisõ & \textsc{ pl ill } \\
\underline{kall}\,+\,eiz & \textsc{ pl ine } \\
\underline{kall}\,+\,eiss & \textsc{ pl ela } \\
\underline{kall}\,+\,eillõ & \textsc{ pl all } \\
\underline{kall}\,+\,eill & \textsc{ pl ade } \\
\underline{kall}\,+\,eilt & \textsc{ pl abl } \\
\underline{kall}\,+\,eissi & \textsc{ pl tra } \\
\underline{kall}\,+\,eissaa & \textsc{ pl ter } \\
\underline{kall}\,+\,eika & \textsc{ pl com } \\
\end{tabular}
\end{sideways}
\captionof{table}{Tüüpsõna \arabic{mallinumber}\,\textit{kalliz} ekstraheeritud muutvormimallid.}
\label{tab:tüüpsõnamall-kalliz}

\end{minipage}

 
\vspace{1em}
\noindent Tüüpsõna ei hõlma teisi lekseeme vormi\-sõnastikus.


\vspace{1.8em}
\begin{minipage}{\textwidth}
\stepcounter{mallinumber}
\textbf{Tüüpsõnamall \arabic{mallinumber}\,\vadja{pal̕l̕õz}}\\

\begin{sideways}
\begin{tabular}{l l}
muutvormimall & tunnused \\
\hline
\underline{pal̕l̕}\,+\,õz & \textsc{ sg nom } \\
\underline{pal̕l̕}\,+\,a & \textsc{ sg gen } \\
\underline{pal̕l̕}\,+\,assõ & \textsc{ sg par } \\
\underline{pal̕l̕}\,+\,asõ & \textsc{ sg ill } \\
\underline{pal̕l̕}\,+\,az & \textsc{ sg ine } \\
\underline{pal̕l̕}\,+\,ass & \textsc{ sg ela } \\
\underline{pal̕l̕}\,+\,allõ & \textsc{ sg all } \\
\underline{pal̕l̕}\,+\,all & \textsc{ sg ade } \\
\underline{pal̕l̕}\,+\,alt & \textsc{ sg abl } \\
\underline{pal̕l̕}\,+\,assi & \textsc{ sg tra } \\
\underline{pal̕l̕}\,+\,assaa & \textsc{ sg ter } \\
\underline{pal̕l̕}\,+\,aka & \textsc{ sg com } \\
\underline{pal̕l̕}\,+\,ad & \textsc{ pl nom } \\
\underline{pal̕l̕}\,+\,ajõ & \textsc{ pl gen } \\
\underline{pal̕l̕}\,+\,ait & \textsc{ pl par } \\
\underline{pal̕l̕}\,+\,aisõ & \textsc{ pl ill } \\
\underline{pal̕l̕}\,+\,aiz & \textsc{ pl ine } \\
\underline{pal̕l̕}\,+\,aiss & \textsc{ pl ela } \\
\underline{pal̕l̕}\,+\,aillõ & \textsc{ pl all } \\
\underline{pal̕l̕}\,+\,aill & \textsc{ pl ade } \\
\underline{pal̕l̕}\,+\,ailt & \textsc{ pl abl } \\
\underline{pal̕l̕}\,+\,aissi & \textsc{ pl tra } \\
\underline{pal̕l̕}\,+\,aissaa & \textsc{ pl ter } \\
\underline{pal̕l̕}\,+\,aika & \textsc{ pl com } \\
\end{tabular}
\end{sideways}
\captionof{table}{Tüüpsõna \arabic{mallinumber}\,\textit{pal̕l̕õz} ekstraheeritud muutvormimallid.}
\label{tab:tüüpsõnamall-pal̕l̕õz}

\end{minipage}

 
\vspace{1em}
\noindent Tüüpsõna hõlmab vormisõnastiku 5 lekseemi: \vadja{\underline{pal̕l̕}õz, \underline{rahv}õz, \underline{taiv}õz, \underline{võõr}õz} ja \vadja{\underline{ahn}õz}.

\paragraph*{\vadja{\underline{pool}õz}}
\vadja{\underline{pool}a}, \vadja{\underline{pool}assõ}, \vadja{\underline{pool}asõ}, \vadja{\underline{pool}ass}, \vadja{\underline{pool}ad}, \vadja{\underline{pool}ajõ}, \vadja{\underline{pool}oit}, \vadja{\underline{pool}oisõ}, \vadja{\underline{pool}oiss}
 \\
Tüüpsõna ei hõlma teisi lekseeme vormi\-sõnastikus.

\paragraph*{\vadja{\underline{ham}mõz}}
\vadja{\underline{ham}pa}, \vadja{\underline{ham}massõ}, \vadja{\underline{ham}pasõ}, \vadja{\underline{ham}pass}, \vadja{\underline{ham}pad}, \vadja{\underline{ham}pajõ}, \vadja{\underline{ham}pait}, \vadja{\underline{ham}paisõ}, \vadja{\underline{ham}paiss}
 \\
Tüüpsõna hõlmab vormisõnastiku lekseeme: \vadja{hammõz, lammõz, ammõz}.


\vspace{1.8em}
\begin{minipage}{\textwidth}
\stepcounter{mallinumber}
\textbf{Tüüpsõnamall \arabic{mallinumber}\,\vadja{lõunõ}}\\

\begin{sideways}
\begin{tabular}{l l}
muutvormimall & tunnused \\
\hline
\underline{lõun}\,+\,õ & \textsc{ sg nom } \\
\underline{lõun}\,+\,a & \textsc{ sg gen } \\
\underline{lõun}\,+\,attõ & \textsc{ sg par } \\
\underline{lõun}\,+\,asõ & \textsc{ sg ill } \\
\underline{lõun}\,+\,az & \textsc{ sg ine } \\
\underline{lõun}\,+\,ass & \textsc{ sg ela } \\
\underline{lõun}\,+\,allõ & \textsc{ sg all } \\
\underline{lõun}\,+\,all & \textsc{ sg ade } \\
\underline{lõun}\,+\,alt & \textsc{ sg abl } \\
\underline{lõun}\,+\,assi & \textsc{ sg tra } \\
\underline{lõun}\,+\,assaa & \textsc{ sg ter } \\
\underline{lõun}\,+\,aka & \textsc{ sg com } \\
\underline{lõun}\,+\,ad & \textsc{ pl nom } \\
\underline{lõun}\,+\,ajõ & \textsc{ pl gen } \\
\underline{lõun}\,+\,ait & \textsc{ pl par } \\
\underline{lõun}\,+\,aisõ & \textsc{ pl ill } \\
\underline{lõun}\,+\,aiz & \textsc{ pl ine } \\
\underline{lõun}\,+\,aiss & \textsc{ pl ela } \\
\underline{lõun}\,+\,aillõ & \textsc{ pl all } \\
\underline{lõun}\,+\,aill & \textsc{ pl ade } \\
\underline{lõun}\,+\,ailt & \textsc{ pl abl } \\
\underline{lõun}\,+\,aissi & \textsc{ pl tra } \\
\underline{lõun}\,+\,aissaa & \textsc{ pl ter } \\
\underline{lõun}\,+\,aika & \textsc{ pl com } \\
\end{tabular}
\end{sideways}
\captionof{table}{Tüüpsõna \arabic{mallinumber}\,\textit{lõunõ} ekstraheeritud muutvormimallid.}
\label{tab:tüüpsõnamall-lõunõ}

\end{minipage}

 
\vspace{1em}
\noindent Tüüpsõna ei hõlma teisi lekseeme vormi\-sõnastikus.

\paragraph*{\vadja{\underline{kan}gõz}}
\vadja{\underline{kan}ka}, \vadja{\underline{kan}gõssõ}, \vadja{\underline{kan}kasõ}, \vadja{\underline{kan}kass}, \vadja{\underline{kan}kad}, \vadja{\underline{kan}kajõ}, \vadja{\underline{kan}kait}, \vadja{\underline{kan}kaisõ}, \vadja{\underline{kan}kaiss}
 \\
Tüüpsõna ei hõlma teisi lekseeme vormi\-sõnastikus.


\vspace{1.8em}
\begin{minipage}{\textwidth}
\stepcounter{mallinumber}
\textbf{Tüüpsõnamall \arabic{mallinumber}\,\vadja{kauniz}}\\

\begin{sideways}
\begin{tabular}{l l}
muutvormimall & tunnused \\
\hline
\underline{kauni}\,+\,z & \textsc{ sg nom } \\
\underline{kauni} & \textsc{ sg gen } \\
\underline{kauni}\,+\,ssõ & \textsc{ sg par } \\
\underline{kauni}\,+\,sõ & \textsc{ sg ill } \\
\underline{kauni}\,+\,z & \textsc{ sg ine } \\
\underline{kauni}\,+\,ss & \textsc{ sg ela } \\
\underline{kauni}\,+\,llõ & \textsc{ sg all } \\
\underline{kauni}\,+\,ll & \textsc{ sg ade } \\
\underline{kauni}\,+\,lt & \textsc{ sg abl } \\
\underline{kauni}\,+\,ssi & \textsc{ sg tra } \\
\underline{kauni}\,+\,ssaa & \textsc{ sg ter } \\
\underline{kauni}\,+\,ka & \textsc{ sg com } \\
\underline{kauni}\,+\,d & \textsc{ pl nom } \\
\underline{kauni}\,+\,jõ & \textsc{ pl gen } \\
\underline{kauni}\,+\,it & \textsc{ pl par } \\
\underline{kauni}\,+\,isõ & \textsc{ pl ill } \\
\underline{kauni}\,+\,iz & \textsc{ pl ine } \\
\underline{kauni}\,+\,iss & \textsc{ pl ela } \\
\underline{kauni}\,+\,illõ & \textsc{ pl all } \\
\underline{kauni}\,+\,ill & \textsc{ pl ade } \\
\underline{kauni}\,+\,ilt & \textsc{ pl abl } \\
\underline{kauni}\,+\,issi & \textsc{ pl tra } \\
\underline{kauni}\,+\,issaa & \textsc{ pl ter } \\
\underline{kauni}\,+\,jka & \textsc{ pl com } \\
\end{tabular}
\end{sideways}
\captionof{table}{Tüüpsõna \arabic{mallinumber}\,\textit{kauniz} ekstraheeritud muutvormimallid.}
\label{tab:tüüpsõnamall-kauniz}

\end{minipage}

 
\vspace{1em}
\noindent Tüüpsõna ei hõlma teisi lekseeme vormi\-sõnastikus.

\paragraph*{\vadja{\underline{angõri}az}}
\vadja{\underline{angõri}a}, \vadja{\underline{angõri}assõ}, \vadja{\underline{angõri}asõ}, \vadja{\underline{angõri}ass}, \vadja{\underline{angõri}ad}, \vadja{\underline{angõri}jõ}, \vadja{\underline{angõri}it}, \vadja{\underline{angõri}isõ}, \vadja{\underline{angõri}iss}
 \\
Sõnatüüp ei hõlma teisi lekseeme vormi\-sõnastikus.


\vspace{1.8em}
\begin{minipage}{\textwidth}
\stepcounter{mallinumber}
\textbf{Tüüpsõnamall \arabic{mallinumber}\,\vadja{raskõz}}\\

\begin{sideways}
\begin{tabular}{l l}
muutvormimall & tunnused \\
\hline
\underline{rask}\,+\,õz & \textsc{ sg nom } \\
\underline{rask}\,+\,a & \textsc{ sg gen } \\
\underline{rask}\,+\,assõ & \textsc{ sg par } \\
\underline{rask}\,+\,asõ & \textsc{ sg ill } \\
\underline{rask}\,+\,az & \textsc{ sg ine } \\
\underline{rask}\,+\,ass & \textsc{ sg ela } \\
\underline{rask}\,+\,allõ & \textsc{ sg all } \\
\underline{rask}\,+\,all & \textsc{ sg ade } \\
\underline{rask}\,+\,alt & \textsc{ sg abl } \\
\underline{rask}\,+\,assi & \textsc{ sg tra } \\
\underline{rask}\,+\,assaa & \textsc{ sg ter } \\
\underline{rask}\,+\,aka & \textsc{ sg com } \\
\underline{rask}\,+\,ad & \textsc{ pl nom } \\
\underline{rask}\,+\,ojõ & \textsc{ pl gen } \\
\underline{rask}\,+\,ait & \textsc{ pl par } \\
\underline{rask}\,+\,aisõ & \textsc{ pl ill } \\
\underline{rask}\,+\,aiz & \textsc{ pl ine } \\
\underline{rask}\,+\,aiss & \textsc{ pl ela } \\
\underline{rask}\,+\,aillõ & \textsc{ pl all } \\
\underline{rask}\,+\,aill & \textsc{ pl ade } \\
\underline{rask}\,+\,ailt & \textsc{ pl abl } \\
\underline{rask}\,+\,aissi & \textsc{ pl tra } \\
\underline{rask}\,+\,aissaa & \textsc{ pl ter } \\
\underline{rask}\,+\,aika & \textsc{ pl com } \\
\end{tabular}
\end{sideways}
\captionof{table}{Tüüpsõna \arabic{mallinumber}\,\textit{raskõz} ekstraheeritud muutvormimallid.}
\label{tab:tüüpsõnamall-raskõz}

\end{minipage}

 
\vspace{1em}
\noindent Tüüpsõna ei hõlma teisi lekseeme vormi\-sõnastikus.


\vspace{1.8em}
\begin{minipage}{\textwidth}
\stepcounter{mallinumber}
\textbf{Tüüpsõnamall \arabic{mallinumber}\,\vadja{kasõ}}\\

\begin{sideways}
\begin{tabular}{l l}
muutvormimall & tunnused \\
\hline
\underline{kas}\,+\,\underline{õ} & \textsc{ sg nom } \\
\underline{kas}\,+\,s\,+\,\underline{õ} & \textsc{ sg gen } \\
\underline{kas}\,+\,\underline{õ}\,+\,ttõ & \textsc{ sg par } \\
\underline{kas}\,+\,s\,+\,\underline{õ}\,+\,sõ & \textsc{ sg ill } \\
\underline{kas}\,+\,s\,+\,\underline{õ}\,+\,z & \textsc{ sg ine } \\
\underline{kas}\,+\,s\,+\,\underline{õ}\,+\,ss & \textsc{ sg ela } \\
\underline{kas}\,+\,s\,+\,\underline{õ}\,+\,llõ & \textsc{ sg all } \\
\underline{kas}\,+\,s\,+\,\underline{õ}\,+\,ll & \textsc{ sg ade } \\
\underline{kas}\,+\,s\,+\,\underline{õ}\,+\,lt & \textsc{ sg abl } \\
\underline{kas}\,+\,s\,+\,\underline{õ}\,+\,ssi & \textsc{ sg tra } \\
\underline{kas}\,+\,s\,+\,\underline{õ}\,+\,ssaa & \textsc{ sg ter } \\
\underline{kas}\,+\,s\,+\,\underline{õ}\,+\,ka & \textsc{ sg com } \\
\underline{kas}\,+\,s\,+\,\underline{õ}\,+\,d & \textsc{ pl nom } \\
\underline{kas}\,+\,s\,+\,\underline{õ}\,+\,jõ & \textsc{ pl gen } \\
\underline{kas}\,+\,s\,+\,\underline{õ}\,+\,it & \textsc{ pl par } \\
\underline{kas}\,+\,s\,+\,\underline{õ}\,+\,isõ & \textsc{ pl ill } \\
\underline{kas}\,+\,s\,+\,\underline{õ}\,+\,iz & \textsc{ pl ine } \\
\underline{kas}\,+\,s\,+\,\underline{õ}\,+\,iss & \textsc{ pl ela } \\
\underline{kas}\,+\,s\,+\,\underline{õ}\,+\,illõ & \textsc{ pl all } \\
\underline{kas}\,+\,s\,+\,\underline{õ}\,+\,ill & \textsc{ pl ade } \\
\underline{kas}\,+\,s\,+\,\underline{õ}\,+\,ilt & \textsc{ pl abl } \\
\underline{kas}\,+\,s\,+\,\underline{õ}\,+\,issi & \textsc{ pl tra } \\
\underline{kas}\,+\,s\,+\,\underline{õ}\,+\,issaa & \textsc{ pl ter } \\
\underline{kas}\,+\,s\,+\,\underline{õ}\,+\,ika & \textsc{ pl com } \\
\end{tabular}
\end{sideways}
\captionof{table}{Tüüpsõna \arabic{mallinumber}\,\textit{kasõ} ekstraheeritud muutvormimallid.}
\label{tab:tüüpsõnamall-kasõ}

\end{minipage}

 
\vspace{1em}
\noindent Tüüpsõna ei hõlma teisi lekseeme vormi\-sõnastikus.


\vspace{1.8em}
\begin{minipage}{\textwidth}
\stepcounter{mallinumber}
\textbf{Tüüpsõnamall \arabic{mallinumber}\,\vadja{vetelüz}}\\

\begin{sideways}
\begin{tabular}{l l}
muutvormimall & tunnused \\
\hline
\underline{vetelü}\,+\,z & \textsc{ sg nom } \\
\underline{vetelü}\,+\,se & \textsc{ sg gen } \\
\underline{vetelü}\,+\,sse & \textsc{ sg par } \\
\underline{vetelü}\,+\,sse & \textsc{ sg ill } \\
\underline{vetelü}\,+\,sez & \textsc{ sg ine } \\
\underline{vetelü}\,+\,sess & \textsc{ sg ela } \\
\underline{vetelü}\,+\,selle & \textsc{ sg all } \\
\underline{vetelü}\,+\,sell & \textsc{ sg ade } \\
\underline{vetelü}\,+\,selt & \textsc{ sg abl } \\
\underline{vetelü}\,+\,sessi & \textsc{ sg tra } \\
\underline{vetelü}\,+\,sessaa & \textsc{ sg ter } \\
\underline{vetelü}\,+\,seka & \textsc{ sg com } \\
\underline{vetelü}\,+\,sed & \textsc{ pl nom } \\
\underline{vetelü}\,+\,ssije & \textsc{ pl gen } \\
\underline{vetelü}\,+\,ssiit & \textsc{ pl par } \\
\underline{vetelü}\,+\,ssiise & \textsc{ pl ill } \\
\underline{vetelü}\,+\,ssiiz & \textsc{ pl ine } \\
\underline{vetelü}\,+\,ssiiss & \textsc{ pl ela } \\
\underline{vetelü}\,+\,ssiille & \textsc{ pl all } \\
\underline{vetelü}\,+\,ssiill & \textsc{ pl ade } \\
\underline{vetelü}\,+\,ssiilt & \textsc{ pl abl } \\
\underline{vetelü}\,+\,ssiissi & \textsc{ pl tra } \\
\underline{vetelü}\,+\,ssiissaa & \textsc{ pl ter } \\
\underline{vetelü}\,+\,ssijka & \textsc{ pl com } \\
\end{tabular}
\end{sideways}
\captionof{table}{Tüüpsõna \arabic{mallinumber}\,\textit{vetelüz} ekstraheeritud muutvormimallid.}
\label{tab:tüüpsõnamall-vetelüz}

\end{minipage}

 
\vspace{1em}
\noindent Tüüpsõna hõlmab vormisõnastiku 2 lekseemi: \vadja{\underline{vetelü}z} ja \vadja{\underline{jäne}z}.


\vspace{1.8em}
\begin{minipage}{\textwidth}
\stepcounter{mallinumber}
\textbf{Tüüpsõnamall \arabic{mallinumber}\,\vadja{põrzõz}}\\

\begin{sideways}
\begin{tabular}{l l}
muutvormimall & tunnused \\
\hline
\underline{põr}\,+\,zõz & \textsc{ sg nom } \\
\underline{põr}\,+\,sa & \textsc{ sg gen } \\
\underline{põr}\,+\,zassõ & \textsc{ sg par } \\
\underline{põr}\,+\,sasõ & \textsc{ sg ill } \\
\underline{põr}\,+\,saz & \textsc{ sg ine } \\
\underline{põr}\,+\,sass & \textsc{ sg ela } \\
\underline{põr}\,+\,sallõ & \textsc{ sg all } \\
\underline{põr}\,+\,sall & \textsc{ sg ade } \\
\underline{põr}\,+\,salt & \textsc{ sg abl } \\
\underline{põr}\,+\,sassi & \textsc{ sg tra } \\
\underline{põr}\,+\,sassaa & \textsc{ sg ter } \\
\underline{põr}\,+\,saka & \textsc{ sg com } \\
\underline{põr}\,+\,sad & \textsc{ pl nom } \\
\underline{põr}\,+\,sojõ & \textsc{ pl gen } \\
\underline{põr}\,+\,soit & \textsc{ pl par } \\
\underline{põr}\,+\,soisõ & \textsc{ pl ill } \\
\underline{põr}\,+\,soiz & \textsc{ pl ine } \\
\underline{põr}\,+\,soiss & \textsc{ pl ela } \\
\underline{põr}\,+\,soillõ & \textsc{ pl all } \\
\underline{põr}\,+\,soill & \textsc{ pl ade } \\
\underline{põr}\,+\,soilt & \textsc{ pl abl } \\
\underline{põr}\,+\,soissi & \textsc{ pl tra } \\
\underline{põr}\,+\,soissaa & \textsc{ pl ter } \\
\underline{põr}\,+\,soika & \textsc{ pl com } \\
\end{tabular}
\end{sideways}
\captionof{table}{Tüüpsõna \arabic{mallinumber}\,\textit{põrzõz} ekstraheeritud muutvormimallid.}
\label{tab:tüüpsõnamall-põrzõz}

\end{minipage}

 
\vspace{1em}
\noindent Tüüpsõna ei hõlma teisi lekseeme vormi\-sõnastikus.


\vspace{1.8em}
\begin{minipage}{\textwidth}
\stepcounter{mallinumber}
\textbf{Tüüpsõnamall \arabic{mallinumber}\,\vadja{ratiz}}\\

\begin{sideways}
\begin{tabular}{l l}
muutvormimall & tunnused \\
\hline
\underline{rat}\,+\,\underline{i}\,+\,z & \textsc{ sg nom } \\
\underline{rat}\,+\,t\,+\,\underline{i} & \textsc{ sg gen } \\
\underline{rat}\,+\,\underline{i}\,+\,ssõ & \textsc{ sg par } \\
\underline{rat}\,+\,t\,+\,\underline{i}\,+\,sõ & \textsc{ sg ill } \\
\underline{rat}\,+\,t\,+\,\underline{i}\,+\,z & \textsc{ sg ine } \\
\underline{rat}\,+\,t\,+\,\underline{i}\,+\,ss & \textsc{ sg ela } \\
\underline{rat}\,+\,t\,+\,\underline{i}\,+\,llõ & \textsc{ sg all } \\
\underline{rat}\,+\,t\,+\,\underline{i}\,+\,ll & \textsc{ sg ade } \\
\underline{rat}\,+\,t\,+\,\underline{i}\,+\,lt & \textsc{ sg abl } \\
\underline{rat}\,+\,t\,+\,\underline{i}\,+\,ssi & \textsc{ sg tra } \\
\underline{rat}\,+\,t\,+\,\underline{i}\,+\,ssaa & \textsc{ sg ter } \\
\underline{rat}\,+\,t\,+\,\underline{i}\,+\,ka & \textsc{ sg com } \\
\underline{rat}\,+\,t\,+\,\underline{i}\,+\,d & \textsc{ pl nom } \\
\underline{rat}\,+\,t\,+\,\underline{i}\,+\,jõ & \textsc{ pl gen } \\
\underline{rat}\,+\,t\,+\,\underline{i}\,+\,it & \textsc{ pl par } \\
\underline{rat}\,+\,t\,+\,\underline{i}\,+\,isõ & \textsc{ pl ill } \\
\underline{rat}\,+\,t\,+\,\underline{i}\,+\,iz & \textsc{ pl ine } \\
\underline{rat}\,+\,t\,+\,\underline{i}\,+\,iss & \textsc{ pl ela } \\
\underline{rat}\,+\,t\,+\,\underline{i}\,+\,illõ & \textsc{ pl all } \\
\underline{rat}\,+\,t\,+\,\underline{i}\,+\,ill & \textsc{ pl ade } \\
\underline{rat}\,+\,t\,+\,\underline{i}\,+\,ilt & \textsc{ pl abl } \\
\underline{rat}\,+\,t\,+\,\underline{i}\,+\,issi & \textsc{ pl tra } \\
\underline{rat}\,+\,t\,+\,\underline{i}\,+\,issaa & \textsc{ pl ter } \\
\underline{rat}\,+\,t\,+\,\underline{i}\,+\,jka & \textsc{ pl com } \\
\end{tabular}
\end{sideways}
\captionof{table}{Tüüpsõna \arabic{mallinumber}\,\textit{ratiz} ekstraheeritud muutvormimallid.}
\label{tab:tüüpsõnamall-ratiz}

\end{minipage}

 
\vspace{1em}
\noindent Tüüpsõna ei hõlma teisi lekseeme vormi\-sõnastikus.

\paragraph*{\vadja{\underline{kat}\underline{õ}}}
\vadja{\underline{kat}t\underline{õ}}, \vadja{\underline{kat}\underline{õ}ttõ}, \vadja{\underline{kat}t\underline{õ}sõ}, \vadja{\underline{kat}t\underline{õ}ss}, \vadja{\underline{kat}t\underline{õ}d}, \vadja{\underline{kat}t\underline{õ}jõ}, \vadja{\underline{kat}t\underline{õ}it}, \vadja{\underline{kat}t\underline{õ}isõ}, \vadja{\underline{kat}t\underline{õ}iss}
 \\
Sõnatüüp ei hõlma teisi lekseeme vormi\-sõnastikus.


\vspace{1.8em}
\begin{minipage}{\textwidth}
\stepcounter{mallinumber}
\textbf{Tüüpsõnamall \arabic{mallinumber}\,\vadja{kuõ}}\\

\begin{sideways}
\begin{tabular}{l l}
muutvormimall & tunnused \\
\hline
\underline{ku}\,+\,\underline{õ} & \textsc{ sg nom } \\
\underline{ku}\,+\,t\,+\,\underline{õ} & \textsc{ sg gen } \\
\underline{ku}\,+\,\underline{õ}\,+\,ttõ & \textsc{ sg par } \\
\underline{ku}\,+\,t\,+\,\underline{õ}\,+\,sõ & \textsc{ sg ill } \\
\underline{ku}\,+\,tt\,+\,\underline{õ}\,+\,z & \textsc{ sg ine } \\
\underline{ku}\,+\,tt\,+\,\underline{õ}\,+\,ss & \textsc{ sg ela } \\
\underline{ku}\,+\,tt\,+\,\underline{õ}\,+\,llõ & \textsc{ sg all } \\
\underline{ku}\,+\,tt\,+\,\underline{õ}\,+\,ll & \textsc{ sg ade } \\
\underline{ku}\,+\,tt\,+\,\underline{õ}\,+\,lt & \textsc{ sg abl } \\
\underline{ku}\,+\,tt\,+\,\underline{õ}\,+\,ssi & \textsc{ sg tra } \\
\underline{ku}\,+\,tt\,+\,\underline{õ}\,+\,ssaa & \textsc{ sg ter } \\
\underline{ku}\,+\,tt\,+\,\underline{õ}\,+\,ka & \textsc{ sg com } \\
\underline{ku}\,+\,tt\,+\,\underline{õ}\,+\,d & \textsc{ pl nom } \\
\underline{ku}\,+\,tt\,+\,\underline{õ}\,+\,jõ & \textsc{ pl gen } \\
\underline{ku}\,+\,tt\,+\,\underline{õ}\,+\,it & \textsc{ pl par } \\
\underline{ku}\,+\,tt\,+\,\underline{õ}\,+\,isõ & \textsc{ pl ill } \\
\underline{ku}\,+\,tt\,+\,\underline{õ}\,+\,iz & \textsc{ pl ine } \\
\underline{ku}\,+\,tt\,+\,\underline{õ}\,+\,iss & \textsc{ pl ela } \\
\underline{ku}\,+\,tt\,+\,\underline{õ}\,+\,illõ & \textsc{ pl all } \\
\underline{ku}\,+\,tt\,+\,\underline{õ}\,+\,ill & \textsc{ pl ade } \\
\underline{ku}\,+\,tt\,+\,\underline{õ}\,+\,ilt & \textsc{ pl abl } \\
\underline{ku}\,+\,tt\,+\,\underline{õ}\,+\,issi & \textsc{ pl tra } \\
\underline{ku}\,+\,tt\,+\,\underline{õ}\,+\,issaa & \textsc{ pl ter } \\
\underline{ku}\,+\,tt\,+\,\underline{õ}\,+\,ika & \textsc{ pl com } \\
\end{tabular}
\end{sideways}
\captionof{table}{Tüüpsõna \arabic{mallinumber}\,\textit{kuõ} ekstraheeritud muutvormimallid.}
\label{tab:tüüpsõnamall-kuõ}

\end{minipage}

 
\vspace{1em}
\noindent Tüüpsõna ei hõlma teisi lekseeme vormi\-sõnastikus.


\vspace{1.8em}
\begin{minipage}{\textwidth}
\stepcounter{mallinumber}
\textbf{Tüüpsõnamall \arabic{mallinumber}\,\vadja{varvõz}}\\

\begin{sideways}
\begin{tabular}{l l}
muutvormimall & tunnused \\
\hline
\underline{var}\,+\,võz & \textsc{ sg nom } \\
\underline{var}\,+\,pa & \textsc{ sg gen } \\
\underline{var}\,+\,vassõ & \textsc{ sg par } \\
\underline{var}\,+\,pasõ & \textsc{ sg ill } \\
\underline{var}\,+\,paz & \textsc{ sg ine } \\
\underline{var}\,+\,pass & \textsc{ sg ela } \\
\underline{var}\,+\,pallõ & \textsc{ sg all } \\
\underline{var}\,+\,pall & \textsc{ sg ade } \\
\underline{var}\,+\,palt & \textsc{ sg abl } \\
\underline{var}\,+\,passi & \textsc{ sg tra } \\
\underline{var}\,+\,passaa & \textsc{ sg ter } \\
\underline{var}\,+\,paka & \textsc{ sg com } \\
\underline{var}\,+\,pad & \textsc{ pl nom } \\
\underline{var}\,+\,pajõ & \textsc{ pl gen } \\
\underline{var}\,+\,pait & \textsc{ pl par } \\
\underline{var}\,+\,paisõ & \textsc{ pl ill } \\
\underline{var}\,+\,paiz & \textsc{ pl ine } \\
\underline{var}\,+\,paiss & \textsc{ pl ela } \\
\underline{var}\,+\,paillõ & \textsc{ pl all } \\
\underline{var}\,+\,paill & \textsc{ pl ade } \\
\underline{var}\,+\,pailt & \textsc{ pl abl } \\
\underline{var}\,+\,paissi & \textsc{ pl tra } \\
\underline{var}\,+\,paissaa & \textsc{ pl ter } \\
\underline{var}\,+\,paika & \textsc{ pl com } \\
\end{tabular}
\end{sideways}
\captionof{table}{Tüüpsõna \arabic{mallinumber}\,\textit{varvõz} ekstraheeritud muutvormimallid.}
\label{tab:tüüpsõnamall-varvõz}

\end{minipage}

 
\vspace{1em}
\noindent Tüüpsõna hõlmab vormisõnastiku 2 lekseemi: \vadja{\underline{var}võz} ja \vadja{\underline{tur}võz}.

\paragraph*{\vadja{\underline{pere}}}
\vadja{\underline{pere}}, \vadja{\underline{pere}tte}, \vadja{\underline{pere}se}, \vadja{\underline{pere}ss}, \vadja{\underline{pere}d}, \vadja{\underline{pere}je}, \vadja{\underline{pere}it}, \vadja{\underline{pere}ise}, \vadja{\underline{pere}iss}
 \\
Tüüpsõna hõlmab vormisõnastiku 3 lekseemi: \vadja{pere, vene} ja \vadja{erne}.

\spacing{1.5}



\subsection{\RN{15} käändkond}

Viieteistkümnes käändkond koondab sõnu nagu \vadja{lühüd}, \vadja{õhud}, \vadja{koollu}, \vadja{ilozuZ}, \vadja{rikkauZ} \cite[51]{ariste_grammar_1968}.

Avatuid küsimusi-tähelepanekuid:
\begin{itemize}
\item 
\end{itemize}


\subsubsection*{Ekstraktmorfoloogia sõnatüübid}
\spacing{1}
\paragraph*{\vadja{\underline{čämmel}}}
\vadja{\underline{čämmel}e}, \vadja{\underline{čämmel}te}, \vadja{\underline{čämmel}äse}, \vadja{\underline{čämmel}ess}, \vadja{\underline{čämmel}ed}, \vadja{\underline{čämmel}ije}, \vadja{\underline{čämmel}iit}, \vadja{\underline{čämmel}iise}, \vadja{\underline{čämmel}iiss}
 \\
Tüüpsõna ei hõlma teisi lekseeme vormi\-sõnastikus.


\vspace{1.8em}
\begin{minipage}{\textwidth}
\stepcounter{mallinumber}
\textbf{Tüüpsõnamall \arabic{mallinumber}\,\vadja{oonõ}}\\

\begin{sideways}
\begin{tabular}{l l}
muutvormimall & tunnused \\
\hline
\underline{oon}\,+\,õ & \textsc{ sg nom } \\
\underline{oon}\,+\,õ & \textsc{ sg gen } \\
\underline{oon}\,+\,õt & \textsc{ sg par } \\
\underline{oon}\,+\,õsõ & \textsc{ sg ill } \\
\underline{oon}\,+\,õz & \textsc{ sg ine } \\
\underline{oon}\,+\,õss & \textsc{ sg ela } \\
\underline{oon}\,+\,õllõ & \textsc{ sg all } \\
\underline{oon}\,+\,õll & \textsc{ sg ade } \\
\underline{oon}\,+\,õlt & \textsc{ sg abl } \\
\underline{oon}\,+\,õssi & \textsc{ sg tra } \\
\underline{oon}\,+\,õssaa & \textsc{ sg ter } \\
\underline{oon}\,+\,õka & \textsc{ sg com } \\
\underline{oon}\,+\,õd & \textsc{ pl nom } \\
\underline{oon}\,+\,ijõ & \textsc{ pl gen } \\
\underline{oon}\,+\,iit & \textsc{ pl par } \\
\underline{oon}\,+\,iisõ & \textsc{ pl ill } \\
\underline{oon}\,+\,iiz & \textsc{ pl ine } \\
\underline{oon}\,+\,iiss & \textsc{ pl ela } \\
\underline{oon}\,+\,iillõ & \textsc{ pl all } \\
\underline{oon}\,+\,iill & \textsc{ pl ade } \\
\underline{oon}\,+\,iilt & \textsc{ pl abl } \\
\underline{oon}\,+\,iissi & \textsc{ pl tra } \\
\underline{oon}\,+\,iissaa & \textsc{ pl ter } \\
\underline{oon}\,+\,ijka & \textsc{ pl com } \\
\end{tabular}
\end{sideways}
\captionof{table}{Tüüpsõna \arabic{mallinumber}\,\textit{oonõ} ekstraheeritud muutvormimallid.}
\label{tab:tüüpsõnamall-oonõ}

\end{minipage}

 
\vspace{1em}
\noindent Tüüpsõna ei hõlma teisi lekseeme vormi\-sõnastikus.


\vspace{1.8em}
\begin{minipage}{\textwidth}
\stepcounter{mallinumber}
\textbf{Tüüpsõnamall \arabic{mallinumber}\,\vadja{kannõl}}\\

\begin{sideways}
\begin{tabular}{l l}
muutvormimall & tunnused \\
\hline
\underline{kannõl} & \textsc{ sg nom } \\
\underline{kannõl}\,+\,õ & \textsc{ sg gen } \\
\underline{kannõl}\,+\,tõ & \textsc{ sg par } \\
\underline{kannõl}\,+\,asõ & \textsc{ sg ill } \\
\underline{kannõl}\,+\,õz & \textsc{ sg ine } \\
\underline{kannõl}\,+\,õss & \textsc{ sg ela } \\
\underline{kannõl}\,+\,õllõ & \textsc{ sg all } \\
\underline{kannõl}\,+\,õll & \textsc{ sg ade } \\
\underline{kannõl}\,+\,õlt & \textsc{ sg abl } \\
\underline{kannõl}\,+\,õssi & \textsc{ sg tra } \\
\underline{kannõl}\,+\,õssaa & \textsc{ sg ter } \\
\underline{kannõl}\,+\,õka & \textsc{ sg com } \\
\underline{kannõl}\,+\,õd & \textsc{ pl nom } \\
\underline{kannõl}\,+\,ijõ & \textsc{ pl gen } \\
\underline{kannõl}\,+\,iit & \textsc{ pl par } \\
\underline{kannõl}\,+\,iisõ & \textsc{ pl ill } \\
\underline{kannõl}\,+\,iiz & \textsc{ pl ine } \\
\underline{kannõl}\,+\,iiss & \textsc{ pl ela } \\
\underline{kannõl}\,+\,iillõ & \textsc{ pl all } \\
\underline{kannõl}\,+\,iill & \textsc{ pl ade } \\
\underline{kannõl}\,+\,iilt & \textsc{ pl abl } \\
\underline{kannõl}\,+\,iissi & \textsc{ pl tra } \\
\underline{kannõl}\,+\,iissaa & \textsc{ pl ter } \\
\underline{kannõl}\,+\,ijka & \textsc{ pl com } \\
\end{tabular}
\end{sideways}
\captionof{table}{Tüüpsõna \arabic{mallinumber}\,\textit{kannõl} ekstraheeritud muutvormimallid.}
\label{tab:tüüpsõnamall-kannõl}

\end{minipage}

 
\vspace{1em}
\noindent Tüüpsõna ei hõlma teisi lekseeme vormi\-sõnastikus.

\paragraph*{\vadja{\underline{peremme}ez}}
\vadja{\underline{peremme}he}, \vadja{\underline{peremme}esse}, \vadja{\underline{peremme}hese}, \vadja{\underline{peremme}hess}, \vadja{\underline{peremme}hed}, \vadja{\underline{peremme}hije}, \vadja{\underline{peremme}hiit}, \vadja{\underline{peremme}hiise}, \vadja{\underline{peremme}hiiss}
 \\
Tüüpsõna hõlmab vormisõnastiku lekseeme: \vadja{peremmeez, meez}.


\vspace{1.8em}
\begin{minipage}{\textwidth}
\stepcounter{mallinumber}
\textbf{Tüüpsõnamall \arabic{mallinumber}\,\vadja{märännü}}\\

\begin{sideways}
\begin{tabular}{l l}
muutvormimall & tunnused \\
\hline
\underline{märänn}\,+\,ü & \textsc{ sg nom } \\
\underline{märänn}\,+\,ü & \textsc{ sg gen } \\
\underline{märänn}\,+\,üt & \textsc{ sg par } \\
\underline{märänn}\,+\,üse & \textsc{ sg ill } \\
\underline{märänn}\,+\,ez & \textsc{ sg ine } \\
\underline{märänn}\,+\,ess & \textsc{ sg ela } \\
\underline{märänn}\,+\,elle & \textsc{ sg all } \\
\underline{märänn}\,+\,ell & \textsc{ sg ade } \\
\underline{märänn}\,+\,elt & \textsc{ sg abl } \\
\underline{märänn}\,+\,essi & \textsc{ sg tra } \\
\underline{märänn}\,+\,essaa & \textsc{ sg ter } \\
\underline{märänn}\,+\,eka & \textsc{ sg com } \\
\underline{märänn}\,+\,ed & \textsc{ pl nom } \\
\underline{märänn}\,+\,ije & \textsc{ pl gen } \\
\underline{märänn}\,+\,eit & \textsc{ pl par } \\
\underline{märänn}\,+\,eise & \textsc{ pl ill } \\
\underline{märänn}\,+\,eiz & \textsc{ pl ine } \\
\underline{märänn}\,+\,eiss & \textsc{ pl ela } \\
\underline{märänn}\,+\,eille & \textsc{ pl all } \\
\underline{märänn}\,+\,eill & \textsc{ pl ade } \\
\underline{märänn}\,+\,eilt & \textsc{ pl abl } \\
\underline{märänn}\,+\,eissi & \textsc{ pl tra } \\
\underline{märänn}\,+\,eissaa & \textsc{ pl ter } \\
\underline{märänn}\,+\,eika & \textsc{ pl com } \\
\end{tabular}
\end{sideways}
\captionof{table}{Tüüpsõna \arabic{mallinumber}\,\textit{märännü} ekstraheeritud muutvormimallid.}
\label{tab:tüüpsõnamall-märännü}

\end{minipage}

 
\vspace{1em}
\noindent Tüüpsõna ei hõlma teisi lekseeme vormi\-sõnastikus.

\paragraph*{\vadja{\underline{čev}äd}}
\vadja{\underline{čev}vä}, \vadja{\underline{čev}ätte}, \vadja{\underline{čev}ättese}, \vadja{\underline{čev}äss}, \vadja{\underline{čev}äd}, \vadja{\underline{čev}vije}, \vadja{\underline{čev}viit}, \vadja{\underline{čev}viise}, \vadja{\underline{čev}viiss}
 \\
Sõnatüüp ei hõlma teisi lekseeme vormi\-sõnastikus.


\vspace{1.8em}
\begin{minipage}{\textwidth}
\stepcounter{mallinumber}
\textbf{Tüüpsõnamall \arabic{mallinumber}\,\vadja{čäčüd}}\\

\begin{sideways}
\begin{tabular}{l l}
muutvormimall & tunnused \\
\hline
\underline{čäč}\,+\,\underline{ü}\,+\,d & \textsc{ sg nom } \\
\underline{čäč}\,+\,č\,+\,\underline{ü} & \textsc{ sg gen } \\
\underline{čäč}\,+\,\underline{ü}\,+\,tt & \textsc{ sg par } \\
\underline{čäč}\,+\,č\,+\,\underline{ü}\,+\,se & \textsc{ sg ill } \\
\underline{čäč}\,+\,č\,+\,\underline{ü}\,+\,z & \textsc{ sg ine } \\
\underline{čäč}\,+\,č\,+\,\underline{ü}\,+\,ss & \textsc{ sg ela } \\
\underline{čäč}\,+\,č\,+\,\underline{ü}\,+\,lle & \textsc{ sg all } \\
\underline{čäč}\,+\,č\,+\,\underline{ü}\,+\,ll & \textsc{ sg ade } \\
\underline{čäč}\,+\,č\,+\,\underline{ü}\,+\,lt & \textsc{ sg abl } \\
\underline{čäč}\,+\,č\,+\,\underline{ü}\,+\,ssi & \textsc{ sg tra } \\
\underline{čäč}\,+\,č\,+\,\underline{ü}\,+\,ssaa & \textsc{ sg ter } \\
\underline{čäč}\,+\,č\,+\,\underline{ü}\,+\,ka & \textsc{ sg com } \\
\underline{čäč}\,+\,č\,+\,\underline{ü}\,+\,d & \textsc{ pl nom } \\
\underline{čäč}\,+\,č\,+\,\underline{ü}\,+\,je & \textsc{ pl gen } \\
\underline{čäč}\,+\,č\,+\,\underline{ü}\,+\,it & \textsc{ pl par } \\
\underline{čäč}\,+\,č\,+\,\underline{ü}\,+\,ise & \textsc{ pl ill } \\
\underline{čäč}\,+\,č\,+\,\underline{ü}\,+\,iz & \textsc{ pl ine } \\
\underline{čäč}\,+\,č\,+\,\underline{ü}\,+\,iss & \textsc{ pl ela } \\
\underline{čäč}\,+\,č\,+\,\underline{ü}\,+\,ille & \textsc{ pl all } \\
\underline{čäč}\,+\,č\,+\,\underline{ü}\,+\,ill & \textsc{ pl ade } \\
\underline{čäč}\,+\,č\,+\,\underline{ü}\,+\,ilt & \textsc{ pl abl } \\
\underline{čäč}\,+\,č\,+\,\underline{ü}\,+\,issi & \textsc{ pl tra } \\
\underline{čäč}\,+\,č\,+\,\underline{ü}\,+\,issaa & \textsc{ pl ter } \\
\underline{čäč}\,+\,č\,+\,\underline{ü}\,+\,ika & \textsc{ pl com } \\
\end{tabular}
\end{sideways}
\captionof{table}{Tüüpsõna \arabic{mallinumber}\,\textit{čäčüd} ekstraheeritud muutvormimallid.}
\label{tab:tüüpsõnamall-čäčüd}

\end{minipage}

 
\vspace{1em}
\noindent Tüüpsõna ei hõlma teisi lekseeme vormi\-sõnastikus.

\paragraph*{\vadja{\underline{ivus}õd}}
\vadja{\underline{ivus}sijõ}, \vadja{\underline{ivus}siit}, \vadja{\underline{ivus}sisõ}, \vadja{\underline{ivus}õss}, \vadja{\underline{ivus}õd}, \vadja{\underline{ivus}sijõ}, \vadja{\underline{ivus}siit}, \vadja{\underline{ivus}siisõ}, \vadja{\underline{ivus}siiss}
 \\
Sõnatüüp ei hõlma teisi lekseeme vormi\-sõnastikus.

\paragraph*{\vadja{\underline{kaat}\underline{s}õd}}
\vadja{\underline{kaat}t\underline{s}ojõ}, \vadja{\underline{kaat}t\underline{s}oit}, \vadja{\underline{kaat}t\underline{s}oisõ}, \vadja{\underline{kaat}\underline{s}õss}, \vadja{\underline{kaat}\underline{s}õd}, \vadja{\underline{kaat}\underline{s}ojõ}, \vadja{\underline{kaat}t\underline{s}oit}, \vadja{\underline{kaat}t\underline{s}oisõ}, \vadja{\underline{kaat}t\underline{s}oiss}
 \\
Tüüpsõna ei hõlma teisi lekseeme vormi\-sõnastikus.

\paragraph*{\vadja{\underline{kooll}ud}}
\vadja{\underline{kooll}õ}, \vadja{\underline{kooll}uttõ}, \vadja{\underline{kooll}õsõ}, \vadja{\underline{kooll}õss}, \vadja{\underline{kooll}õd}, \vadja{\underline{kooll}ujõ}, \vadja{\underline{kooll}uit}, \vadja{\underline{kooll}uisõ}, \vadja{\underline{kooll}uiss}
 \\
Tüüpsõna ei hõlma teisi lekseeme vormi\-sõnastikus.

\paragraph*{\vadja{\underline{koor}r\underline{õ}}}
\vadja{\underline{koor}r\underline{õ}}, \vadja{\underline{koor}r\underline{õ}t}, \vadja{\underline{koor}t\underline{õ}sõ}, \vadja{\underline{koor}r\underline{õ}ss}, \vadja{\underline{koor}r\underline{õ}d}, \vadja{\underline{koor}t\underline{õ}jõ}, \vadja{\underline{koor}t\underline{õ}it}, \vadja{\underline{koor}t\underline{õ}isõ}, \vadja{\underline{koor}t\underline{õ}iss}
 \\
sõnatüüp ei hõlma teisi lekseeme


\vspace{1.8em}
\begin{minipage}{\textwidth}
\stepcounter{mallinumber}
\textbf{Tüüpsõnamall \arabic{mallinumber}\,\vadja{kõlmõd}}\\

\begin{sideways}
\begin{tabular}{l l}
muutvormimall & tunnused \\
\hline
\underline{kõlm}\,+\,õd & \textsc{ sg nom } \\
\underline{kõlm}\,+\,õ & \textsc{ sg gen } \\
\underline{kõlm}\,+\,a & \textsc{ sg par } \\
\underline{kõlm}\,+\,õsõ & \textsc{ sg ill } \\
\underline{kõlm}\,+\,õz & \textsc{ sg ine } \\
\underline{kõlm}\,+\,õss & \textsc{ sg ela } \\
\underline{kõlm}\,+\,õllõ & \textsc{ sg all } \\
\underline{kõlm}\,+\,õll & \textsc{ sg ade } \\
\underline{kõlm}\,+\,õlt & \textsc{ sg abl } \\
\underline{kõlm}\,+\,õssi & \textsc{ sg tra } \\
\underline{kõlm}\,+\,õssaa & \textsc{ sg ter } \\
\underline{kõlm}\,+\,õka & \textsc{ sg com } \\
\underline{kõlm}\,+\,õd & \textsc{ pl nom } \\
\underline{kõlm}\,+\,ijõ & \textsc{ pl gen } \\
\underline{kõlm}\,+\,iit & \textsc{ pl par } \\
\underline{kõlm}\,+\,iisõ & \textsc{ pl ill } \\
\underline{kõlm}\,+\,iiz & \textsc{ pl ine } \\
\underline{kõlm}\,+\,iiss & \textsc{ pl ela } \\
\underline{kõlm}\,+\,iillõ & \textsc{ pl all } \\
\underline{kõlm}\,+\,iill & \textsc{ pl ade } \\
\underline{kõlm}\,+\,iilt & \textsc{ pl abl } \\
\underline{kõlm}\,+\,iissi & \textsc{ pl tra } \\
\underline{kõlm}\,+\,iissaa & \textsc{ pl ter } \\
\underline{kõlm}\,+\,ijka & \textsc{ pl com } \\
\end{tabular}
\end{sideways}
\captionof{table}{Tüüpsõna \arabic{mallinumber}\,\textit{kõlmõd} ekstraheeritud muutvormimallid.}
\label{tab:tüüpsõnamall-kõlmõd}

\end{minipage}

 
\vspace{1em}
\noindent Tüüpsõna ei hõlma teisi lekseeme vormi\-sõnastikus.


\vspace{1.8em}
\begin{minipage}{\textwidth}
\stepcounter{mallinumber}
\textbf{Tüüpsõnamall \arabic{mallinumber}\,\vadja{olud}}\\

\begin{sideways}
\begin{tabular}{l l}
muutvormimall & tunnused \\
\hline
\underline{olu}\,+\,d & \textsc{ sg nom } \\
\underline{olu} & \textsc{ sg gen } \\
\underline{olu}\,+\,ttõ & \textsc{ sg par } \\
\underline{olu}\,+\,sõ & \textsc{ sg ill } \\
\underline{olu}\,+\,z & \textsc{ sg ine } \\
\underline{olu}\,+\,ss & \textsc{ sg ela } \\
\underline{olu}\,+\,llõ & \textsc{ sg all } \\
\underline{olu}\,+\,ll & \textsc{ sg ade } \\
\underline{olu}\,+\,lt & \textsc{ sg abl } \\
\underline{olu}\,+\,ssi & \textsc{ sg tra } \\
\underline{olu}\,+\,ssaa & \textsc{ sg ter } \\
\underline{olu}\,+\,ka & \textsc{ sg com } \\
\underline{olu}\,+\,d & \textsc{ pl nom } \\
\underline{olu}\,+\,jõ & \textsc{ pl gen } \\
\underline{olu}\,+\,it & \textsc{ pl par } \\
\underline{olu}\,+\,isõ & \textsc{ pl ill } \\
\underline{olu}\,+\,iz & \textsc{ pl ine } \\
\underline{olu}\,+\,iss & \textsc{ pl ela } \\
\underline{olu}\,+\,illõ & \textsc{ pl all } \\
\underline{olu}\,+\,ill & \textsc{ pl ade } \\
\underline{olu}\,+\,ilt & \textsc{ pl abl } \\
\underline{olu}\,+\,issi & \textsc{ pl tra } \\
\underline{olu}\,+\,issaa & \textsc{ pl ter } \\
\underline{olu}\,+\,ika & \textsc{ pl com } \\
\end{tabular}
\end{sideways}
\captionof{table}{Tüüpsõna \arabic{mallinumber}\,\textit{olud} ekstraheeritud muutvormimallid.}
\label{tab:tüüpsõnamall-olud}

\end{minipage}

 
\vspace{1em}
\noindent Tüüpsõna ei hõlma teisi lekseeme vormi\-sõnastikus.


\vspace{1.8em}
\begin{minipage}{\textwidth}
\stepcounter{mallinumber}
\textbf{Tüüpsõnamall \arabic{mallinumber}\,\vadja{toho}}\\

\begin{sideways}
\begin{tabular}{l l}
muutvormimall & tunnused \\
\hline
\underline{toh}\,+\,o & \textsc{ sg nom } \\
\underline{toh}\,+\,o & \textsc{ sg gen } \\
\underline{toh}\,+\,tõ & \textsc{ sg par } \\
\underline{toh}\,+\,tosõ & \textsc{ sg ill } \\
\underline{toh}\,+\,oz & \textsc{ sg ine } \\
\underline{toh}\,+\,oss & \textsc{ sg ela } \\
\underline{toh}\,+\,ollõ & \textsc{ sg all } \\
\underline{toh}\,+\,oll & \textsc{ sg ade } \\
\underline{toh}\,+\,olt & \textsc{ sg abl } \\
\underline{toh}\,+\,ossi & \textsc{ sg tra } \\
\underline{toh}\,+\,ossaa & \textsc{ sg ter } \\
\underline{toh}\,+\,oka & \textsc{ sg com } \\
\underline{toh}\,+\,od & \textsc{ pl nom } \\
\underline{toh}\,+\,ojõ & \textsc{ pl gen } \\
\underline{toh}\,+\,oit & \textsc{ pl par } \\
\underline{toh}\,+\,oisõ & \textsc{ pl ill } \\
\underline{toh}\,+\,oiz & \textsc{ pl ine } \\
\underline{toh}\,+\,oiss & \textsc{ pl ela } \\
\underline{toh}\,+\,oillõ & \textsc{ pl all } \\
\underline{toh}\,+\,oill & \textsc{ pl ade } \\
\underline{toh}\,+\,oilt & \textsc{ pl abl } \\
\underline{toh}\,+\,oissi & \textsc{ pl tra } \\
\underline{toh}\,+\,oissaa & \textsc{ pl ter } \\
\underline{toh}\,+\,oika & \textsc{ pl com } \\
\end{tabular}
\end{sideways}
\captionof{table}{Tüüpsõna \arabic{mallinumber}\,\textit{toho} ekstraheeritud muutvormimallid.}
\label{tab:tüüpsõnamall-toho}

\end{minipage}

 
\vspace{1em}
\noindent Tüüpsõna hõlmab vormisõnastiku 2 lekseemi: \vadja{\underline{toh}o} ja \vadja{\underline{roh}o}.

\spacing{1.5}


% \subsection{Ekstraktmorfoloogiaga leitud tüüpsõnad}
% 
% See alaosa loendab leitud tüüpsõnad sõnaliigiti. Analüüsitakse tüüpsõnade alla kuuluvaid sõnu struktuurselt (kui mitu silpi, silpide struktuur).
% 
% (Analüüsidest on võimalik luua arvutikirjeldus hüpoteetilise vadjakeelse sõna üle õigekirjakontrollija jaoks.)
% 
% \paragraph*{bad̕d̕õ} kuulub käändkonda \RN{3} sest \msd{pl} -oi-
% \paragraph*{airo} kuulub käändkonda \RN{2} sest \msd{sg par} lõpp on -oa või -ua
% \paragraph*{bagaži} kuulub käändkonda \RN{2} ja \msd{pl} on -ii-
% \paragraph*{bank} kuulub käändkonda \RN{3} kuigi esimeses silbis esineb ka \vadja{u} peale \vadja{a, õ ja i} NB! \vadja{bank} ehk ei kuulugi siia või on paha nimetaja
% \paragraph*{čimolain} kuulub käändkonda \RN{12}
% \paragraph*{fartukkõ} kuulub käändkonda \RN{3} sest \msd{pl} -oi-
% \paragraph*{duumõ} kuulub käändkonda \RN{5} aga mida teha \msd{pl} -ii- (Aristel on -õi-)
% \paragraph*{flakku} kuulub käändkonda \RN{2} sest \msd{sg par} lõpp on -oa või -ua
% \paragraph*{baldõhina} kuulub käändkonda \RN{3} aga \vadja{suma} ei peaks siin olema?
% \paragraph*{bankrutti} kuulub käändkonda \RN{2}
% \paragraph*{eine} kuulub käändkonda \RN{8} -ä-tüvevokaaliga NB! aga Aristel \vadja{eine} hoopis \RN{4}, kus \msd{pl} on tagapoolne -oi-
% \paragraph*{greebeni} kuulub käändkonda \RN{2}
% \paragraph*{ivuz} kuulub käändkonda \RN{11} ja Jõgõperä moodi on vokaalitüvel -s-, mitte -hs- ega -ss-
% \paragraph*{aikõ} kuulub käändkonda \RN{3} 
% \paragraph*{hattu} kuulub käändkonda \RN{2}
% \paragraph*{sarvi} kuulub käändkonda \RN{7} aga -i:-õ:-ia, Aristel -i:-õ:-õa
% \paragraph*{alku} kuulub käändkonda \RN{2}
% \paragraph*{koivuin} kuulub käändkonda \RN{12}
% \paragraph*{lako} kuulub käändkonda \RN{2}
% \paragraph*{hoolitoi} kuulub käändkonda \RN{13}
% \paragraph*{irvi} kuulub käändkonda \RN{7}
% \paragraph*{kokki} kuulub käändkonda \RN{2} aga \msd{pl} on -ii-?
% \paragraph*{juuri} kuulub käändkonda \RN{10}
% \paragraph*{kraaskõ} kuulub käändkonda \RN{3}
% \paragraph*{jaanikukkõ} kuulub käändkonda \RN{5} aga mida teha -ii- Aristel on -õi-
% \paragraph*{iiri} kuulub käändkonda \RN{10}
% \paragraph*{eglin} kuulub käändkonda \RN{12}
% \paragraph*{liivõkõz} kuulub käändkonda \RN{14}(?)
% \paragraph*{lamppi} kuulub käändkonda \RN{2}
% \paragraph*{mato} kuulub käändkonda \RN{2} (või \RN{9}?)
% \paragraph*{kittsi} kuulub käändkonda \RN{2}
% \paragraph*{inostranttsõ} kuulub käändkonda \RN{3} NB! \msd{sg ill} valesti -sasõ?
% \paragraph*{karjušši} kuulub käändkonda \RN{2}
% \paragraph*{ikolookkõ} kuulub käändkonda \RN{5}
% \paragraph*{kanka} kuulub käändkonda \RN{6} ehk -\vadja{õa}-lõpulised
% \paragraph*{jäin} kuulub käändkonda \RN{12}
% \paragraph*{kompjutera} kuulub käändkonda \RN{5} kas ühtlustada -õ:-a:-a
% \paragraph*{murhõ} kuulub käändkonda \RN{10}
% \paragraph*{lento} kuulub käändkonda \RN{2}
% \paragraph*{lippu} kuulub käändkonda \RN{2}
% \paragraph*{riittõ} kuulub käändkonda \RN{3}
% \paragraph*{rusko} kuulub käändkonda \RN{2}
% \paragraph*{tüü} kuulub käändkonda \RN{1}
% \paragraph*{pal̕l̕õz} kuulub käändkonda \RN{14}
% \paragraph*{hammõz} kuulub käändkonda \RN{14}
% \paragraph*{pää} kuulub käändkonda \RN{1}
% \paragraph*{magnetti} kuulub käändkonda \RN{2}
% \paragraph*{mõiznikkõ} kuulub käändkonda \RN{3}
% \paragraph*{pere} kuulub käändkonda \RN{14} või \RN{6}(?) (\vadja{erne} ja \vadja{pere} kindlasti eri etümoloogiatega)
% \paragraph*{valka} kuulub käändkonda \RN{6}
% \paragraph*{kotko} kuulub käändkonda \RN{2}
% \paragraph*{propkõ} kuulub käändkonda \RN{5}
% \paragraph*{pehmiä} kuulub käändkonda \RN{6}
% \paragraph*{pihlpuu} kuulub käändkonda \RN{1}
% \paragraph*{meri} kuulub käändkonda \RN{10} NB! ühtlustada \msd{sg par} vokaaliga
% \paragraph*{süsi} kuulub käändkonda \RN{7}
% \paragraph*{sata} kuulub käändkonda \RN{3}
% \paragraph*{rapa} kuulub käändkonda \RN{3}
% \paragraph*{rätte} kuulub käändkonda \RN{2}
% \paragraph*{rantõ} kuulub käändkonda \RN{3}
% \paragraph*{trubõ} kuulub käändkonda \RN{3}
% \paragraph*{l̕iitkõ} kuulub käändkonda \RN{3}
% \paragraph*{rissisä, česä} kuulub käändkonda \RN{11} NB! \vadja{rissisä} teha \vadja{isä} järgi
% \paragraph*{musikõz} kuulub käändkonda \RN{14} või peaks ümber tegema \RN{15} järgi?
% \paragraph*{nenä, čülä} kuulub käändkonda \RN{8}
% \paragraph*{läkine} kuulub käändkonda \RN{8}
% \paragraph*{rissimä, emä} kuulub käändkonda \RN{8}
% \paragraph*{pliittõ} kuulub käändkonda \RN{3}
% \paragraph*{kaamenšikka} kuulub käändkonda \RN{3}
% \paragraph*{pähčen, ičäv} kuulub käändkonda \RN{8} ???
% \paragraph*{ülči, jälči} kuulub käändkonda \RN{2}
% \paragraph*{vetelüz, jänez} kuulub käändkonda \RN{14}
% \paragraph*{löülü, jürü} kuulub käändkonda \RN{2}
% \paragraph*{pen̕sioner} kuulub käändkonda \RN{3} (mil moel erineb \vadja{bank}ast?)
% \paragraph*{mussõ} kuulub käändkonda \RN{5} kuigi on -õi- mitte -ii-
% \paragraph*{liippõ} kuulub käändkonda \RN{3}
% \paragraph*{koomikk} kuulub käändkonda \RN{3} mis \msd{sg nom} lõpuga teha?
% \paragraph*{või} kuulub käändkonda \RN{1}
% \paragraph*{passi} kuulub käändkonda \RN{5}
% \paragraph*{moškõ} kuulub käändkonda \RN{5}
% \paragraph*{õnki} kuulub käändkonda \RN{5}
% \paragraph*{kõva} kuulub käändkonda \RN{6} (\vadja{kõrka} kuulub, aga \vadja{kõva} \msd{sg par} võiks muuta?
% \paragraph*{partõ} kuulub käändkonda \RN{3}
% \paragraph*{tečejä} kuulub käändkonda \RN{8}
% \paragraph*{uhsi} kuulub käändkonda \RN{10}
% \paragraph*{poutõ} kuulub käändkonda \RN{3}
% \paragraph*{seppe} kuulub käändkonda \RN{8}
% \paragraph*{suukkurliivõ} kuulub käändkonda \RN{3}
% \paragraph*{noori, lõhi} kuulub käändkonda \RN{10}
% \paragraph*{poikõ} kuulub käändkonda \RN{5}
% \paragraph*{lähe} kuulub käändkonda \RN{14}
% \paragraph*{lähe} kuulub käändkonda \RN{14} valida emb-kumb
% \paragraph*{läsijõ} kuulub käändkonda \RN{8} kas ühtlustada eespoolseks niku \vadja{tečejä}
% \paragraph*{läsijõ} kuulub käändkonda \RN{8} või ühtlustada tagapoolseks \vadja{-ja} liiteks?
% \paragraph*{muna} kuulub käändkonda \RN{5} kuigi -õi- mitte -ii-
% \paragraph*{musikko} kuulub käändkonda \RN{2}
% \paragraph*{peremmeez} kuulub käändkonda \RN{15}
% \paragraph*{trubačist, mokom} kuulub käändkonda \RN{5}
% \paragraph*{põlto} kuulub käändkonda \RN{2}
% \paragraph*{märännü} kuulub käändkonda \RN{15} või
% \paragraph*{märännü} kuulub käändkonda \RN{2} emb-kumb valida
% \paragraph*{siso} kuulub käändkonda \RN{2}
% \paragraph*{sooli} kuulub käändkonda \RN{10} ühtlustada \msd{sg par} lõpud
% \paragraph*{vaahto} kuulub käändkonda \RN{2}
% \paragraph*{vilppi} kuulub käändkonda \RN{2}
% \paragraph*{võõrõz} kuulub käändkonda \RN{14}
% \paragraph*{tauti} kuulub käändkonda \RN{2} vali emb-kumb tüvemuutus
% \paragraph*{tauti} kuulub käändkonda \RN{2} vali emb-kumb
% \paragraph*{vattsõ} kuulub käändkonda \RN{3}
% \paragraph*{voosi} kuulub käändkonda \RN{10}
% \paragraph*{aapõ} kuulub käändkonda \RN{3}
% \paragraph*{ahas} kuulub käändkonda \RN{14}
% \paragraph*{aitõ} kuulub käändkonda \RN{3}
% \paragraph*{aloi} kuulub käändkonda \RN{8} ??? \RN{5} jään hätta \msd{sg nom} \vadja{alojõ}? vrd \vadja{kitai}
% \paragraph*{alõin} kuulub käändkonda \RN{12} ???
% \paragraph*{angõriaz} kuulub käändkonda \RN{14} ühtlüstada tüvevokaali
% \paragraph*{auči} kuulub käändkonda \RN{2}
% \paragraph*{baabukõz} kuulub käändkonda \RN{14} ühtlüstada tüvevokaali
% \paragraph*{bašmuk} kuulub käändkonda \RN{3}
% \paragraph*{bašn̕i} kuulub käändkonda \RN{2} sest on palataliseeritud???
% \paragraph*{biblioteek} kuulub käändkonda \RN{5}
% \paragraph*{biskvittõ} kuulub käändkonda \RN{5}
% \paragraph*{borovikkõ} kuulub käändkonda \RN{3}
% \paragraph*{bruuss} kuulub käändkonda \RN{14} või \vadja{bruussõ} \RN{3}?
% \paragraph*{bukvõ} kuulub käändkonda \RN{5}
% \paragraph*{bul̕bukõz} kuulub käändkonda \RN{14}
% \paragraph*{čenče} kuulub käändkonda \RN{8}
% \paragraph*{čeväd} kuulub käändkonda \RN{15}
% \paragraph*{čämmel} kuulub käändkonda \RN{15}
% \paragraph*{čäsi} kuulub käändkonda \RN{10}
% \paragraph*{čäčüd} kuulub käändkonda \RN{15} Ariste mainib, et Jõgõperäl \msd{sg nom} vorm \vadja{čäčü}
% \paragraph*{čümme} kuulub käändkonda \RN{13} või \RN{10} NB! aga \msd{sg par} muuta \vadja{-ntä}? ja \msd{sg gen} peaks lõppema -ne?
% \paragraph*{čümmenäz} kuulub käändkonda \RN{13} NB! ordinaalid >3 ühtlustada selle käändkonna järgi
% \paragraph*{dovariššõ} kuulub käändkonda \RN{3} kas \msd{sg ine} ka gemineerub?
% \paragraph*{enči} kuulub käändkonda \RN{7}
% \paragraph*{esimein} kuulub käändkonda \RN{12}
% \paragraph*{famil̕} kuulub käändkonda \RN{2} sest on palataliseeritud???
% \paragraph*{fartõl} kuulub käändkonda \RN{3} NB! \msd{sg ill} peab valesti olema?
% \paragraph*{fookusnik} kuulub käändkonda \RN{3}
% \paragraph*{fotokartočka} kuulub käändkonda \RN{3} või peaks \msd{pl} muutma \RN{5} järgi?
% \paragraph*{fraak} kuulub käändkonda \RN{5}
% \paragraph*{frikad̕el̕k} kuulub käändkonda \RN{3} NB! kas muuta astmevahelduslikuks?
% \paragraph*{haisu} kuulub käändkonda \RN{2}
% \paragraph*{hapo} kuulub käändkonda \RN{2} aga \msd{sg nom} nõrgas astmes sest *<~hapan?
% \paragraph*{häülütüs} kuulub käändkonda \RN{14}
% \paragraph*{hüppü} kuulub käändkonda \RN{2}
% \paragraph*{ičä} kuulub käändkonda \RN{8} kuigi võiks olla \RN{9}
% \paragraph*{iloin} kuulub käändkonda \RN{12}
% \paragraph*{itikkõ} kuulub käändkonda \RN{3}
% \paragraph*{ivusõd} kuulub käändkonda \RN{15} ??? aga mul on plurale tantum?
% \paragraph*{joožikkõ} kuulub käändkonda \RN{3} NB! mul on \msd{pl} valesti nõrgas astmes?
% \paragraph*{jõki} kuulub käändkonda \RN{7} aga miks mitte \RN{2}?
% \paragraph*{jõutu} kuulub käändkonda \RN{3}
% \paragraph*{järčü} kuulub käändkonda \RN{3}
% \paragraph*{kaani} kuulub käändkonda \RN{10} aga kas \msd{sg gen} tüvevokaal muutub?
% \paragraph*{kaatsõd} kuulub käändkonda \RN{15} ??? aga mul on plurale tantum?
% \paragraph*{kahs} kuulub käändkonda \RN{10} kas parem oleks siiski \vadja{kahsi}?
% \paragraph*{kahõsa} kuulub käändkonda \RN{13} ?? vt ka Rozhanskiyst üle
% \paragraph*{kal̕indora} kuulub käändkonda \RN{5} või \RN{9} aga -õi- mitte -ii-? kas kõik -õi- hoopis \RN{9} alla??
% \paragraph*{kalliz} kuulub käändkonda \RN{14}
% \paragraph*{kamal̕ikka} kuulub käändkonda \RN{3}
% \paragraph*{kand̕idaat} kuulub käändkonda \RN{3}
% \paragraph*{kangõz} kuulub käändkonda \RN{14}
% \paragraph*{kanki} kuulub käändkonda \RN{2}
% \paragraph*{kannõl} kuulub käändkonda \RN{14}
% \paragraph*{kant} kuulub käändkonda \RN{3}
% \paragraph*{kasõ} kuulub käändkonda \RN{14}
% \paragraph*{katol̕ikk} kuulub käändkonda \RN{3}
% \paragraph*{katõ} kuulub käändkonda \RN{14}
% \paragraph*{kauniz} kuulub käändkonda \RN{14} NB! muuta Tsvetkovi -e-lõpp -i vastu nagu (Konkovalgi) AGA mida teha \msd{pl} tüvevokaaliga?
% \paragraph*{kaõ} kuulub käändkonda \RN{10} NB! kuigi \msd{sg nom} pole -i-lõpuline!
% \paragraph*{kerkä} kuulub käändkonda \RN{6}
% \paragraph*{keskolin} kuulub käändkonda \RN{12}
% \paragraph*{kitai} kuulub käändkonda \RN{8}
% \paragraph*{klaass} kuulub käändkonda \RN{3}
% \paragraph*{koffi} kuulub käändkonda \RN{2}
% \paragraph*{kolaus} kuulub käändkonda \RN{5}
% \paragraph*{komit̕et} kuulub käändkonda \RN{2}
% \paragraph*{koollud} kuulub käändkonda \RN{15} aga kas tüvevokaali peaks kuidagi ühtlustama?
% \paragraph*{koori} kuulub käändkonda \RN{2}
% \paragraph*{koorrõ} kuulub käändkonda \RN{15} ??? mida \msd{sg par} lõpuga teha?
% \paragraph*{kuha} kuulub käändkonda \RN{5} ???
% \paragraph*{kultõ} kuulub käändkonda \RN{5} või \RN{9} ??
% \paragraph*{kumpõ} kuulub käändkonda \RN{5} aga \msd{pl}-tüvi ühtlustada pikaks?
% \paragraph*{kurkku} kuulub käändkonda \RN{2}
% \paragraph*{kurp} kuulub käändkonda \RN{5}
% \paragraph*{kursi} kuulub käändkonda \RN{2}
% \paragraph*{kusi} kuulub käändkonda \RN{10} NB! ühtlusta \msd{sg ill}
% \paragraph*{kutõ} kuulub käändkonda \RN{14} ??
% \paragraph*{kuus} kuulub käändkonda \RN{7}
% \paragraph*{kuusi} kuulub käändkonda \RN{7}
% \paragraph*{kuuvvaiz} kuulub käändkonda \RN{13} NB! ühtlustada
% \paragraph*{kuõ} kuulub käändkonda \RN{14}
% \paragraph*{kõik} kuulub käändkonda \RN{5}
% \paragraph*{kõlmaz} kuulub käändkonda \RN{13} NB! ühtlustada
% \paragraph*{kõlmõd} kuulub käändkonda \RN{15} NB! \msd{sg par} muuta \vadja{kõlmõttõ}
% \paragraph*{lafkõ} kuulub käändkonda \RN{3}
% \paragraph*{lahti} kuulub käändkonda \RN{2} NB! lühike illatiiv muuta \vadja{lahtisõ}?
% \paragraph*{laki} kuulub käändkonda \RN{5}
% \paragraph*{lehto} kuulub käändkonda \RN{2}
% \paragraph*{leipe} kuulub käändkonda \RN{8}
% \paragraph*{lisä} kuulub käändkonda \RN{8} vrd \vadja{isä}?
% \paragraph*{lootõ} kuulub käändkonda \RN{5}
% \paragraph*{luiskõ} kuulub käändkonda \RN{5}
% \paragraph*{lumi} kuulub käändkonda \RN{10}
% \paragraph*{luukkõ} kuulub käändkonda \RN{9}?? -õi-mitmus
% \paragraph*{lõunõ} kuulub käändkonda \RN{14}??
% \paragraph*{läikk} kuulub käändkonda \RN{2}
% \paragraph*{läikkiv} kuulub käändkonda \RN{8}
% \paragraph*{läsive} kuulub käändkonda \RN{8}
% \paragraph*{läsü} kuulub käändkonda \RN{2}
% \paragraph*{lühüd} kuulub käändkonda \RN{15}
% \paragraph*{maa} kuulub käändkonda \RN{1}
% \paragraph*{magnettiin} kuulub käändkonda \RN{12}?
% \paragraph*{mahsu} kuulub käändkonda \RN{2}
% \paragraph*{mahsõ} kuulub käändkonda \RN{3}
% \paragraph*{main} kuulub käändkonda \RN{12}
% \paragraph*{makka} kuulub käändkonda \RN{6}
% \paragraph*{makuz} kuulub käändkonda \RN{11}
% \paragraph*{mansikõz} kuulub käändkonda \RN{14}
% \paragraph*{mettse} kuulub käändkonda \RN{8}
% \paragraph*{moodõ} kuulub käändkonda \RN{9}?? -õi-mitmus?
% \paragraph*{mõlõpi} kuulub käändkonda \RN{} 
% \paragraph*{mäči} kuulub käändkonda \RN{7}
% \paragraph*{märče} kuulub käändkonda \RN{8}
% \paragraph*{mätä} kuulub käändkonda \RN{8}
% \paragraph*{müüjõ} kuulub käändkonda \RN{8} kas ühtlustada -jõ eespoolseks?
% \paragraph*{nagriz} kuulub käändkonda \RN{11}
% \paragraph*{nagru} kuulub käändkonda \RN{2}
% \paragraph*{nel̕l̕äz} kuulub käändkonda \RN{13}
% \paragraph*{nenäkõz} kuulub käändkonda \RN{14} kas ühtlustada eespoolseks?
% \paragraph*{nõki} kuulub käändkonda \RN{12}
% \paragraph*{olud} kuulub käändkonda \RN{15}
% \paragraph*{ooli} kuulub käändkonda \RN{10} lisada \msd{sg par} lõpuvokaal
% \paragraph*{oonõ} kuulub käändkonda \RN{9}??
% \paragraph*{paganus} kuulub käändkonda \RN{11}
% \paragraph*{paha} kuulub käändkonda \RN{3},???
% \paragraph*{parad} kuulub käändkonda \RN{2}
% \paragraph*{partõin} kuulub käändkonda \RN{12}
% \paragraph*{parõpi} kuulub käändkonda \RN{}???
% \paragraph*{peenepi} kuulub käändkonda \RN{}???
% \paragraph*{pesä} kuulub käändkonda \RN{8}
% \paragraph*{pii} kuulub käändkonda \RN{1}
% \paragraph*{poduškõ} kuulub käändkonda \RN{3}
% \paragraph*{pojo} kuulub käändkonda \RN{9} aga mitmuse tüvi -ai-?
% \paragraph*{poolõz} kuulub käändkonda \RN{14}
% \paragraph*{Portugaalija} kuulub käändkonda \RN{6}???
% \paragraph*{poštaljon} kuulub käändkonda \RN{2} aga kas \msd{sg nom} -i-lõpuga?
% \paragraph*{programmõ} kuulub käändkonda \RN{5} NB! muuda mitmusetüvi pikaks -ii-
% \paragraph*{puhaz} kuulub käändkonda \RN{14}
% \paragraph*{põrzõz} kuulub käändkonda \RN{14}
% \paragraph*{põski} kuulub käändkonda \RN{7}
% \paragraph*{pühä} kuulub käändkonda \RN{8}
% \paragraph*{püütö} kuulub käändkonda \RN{2}
% \paragraph*{raadio} kuulub käändkonda \RN{6}??
% \paragraph*{rakõ} kuulub käändkonda \RN{14}
% \paragraph*{raskõz} kuulub käändkonda \RN{14}
% \paragraph*{ratis} kuulub käändkonda \RN{14} aga Aristel on \vadja{ratiz}?
% \paragraph*{rihenneüz} kuulub käändkonda \RN{11} aga see pole ilus liitsõna
% \paragraph*{rikaz} kuulub käändkonda \RN{14}
% \paragraph*{rissi} kuulub käändkonda \RN{2}
% \paragraph*{roho} kuulub käändkonda \RN{15}? NB! \msd{sg par} lõpuvokaal lisada
% \paragraph*{rookõ} kuulub käändkonda \RN{5} või \RN{9}
% \paragraph*{rooppõ} kuulub käändkonda \RN{5} või \RN{9}
% \paragraph*{ruska} kuulub käändkonda \RN{6}
% \paragraph*{räpäl} kuulub käändkonda \RN{8}
% \paragraph*{rüiz} kuulub käändkonda \RN{14}??
% \paragraph*{seemen} kuulub käändkonda \RN{13}
% \paragraph*{selče} kuulub käändkonda \RN{8}
% \paragraph*{sese} kuulub käändkonda \RN{2}????
% \paragraph*{siipi} kuulub käändkonda \RN{2}??
% \paragraph*{siitiä} kuulub käändkonda \RN{6}
% \paragraph*{sika} kuulub käändkonda \RN{3}? aga miks geminatsioon \msd{sg par}?
% \paragraph*{siltõ} kuulub käändkonda \RN{3}
% \paragraph*{sinin} kuulub käändkonda \RN{12}
% \paragraph*{slona} kuulub käändkonda \RN{8}??
% \paragraph*{soo} kuulub käändkonda \RN{1}
% \paragraph*{sopuin} kuulub käändkonda \RN{12}
% \paragraph*{sorsõ} kuulub käändkonda \RN{5}
% \paragraph*{susi} kuulub käändkonda \RN{7}
% \paragraph*{süčüzü} kuulub käändkonda \RN{2}
% \paragraph*{sünti} kuulub käändkonda \RN{2}
% \paragraph*{süsiin} kuulub käändkonda \RN{12}??
% \paragraph*{süä} kuulub käändkonda \RN{13}?? kuigi Tsetkovil pole \msd{sg gen} -me-lõpuline? või on see \RN{6} järgi?
% \paragraph*{süüčči} kuulub käändkonda \RN{2}
% \paragraph*{štanad} kuulub käändkonda \RN{3} aga on plurale tantum
% \paragraph*{taimi} kuulub käändkonda \RN{7}
% \paragraph*{talviin} kuulub käändkonda \RN{12}
% \paragraph*{tarkuz} kuulub käändkonda \RN{11}
% \paragraph*{tee} kuulub käändkonda \RN{1}
% \paragraph*{terve} kuulub käändkonda \RN{6}
% \paragraph*{toho} kuulub käändkonda \RN{15}? NB! \msd{sg par} lõpuvokaal lisada
% \paragraph*{tuhattõ} kuulub käändkonda \RN{3}?
% \paragraph*{turvõz} kuulub käändkonda \RN{14}
% \paragraph*{täti} kuulub käändkonda \RN{2}
% \paragraph*{tühjõ} kuulub käändkonda \RN{4} aga see käändkond on nii ebaproduktiivne, et tõsta ümber \RN{8}  NB! muuda \vadja{tühje}??
% \paragraph*{tükkü} kuulub käändkonda \RN{2}
% \paragraph*{tünke} kuulub käändkonda \RN{8}
% \paragraph*{tütär} kuulub käändkonda \RN{13}???
% \paragraph*{usa} kuulub käändkonda \RN{5}
% \paragraph*{vahti} kuulub käändkonda \RN{2}
% \paragraph*{vaka} kuulub käändkonda \RN{6}
% \paragraph*{varsi} kuulub käändkonda \RN{10}
% \paragraph*{varvõz} kuulub käändkonda \RN{14}
% \paragraph*{velosipedõ} kuulub käändkonda \RN{5}
% \paragraph*{vihtõ} kuulub käändkonda \RN{3}
% \paragraph*{vikahtõ} kuulub käändkonda \RN{3}
% \paragraph*{viki} kuulub käändkonda \RN{2}
% \paragraph*{viks} kuulub käändkonda \RN{3}
% \paragraph*{villõ} kuulub käändkonda \RN{3}
% \paragraph*{vimpõ} kuulub käändkonda \RN{5}
% \paragraph*{vipu} kuulub käändkonda \RN{2}
% \paragraph*{väči} kuulub käändkonda \RN{7}
% \paragraph*{õgaz} kuulub käändkonda \RN{14}
% \paragraph*{ohsõ} kuulub käändkonda \RN{5}
% \paragraph*{õhtõgoin} kuulub käändkonda \RN{12}
% \paragraph*{õja} kuulub käändkonda \RN{8}
% \paragraph*{õma} kuulub käändkonda \RN{8}
% \paragraph*{õmpõjõ} kuulub käändkonda \RN{8}
% \paragraph*{õmpõlia} kuulub käändkonda \RN{8} -lia aga -lija?
% \paragraph*{õnnõkõz} kuulub käändkonda \RN{14}
% \paragraph*{õnnõliin} kuulub käändkonda \RN{12}
% \paragraph*{õnnõtoi} kuulub käändkonda \RN{13}
% \paragraph*{õpõin} kuulub käändkonda \RN{12}
% \paragraph*{õpõttõja} kuulub käändkonda \RN{8}
% \paragraph*{õttsõ} kuulub käändkonda \RN{3}
% \paragraph*{änte} kuulub käändkonda \RN{8}
% \paragraph*{ärčä} kuulub käändkonda \RN{8} aga võiks ka \RN{7}
% \paragraph*{ääri} kuulub käändkonda \RN{7} aga \msd{sg par} \vadja{ääreä}?
% \paragraph*{ühs} kuulub käändkonda \RN{10}
% \paragraph*{üvä} kuulub käändkonda \RN{8}




%\subsection{Ekstraktmorfoloogia üldistatud muuttüüpide algoritm}
%\label{sec:muuttüüpide-süsteem}

% tundub, et algoritm on kallutatud regulaarsustele, mis esinevad kirjakeeltes-ühiskeeltes, aga ei tööta nii hästi variatiivse keeleainese peal

%Silfverberg ja Hulden (\citeyear{silfverberg_computational_2018}) on kirjeldanud üht formaalset viisi, kuidas ekstrakt\-morfoloogia tüüpsõnu kokku grupeerida ja seega nende arvu vähendada. Siin alaosas rakendatakse meetodit leitud tüüpsõnadele ja esitatakse selle põhjal loodud vadja muuttüübistik ja võrreldatakse leitud muuttüübistikku Ariste käändkondadega.



% Eesti muuttüüpide traditsioonist on kirjutanud mh \cite{viks_muuttuubid_nodate}.



%\subsubsection{Muuttüüp \RN{1}}
% Muuttüüp \RN{1} koondab enda alla kõige suurema hulga sõnu (303) ja tüüpsõnu (??). Hõlmatud tüüpsõnad jagunevad järgnevatesse \cite[85]{__2011} välja toodud muutkondadesse (klassi).
% 
% \paragraph*{Klass 1} \vadja{aapõ}, \vadja{aikõ}, \vadja{aitõ}, \vadja{butkõ}, \vadja{čenče}, \vadja{dovariššõ(?)}, \vadja{enči}, \vadja{propkõ}, \vadja{ülči}, \vadja{laiskõ}, \vadja{kultõ}, \vadja{kumpõ}, \vadja{õnki}, \vadja{kurp}, \vadja{partõ}, \vadja{lafkõ}, \vadja{lahti}, \vadja{lautõ}, \vadja{leipe}, \vadja{lootõ}, \vadja{luiskõ}, \vadja{poikõ}, \vadja{märče}, \vadja{selče}, \vadja{mätä}, \vadja{rantõ}, \vadja{põlvi}(?), \vadja{kanki}, \vadja{kant}, \vadja{põski}, \vadja{rookõ}, \vadja{siipi}, \vadja{siltõ}, \vadja{sorsõ}, \vadja{taimi}(?), \vadja{tünko}, \vadja{vihtõ}, \vadja{vimpõ}, \vadja{õhsõ}, \vadja{änte}, \vadja{ääri}(?)
% 
% \paragraph*{Klass 1a} \vadja{kukkõ}, \vadja{biskvittõ}, \vadja{borovikkõ}, \vadja{bruuss}, \vadja{mõiznikkõ}, \vadja{fartukkõ}, \vadja{pliittõ}, \vadja{ikolookkõ}, \vadja{itikkõ}, \vadja{joožikkõ}, \vadja{katol̕ikk}, \vadja{liippõ}, \vadja{koomikk}, \vadja{klaass}, \vadja{riittõ}, \vadja{seppe}, \vadja{luukkõ}, \vadja{mahsõ}, \vadja{musikko}, \vadja{rooppõ}, \vadja{tükku}, 
% 
% \paragraph*{Klass 2}
% 
% \paragraph*{Klass 2$^\ast$} \vadja{česä}, \vadja{ičä}, \vadja{jõki}, \vadja{sõta(?)}, \vadja{mäči}, \vadja{nõki}, \vadja{rapa}, \vadja{sika}, \vadja{usa}, \vadja{väči} 
% 
% \paragraph*{Klass 3} \vadja{iiri}, \vadja{irvi(?)}, \vadja{suuri}, \vadja{kahs}, \vadja{kahõsa(?)}, \vadja{koollud(?)}, \vadja{koori}, \vadja{kuus(?)}, \vadja{uusi}(?), \vadja{kuusi}, \vadja{lumi}, \vadja{noori}, \vadja{lõunõ}, \vadja{läsijõ}, \vadja{lühüd}, \vadja{meri}, \vadja{peeni}, \vadja{meeli}, \vadja{meez}, \vadja{ooli}, \vadja{poduškõ}(?), \vadja{pooli}, \vadja{uhsi}, \vadja{varsi}, \vadja{õmpõlia}, \vadja{ühs}
% 
% \paragraph*{Klass 3$^\ast$} \vadja{čäsi}, \vadja{laki}, \vadja{süsi}, \vadja{vesi}, \vadja{susi}
% 
% \paragraph*{Klass 4} \vadja{pal̕l̕õz}, \vadja{alõin}, \vadja{ammõz}, \vadja{čimolain}, \vadja{baabukõz}, \vadja{bul̕bukõz}, \vadja{musikõz}, \vadja{eglin}, \vadja{čümmenäz}, \vadja{esimein}, \vadja{iirikkõin}, \vadja{iloin}, \vadja{jäin}, \vadja{liivõkõz}, \vadja{kangõz}, \vadja{kauniz}, \vadja{keskolin}, \vadja{kultõin}, \vadja{kuuvvaiz}, \vadja{kõlmaz}, \vadja{makuz}, \vadja{mansikõz}, \vadja{nagriz}, \vadja{nel̕l̕äz}, \vadja{nenäkõz}, \vadja{nain}, \vadja{poolõz}, \vadja{põrzõz}, \vadja{raskõz}, \vadja{rihenneüz}, \vadja{rüiz}, \vadja{sinin}, \vadja{sopuin}, \vadja{süsiin}, \vadja{võõrõz}, \vadja{talviin}, \vadja{tarkuz}, \vadja{turvõz}, \vadja{varvõz}, \vadja{õhtõgoin}, \vadja{õnnõkõz}, \vadja{õnnõliin}



%\subsection{Põhivormid ja analoogiavormid}

%Selles osas selgitatakse välja vadja keele tüüpsõnade põhi- ja analoogiavormid sõnaliigiti. Seda püütakse teha formaalselt põhinedes vaid ekstrakt\-morfoloogiaga leitud tüüpsõnamallidele.

%\cite{erelt_eesti_2007} järgi ``[p]õhivormid on need vormid, mida pole võimalik teiste vormide alusel tuletada ning mille moodustamiseks tuleb iga sõnatüübi korral anda vastavad reeglid.'' ja ``[a]naloogiavormid on vormid, mida saab moodustada mingi põhivormi analoogial.''

%Tegelikult on ekstraktmorfoloogia leitud LCS ainus põhivorm ja kõik muutvormid on sellest tuletatud analoogiavormid. Kuna aga läänemeresoome keelte keeleteaduses ei ole katkendlike põhivormide kasutamine traditsioonis (nagu seda on nt araabia keelte puhul), püütakse siin leida traditsioonilise käsitluse järgi põhi- ja analoogiavormid.

%\subsubsection{Käändsõnad}
% Eesti keele käändsõna põhivormid on ainsuse nimetav, ainsuse omastav, ainsuse osastav, mitmuse omastav ja mitmuse osastav. Põhivormiks tuleb tingimisi lugeda ka ainsuse lühikest sisseütlevat.

%\subsubsection{Tegusõnad}



%\subsection{Muuttüüpide produktiivsus}
%
%Kristiina Kross (Ross) nimetab produktiivsuseks ``mingi morfoloogilise nähtuse võimet allutada endale uusi sõnu'' (\cite{kross_eesti_1984}). Siin allosas seatakse eelmises osas leitud muuttüübid pingeritta selle järgi, kui mitu tüüpsõna nendele allub.
%
%Kas selleks on vaja defineerida, mis on \textit{uus sõna}? Näiteks kõik uuemad vene keele laenud.
%
%Kas produktiivsuse pingerida on võimalik jagada mingi kriteeriumi järgi avatuteks ja suletuteks muuttüüpideks?





\newpage
\section{Programmkoodi tuletamine}
\label{sec:programmkoodi-tuletamine}


Programmkoodi tuletamise (ingl. \textit{source code generation}) all peetakse siin töös silmas mistahes protsessi, mille käigus tuletatakse mingi üldisema kirjelduse põhjal programmkoodi ühe või mitme konkreetse programmeerimiskeskkona jaoks.

Üldine kirjeldus (või teisisõnu ontoloogia) kirjeldab faktuaalset \textit{mis}-laadi teadmist ning tuletatud programmkood kirjeldab konkreetselt \textit{kuidas} neid teadmisi rakendada.

Töös kasutatakse keskseks kirjelduseks vormisõnastikku ja sellest ekstrakt\-morfoloogiaga eraldatud sõnatüübi\-malle. Kirjeldus on vormistatud rahvusvahelise standardi \textit{Lexical Markup Framework} (\cite{iso/tc_37/sc_4_language_2007}) järgi leksikaalseks ressurssiks, mis on XML vormingus.

Programmkoodi tuletavad nn generaatorid. Töös esitatakse kaht generaatorit, üks programmeerimiskeele Grammatical Framework jaoks ning teine Giella keeletehnoloogilise taristu integreerimise jaoks. Generaatorid on kirjutatud XQuery programmeerimiskeeles.

Mõlema programmkoodi\-generaatori ühine arhitektuurne omadus on terminite tõlke\-tabelite kasutamine. Tõlke\-tabelite põhjal asendatakse leksikaalses ressursis kasutatud terminid sihtkeeles kasutatud terminitega. See võimaldab järgida ja austada eri taristute terminoloogilisi traditsioone. Näiteks nimetatakse sõnatüüpe LMFis ingliskeelse prefiksiga \textit{as} (kui \textit{asKatto}), aga Grammatical Framework'is eesliitega \textit{mk} (inglis\-keelsest tegusõnast \textit{make}) kui \textit{mkKatto} ja Giella taristus hoopis sõnaliigiga (\textit{N\_KATTO}).

Samuti erinevad eri taristutes kasutatud grammatiliste tunnuste märgendid (nt ainsuse nominatiiv on LMFis ja GFis \textit{singular nominative}, aga Giellas \textit{+SG+NOM}).
% A shared architectural feature of both code generators is the use of translation
% tables for translating terms used in the LMF ontology to their corresponding terms
% used in the host environment. For example the names of paradigms are prefixed with
% as in the LMF (e.g. asTšiutto), but named like actions in the GF (mkTšiutto), and
% prefixed by their part of speech in Giellatekno (N_TŠIUTTO). Also the terminology for
% grammatical features differ between the environments.
% In this way different terminological traditions are supported and respected.



\subsection{Keskne kirjeldus Lexical Markup Framework vormingus}
\label{sec:lmf}

% 3.1
% Lexical Markup Framework
% The Lexical Markup Framework (LMF, ISO 24613:2008) is an ISO standard for natural
% language processing lexicons and machine readable dictionaries. It provides a com-
% mon model for managing exchange of data and enables merging of different resources.
% (Francopoulo, 2013).
% The LMF standard consists of a core model and several extensions. In our work we
% use two extensions: the NLP Morphological Pattern extension to model the extracted
% paradigm information, and the Morphology extension to represent each lexeme’s in-
% flected wordforms.
% Representing the lexeme’s morphology with both paradigms (describing in inten-
% sion) and inflected wordform tables (describing in extension) might seem superfluous.
% But both representations serve their own purpose.
% Listing extensionally all inflectional wordforms for each lexeme is in this work
% considered part of documentation and what adds value to the work’s 50-year per-
% spective (discussed in section 4).
% Recording lexeme’s all inflected wordforms extensionally also creates the possi-
% bility to further annotate the individual wordforms, such as real attested corpus at-
% testations, or other meta-linguistic information such as judgements.
% Listing wordform information explicitly is also beneficial for the dictionary sys-
% tem, enabling quick searches and statistics.
% Next, we will introduce our data and how it is represented in the LMF.

% sissejuhatav tekst
Lexical Markup Framework (LMF) on loomuliku keeletöötluse leksikonidele ja masin\-loetavatele sõna\-raamatutele mõeldud rahvusvaheline ISO-standard. Standard koosneb märgenduskeelega defineeritud ühisest mudelist andmevahetuse juhtimiseks ja võimaldab erinevate ressursside ühendamist. (\cite[1]{francopoulo_lmf_2013}).
% TODO kas Francopoulo peatükist ei leidu natuke pikemalt lahtivõetud kirjeldus?

% laiendimoodulid
Standard koosneb põhimoodulist ja mitmest eri\-otstarbelisest laiendi\-moodulist (\cite[22]{francopoulo_lmf_2013}). Magistritöös kasutatakse peale põhimoodulit veel kahte moodulit: morfoloogia moodul (\textit{LMF Morphology Extension}) ja morfoloogiliste paradigmade moodul (\textit{LMF Morphological Pattern Extension}). Valitud moodulitega on olnud võimalik kirjeldada ja salvestada nii vormisõnastiku andmed (lekseemide muutvormid) kui ka ekstrakt\-morfoloogiaga eraldatud andmed (sõnatüüpide protsessuaalsed koostamis\-mallid ja tehniliste tüvede muutujad).

% morfoloogiamoodul eesmärk
Morfoloogiamooduli eesmärgiks on kirjeldada morfoloogiat mahu kaudu, s.o kirjeldada lekseemi loendades kõik selle muutvormid.

% paradigmamooduli eesmärk
Morfoloogiliste paradigmade mooduli eesmärgiks on seevastu kirjeldada morfoloogiat sisu kaudu, s.o kirjeldada neid kriteeriume ja reegleid, millega saab moodustada kõik ühe lekseemi muutvormid. Selles töös kirjeldatakse ekstrakt\-morfoloogia sõnatüübi\-mallid antud mooduliga.

% topeltkirjeldus ju liigne?
Sama nähtuse kirjeldamine nii mahus kui ka sisus võib tunduda liigsena, ent mõlemal kirjeldusviisil on omad head küljed. Lekseemide iga muutvormi loendamist peetakse magistritöös eeskätt dokumenteeriva ja arhiveeritava väärtusena, mis ei kahane pikaajalises perspektiivis -- muutvormide loendamine on ka 50 või 150 aasta pärast informatiivne, keele\-tehnoloogia programm\-kood ei pruugi olla arusaadav ega jooksutatav. Veel võimaldab muutvormide loendamine edaspidises sõnastikutöös ka igale muutvormile lisada informatsiooni, nt selle reaalsete korpusesinemuste kohta. %Samas on just loendatud muutvormid ekstraktmorfoloogia meetodi sisend, ja võimaldab .

% TODO sõnatüübimalli hoidmise häid külgi pole kirjeldatud
Ekstraktmorfoloogiaga eraldatud sõnatüübimallide kirjeldamine samas vormingus võimaldab neid protsessuaalseid koostamis\-eeskirju teisendada programm\-koodi generaatoritega. Generaatoreid võib hiljem lisada juurde uue keeletehnoloogia tuletamise jaoks ilma selleta, et lähte\-andmestikku peab muutma. See teebki LMFi kirjelduse nö tehnoloogia\-neutraalseks standardiks.

% mis seal veel hoitakse?
Peale sõnaartiklite ja morfoloogilise informatsiooni hoitakse leksikaalses ressursis ka globaalset informatsiooni, nagu keele nimetust ja keele\-koodi, mida kasutatakse tuletatud programmkoodis peamiselt failide nimetamisel.

% TODO kirjutada outro ja intro järgmise kahe alatüki jaoks


\subsubsection{Sõnaartikli esitamine LMFis}
Lekseemide sõnaartiklid LMFi esituses koosnevad muutvormide loendist, lemmaks valitud muutvormist, sõnaliigist ja sõnatüübist. Paralleelvormide puhul on sõnatüüpe rohkem kui üks.

% TODO
Andmete struktuuri näitlikustav sõnaartikkel on LMFi XML märgendus\-formaadis esitatud joonises~\ref{code:lmf-lexicalentry}.
% TODO 'joonis' on siin pigem 'näidis'

% TODO allpool peak olema commonNoun, kopeeri uus kood!
\spacing{1}
\begin{figure}[h]
  \center
\begin{minted}[frame=single,fontsize=\small,framesep=10pt]{XML}
<LexicalEntry morphologicalPatterns="asKatto">
  <feat att="partOfSpeech" val="noun"/>
  <Lemma>
    <feat att="writtenForm" val="katto"/>
  </Lemma>
  <WordForm>
    <feat att="writtenForm" val="katto"/>
    <feat att="grammaticalNumber" val="singular"/>
    <feat att="grammaticalCase" val="nominative"/>
  </WordForm>
  <WordForm>
    <feat att="writtenForm" val="katod"/>
    <feat att="grammaticalNumber" val="plural"/>
    <feat att="grammaticalCase" val="nominative"/>
  </WordForm>
</LexicalEntry>
\end{minted}
\caption{Sõnaartikli \textit{katto} esitamine LMFis (muutvormid on kajastatud vaid osaliselt).
  \label{code:lmf-lexicalentry}}
\end{figure}
\spacing{1.5}

Sõnatüübimall \textit{asKatto} on märgitud \textit{morphological\-Patterns} atribuudis. Mitme sõnatüübi puhul loendatakse need tühikutega eraldatuna. Sõnaliik on märgitud \textit{part\-Of\-Speech} elemendis (\textit{noun}). Element \textit{Lemma} sisaldab lemmaks valitud sõnavormi. \textit{Word\-Form}-elemendid loendavad muutvorme koos nende grammatiliste tunnustega: arv (\textit{grammatical\-Number} ja kääne (\textit{grammatical\-Case}).

Sõnavormide modaalsust on täpsustatud kirjakeelseks kujuks (\textit{writtenForm}).




\subsubsection{Sõnatüübi malli esitamine LMFis}

%The extracted paradigms are represented as LMF Morphological Patterns. These hold
%information about their part of speech and are name-tagged with an ID. The names
%follow the LMF tradition and are prefixed with as, such as asTšiutto.
Ekstraktmorfoloogiaga eraldatud sõnatüübid kirjeldatakse LMFi morfoloogiliste paradigmade mooduli elementidega (\textit{LMF Morphological Pattern Extension}, varem nimetatud ka \textit{LMF Paradigm Pattern}). Iga sõnatüüp on märgitud identifitseeriva nimetusega, mille puhul on järgitud LMFi tava lisada tüüpsõna ette ingliskeelne eeslide \textit{as} (nt \textit{asKatto}).

%The LMF Morphological Patterns model all the information extracted by Extract
%Morphology. The attested variable values, i.e. the technical stems, of all lexemes
%added in the Morphology Lab is saved. This information could be utilized to inte-
%grate prediction models into the generated source code, as have been demonstrated
%by Forsberg and Hulden (2016). This has not been done, as the work has focused on
%integrating the lexical resources as a first stage.
Peale nimetuse ja sõnaliigi kirjeldatakse LMF morfoloogiliste paradigmadega veel kõik ekstraktmorfoloogiaga eraldatud informatsioon: salvestatakse lekseemide tehniliste tüvede muutujad ja muutvormimallid.

Muutvormallid on LMFis kirjeldatud nn transformatsiooni\-elementidega (\textit{TransformSet}), mis koosnevad muutvormi morfoloogistest tunnustest (\textit{GrammaticalFeatures}) ja muutvormi koostamiseks vajalikest protsessidest (\textit{Process}). Koostamis\-protsessid osundavad lihtsaid, järjestatud konkatenatsioni\-tehteid (\textit{addAfter}), mis liidavad lükkimise teel tehnilise tüve muutujad ja muutvormi ülejäänud tähtkoostised kokku.

Andmete struktuuri näitlikustav sõnatüüp on LMFi XML märgendus\-formaadis esitatud joonises~\ref{code:lmf-paradigmpattern} (lk~\pageref{code:lmf-paradigmpattern}).

\spacing{1}
\begin{figure}[h]
  \center
  % TODO muuda kood asKatto järgi!
  % TODO muuda tšiutto õigekirja järgi
\begin{minted}[frame=single,fontsize=\footnotesize,framesep=10pt]{XML}
<MorphologicalPattern>
  <feat att="id" val="asTšiutto"/>
  <feat att="partOfSpeech" val="nn"/>
  <TransformSet>
    <GrammaticalFeatures>
      <feat att="grammaticalNumber" val="singular"/>
      <feat att="grammaticalCase" val="nominative"/>
    </GrammaticalFeatures>
    <Process>
      <feat att="operator" val="addAfter"/>
      <feat att="processType" val="pextractAddVariable"/>
      <feat att="variableNum" val="1"/>
    </Process>
    <Process>
      <feat att="operator" val="addAfter"/>
      <feat att="processType" val="pextractAddConstant"/>
      <feat att="stringValue" val="t"/>
    </Process>
    <Process>
      <feat att="operator" val="addAfter"/>
      <feat att="processType" val="pextractAddVariable"/>
      <feat att="variableNum" val="2"/>
    </Process>
  </TransformSet>
  <TransformSet>
    <GrammaticalFeatures>
      <feat att="grammaticalNumber" val="plural"/>
      <feat att="grammaticalCase" val="nominative"/>
    </GrammaticalFeatures>
    <Process>
      <feat att="operator" val="addAfter"/>
      <feat att="processType" val="pextractAddVariable"/>
      <feat att="variableNum" val="1"/>
    </Process>
    <Process>
      <feat att="operator" val="addAfter"/>
      <feat att="processType" val="pextractAddVariable"/>
      <feat att="variableNum" val="2"/>
    </Process>
    <Process>
      <feat att="operator" val="addAfter"/>
      <feat att="processType" val="pextractAddConstant"/>
      <feat att="stringValue" val="d"/>
    </Process>
  </TransformSet>
<MorphologicalPattern>
\end{minted}
\caption{Tüüpsõnamalli \texttt{tšiutto} (mille alla kuuluvad mh \textit{tšiutto} ja \textit{katto}) esitus LMF\-is. Esitus mudeldab muutvormimalle $x_1 \oplus \textbf{t} \oplus x_2$ ning $x_1 \oplus x_2 \oplus \textbf{d}$.
  \label{code:lmf-paradigmpattern}}
\end{figure}
\spacing{1.5}

Morfoloogiliste paradigmade \textit{Process}-elemendid on üks-üheses vastavuses ekstrakt\-morfoloogia meetodiga eraldatud sõnatüübi\-mallidega.

% TODO process elementide processType kirjeldus!

% TODO kas ma lihtsalt 'unustan' tehniliste tüvede kirjeldamise?


\FloatBarrier
\subsection{Integreerimine Grammatical Framework'iga}
\label{sec:gf}

Grammatical Framework (GF) on eriotstarbeline programmeerimis\-keel, mis on loodud loomulike (ja tehislike) keelte grammatikate kirjeldamiseks. GFi iseloomustab muuhulgas see, et see kuulub funktsionaalsete programmeerimis\-keelte paradigmasse ja on rajatud tüübiteooriale. (\cite[\RN{7}]{ranta_grammatical_2011})

Teoreetiliselt on GF formaal\-grammatikana väljendus\-rikkuse poolest ekvivalentne PMCFG tüüpi grammatikaga (\textit{Parallell Multiple Context-Free Grammar}), mis jääb mõõdukalt konteksti\-tundlike \textit{mildly context-sensitive} ja täielkult konteksti\-tundlike \textit{fully context-sensitive} grammatikate vahele (\cite[10]{ranta_grammatical_2011}).

Grammatical Framework toetab on keele\-spetsiifilise koodi eraldamist teekidesse, mida nimetatakse ressursi\-teekideks (\cite{ranta_grammars_2008}, \cite[97]{ranta_grammatical_2011}). Üle 32~keele on oma ressursi\-teegiga toetatud. Eesti keele ressursi\-teegi on loonud \cite{listenmaa_computational_2014}. %Teegid võivad ka jagada ühiseid grammatilisi jooni eriti süntaksis. On olemas soome süntaksi 

Siin esitatud programm\-koodi generaator loob vadja keele ressursi\-teegi morfoloogia ja leksikoni moodulid. Loodud vadja ressursi\-teek on internetis saadaval\footnote{https://github.com/keeleleek/GF-Votic}. % Usually the best way to start resource grammar writing is with the morphology. \cite[209]{ranta_grammatical_2011}

% TODO järgmine kopeerida ka programmkoodi genereerimise intro alla
Loodud generaator näitlikustab üht integreerimis\-viisi, mida iseloomustab tarkvara arendusmeetod pidev paigaldamine (\textit{Continuous Integration}). Vormisõnastiku täiendamist (ja ka muutmist) saab kasutada eesmärgipäraselt selleks, et viia täiendused sisse Grammatical Framework'i vadja morfoloogia\-teeki ja sealt\-kaudu sellele põhinevatele rakendustele.

Generaator on jaotatav kaheks osaks, millest üks genereerib vadja morfoloogia\-mooduli ja teine vadja leksikoni\-mooduli.

Generaator koostatud võimalikult universaalseks ja ei sisalda midagi vadja keelele spetsiifilist. Lemmavormi valikut ei ole generaatorisse jäigalt sissekodeeritud, vaid on vahetatav generaatori parameetritega. Veel koostatakse moodulite failinimed vastavalt leksikaalses ressursist antud keele\-koodi järgi.



\subsubsection{Morfoloogia moodul}
\label{sec:gf-tüüpsõnad}
Grammatical Framework'i morfoloogiamoodul koosneb nn paradigma\-funktsioonidest, mille ülesandeks on luua sõnade paradigmad ehk muutvormitabelid (\cite[248]{ranta_grammatical_2011}, \cite[645]{detrez_smart_2012}).

Ekstrakt\-morfoloogiaga eraldatud sõnatüübi\-mallide funktsioonid on olnud lihtne teostada paradigma\-funktsioonidena. Igale sõnatüübile on teostatud kaks eraldi funktsiooni: üks üldine, mis võtab sisendiks sõna muutvormi tervikuna ja väljastab sõna tehnilise tüve osad, teine on konkreetne ja selle sisendiks tehnilise tüve osad ja väljundiks on muutvormide tabel. Kuigi on võimalik kahest funktsioonist üks moodustada, on kaks funktsiooni eraldi hoitud võimalike tarkvara\-vigade otsimise tarbeks, et arendajal on võimalik testida enda moodustatud tehniliste tüvedega.

% To generate the function for an extracted paradigm description, we need to specify its name, interface and body.
Sõnatüübist paradigmafunktsiooni genereerimiseks on vaja määrata funktsiooni nimi, liides (\textit{interface}) ja keha (\textit{body}). Genereeritud programm\-koodi illustreeritakse joonises~\ref{code:gf-morfoloogia}. % TODO joonis?

Morfoloogiamooduli nimi on koostatud leksikaalses ressursis oleva vadja keelekoodiga vastavalt GF mooduli\-nimetustele \textit{MorphoVot}.

% TODO muuda koodis olevad sõnad katto vastu
\spacing{1}
\begin{figure}[H]
  \center
\begin{minted}[frame=single,fontsize=\small,framesep=10pt]{haskell}
resource MorphoVot = {

param
  Number = singular | plural ;
  Case = nominative ;
  NForm = NF Number Case ;

oper
  Noun : Type = {s : NForm => Str} ;

------------------------------------------------
-- Start of Noun section
------------------------------------------------

  mkTšiutto : Str -> Noun = \tšiutto -> 
    case tšiutto of {
      tšiut + "t" + o@(-(_+"t"+_)) => mkTšiuttoConcrete tšiut o ;
      _ => Predef.error "Unsuitable lemma for mkTšiutto"
    } ;
  
  mkTšiuttoConcrete : Str -> Str -> Noun = \tšiut,o -> 
    { s =
      table {
        NF singular nominative => tšiut + "t" + o ;
        NF plural nominative => tšiut + o + "d"
      }
    } ;
}
\end{minted}
\caption{Vadja keele GF morfoloogia\-mooduli genereeritud programmkood (näites piiratud käänetega).
  \label{code:gf-morfoloogia}}
\end{figure}
\spacing{1.5}

Funktsiooni nimetuseks on sõnatüübi identifikaator LMFi leksikaalsest ressursist, mille eesliide \textit{as} vahetatakse GFi traditsiooni järgi \textit{mk}.

% The function’s interface is declared in the oper section and parameter types in
% the param section. The names and values of the parameters reflect the attributes and values of the grammatical features in the lMF, with minor modifications. The oper
% definition is largely templatic, reflecting only the parts of speech from the LMF.
Funktsiooni liides on deklareeritud \textit{oper} osas ja parameetrite tüübid \textit{param} osas. Parameetrite nimetused ja kategooriad peegeldavad grammatilisi tunnuseid LMFis.

% To help readers navigate the code, the paradigms are ordered by their part of
% speech and headers are generated in the form of comments for each part of speech
% section. These are the only comments generated at the moment.
Kuigi vadja vormisõnastik sisaldab magistritöö valmimise hetkel ainult nimisõnu, arvestab koodi\-generaator teiste sõna\-liikidega ja grupeerib sõnatüüpide paradigma\-funktsioonid vastavateks sektsioonideks, mille ette lisatakse kommentaar sõnaliigi kohta. Koodi lugemise hõlbustamiseks järjestatakse veel sõnatüüpide funktsioonid nende nimede järgi tähestikuliselt. 

% Every paradigm is split into two separate functions a high-level dispatch function
% (mkTšiutto, and low-level to mkTšiuttoConcrete). This is to allow a developer to
% use the low-level function for debugging or testing purposes.

% The high-level function takes a string with the chosen lemma form as its input and
% splits it into the technical stem parts, which are then simply delegated to the low-level
% function. An error message is thrown in case the input string is not able to match.
Üldisem, lemmavormil opereeriv funktsioon eraldab sisestatud muutvormi selle tehnilise tüve osadeks, mis seejärel edastatakse konkreetsele paradigma\-funktsioonile.
% Note that GF’s implementation of regular expressions is non-greedy as opposed to
% many other programming languages. Because of this the expression @(-(_+"t"+_))
% is appended to the last part of the pattern. This expression is automatically created
% and makes the `t` match the last letter t in the wordform.
Võib märkida, et GFi regulaaravaldised töötavad mitte-aplalt (ingl. \textit{non-greedy}), vastupidiselt paljude teiste programmeerimis\-keeltele. Seetõttu lisatakse nt \textit{@(-(\_+"t"+\_))} avaldisesse, mis muudab \textit{t}-tähega ühtimise aplaks.

% The names of the variables holding the technical stem parts are taken from the
% paradigm’s name.
Tehnilise tüve muutujate nimetused saadakse sõnatüübi nimetusest.

% The low-level, or concrete, function is the one that generates the inflection table.
% On the left-hand side of the table is the grammatical features and on the right-hand
% side are the concatenation patterns that instantiates the wordforms.
Konkreetse funktsiooni nimetuse järele lisatakse \textit{Concrete} ja see funktsioon tagastab muutvormide tabeli. Tabeli vasakul poolel on grammatilised tunnused ja paremal poolel on muutvormide mallid.




\subsubsection{Leksikoni moodul}
\label{sec:gf-leksikon}

Grammatical Framework on oma olemuselt mitmekeelne ning iga rakendus peaks täitma nende oma semantika abstraktse grammatikaga. Kuna vormisõnastik ei sisalda hetkel tõlkeid, genereeritakse lihtne sõnaloend.
%GF is a multilingual framework and each application is expected to define their own semantics in an abstract grammar. No attempt is made to include semantic pivots to the Votic resource, instead a simple monolingual word list is generated as a lexicon.

Iga sõnaartikli puhul koostatakse sõnaloendisse kirje, mille nimeks on lemma ja sõnaliigi\-tähis ning väljakutse vastava sõnatüübi paradigma\-funktsioonile. Juhul, kui sõna\-artiklil on mitu sõnatüüpi, määratakse need kirje variantideks.
%This lexicon specifies for each entry in the resource its lemma, which is appended with its part of speech and a call to the paradigm function. In the case a lemma has multiple paradigms, each one is declared as variants.

%Rakenduste loomisel võib GF arendaja tugineda genereeritud sõnaloendile.
%The GF developer could use this word list when translating the application-specific vocabulary names of the abstract grammar.

Genereeritud sõnaloendi programm\-koodi illustreeritakse joonises~\ref{code:gf-leksikon}. % TODO joonis?
%The source code of the generated word list is illustrated in figure~\ref{code:gf-lexikon}.

\begin{figure}[ht]
  \center
  \begin{minted}[frame=single,fontsize=\small,framesep=10pt]{haskell}
fun
  lin katto_N = mkKatto "katto"
  lin čiutto_N = mkKatto "čiutto" ;
\end{minted}
\caption{Generated source code for the Votic GF lexicon.
  \label{code:gf-lexikon}}
\end{figure}


%\subsubsection{Arutelu}
%\label{sec:gf-arutelu}

%Loodud morfoloogiakomponenti on kasutatud interaktiivses vadja-vene-vadja vestmikus.





\subsection{Integreerimine Giella-taristuga}

% pmst on see 
% cat root.lexc affixes/nouns.lexc stems/nouns.lexc > vot.lexc
% hfst-xfst
% read lexc /tmp/vot.lexc
% define Lexicon
% source paradigms.xfscript
% define Paradigms
% define Speller Lexicon ?+ .o. Paradigms
% 
% ! kusjuures ?+ katab/ühtib käänded
% NB! veel juurde on testide genereerimine!


% mis on Giellatekno infra
The Giellatekno infrastructure has been characterized in \cite{moshagen_building_2013} to be a \textit{development environment infrastructure} (as opposed to a resource infrastructure), offering a framework for building language-specific analysers and directly turn them into a wide range of useful programs.

From the point of view of our work on Votic morphology, the programs of interest are proofing tools and morphological analyzers.

% minu töö rasked kohad
For the integration with Giellatekno's infrastructure, several components are needed: a lexicon, paradigm descriptions in FST, automatic test declarations, and a Makefile that binds all the components together.

At the work's current stage all but the Makefile is being generated. Because of this, each component has only been tested on its own but not integrated in the infrastructure.

We don't include any examples of generated code for the Giellatekno infrastructure because of space considerations. We will only present and discuss the main design choices made.



Keeletehnoloogilise taristuga Giella integreeritakse selles töös peamiselt selleks, et saada kätte õigekirja\-kontrollija. Giella-taristu koosneb veel võimalustest. Taristut kasutavad peamiselt Giellatekno ja Divvun.

Integreerimine on jagatav kaheks peamiseks osaks: leksikoni integreerimeerimine ja tüüpsõnamallide integreerimine. Seejärel kirjeldatakse taristu poolt loodud õigekirjakontrollija tööpõhimõtet ja lõpetuseks on arutelu.

\subsubsection{Tüüpsõnad}
\label{sec:giella-tüüpsõnad}

Paradigmade ehk tüüpsõnamallide esitus FST formalismis põhineb suuresti Forsbergi ja Huldeni (\citeyear{forsberg_learning_2016}) tööle.

Paradigmad esitatakse relatsioonidena sõnavormi ja lemma koos analüüsiga vahel. Sellised relatsioonid sisaldavad lõpmatut hulka sõnalemmasid, millest mõistagi pole suurem osa vadjakeelsed. Mis on siiski tähtis, on see, et relatsioonid mudeldavad paradigmasid.

Sõnade lõpmatu hulk piiratakse leksikonis antuga ja niiviisi saadakse leksikonis sisalduvate sõnade kõik sõnavormid. Nendest ja ainult nendest sõnavormidest koosnebki esialgne vadja õigekirja\-kontrollija.

\subsubsection{Leksikon}
\label{sec:giella-leksikon}

``Formally, the lexc language is a kind of right-recursive phrase-structure 
grammar.'' ja ``A lexc description compiles into a standard Xerox finite-state network, either a simple automaton or a transducer.'' (\cite[lk~203]{beesley_finite_2003}).

Kuigi lexc fraasi\-struktuuri\-grammatikatega on võimalik paradigmasid (tüüpsõnamalle) mudeldada, ja tavaliselt selleks seda kasutataksegi Giella taristus, võtab see töö teise lähenemisnurga ja lihtsustab võimalikult palju leksikoni struktuuri.

Leksikon koosneb selles töös ainult kahest andest: \textit{lemma} ja \textit{tüüpsõna}. 


% Instead of generating the lexicon straight into FST representation, we use Giellatekno's Sanat XML format. This is achieved by simple XML transformations from the LMF format.
% 
% The Giellatekno infrastructure uses the Sanat XML to generate its own source codes.
% 
% In its essence, the Sanat XML contains the lexical entry's lemma, the name of the paradigm (e.g a FST continuation class) and the technical stem needed by the continuation class.
% 
% What is crucially missing from our implementation is the Finnish translation equivalents used as interlingual pivots in the Giellatekno infrastructure.
% These equivalents will be generated when our work on the Votic morphological dictionary resource has reached the stage of adding them. 
% 

\subsubsection{Õigekirjakontrollija}
\label{sec:giella-õigekirjakontrollija}

Eelnevalt kirjeldatud integreerimine Giella-taristusse võimaldab taristul luua õigekirjakontrollija. Mis on õigekirjakontrollija, kus seda kasutatakse ja mida see kontrollib?

%\subsubsection{Arutelu}
%\label{sec:giella-arutelu}

Loodud õigekirjakontrollija on eesmärgipäraselt jäetud lihtsakoeliseks. See märgib kõik sõnad valeks, mis ei sisaldu sõnastikus. See on lühiajaliseks kasutamiseks ja mõeldud ärgitama kasutajaid ise pakkuma täiendusi ja sõnaloomet vadja sõnastikusse.



%\newpage
%\section{Arutelu}

%Arutelu struktuur peaks järgima üks-ühele sissejuhatuses väljatoodut, ent sellele siis lisama arutelu (sissejuhatus ainult nentingud).

% ARUTELU POLE, SISSEKIRJUTATUD MUJALE NING KOKKUVÕTE ALL VÕIB MAINIDA MIS OOTUSED OLID JA MIS SAADUD RESULTAAT SELLEST ERINEB

\newpage
\section{Kokkuvõte}

Magistritöö on kirjeldanud süsteemi, millega on ühelt poolt defineeritud vadja keele normatiivne morfoloogia ja mille põhjal teisalt tuletatakse automaatselt morfoloogiline keeletehnoloogia.

Morfoloogilise normatiivi vajadust ajendab Heinike Heinsoo läbiviidud kursused keelekümbluskoolis Ämmesse Vunukassaa ja normatiiv on hõlpsasti muudetav-parendatav ilma programmeerimisoskusteta.

Saadud morfoloogilist tüübistikku on analüüsitud vadja keele grammatikatega ja põhjendatud ajaloolise morfoloogiaga.

---

% töö tuumik on tüüpsõnakirjeldused
Töö keskseks osaks on ekstraktmorfoloogiameetodiga saadud tüüpsõnakirjeldused.
% kirjeldused LMFi ja neist genereeritakse kood
Kirjeldused kodeeritakse koos sõnastikuga ümber standardsesse vormingusse ja saadud leksikaalse ressursi järgi tuletatakse automaatselt programmkoodi kahe keeletehnoloogilise platvormi jaoks, ja tagatakse seega vadja keele tugi nendes platvormides.

% seetõttu töötab ekstraktmorfoloogia liidesena
Niivisii kasutatakse ekstraktmorfoloogia meetodit kasutaja\-liidesena, mille abil koostatakse arvutimorfoloogia ainult tüüpsõnade muutvormitabeleid sedastades -- mitte programmeerides.

% kirjeldus kesksel kohal, parandused õigesse kohta
Magistritöös esitatud töövoog paneb leksikaalse ressursi kesksele kohale ja tuletatud tehnoloogia sellest teiseseks. Uue sõnavara ja vigade parandused tehakse ressursis, mitte mitmes tehnoloogias eraldi.

% standardi kasutamine tagab pikaajalise loetavus
Kuna nii tüüpsõnade kirjeldused, kui ka ülejäänud sõnastik kodeeritakse rahvusvahelise standardi Lexical Markup Framework vormingusse, tagatakse võimaluse ressursi pikaajaliseks arhiveerimiseks. Leksikaalne ressurss on loetav ja arusaadav palju kauem, kui seda on programmeerimiskood.

% panus dokumentaalsele lingvistikale
Viimase tõttu püüab magistritöö ühendada arvutuslingvistika ja dokumenteeriva lingvistika valdkondi.







\newpage
\section{Põhimõisted ja lühendid}
\label{sec:põhimõisted}
Siin loetletakse töös kasutatud mõisted ja lühendid koos nende tähendustega.

\spacing{1}
\glsaddall
\small{
  \printglossary[title={},toctitle={}]
}
\spacing{1.5}








\newpage
\section{Kirjandus}
\label{sec:kirjandus}
% @TODO: kuidas määrata biblatex-i keele eesti keelele?
\spacing{1}
{
  \renewcommand*{\bibfont}{\small}
  \printbibliography[heading=none]
}
\spacing{1.5}







\newpage
\section{The use of Extract Morphology for Automatic Derivation of Language Technology for Votic}

An English language summary of this work.







\newpage
\section{Lisad}

Siin on esitatud kõik ekstraheeritud tüüpsõnamallide tabelid.

%\input{lmf-paradigms}


\end{document}

% Local Variables:
% TeX-engine: xelatex
% End:
