\documentclass[12pt,a4paper]{article}

\usepackage[top=4cm, bottom=3cm, left=4cm, right=2.5cm]{geometry}
\usepackage{setspace}

\usepackage{float}

\usepackage{enumerate}

\usepackage{booktabs}

% polyglossia
\usepackage{polyglossia}
\usepackage{fontspec}
\usepackage{xunicode}
\usepackage{xltxtra}
\usepackage{url}
\usepackage{expex}

% Use a Free/Libre font with Finnish–Hungarian-Cyrillic-UPA coverage
\setmainfont[Mapping=tex-text]{Linux Libertine O}
% set languages to use
\setmainlanguage{estonian}
\setotherlanguages{english}

\usepackage{csquotes}

% Polyglossias miskipärast ei tööta poolitamine
% \usepackage{polyglossia}
% \setdefaultlanguage{german}

% @todo: vormistada kirjanduse loetelu õigesti
% @todo: vormistada tekstis viited õigesti (nt lk kooloni järel jne)
\usepackage[backend=biber,style=authoryear]{biblatex}
\addbibresource{bibliography-mathesis.bib}

\usepackage{hyperref}
\urlstyle{}

% koodiplokkide esitamiseks
\usepackage{minted}

% Sõnastik
\usepackage[nopostdot,xindy]{glossaries}
\makeglossaries

% sisukord
% Bakalaureusetöös on soovitatav piirduda kolmeastmelise hierarhiaga, magistritöös neljaastmelisega
\setcounter{tocdepth}{4}


% kasutatud mõisted ja lühendid
\newglossaryentry{tüüpsõnamall}{
  name=Tüüpsõnamall,
  description={on ekstraktmorfoloogiaga leitud tüüpsõna paradigma kirjeldus, mis koosneb iga muutvormi koostamismallidest ehk muutvormimallidest. Tüüpsõnamall on relatsioon tehnilise tüve ja kõigi selle paradigmasse kuuluvate muutvormide vahel.}
}
\newglossaryentry{muutvormimall}{
  name=Muutvormimall,
  description={kirjeldab üksiku muutvormi koostamisskeemi ja kannab selle grammatilised tunnused. On integraalne osa tüüpsõnamallist. Koostamisskeem koosneb muutujatest ja konstantidest, mille tähtkoostised lükitakse üks-teise järele. Muutujate tähtkoostised võivad olla mingil moel piiratud.}
}
\newglossaryentry{lemma}{
  name=Lemma,
  description={on suvaliselt valitud grammatiliste tunnuste komplekt, mida kasutatakse lekseemi viitamiseks.}
}
\newglossaryentry{tehniline-tüvi}{
  name=Tehniline tüvi,
  description={on tähtkoostiste järjend, millega saab tüüpsõnamalli muutvormide muutujad asendada elik väärtustada ja niiviisi koostada ühe konkreetse sõna kõik vormid.}
}
\newglossaryentry{mikrostruktuur}{
  name=Mikrostruktuur,
  description={on sõnastiku sõnaartikli sisemine struktuur.}
}
\newglossaryentry{konkatenatsioon}{
  name=Konkatenatsioon,
  description={ehk \ensuremath{\oplus} on tähtede ja tähtjärjendite lükkimine teine-teise järele, et moodustada uus tähtjärjend. Näiteks \textit{aa} \ensuremath{\oplus} \textit{be} moodustab \textit{aabe}.},
  symbol=ensuremath{\oplus}
}



\begin{document}


% @todo: lisada magistritöö, ülikool ja juhendaja jne
% vaata siit https://tex.stackexchange.com/questions/184848/how-to-add-the-name-of-the-supervisor-in-a-thesis-field#184878
% @todo: vormistada esileht eraldi
\title{Ekstraktmorfoloogia meetodiga tuletatud keeletehnoloogia vadja sõnavara näitel}
\author{Kristian Kankainen}
% \supervisor{Heinike Heinsoo, Külli Prillop}
% \university{Tartu Ülikool}
% \department{Eesti ja üldkeeleteaduse õppetool}
\date{2019}
\maketitle


\newpage
\tableofcontents


% TODO kust pärineb järgmine tsitaat? "computational methodology for linguistic analysis is not the same thing as computational linguistics" ()


\newpage
\spacing{1.5}
\section{Sissejuhatus}

Magistritöö esimene eesmärk on luua H. Heinsoo Sõnakopittöjas esitatud sõnavarast morfoloogiline sõnastik, mis sisaldab sõnavara kõiki muutvorme. Selleks vajalik arvutimorfoloogiline kirjeldus ehitatakse sellisel moel, et see taandub tüüpsõnade muutvormitabelite esitamisele, mitte grammatiliste reeglite esitamisele. Niiviisi ehitatud teooriavaba(m) arvutimorfoloogiline kirjeldus võimaldab luua erinevaid keeletehnoloogiaid automaatselt programmkoodi tuletamise teel. Esitatakse kolme tehnoloogia automaatset tuletamist: 1)~ühe keeletehnoloogilise taristusse integreerimise kaudu õigekirjakontrollija, 2)~vadja keele arvutimorfoloogia moodul ühe loomulike keelte grammatikate koostamiseks mõeldud programmeerimiskeelele ja 3)~morfoloogia tehnoloogiaülene kirjeldus ühe rahvusvahelise standardi abil.

Kuna kõik tuletatud keeletehnoloogia edaspidine täiendamine ja täpsustamine käib ainult lekseemide muutvormitabelite täiendamise ja täpsustamise kaudu, peab esimese eesmärgi juurde lisama seda, et magistritöös loodud leksikograafiline süsteem võimaldab keeleaktivistide rühmal töötada oma sõnavara ja keeletehnoloogia kallal edaspidi ka ilma spetsialistist keeleteadlase ja keeletehnoloogi abil. Kas seda vadja keele puhul ka juhtub, jääb tuleviku näidata.

Magistritöö teine eesmärk on analüüsida leitud tüüpsõnad mitmel viisil: 1)~kirjeldada nende morfofonoloogiat keeleajalooliste arengute taustal, 2)~leida tüüpsõnade põhivormid ja analoogiavormid, 3)~esitada üks võimalik muuttüüpide süsteem ja võrrelda seda seni esitatutega ja viimalt 4)~analüüsida muuttüüpide produktiivsust.


---


Magistritöö loob viisi ehitada arvutimorfoloogia puhtalt lekseemide sõnavormide esitamise teel ning teisendada ehitatud arvutimorfoloogilise mudeli automaatselt kahte keeletehnoloogilisse raamistikku.

Magistritöö kasutab loodud süsteemi selleks, et kirjeldada vadja keele normatiivsed morfoloogilised tüüpsõnad.

Tööd ajendab mõtteviis minimeerida tööd: loodud normatiivne morfoloogiline tüübistik on aluseks automaatselt tuletatud keeletehnoloogiale, kui normatiiv muutub, muutub ka keeletehnoloogia. Töö paneb leksikaalse ressursi esikohale ja kõik leitud sisulised vead õiendatakse otse ressursis, mitte keeletehnoloogilistes tarkvarades eraldi.




\newpage
\section{Teoreetilised lähtekohad}
Kuna töö opereerib arvutilingvistika, deskriptiivse ja dokumentaalse lingvistika ääremail, peame selgitama töö teoreetilised lähtekohad. Siinsele kompendiumiks on ka põhimõisted seletatud pt~\ref{sec:põhimõisted} \nameref{sec:põhimõisted}.

Töö püüdleb olla võimalikult teooriavaba, lastes vadja lekseemide sõnavormide tähtkoostised ise määrata nende paradigmade koostamis\-reeglid. (See on olla deduktiivne esialgse morfoloogia postuleerimises, vastandudes induktiivsele, s.o mingist grammatilisest kirjeldusest lähtudes.)

Tööl on siiski teoreetilised lähtekohad, mis tulenevad ühelt poolt arvutimorfoloogia nõuetest ja teisalt klassikalisest paradigmaatilisest morfoloogiakäsitlusest. Järgmiselt püüan argumenteerida, et arvutimorfoloogia ei pea olema mingist lingvistisest teooriast ajendatud. Seejärel tutvustan tööle kõige lähedamini asetsevat morfoloogilist käsitlust.


\subsection{Vadja kirjakeel ja korpusplaneerimine}
% -- tõsta ja sobitada kirjakeele peatükk ka siia alla

Selles peatükis seletan vadja kirjakeeleks valitud aluseid ning korpusplaneerimise mõistet.

% korpusplaneerimine on alamosa keeleplaneerimisest?
Korpusplaneerimine on mis, miks seda tehakse ja kes seda teevad (kasuta Cooper 1996: 45). Korpusplaneerimine koosneb (Kloss 1968) kolmest osast:
\spacing{1}
\begin{enumerate}
\item kirjamine (ingl. \textit{graphization}) ehk ortograafia paikapanemine
\item morfoloogia ühtlustamine ja standardiseerimine
\item sõnavara moderniseerimine ehk rikastamine
\end{enumerate}
\spacing{1.5}

Neid osi ei pea vaatama järjestiku etappidena moderniseerimise poole, (Coulmas 1989) on nimetanud seda pidevaks adapteerimiseks.

Käesoleva töö fookuses on morfoloogia ühtlustamine ja vormisõnastiku moodustamine ning hõlmab viimase realiseerimise keeletehnoloogiana (mh õigekirja\-kontrollijana).
Morfoloogia ühtlustamine mahutab siinses töös järgnevaid etappe:
% kas see jutt kuulub siia?
\spacing{1}
\begin{enumerate}
\item lekseemide ühtlustatud paradigmade fikseerimine (kusjuures minimeeritakse paralleelvorme)
\item lekseemide paradigmade esitamine vormisõnastikuna
\item lekseemide grupeerimine muuttüüpidesse
\end{enumerate}
\spacing{1.5}

% TODO kuidas see ümberkirjutada ja selgeks teha?
Kuna see töö loob ühtse töökeskkonna, mille keskmeks on vormisõnastik mis määrab morfoloogia, siis loodab autor ka sellele, et ka korpusplaneerimise antakse üle vadjalastele -- muuda sõna muutvormi vormisõnastikus ja see muutub ka õigekirjakontrollijas ja mujal keeletehnoloogias.

Vormisõnastikust ja arvutimorfoloogiast lähemalt järgnevas peatükis.


\subsection{Arvutimorfoloogia eesmärk ja lingvistiline motiveeritus}

% TODO leia allpool õige koht sellele: Kui lingvistika üldine eesmärk on leida ja kirjeldada keelenähtuste reeglipärasusi, siis jääb lahtiseks küsimus mis on arvuti\-lingvistika eesmärk -- kas see on formaliseerida lingvistika poolt leitud reeglipärasused või võib see mahutada ka nende reeglipärasuste leidmist? See töö lähtub arusaamast, et arvuti on abivahend lingvistile, mitte ei ole ainult lingvisti(ka) formalisatsioon.
% Seda tehakse kahel põhjusel -- esiteks puudub täielik morfoloogiakirjeldus vadja keelele. Teine põhjus on arvutiinsenerlik põhjus. Siinne töö maht ei luba täielikku morfoloogiakirjeldust, pealegi on keeled ajas muutuvad ja tuleb täiendusi teha. Seetõttu ei taha siinne töö esitada sellist formaalset morfoloogiakirjeldust, mida keegi teine ei saaks täiendada. Iga formalismi puhul kaasneb õpikõver, et sellest üldse aru saada, siinne töö esitab arvutimorfoloogia võimalikult formalismivabalt --- vormisõnastiku kujul --- milles sisalduva sõnavara on võimalik igal-ühel täiendada ja muuta ainult sellekaudu, et muuta konkreetse sõna sõnavormi.
% Tehnoloogiavaba(m) kirjeldus on standartne, ja keeletehnoloog, kes tahab lisada uue ... võib seda teha, kirjelduse formaat on kirjeldatud rahvusvahelise standardi dokumentatsioonis.

% Ehk alustada sellega: Kuna töö kasutab läbivalt sõna 'morfoloogia' natuke teistsuguses tähenduses kui see tavaliselt keeleteaduses kannab, tuleb selle tähendust kõigepealt lahti seletada. . Töö loobub morfeemist kui keele väikseim tähenduslik üksus ja seega ei tähenda siin töös morfoloogia morfeemilist morfoloogiat. Aronoff on pööranud tähelepanu tõigale, et 'morfoloogia' kannab teistsugust tähendust keeleteaduses kui ta seda kannab teistes teadusharudes. Enamik teadusharudes tähendab 'morfoloogia' umbes "võimalike vormide uurimine-kaardistamine". Keeleteaduses hoopis ...

Arvutimorfoloogia eesmärgiks on siin töös valitud moodustada elektroonse vormisõnastiku:
\begin{quote}
  ``Üks täielik vormisõnastik peaks esitama kõigi sõnade kõik muutevormid koos vastava grammatilise iseloomustusega. Ainult siis saab kasutaja sõnastikust ilma mingi vaevata ja täiesti kindlalt teada, milline on vajalik vorm antud sõnast või millise sõna millise vormiga on tegemist tundmatu sõnavormi puhul.'' (\cite{viks_vaike_1992} lk 7).
\end{quote}

% Elektroonne vormisõnastik moodustab seega iga lekseemi jaoks relatsiooni ehk seose $(tsitaatvorm, {muutevormid koos vastava grammatilise iseloomustusega})$.
Sellist vormisõnastikku võib moodustada erinevatel viisidel. Näiteks leksikaalse andme\-baasina, kus iga lekseemi puhul on nenditud kõik selle muutevormid koos vastava grammatilise iseloomustusega, või näiteks reeglite komplektina, mida rakendades saab koostada lekseemi muutvorme vastavalt nende grammatilistele iseloomustustele.

Matemaatilises mõttes kujutab vormisõnastik vaid \textit{seost} muutevormide ja nende vastavate grammatiliste iseloomustuste vahel. 

% arvutimorfoloogia kui arbitraarne valik realiseerimaks vormisõnastiku sisu (kas info pakkimismehanism või lingvistiliselt motiveeritud)
Arvutimorfoloogiad võivad seda seost (või vormisõnastiku funktsionaalsust) realiseerida arvutuslikult erinevatel viisidel ja ei pea olema lingvistilis-grammatiliselt motiveeritud. Kuna üks täielik vormisõnastik on mahult niivõrd suur (kui mitte lõpmatult suur), on selle mahu kompaktsem ja ülevaatlikum esitus peamiseks motivatsiooniks organiseerida selle koostamise reeglite abil, mis on ühel või teisel moel põhjendatud lingvistiliste-grammatiliste reeglipärasustega.

% pilk ajalukku (mida hiljem ümberlükata kui strukturalistlik-morfeemiline, aga mis on jätnud jälje ka arvutimorfoloogiate mõtemaailmale kui lingvistiliselt realistlikud)
Eelmise sajandi keskpaiku jagas Charles Francis Hockett kõik seni Ameerikas sajandi algusest saadik ilmunud grammatikad kahe üldise mudeli järgi, IA (ingl. \textit{Item-and-Arrangement}, üksus ja distributsioon v järjestus v korraldus) ja IP (ingl. \textit{Item-and-Process}, üksus ja protsess ehk protsessi\-morfoloogia). Kõrvalmärkusena tõi ta välja ka kolmanda, ``vanema ja väärikama'' mudeli, WP (ingl. \textit{Word-and-Paradigm}, sõna ja paradigma), aga jättis selle oma käsitlusest välja (\cite[210]{hockett_two_1954}). Hockett võrdleb IA ja IP mudelite eeliseid ja argumenteerib, et IA toonane populaarsus seisneb eeskätt selles, et ajastu eelistab formaalseid mudeleid. Kuna IA-mudel oli juba formaliseeritud tahtis Hockett nüüd formaliseerida sellest vanema IP-mudeli (\cite[214]{hockett_two_1954}) ning sellest sai hiljem, Fred Karlssoni sõnade järgi, generatiivse lingvistika peamiseks mudeliks (\cite[126]{karlsson_uldkeeleteadus_2002}).

IP-mudel põhineb (morfoloogilise) protsessi mõistel, millega ühest algvormiks valitud kujust (ingl. \textit{base}) luuakse teine vorm (\cite[210]{hockett_two_1954}). IA tekkis vastureaktsioonina IP protsessi\-mõiste suunalisusele -- enam ei tahetud tõsta esile üht vormi algsemaks teistest vormidest (\cite[211]{hockett_two_1954}). IA põhineb morfeemi mõistel, mida Hockett iseloomustab kui keele väikseimat grammatiliselt olulist üksust, ja selle distributsiooni määramisel (\cite[212]{hockett_two_1954}). Hocket nendib, et ka IA mudeli puhul tuleb siiski teha kohati suvalisi valikuid selle üle, mis kuulub morfeemi tasandile ja mis kuulub distributsiooni tasandile (\cite[212]{hockett_two_1954}).

% Stumpi neljamõõtmeline jaotus
Gregory Stump on arendanud Hocketti IP ja IA kaheks\-jagamise klassifikatsiooni edasi tänapäevaste morfoloogiliste teooriate põhjal. Nimetades IAd ümber leksikaalseks (ingl. \textit{lexical}) ja IPd inferentsiaalseks (ingl. \textit{inferential}) lisab ta klassifikatsioonile veel sisemise telje: inkrementaalsed (ingl. \textit{incremental}) ja realiseerivad (ingl \textit{realizational}) teooriad. (\cite{stump_inflectional_2001}, lk 1-2)

Inkrementaalsete teooriate järgi lisandub iga (olgu IA puhul leksikaalselt loetletud või IP puhul inferentsiaalse reegliga tuletatud) morfosüntaktilise tunnuse puhul sõnale ka selle vormiline eksponent (\cite[2]{stump_inflectional_2001}). Vormilised eksponendid on üks-üheses seoses grammatiliste tunnustega ja need väljenduvad ükshaaval elik inkrementaalselt.

Realiseerivate teooriate juures ei pea vormiline eksponent iga morfosüntaktilise tunnuse puhul eraldi ja koheselt väljenduma, vaid vormiline väljendus võib realiseeruda tunnuste suuremate komplektide puhul või üldse kui sõna kõik tunnused on teada (\cite[2]{stump_inflectional_2001}).

Realiseerivad teooriad võimaldavad niiviisi suurema paindlikkuse vormiliste väljendujate \textit{realiseerimisel}, loobudes vormiliste väljendujate üks-ühesest seosest morfosüntaktiliste tunnustega.

Stumpi jagab oma klassifikatsiooni järgi Lieberi morfoloogilise teooria leksikaalseks ja inkementaalseks. Halle ja Marantzi distributsioonilise morfoloogia teooria leksikaalseks ja realiseerivaks. Steele'i artikuleeritud morfoloogia teooria esindab inferentsiaalset ja inkrementaalset suunda. (\cite[2--3]{stump_inflectional_2001}).

Stumpi enda ja Matthewsi, Zwicky ning Andersoni teooriaid nimetab ta WP teooriateks, mis on inferentsiaalsed ja realiseerivad (\cite[3]{stump_inflectional_2001}).


% Finally, Word-and-Paradigm theories of inflection (e.g. those proposed
% by Matthews (), Zwicky (a), and Anderson ()) are of the
% inferential–realizational type. In inferential–realizational theories, an
% inflected word’s association with a particular set of morphosyntactic prop-
% erties licenses the application of rules determining the word’s inflectional
% form; likes, for example, arises by means of a rule appending -s to any verb
% stem associated with the properties ‘sg subject agreement’, ‘present tense’,
% and ‘indicative mood’.

Robert Beard on nimetanud ülaltoodud viimaste autorite arendatud realiseerivaid teooriaid eru-morfoloogiaks (ingl. \textit{'split' morphology}) (\cite[20]{beard_morpheme_1987}) ja pakkunud välja morfoloogia veel võimsama eraldamise, mis põhineb tema morfoloogia lahususe hüpoteesil (ingl. \textit{Separation Hypothesis}) (\cite{beard_lexeme-morpheme_1995}).

% ird-morfoloogia VS morfoloogia lahususe hüpotees
Morfoloogia lahususe hüpoteesil põhinevate teooriate ja realiseerivate (eru-)morfoloogia\-teooriate vahe on fundamentaalne ja lähtub nende käsitlusest süntaksi ja semantika vahekorrast. Kõige ilmekalt paistab nende vahe morfeemi definitsioonis, küsimuses kas morfeem on keele väikseim vormiline tähenduslik üksus või mitte.

Beardi teoorias ei ole morfeem grammatiliselt tähenduslik, vaid defineeritud kui mistahes muutusena lekseemi fonoloogilises kujus (\cite[31]{beard_morpheme_1987}). Seega on tema teoorias ainult lekseemid tähenduslikud märgid ning grammatilised afiksid (morfeemid) on seda vaid sattumuslikult (\cite[17]{beard_morpheme_1987}).

Käesolevas magistritöös rakendatud ekstraktmorfoloogia on oma organisatsiooni suhtes sõna ja paradigma mudel, aga selle käsitus morfeemist on lähedasem Beardi teooriale.

% Karttust ja Koskenniemit tuua mängu sisse alles siis, kui on vaja näidata kuidas arvuti poolel on hakatud asju tõlgendama enda mõtteviisi järgi
Arvutilingvistikas on arvutimorfoloogiat üldiselt organiseeritud klassikalise morfeemi\-käsituse järgi. Seda ilmestab hästi 
% Karttuneni väljakutsed arvutimorfoloogias
Lauri Karttunen, kes nendib inimkeele mudeldamise puhul arvutimorfoloogias kaks väljakutset: 1)\nobreakspace morfotaktika ehk sõnast väiksemate üksuste kombineerumine ja 2)\nobreakspace morfoloogilised vaheldused ehk sõnast väiksemate üksuste kuju olenemine nende ümbritsevast kontekstist (\cite{karttunen_computing_2003}).

Mille mõlemad väljakutsed viitavad otseselt klassikalisele morfeemi\-käsitusele.

Karttuneni artikkel on vastus Stumpi teooriale ja ta näitlikustab selles kuidas Stumpi teooria on võimalik rakendada kasutades lõplike automaatide formalismi.

Karttunen toob välja olukorra, et arvuti\-morfoloogiad põhinevad arvutuslikel formalismidel, millega nad implementeerivad morfoloogiaid ja mitte ei põhine otse mingil lingvistilisel teoorial. Ta ütleb et morfoloogia\-uurija üllitiste peamine eesmärk on olla veenev, et tema teooria annab läbinägelikuma (ingl. \textit{insightful}) ja elegantsema kirjelduse kui teised teooriad ja formalismid (\cite[2]{karttunen_computing_2003}). Praktilised küsimused nagu sõnavaraline katvus, arvutus\-kiirus ja mälu\-maht ei ole relevantsed akadeemilisele morfoloogia\-uurijale (\cite[2]{karttunen_computing_2003}).

Seega võib öelda, et arvutimorfoloogia on laiem kui lingvistiline morfoloogia, kuna esimest ei piira mitte teooria, vaid arvutusliku meetodi võimsus. Karttunen tõestab artiklis, et Stumpi inferentsiaalne-realiseeriv teooria on taandatav lõplike automaatide formalismi arvutusvõimsusele.% (\cite{karttunen_computing_2003}). % TODO vaata kas viide tõesti hõlmab tervet paragrahvi

Sellest võib järeldada, et arvutilingvistikas on lingvistilise teooria roll pigem olla ajendiks kui tõetruuks postulaadiks, kuigi kindlasti on teooria ja selle implementatsioonilise praktika vahekord raskesti eraldatavad ja ajas muutuvad. Kuigi tendentsi tõetruuduse vähenemisele võib siiski täheldada tänapäeval ka Kimmo Koskenniemi töös, kus ta on hiljuti oma taandatud kahetasemelises morfoloogiamudelis püüdnud morfofoneemi mõiste juures loobuda selle tähendusliku külje lingvistilisest realismist, omastades seda puhtalt vormile:
\begin{quote}
  ``\textit{Morphophonemes} are represented just as the \textit{combinations of the corresponding letters} (or phonemes) which we can observe in the surface forms. On the one hand, such an interpretation of morphophonemes is crude, but on the other hand, it is a fact that anybody can observe.'' (\cite{koskenniemi_informal_2013}, lk 157)
\end{quote}

% induktsioon ja deduktsioon
Sügavama epistemoloogilise põhjuse, miks arvutimorfoloogiaid on ajendanud pigem lingvistiline motivatsioon ja mitte arvutusteoreetilised võimalused, arvab siinkirjutaja leiduvat strukturaalse lingvistika formaliseerimisperioodi alguses, mis algas enne arvutusmasinate leiutamist (1940.--1960.-ndateil aastatel) ja ammu enne arvutite arvutus- ja mälumahtuvuse võimsuse plahvatuslikku suurenemist (1980.--2000.-ndail). % TODO lisada Karttuneni ajalooline ülevaade fst morfoloogia arengutest

% lingvistilise teooria formaliseerimine
Formaalseid teooriaid ja seega teooriate formaliseerimist peetakse teaduse lipulaevaks (\cite[2026]{auroux_history_2006}). Teooriate formaliseerimis\-protsessi jagab Pieter Seuren neljaks etapiks, kus esimene koosneb uuritava ainese tüüpide (ehk kategooriate) leidmisest ning nendele esitus\-kuju määramisest (\cite[2027]{auroux_history_2006}). (Teisisõnu tegeleb see \textit{type-token distinction}'i probleemiga). Teine etapp käib sageli käsi-käes esimese etapiga ja hõlmab tüüpide taksonoomia määramist, ehk selle määramist, mis andmed kuuluvad mis tüübi alla millal ja mis tingimustes (\cite[2027]{auroux_history_2006}). Kolmas etapp koosneb struktuuri määramisest tüüpide esinemisele, elik kuidas kategooriaid on võimalik omavahel kombineerida (\cite[2027 jj]{auroux_history_2006}) näiteks puu- või sõltuvus\-struktuuride abil. Neljas ja viimane etapp koosneb ühe ennustava ja kirjeldava väärtusega formaalse teooria ülesseadmisest algoritmina ehk sammsammulise tegevusjuhisena (\cite[2031]{auroux_history_2006}).
%final Stage 4, which consists in the setting up of a formal predictive and explanatory theory that has the precision of an algorithmic procedure.

% arvutimorfoloogia on arvutiprogramm mis on formaalne aparaat
Arvutimorfoloogia on arvutiprogramm (või mitme programmi komplekt), mis tahest-tahtmata hõlmab seelaadset formaalset sammsammulist tegevusjuhist.

% probleem asub 3 ja 4 etapi vahel -- kas teha deduktiivselt või induktiivselt?
Probleem, miks arvutimorfoloogiad juhinduvad lingvistilistest teooriatest ja mitte puht-arvutuslikest võimalustest asub formaliseerimis\-protsessi 3. ja 4. etapi vahel. Millisel viisil tuleb põhjendada struktuuri määravaid reegleid?

% zellig
% Generatiivse lingvistika suurkuju Noam Chomsky juhendaja Zellig Harris kirjutab
Zellig Harris (kes oli Noam Chomsky juhendaja) kirjeldab oma \textit{magnum opus} teoses grammatika formaliseerimise lähenemist, mis põhjendab strukturaalsete reeglite määramise ühe formaalse avastamis\-menetluse abil keeleainese korpus\-esinemustest. See on, formaalse teooria sammsammulised reeglid tuletatakse puhtalt struktuuride esinemistest korpusanalüüsi teel. Selline väga töömahukas grammatika loomise menetlusviis sai tema kaasaegsetelt kõva kriitikat olles nii ilmselgelt ebarealistlik ja ebapraktiline. Harris oli tundlik kriitikale ja mainib oma raamatu lõpus viisi, kuidas korpus\-esinemustest eraldi püstitatud reegleid saab hoopis vastupidises suunas \textit{testida} korpustekstide peal. See pani aluse generatiivsele grammatikale, mida arendas edasi tema kasvandik Noam Chomsky teoses \textit{Syntactic Structures} (\citeyear{chomsky_syntactic_1957}). (\cite[2031]{auroux_history_2006}).

% seletav tekst miks ma jauran
Eelnevaga olen ma tahtnud öelda seda, et arvutimorfoloogiate koostamis\-põhimõtted põhineda morfeemil ja morfotaktilistel reeglitel ja mitte puhtalt muutvormide nentimisel korpuse põhjal, on eeskätt ajalooliste traditsioonide järjepidevus. Käesolev töö ei järgi neid traditsioone.

See traditsioon on kristalliseerunud ka pealkirjas ``This volume grows out of a special session that we organized at the January 2009 Annual Meeting of the Linguistic Society of America entitled ``Computational Linguistics: Implementation of Analyses against Data''.'' (\cite{bender_computational_2010}).

% Matthews kirjeldab WP ``mudelit'' kui ... ja Karttuneni järgi kronoloogia Zwicky kaudu Stumpini, aga me ei peagi laskuma WP mudeli arvutusliku külge juurde -- see on väga lihtsal moel lahendatud ekstraktmorfoloogias. Ja Karttuneni konstateering, et Stumpi mudel on arvutuslikult samaväärne FSTga. Ja Roark ja Sproat argumenteerivad, et X ja Y on samuti taandatavad FST-le ja seega samaväärsed. Olen valmis oma argumentatsiooniga siin peatükis, et arvutimorfoloogiad võivad vormisõnastiku funktsionaalsuse realiseerida ilma lingvistilise motivatsioonita. Nagu Stump arvab, et tuleb eelistada ``A theory of inflectional morphology must be preferred to the extent that it minimizes any dependence on theoretical distinctions which are not empirically motivated.'' (\cite{stump_inflectional_2001} lk 9).




\subsection{Paradigmaatiline morfoloogia}

Matthews tõstab esile kaks WP-mudelit: klassikaline ja uus. Klassikaline kuulub X ajastusse ja selle kõrghetk oli õpikutes X sajandil, ajal kui keelt õpetati kooloniates (\cite{matthews_morphology_1991} lk X). Uue rajas Matthews ja seda on Karttuneni sõnul edasi arendanud Zwicky, Anderson ja Stump (\cite{karttunen_computing_2003} lk 2).

Klassikaline on lähedasem siinses magistritöös rakendatud ekstraktmorfoloogiale. Klassikalise ja strukturalistlike lähenemiste suurim vahe seisneb kahes asjaolus. Eeskätt ei näinud klassikalised grammatikud sõnast väiksemat ja tähenduslikku üksust. Arvestati ainult vormikülje üksustega, tähtede-foneemide ja silpidega. Sõna oli väikseim tähenduslik element keeles ja seda nähti tervikuna, hõlmates terve oma paradigma kuuluvaid vorme. (\cite{matthews_morphology_1991} lk X--Y)

Matthews toob välja (klassikalise) paradigmaatilise suuna kolm head omadust. ....

Õpikutes ja grammatikates välja toodud reeglid, mille abil sai ühe lekseemi paradigma moodustada, opereerisid ainult sõnavormide tähtkoostisel. Näiteks võidi ühe sõnavormi lõputähti asendada teiste tähtedega, et saada teine sõnavorm. Et asendatavatele tähtkoostistele ei omandatud mingit tähendust, näitlikustab see, et mõne reegli algvormiks võidi valida selline sõnavorm, mis oma tähtkoostise poolest kõige paremini sobis. (\cite{matthews_morphology_1991} lk X--Y).





\subsection{Vadja kirjakeel ja normatiiv}
\label{sec:kirjakeel}

% vadjal on palju ressursse, grammatikaid jne, aga pole palju mis vastab
% keeleõppija küsimustele kuidas?
Vadja keelele ei loodud kirjandust 1930-ndateil, nagu seda tehti Nõukogude Liidus näiteks karjala, vepsa ja isuri keele jaoks. 

Siiski on vadja keelel hulganisti lingvistilisi kirjeldusi, nagu grammatikaid (mh \cites{ahlqvist_wotisk_1856}{airila_vatjan_1934}{tsvetkov_vadja_2008}{ariste_grammar_1968}{__2011}), sõnaraamatuid (mh \cites{tsvetkov_vatjan_1995}{ariste_vadja_1943}{laakso_vatjan_1989}{raag_dictionary_1982}{pomberg_vadja_1991}{grunberg_vadja_2013}{heinsoo_vadsonakopittoja_2015}) ja ka etnograafilisi töid (mh \cites{kass_kasitoo-_1961}{malk_vadja_1977}).

Kirjeldused ei aita siiski kaasa tänapäeva keeleõppija küsimustele \textit{kuidas kirjutatakse sõna \textsc{tüttö} mitmuse omastavas?}. Selleks on vaja tänapäevase vadja keele morfoloogia standardiseerimist ehk normatiivset kirjeldust.

Käesolev töö ei pürgi looma lõplikku normatiivi, kuivõrd ta loob süsteemi, mis oskab vastata morfoloogilistele küsimustele. Aga loodud süsteemi peamine eesmärk on siiski võimaldada muuta ja jätkata tööd normatiivi arendamiseks ja mille ümber saaks keeleaktivistid ise koonduda, ilma et selleks oleks niivõrd vaja ei lingvistilist ega keeletehnoloogilist spetsialisti.

Püüd luua vadja morfoloogiale normatiivne alus lihtsustab paljudele küsimustele vastusi leida, nt mis käändeid arvestada. Siiski on tööga loodud \textit{keele\-tehnoloogia tuletamise süsteem} avatud ka teistsugustele lähenemistele keeleainesele.

\subsubsection{Noomeni käänded}

Siin käin läbi kust mu tabeli sõnavormid pärinevad, või kuidas need olen rekonstrueerinud. Tüved tulevad Tsvetkovi sõnastikust või Heinsoo või Konkova sõnastikest, kui need on Tsvetkovil puudu. Käändelõppude puhul on järgitud Konkova õpikut.

Vokaale on püütud ühtlustada Heinsoo ja Konkova antud järgi.

\paragraph{Inessiiv}

``A characteristic feature of the Votic inessive is the fact that
geminate stops -kk-, -pp-, -tt-, the geminate affricates -tts-.
-ttš-, the geminate -ss-, and the consonant cluster -hs- always
are in the strong grade before this case marker'' (\cite[23]{ariste_grammar_1968}).

Tegin nii, et kui geminaat esines \textsc{sg nom} vormis, siis muutsin. Konsonant\-klustri -hs- puhul ei vahetanud \textit{ühs} ja \textit{kahs}, sest pole geminaadid.

\subsection{Sõnavara}
\label{sec:sõnavara}
Töös on piirdutud Heinsoo Sõnakopittõjas esitatud noomenite ja adjektiividega. Sõnade paradigmasidd on täiendatud Laakso toimetatud Tsvetkovi sõnaraamatus esitatuga, kusjuures on eelistatud pikemaid muutvorme (TODO kirjelda miks on seda eelistatud). Mõningaid Sõnakopittõjas mitte-esinevaid sõnu on siiski Laakso sõnastikust tööle lisatud, selle eesmärgiga, et laiendada vadja õigekirjakontrollija sõnavara.


\subsection{Ortograafia}
\label{sec:ortograafia}

On järgitud Heinsoo loodud ortograafiat mille jaoks on Kankainen teinud vadja klaviatuuripaigutise (Kankainen, ilmumas). Vadja erinevatest kirjaviisidest on kirjutanud \cite{ernits_vadja_2010} ja erinevatest kirjakeele loomise pürgimistest \cite{ernits_vadja_2006}.


\subsection{Morfofonoloogia}
Tavaliselt jagatakse arvutilingvistikas morfoloogia ja fonoloogia eraldi nii, et morfoloogia tasandil on abstraktne esitus, nn morfofoneemid, mille pindesitused tulenevad eraldi fonoloogilistest reeglitest.

Niiviisi saaks esitada mõlemad fonoloogilised vormid \textit{tšiuttoa} ja \textit{tüttöä} ühe ja sama morfoloogilise kujuga \textsc{tšiutto+a} ja \textsc{tüttö+a}. Kusjuures käändelõpu \textsc{+a} pindesitus \textit{a}-na või \textit{ä}-na sõltuks vastavalt sellest, kas lemmas esineb tagapoolsed või eespoolsed vokaalid.

See töö ei arvesta morfofonoloogilise tasandiga. Peatükis \ref{sec:muuttüüpide-süsteem} näidatakse üht võimalikku viisi koondada tüüpsõnu kokku abstraktsemal tasandil, mis mingil määral arvestab ka morfofonoloogilisi reeglipärasusi.


\subsection{Klassikaline paradigmaatiline morfoloogia}
Sõna kui selle vormide tervik; pedagoogiline praktika ja paradigma üldistuse ülekantavus uutele sõnadele \cite{matthews_morphology_1991}. Matthewski jätab mudeli vormipõhiseks ja mitte morfeemipõhiseks, selle kohta edasi järgmises allosas.


\subsection{Morfeemi staatus ja definitsioon}
Morfeemi ei käsitleta siin töös levinud lingvistilisest seisukohast kui \textit{väikseimat tähenduslikku üksust}, vaid klassikalistele paradigmaatilistele lähenemistele omaselt kui \textit{mistahes tähtkoostise muutust, millega kaasneb tähenduslik muutus} (\cites{beard_morpheme_1987}{beard_lexeme-morpheme_1995}). Morfeemipõhist suunda ajab nt \cite{stump_inflectional_2001}.


\subsection{Muuttüüp, tüüpsõna ja muutkond}
Eesti traditsiooni järgi on muuttüüp tüüpsõnast üldisem. Kuidas siin töös terminoloogiliselt ümber käia, kas \textit{muuttüüp} või \textit{tüüpsõna\-mall}?

Muuttüübistik sõltub selle aluseks võetud klassifikatsioonist, ekstrakt\-morfoloogiat võiks vaadata kui lihtsalt üht väga formaalselt defineeritud muuttüübistikku.

Huldenil on omakorda üks väga formaalne viis, kuidas vähendada ekstrakt\-morfoloogiaga leitud muuttüüpide arvu. Kas see on hoopis muuttüübistik?




\newpage
\section{Ekstraktmorfoloogia meetod}
\label{sec:ekstraktmorfoloogia-meetod}
See osa kirjeldab töös rakendatud ekstraktmorfoloogia meetodit. Töö kasutab ekstraktmorfoloogiat kaheks otstarbeks, esiteks vadja keele morfoloogiliste tüüpsõnade väljaselgitamiseks ja kirjeldamiseks ja teisalt programmkoodi automaatseks tuletamiseks saadud kirjelduse põhjal. Neid kahte rakendust kirjeldatatakse lähemalt vastavates peatükkides \textit{\nameref{sec:analüüs}} ja \textit{\nameref{sec:programmkoodi-tuletamine}}.




\subsection{Sissejuhatus}
\label{sec:ekstraktmorfoloogia-sissejuhatus}

% juhendatud masinõppe meetod -- aga masinõppe ei kõla minu jaoks hästi
Ekstraktmorfoloogia on juhendatud masinõppe meetod, mis üldistab lekseemide muutvormitabeleid ja eraldab neist \glslink{tüüpsõnamall}{tüüpsõnamallid}. See on \textit{juhendatud}, sest sisendiks olevad muutvormitabelid peavad olema korrektselt koostatud. %Ja see on \textit{masinõppe}, sest sisendist analüüsitud mudel on laiendatav ka uuele ja tundmatule sisendile.

% masinõppe asemel kasutan ma seda kirjeldusena ja pealdisena paradigmadele ning liidesena arvutimorfoloogia deklareerimiseks
Selles töös käsitletakse meetodi abil saadud mudelit siiski pigem lihtsa kirjeldusena. See on tüüpsõnakirjeldus, mis on osa sõnastikust -- lekseemi paradigma kirjeldusena. Ja sellest kirjeldusest 
% kirjeldus on inimloetav (viidatav fakt)

% tüüpsõnamalli kohta
Tüüpsõna\-mall koosneb muutvormide mallidest ja vastab seega morfoloogilise paradigma mõistele. 
Tüüpsõna\-malli abil on võimalik moodustada ka tundmatu sõna kõik muutvormid.
% produktiivsusmalli kohta
Kuna kaks või enam lekseemi võivad jagada üht ja sama tüüpsõna\-malli (s.o kuuluda sama paradigmasse), on võimalik ekstrakt\-morfoloogia meetodiga üldistada lekseemide iseärasusi ja luua nendest tüüpsõnade produktiivsuse mudeli. Produktiivsus\-mudeliga on võimalik ennustada uue ja tundmatu sõnavormi kuuluvust ühe või teise tüüpsõna alla. 


Veel ilma detailidesse takerdumata näitlikustatakse siinkohal lugejale meetodi sisendit ja väljundit. Sisendiks on ühe lekseemi muutvormitabel tervikuna (vt tabel\nobreakspace \ref{tab:sisendtabel_katto}). Väljundiks on meetodi poolt leitud tüüpsõnamall (vt tabel\nobreakspace \ref{tab:väljundtabel_katto}). Tabelitele viidatakse alljärgnevas tekstis mitmel korral.

\spacing{1.0}
\begin{table}[H] %[!htbp] % kuvab tabelit definiitselt enne neile järgnevat teksti
      \footnotesize
  \begin{minipage}[t]{.40\textwidth}
%    \centering
    \begin{tabular}[t]{l l}
      muutvorm            & tunnused \\ \hline
      \textit{katto}      & \textsc{sg nom} \\
      \textit{katod}      & \textsc{pl nom} \\
      \textit{kato}       & \textsc{sg gen} \\
      \textit{kattoi}     & \textsc{pl gen} \\
      \textit{kattoje}    & \textsc{pl gen} \\
      \textit{kattoa}     & \textsc{sg part} \\
      \textit{kattoi}     & \textsc{pl part} \\
      \textit{kattoite}   & \textsc{pl part} \\
      \textit{kattose}    & \textsc{sg ill} \\
      \textit{kattoise}   & \textsc{pl ill} \\
      \textit{kattoz}     & \textsc{sg ine} \\
      \textit{kattoiz}    & \textsc{pl ine} \\
      \textit{katosse}    & \textsc{sg ela} \\
      \textit{kattoisse}  & \textsc{pl ela} \\
      \textit{katolle}    & \textsc{sg all} \\
      \textit{kattoille}  & \textsc{pl all} \\
      \textit{katol}      & \textsc{sg ade} \\
      \textit{kattoil}    & \textsc{pl ade} \\
      \textit{katolte}    & \textsc{sg abl} \\
      \textit{kattoilte}  & \textsc{pl abl} \\
      \textit{katossi}    & \textsc{sg tran} \\
      \textit{kattoissi}  & \textsc{pl tran} \\
      \textit{kattossaa}  & \textsc{sg term} \\
      \textit{kattoissaa} & \textsc{pl term} \\
      \textit{katoka}     & \textsc{sg com} \\
      \textit{kattoika}   & \textsc{pl com} \\
    \end{tabular}
    \caption{Sisendi muutvormide tabel koos morfo\-loogiliste tunnustega.}
    \label{tab:sisendtabel_katto}
  \end{minipage}
  \hfill
  \begin{minipage}[t]{.55\textwidth}
    \centering
    \begin{tabular}[t]{l l l}
      ühisosajada                     & muutvormi\-mall           & tunnused \\
      \hline
      \underline{kat} t \underline{o}       & $x_1$ + t + $x_2$         & \textsc{sg nom} \\
      \underline{kat}   \underline{o} d     & $x_1$ + $x_2$ + d         & \textsc{pl nom} \\
      \underline{kat}   \underline{o}       & $x_1$ + $x_2$             & \textsc{sg gen} \\
      \underline{kat} t \underline{o} i     & $x_1$ + t + $x_2$ + i     & \textsc{pl gen} \\
      \underline{kat} t \underline{o} je    & $x_1$ + t + $x_2$ + je    & \textsc{pl gen} \\
      \underline{kat} t \underline{o} a     & $x_1$ + t + $x_2$ + a     & \textsc{sg part} \\
      \underline{kat} t \underline{o} i     & $x_1$ + t + $x_2$ + i     & \textsc{pl part} \\
      \underline{kat} t \underline{o} ite   & $x_1$ + t + $x_2$ + ite   & \textsc{pl part} \\
      \underline{kat} t \underline{o} se    & $x_1$ + t + $x_2$ + se    & \textsc{sg ill} \\
      \underline{kat} t \underline{o} ise   & $x_1$ + t + $x_2$ + ise   & \textsc{pl ill} \\
      \underline{kat} t \underline{o} z     & $x_1$ + t + $x_2$ + z     & \textsc{sg ine} \\
      \underline{kat} t \underline{o} iz    & $x_1$ + t + $x_2$ + iz    & \textsc{pl ine} \\
      \underline{kat}   \underline{o} sse   & $x_1$ + $x_2$ + sse       & \textsc{sg ela} \\
      \underline{kat} t \underline{o} isse  & $x_1$ + t + $x_2$ + isse  & \textsc{pl ela} \\
      \underline{kat}   \underline{o} lle   & $x_1$ + $x_2$ + lle       & \textsc{sg all} \\
      \underline{kat} t \underline{o} ille  & $x_1$ + t + $x_2$ + ille  & \textsc{pl all} \\
      \underline{kat}   \underline{o} l     & $x_1$ + $x_2$ + l         & \textsc{sg ade} \\
      \underline{kat} t \underline{o} il    & $x_1$ + t + $x_2$ + il    & \textsc{pl ade} \\
      \underline{kat}   \underline{o} lte   & $x_1$ + $x_2$ + lte       & \textsc{sg abl} \\
      \underline{kat} t \underline{o} ilte  & $x_1$ + t + $x_2$ + ilte  & \textsc{pl abl} \\
      \underline{kat}   \underline{o} ssi   & $x_1$ + $x_2$ + ssi       & \textsc{sg tran} \\
      \underline{kat} t \underline{o} issi  & $x_1$ + t + $x_2$ + issi  & \textsc{pl tran} \\
      \underline{kat} t \underline{o} ssaa  & $x_1$ + t + $x_2$ + ssaa  & \textsc{sg term} \\
      \underline{kat} t \underline{o} issaa & $x_1$ + t + $x_2$ + issaa & \textsc{pl term} \\
      \underline{kat}   \underline{o} ka    & $x_1$ + $x_2$ + ka        & \textsc{sg com} \\
      \underline{kat} t \underline{o} ika   & $x_1$ + t + $x_2$ + ika   & \textsc{pl com} \\
    \end{tabular}
    \caption{Väljundi tüüpsõnamall (kus\-juures $x_1 = $ \textit{kat} ja $x_2 = $ \textit{o} vastab mallist leitud ühisosajadale).}
    \label{tab:väljundtabel_katto}
  \end{minipage}
\end{table}
\spacing{1.5}








\newpage
\section{Vadja morfoloogiliste tüüpsõnade analüüs}
\label{sec:analüüs}

See osa kirjeldab ekstraktmorfoloogiaga leitud vadja keele morfoloogilisi tüüpsõnu ja analüüsib neid muuttüüpidesse kasutades üht ekstraktmorfoloogia üldistust. Leitud muuttüüpe analüüsitakse seejärel vadja grammatikate ja läänemeresoome ajaloolise morfoloogia seisukohalt.




\subsection{Ekstraktmorfoloogiaga leitud tüüpsõnad}

See alaosa loendab leitud tüüpsõnad sõnaliigiti. Analüüsitakse tüüpsõnade alla kuuluvaid sõnu struktuurselt (kui mitu silpi, silpide struktuur).

(Analüüsidest on võimalik luua arvutikirjeldus hüpoteetilise vadjakeelse sõna üle õigekirjakontrollija jaoks.)




\subsection{Üks võimalik muuttüüpide süsteem}
\label{sec:muuttüüpide-süsteem}

Silfverberg ja Hulden (\citeyear{silfverberg_computational_2018}) on kirjeldanud üht formaalset meetodit, kuidas ekstrakt\-morfoloogia tüüpsõnade arvu vähendada. Siin alaosas rakendatakse meetodit leitud tüüpsõnadele ja esitatakse selle põhjal loodud vadja muuttüübistik ja võrreldatakse leitud muuttüübistikku .

Eesti muuttüüpide traditsioonist on kirjutanud mh \cite{viks_muuttuubid_nodate}.




\subsection{Põhivormid ja analoogiavormid}

Selles osas selgitatakse välja vadja keele tüüpsõnade põhi- ja analoogiavormid sõnaliigiti. Seda püütakse teha formaalselt põhinedes vaid ekstrakt\-morfoloogiaga leitud tüüpsõnamallidele.

\cite{erelt_eesti_2007} järgi ``[p]õhivormid on need vormid, mida pole võimalik teiste vormide alusel tuletada ning mille moodustamiseks tuleb iga sõnatüübi korral anda vastavad reeglid.'' ja ``[a]naloogiavormid on vormid, mida saab moodustada mingi põhivormi analoogial.''

Tegelikult on ekstraktmorfoloogia leitud LCS ainus põhivorm ja kõik muutvormid on sellest tuletatud analoogiavormid. Kuna aga läänemeresoome keelte keeleteaduses ei ole katkendlike põhivormide kasutamine traditsioonis (nagu seda on nt araabia keelte puhul), püütakse siin leida traditsioonilise käsitluse järgi põhi- ja analoogiavormid.

\subsubsection{Käändsõnad}
% Eesti keele käändsõna põhivormid on ainsuse nimetav, ainsuse omastav, ainsuse osastav, mitmuse omastav ja mitmuse osastav. Põhivormiks tuleb tingimisi lugeda ka ainsuse lühikest sisseütlevat.

\subsubsection{Tegusõnad}



\subsection{Muuttüüpide produktiivsus}

Kristiina Kross (Ross) nimetab produktiivsuseks ``mingi morfoloogilise nähtuse võimet allutada endale uusi sõnu'' (\cite{kross_eesti_1984}). Siin allosas seatakse eelmises osas leitud muuttüübid pingeritta selle järgi, kui mitu tüüpsõna nendele allub.

Kas selleks on vaja defineerida, mis on \textit{uus sõna}? Näiteks kõik uuemad vene keele laenud.

Kas produktiivsuse pingerida on võimalik jagada mingi kriteeriumi järgi avatuteks ja suletuteks muuttüüpideks?





\newpage
\section{Programmkoodi tuletamine}
\label{sec:programmkoodi-tuletamine}

Programmkoodi tuletamise all peetakse siin töös silmas mistahes protsessi, mille käigus tuletatakse mingi üldisema kirjelduse põhjal programmkoodi ühe või mitme konkreetse programmeerimiskeskkona jaoks.

Üldine kirjeldus (või teisisõnu ontoloogia) kirjeldab faktuaalselt \textit{mida} ning tuletatud programmkood kirjeldab konkreetselt \textit{kuidas} seda teadmist rakendada.

Töös kasutatakse keskseks kirjelduseks leksikaalset ressurssi, mille peamine osa koosneb ekstraktmorfoloogiaga leitud tüüpsõnade mallidest.

Keskse kirjelduse leksikaalset ressurssi hoitakse rahvusvahelise standardi vormingus \textit{Lexical Markup Framework} (\cite{iso/tc_37/sc_4_language_2007}).

Programmkoodi tuletavad nn generaatorid. Töös esitatakse kaht generaatorit, üks programmeerimiskeele Grammatical Framework jaoks ning teine Giella keeletehnoloogilise taristu integreerimise jaoks. Generaatorid on kirjutatud programmeerimiskeeles XQuery.




\subsection{Keskne kirjeldus Lexical Markup Framework vormingus}

% sissejuhatav tekst
Sissejuhatav tekst, mis on e-sõnastike ja leksikaalsete andmebaaside rahvusvaheline standard Lexical Markup Framework (\cite{iso/tc_37/sc_4_language_2007}) ja milleks seda kasutatakse. (märksõnu: semantika eeldefineeritud märgenduskeel; koostöövõime)

% laiendimoodulid
Standardi märgenduskeel koosneb mitmest eriotstarbelisest laiendimoodulist (vt nt \cite{francopoulo_lmf_2013}). Siinne töö kasutab kahte: morfoloogia moodul (\textit{LMF Morphology Extension}) ja morfoloogiliste paradigmade moodul (\textit{LMF Morphological Pattern Extension}).

% morfoloogiamoodul eesmärk
Morfoloogiamooduli eesmärgiks on kirjeldada morfoloogiat mahu kaudu, s.o kirjeldada lekseemi loendades kõiki selle muutvorme.

% paradigmamooduli eesmärk
Morfoloogiliste paradigmade mooduli eesmärgiks on seevastu kirjeldada sisu kaudu, s.o kirjeldada neid kriteeriume ja reegleid, millega saab moodustada kõik ühe lekseemi muutvormid. Selles töös kirjeldatakse ekstraktmorfoloogia tüüpsõnamalle antud mooduliga.

% topeltkirjeldus ju liigne?
Sama nähtuse kirjeldamine nii mahus kui ka sisus võib tunduda liigsena, ent niiviisi võimaldatakse rohkem informatsiooni hoidmist.

% aga ei ole -- kuigi kõlab rohkem nagu diskussiooni alla kuuluvat?
Näiteks võib iga lekseemi muutvormi kohta hoida informatsiooni nende reaalsetest korpusesinemustest. Niiviisi on võimalik klassifitseerida tüüpsõnade teoretiseeringutaset, kui ühe ja sama tüüpsõna alla kuuluvate lekseemide korpusleiud kinnitavad igat selle muutvormi, ei ole see teoretiseeritud.

% @TODO: mis veel häid külgi topeltkirjeldamisega on?

% mis seal veel hoitakse?
Peale sõnaartiklite ja morfoloogilise informatsiooni hoitakse leksikaalses ressursis ka globaalset informatsiooni, nagu keele nimetus ja kood.




\subsubsection{Sõnaartiklite esitamine LMFis}
Iga sõnaartikkel ehk leksikaalne kirje kannab informatsiooni lekseemi sõnaliigi kohta, selle valitud lemma vorm ning morfoloogiamooduliga esitatud muutvormitabeli.

\begin{figure}[p]
  \center
\begin{minted}[frame=single,fontsize=\small,framesep=10pt]{XML}
<LexicalEntry morphologicalPatterns="asKatto">
  <feat att="partOfSpeech" val="nn"/>
  <Lemma>
    <feat att="writtenForm" val="katto"/>
  </Lemma>
  <WordForm>
    <feat att="writtenForm" val="katto"/>
    <feat att="grammaticalNumber" val="singular"/>
    <feat att="grammaticalCase" val="nominative"/>
  </WordForm>
  <WordForm>
    <feat att="writtenForm" val="katod"/>
    <feat att="grammaticalNumber" val="plural"/>
    <feat att="grammaticalCase" val="nominative"/>
  </WordForm>
</LexicalEntry>
\end{minted}
\caption{Sõnaartikli \textit{katto} esitamine LMFis (muutvormid kajastatud vaid osaliselt).
  \label{code:lmf-lexicalentry}}
\end{figure}





\subsubsection{Tüüpsõnamallide esitamine LMFis}

\begin{figure}[p]
  \center
\begin{minted}[frame=single,fontsize=\footnotesize,framesep=10pt]{XML}
<MorphologicalPattern>
  <feat att="id" val="asTšiutto"/>
  <feat att="partOfSpeech" val="nn"/>
  <TransformSet>
    <GrammaticalFeatures>
      <feat att="grammaticalNumber" val="singular"/>
      <feat att="grammaticalCase" val="nominative"/>
    </GrammaticalFeatures>
    <Process>
      <feat att="operator" val="addAfter"/>
      <feat att="processType" val="pextractAddVariable"/>
      <feat att="variableNum" val="1"/>
    </Process>
    <Process>
      <feat att="operator" val="addAfter"/>
      <feat att="processType" val="pextractAddConstant"/>
      <feat att="stringValue" val="t"/>
    </Process>
    <Process>
      <feat att="operator" val="addAfter"/>
      <feat att="processType" val="pextractAddVariable"/>
      <feat att="variableNum" val="2"/>
    </Process>
  </TransformSet>
  <TransformSet>
    <GrammaticalFeatures>
      <feat att="grammaticalNumber" val="plural"/>
      <feat att="grammaticalCase" val="nominative"/>
    </GrammaticalFeatures>
    <Process>
      <feat att="operator" val="addAfter"/>
      <feat att="processType" val="pextractAddVariable"/>
      <feat att="variableNum" val="1"/>
    </Process>
    <Process>
      <feat att="operator" val="addAfter"/>
      <feat att="processType" val="pextractAddVariable"/>
      <feat att="variableNum" val="2"/>
    </Process>
    <Process>
      <feat att="operator" val="addAfter"/>
      <feat att="processType" val="pextractAddConstant"/>
      <feat att="stringValue" val="d"/>
    </Process>
  </TransformSet>
<MorphologicalPattern>
\end{minted}
\caption{Tüüpsõnamalli \texttt{tšiutto} (mille alla kuuluvad mh \textit{tšiutto} ja \textit{katto}) esitus LMFis. Esitus mudeldab muutvormimalle $x_1 \oplus \textbf{t} \oplus x_2$ ning $x_1 \oplus x_2 \oplus \textbf{d}$.
  \label{code:lmf-paradigmpattern}}
\end{figure}




\subsection{Grammatical Framework morfoloogiakomponent}

Mis on see, mida mina teen. Seejärel, mis on programmeerimiskeel Grammatical Framework ja milleks seda kasutatakse.

Morfoloogiakomponendi programmkood on jaotatav kaheks tükiks, leksikoniks ja muuttüüpide funktsioonid. Järgnevalt neist detailsemalt. Viimases alaosas on arutelu GFide võimalustest ja edasiarendusvõimalustest.

\subsubsection{Leksikon}
\label{sec:gf-leksikon}

\subsubsection{Muuttüübid}
\label{sec:gf-muuttüübid}

\subsubsection{Arutelu}
\label{sec:gf-arutelu}

Loodud morfoloogiakomponenti on kasutatud interaktiivses vadja-vene-vadja vestmikus.





\subsection{Integreerimine Giella-taristuga}

Keeletehnoloogilise taristuga Giella integreeritakse selles töös peamiselt selleks, et saada kätte õigekirja\-kontrollija. Giella-taristu koosneb veel võimalustest. Taristut kasutavad peamiselt Giellatekno ja Divvun.

Integreerimine on jagatav kaheks peamiseks osaks: leksikoni integreerimeerimine ja tüüpsõnamallide integreerimine. Seejärel kirjeldatakse taristu poolt loodud õigekirjakontrollija tööpõhimõtet ja lõpetuseks on arutelu.

\subsubsection{Leksikon}
\label{sec:giella-leksikon}

``Formally, the lexc language is a kind of right-recursive phrase-structure 
grammar.'' ja ``A lexc description compiles into a standard Xerox finite-state network, either a simple automaton or a transducer.'' (\cite[lk~203]{beesley_finite_2003}).

Kuigi lexc fraasi\-struktuuri\-grammatikatega on võimalik paradigmasid (tüüpsõnamalle) mudeldada, ja tavaliselt selleks seda kasutataksegi Giella taristus, võtab see töö teise lähenemisnurga ja lihtsustab võimalikult palju leksikoni struktuuri.

Leksikon koosneb selles töös ainult kahest andest: \textit{lemma} ja \textit{tüüpsõna}. 

\subsubsection{Muuttüübid}
\label{sec:giella-muuttüübid}

Paradigmade ehk tüüpsõnamallide esitus FST formalismis põhineb suuresti Forsbergi ja Huldeni (\citeyear{forsberg_learning_2016}) tööle.

Paradigmad esitatakse relatsioonidena sõnavormi ja lemma koos analüüsiga vahel. Sellised relatsioonid sisaldavad lõpmatut hulka sõnalemmasid, millest mõistagi pole suurem osa vadjakeelsed. Mis on siiski tähtis, on see, et relatsioonid mudeldavad paradigmasid.

Sõnade lõpmatu hulk piiratakse leksikonis antuga ja niiviisi saadakse leksikonis sisalduvate sõnade kõik sõnavormid. Nendest ja ainult nendest sõnavormidest koosnebki esialgne vadja õigekirja\-kontrollija.

\subsubsection{Õigekirjakontrollija}
\label{sec:giella-õigekirjakontrollija}

Eelnevalt kirjeldatud integreerimine Giella-taristusse võimaldab taristul luua õigekirjakontrollija. Mis on õigekirjakontrollija, kus seda kasutatakse ja mida see kontrollib?

\subsubsection{Arutelu}
\label{sec:giella-arutelu}

Loodud õigekirjakontrollija on eesmärgipäraselt jäetud lihtsakoeliseks. See märgib kõik sõnad valeks, mis ei sisaldu sõnastikus. See on lühiajaliseks kasutamiseks ja mõeldud ärgitama kasutajaid ise pakkuma täiendusi ja sõnaloomet vadja sõnastikusse.



\newpage
\section{Arutelu}

Arutelu struktuur peaks järgima üks-ühele sissejuhatuses väljatoodut, ent sellele siis lisama arutelu (sissejuhatus ainult nentingud).


\newpage
\section{Kokkuvõte}

Magistritöö on kirjeldanud süsteemi, millega on ühelt poolt defineeritud vadja keele normatiivne morfoloogia ja mille põhjal teisalt tuletatakse automaatselt morfoloogiline keeletehnoloogia.

Morfoloogilise normatiivi vajadust ajendab Heinike Heinsoo läbiviidud kursused keelekümbluskoolis Ämmesse Vunukassaa ja normatiiv on hõlpsasti muudetav-parendatav ilma programmeerimisoskusteta.

Saadud morfoloogilist tüübistikku on analüüsitud vadja keele grammatikatega ja põhjendatud ajaloolise morfoloogiaga.

---

% töö tuumik on tüüpsõnakirjeldused
Töö keskseks osaks on ekstraktmorfoloogiameetodiga saadud tüüpsõnakirjeldused.
% kirjeldused LMFi ja neist genereeritakse kood
Kirjeldused kodeeritakse koos sõnastikuga ümber standardsesse vormingusse ja saadud leksikaalse ressursi järgi tuletatakse automaatselt programmkoodi kahe keeletehnoloogilise platvormi jaoks, ja tagatakse seega vadja keele tugi nendes platvormides.

% seetõttu töötab ekstraktmorfoloogia liidesena
Niivisii kasutatakse ekstraktmorfoloogia meetodit kasutaja\-liidesena, mille abil koostatakse arvutimorfoloogia ainult tüüpsõnade muutvormitabeleid sedastades -- mitte programmeerides.

% kirjeldus kesksel kohal, parandused õigesse kohta
Magistritöös esitatud töövoog paneb leksikaalse ressursi kesksele kohale ja tuletatud tehnoloogia sellest teiseseks. Uue sõnavara ja vigade parandused tehakse ressursis, mitte mitmes tehnoloogias eraldi.

% standardi kasutamine tagab pikaajalise loetavus
Kuna nii tüüpsõnade kirjeldused, kui ka ülejäänud sõnastik kodeeritakse rahvusvahelise standardi Lexical Markup Framework vormingusse, tagatakse võimaluse ressursi pikaajaliseks arhiveerimiseks. Leksikaalne ressurss on loetav ja arusaadav palju kauem, kui seda on programmeerimiskood.

% panus dokumentaalsele lingvistikale
Viimase tõttu püüab magistritöö ühendada arvutuslingvistika ja dokumenteeriva lingvistika valdkondi.







\newpage
\section{Põhimõisted ja lühendid}
\label{sec:põhimõisted}
Siin loetletakse töös kasutatud mõisted ja lühendid koos nende tähendustega.

\spacing{1}
\glsaddall
\small{
  \printglossary[title={},toctitle={}]
}
\spacing{1.5}








\newpage
\section{Kirjandus}
\label{sec:kirjandus}
% @TODO: kuidas määrata biblatex-i keele eesti keelele?
\spacing{1}
{
  \renewcommand*{\bibfont}{\small}
  \printbibliography[heading=none]
}
\spacing{1.5}







\newpage
\section{The use of Extract Morphology for Automatic Derivation of Language Technology for Votic}

An English language summary of this work.







\newpage
\section{Lisad}

Siin on esitatud kõik ekstraheeritud tüüpsõnamallide tabelid.

\input{lmf-paradigms}


\end{document}

% Local Variables:
% TeX-engine: xelatex
% End:
