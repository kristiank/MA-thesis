\documentclass[12pt,a4paper]{article}

\usepackage[top=4cm, bottom=3cm, left=4cm, right=2.5cm]{geometry}
\usepackage{setspace}

\usepackage{float}

\usepackage{enumerate}

\usepackage{booktabs}

% Babel
\usepackage[utf8]{inputenc}
\usepackage[T1]{fontenc}
\usepackage[estonian]{babel}

\usepackage{csquotes}

% Polyglossias miskipärast ei tööta poolitamine
% \usepackage{polyglossia}
% \setdefaultlanguage{german}

% @todo: vormistada kirjanduse loetelu õigesti
% @todo: vormistada tekstis viited õigesti (nt lk kooloni järel jne)
\usepackage[backend=biber,style=authoryear,natbib=true]{biblatex}
\addbibresource{bibliography-mathesis.bib}

\usepackage{hyperref}

% Sõnastik
\usepackage[toc,nopostdot,xindy]{glossaries}
\makeglossaries

% sisukord
% Bakalaureusetöös on soovitatav piirduda kolmeastmelise hierarhiaga, magistritöös neljaastmelisega
\setcounter{tocdepth}{4}


% kasutatud mõisted ja lühendid
\newglossaryentry{tüüpsõnamall}{
  name=tüüpsõnamall,
  description={on ekstraktmorfoloogiaga leitud paradigma kirjeldus, mis koosneb iga muutvormi koostamisskeemist}
}
\newglossaryentry{mikrostruktuur}{
  name=mikrostruktuur,
  description={on sõnastiku sõnaartikli sisemine struktuur}
}



\begin{document}


% @todo: lisada magistritöö, ülikool ja juhendaja jne
% vaata siit https://tex.stackexchange.com/questions/184848/how-to-add-the-name-of-the-supervisor-in-a-thesis-field#184878
% @todo: vormistada esileht eraldi
\title{Ekstraktmorfoloogia meetodiga tuletatud keeletehnoloogia vadja sõnavara näitel}
\author{Kristian Kankainen}
% \supervisor{Heinike Heinsoo, Külli Prillop}
% \university{Tartu Ülikool}
% \department{Eesti ja üldkeeleteaduse õppetool}
\date{2019}
\maketitle


\newpage
\tableofcontents



\newpage
\spacing{1.5}
\section{Sissejuhatus}

Magistritöö loob viisi ehitada arvutimorfoloogia puhtalt lekseemide sõnavormide esitamise teel ning teisendada ehitatud arvutimorfoloogilise mudeli automaatselt kahte keeletehnoloogilisse raamistikku.

Magistritöö kasutab loodud süsteemi selleks, et kirjeldada vadja keele normatiivsed morfoloogilised tüüpsõnad.

Tööd ajendab mõtteviis minimeerida tööd: loodud normatiivne morfoloogiline tüübistik on aluseks automaatselt tuletatud keeletehnoloogiale, kui normatiiv muutub, muutub ka keeletehnoloogia. Töö paneb leksikaalse ressursi esikohale ja kõik leitud sisulised vead õiendatakse otse ressursis, mitte keeletehnoloogilistes tarkvarades eraldi.




\subsection{Teoreetilised lähtekohad}

Morfeemi ei käsitleta siin töös levinud lingvistilisest seisukohast kui \textit{väikseimat tähenduslikku üksust}, vaid klassikalistele paradigmaatilistele lähenemistele omaselt kui \textit{mistahes tähtkoostise muutust, millega kaasneb tähenduslik muutus} (\cite{beard_morpheme_1987}, \cite{beard_lexeme-morpheme_1995}).




\newpage
\section{Ekstraktmorfoloogia meetod}
\label{sec:ekstraktmorfoloogia-meetod}
See osa kirjeldab töös rakendatud ekstraktmorfoloogia meetodit. Töö kasutab ekstraktmorfoloogiat kaheks otstarbeks, esiteks vadja keele morfoloogiliste tüüpsõnade väljaselgitamiseks ja kirjeldamiseks ja teisalt programmkoodi automaatseks tuletamiseks saadud kirjelduse põhjal. Neid kahte rakendust kirjeldatatakse lähemalt vastavates peatükkides \textit{\nameref{sec:analüüs}} ja \textit{\nameref{sec:programmkoodi-tuletamine}}.




\subsection{Sissejuhatus}
\label{sec:ekstraktmorfoloogia-sissejuhatus}

% juhendatud masinõppe meetod -- aga masinõppe ei kõla minu jaoks hästi
Ekstraktmorfoloogia on juhendatud masinõppe meetod, mis üldistab lekseemide muutvormitabeleid ja eraldab neist \glslink{tüüpsõnamall}{tüüpsõnamallid}. See on \textit{juhendatud}, sest sisendiks olevad muutvormitabelid peavad olema korrektselt koostatud. %Ja see on \textit{masinõppe}, sest sisendist analüüsitud mudel on laiendatav ka uuele ja tundmatule sisendile.

% masinõppe asemel kasutan ma seda kirjeldusena ja pealdisena paradigmadele ning liidesena arvutimorfoloogia deklareerimiseks
Selles töös käsitletakse meetodi abil saadud mudelit siiski pigem lihtsa kirjeldusena. See on tüüpsõnakirjeldus, mis on osa sõnastikust -- lekseemi paradigma kirjeldusena. Ja sellest kirjeldusest 
% kirjeldus on inimloetav (viidatav fakt)

% tüüpsõnamalli kohta
Tüüpsõna\-mall koosneb muutvormide mallidest ja vastab seega morfoloogilise paradigma mõistele. 
Tüüpsõna\-malli abil on võimalik moodustada ka tundmatu sõna kõik muutvormid.
% produktiivsusmalli kohta
Kuna kaks või enam lekseemi võivad jagada üht ja sama tüüpsõna\-malli (s.o kuuluda sama paradigmasse), on võimalik ekstrakt\-morfoloogia meetodiga üldistada lekseemide iseärasusi ja luua nendest tüüpsõnade produktiivsuse mudeli. Produktiivsus\-mudeliga on võimalik ennustada uue ja tundmatu sõnavormi kuuluvust ühe või teise tüüpsõna alla. 


Veel ilma detailidesse takerdumata näitlikustatakse siinkohal lugejale meetodi sisendit ja väljundit. Sisendiks on ühe lekseemi muutvormitabel tervikuna (vt tabel\nobreakspace \ref{tab:sisendtabel_katto}). Väljundiks on meetodi poolt leitud tüüpsõnamall (vt tabel\nobreakspace \ref{tab:väljundtabel_katto}). Tabelitele viidatakse alljärgnevas tekstis mitmel korral.

\spacing{1.0}
\begin{table}[H] %[!htbp] % kuvab tabelit definiitselt enne neile järgnevat teksti
      \footnotesize
  \begin{minipage}[t]{.40\textwidth}
%    \centering
    \begin{tabular}[t]{l l}
      muutvorm            & tunnused \\ \hline
      \textit{katto}      & \textsc{sg nom} \\
      \textit{katod}      & \textsc{pl nom} \\
      \textit{kato}       & \textsc{sg gen} \\
      \textit{kattoi}     & \textsc{pl gen} \\
      \textit{kattoje}    & \textsc{pl gen} \\
      \textit{kattoa}     & \textsc{sg part} \\
      \textit{kattoi}     & \textsc{pl part} \\
      \textit{kattoite}   & \textsc{pl part} \\
      \textit{kattose}    & \textsc{sg ill} \\
      \textit{kattoise}   & \textsc{pl ill} \\
      \textit{kattoz}     & \textsc{sg ine} \\
      \textit{kattoiz}    & \textsc{pl ine} \\
      \textit{katosse}    & \textsc{sg ela} \\
      \textit{kattoisse}  & \textsc{pl ela} \\
      \textit{katolle}    & \textsc{sg all} \\
      \textit{kattoille}  & \textsc{pl all} \\
      \textit{katol}      & \textsc{sg ade} \\
      \textit{kattoil}    & \textsc{pl ade} \\
      \textit{katolte}    & \textsc{sg abl} \\
      \textit{kattoilte}  & \textsc{pl abl} \\
      \textit{katossi}    & \textsc{sg tran} \\
      \textit{kattoissi}  & \textsc{pl tran} \\
      \textit{kattossaa}  & \textsc{sg term} \\
      \textit{kattoissaa} & \textsc{pl term} \\
      \textit{katoka}     & \textsc{sg com} \\
      \textit{kattoika}   & \textsc{pl com} \\
    \end{tabular}
    \caption{Sisendi muutvormide tabel koos morfo\-loogiliste tunnustega.}
    \label{tab:sisendtabel_katto}
  \end{minipage}
  \hfill
  \begin{minipage}[t]{.55\textwidth}
    \centering
    \begin{tabular}[t]{l l l}
      ühisosajada                     & muutvormi\-mall           & tunnused \\
      \hline
      \textbf{kat} t \textbf{o}       & $x_1$ + t + $x_2$         & \textsc{sg nom} \\
      \textbf{kat}   \textbf{o} d     & $x_1$ + $x_2$ + d         & \textsc{pl nom} \\
      \textbf{kat}   \textbf{o}       & $x_1$ + $x_2$             & \textsc{sg gen} \\
      \textbf{kat} t \textbf{o} i     & $x_1$ + t + $x_2$ + i     & \textsc{pl gen} \\
      \textbf{kat} t \textbf{o} je    & $x_1$ + t + $x_2$ + je    & \textsc{pl gen} \\
      \textbf{kat} t \textbf{o} a     & $x_1$ + t + $x_2$ + a     & \textsc{sg part} \\
      \textbf{kat} t \textbf{o} i     & $x_1$ + t + $x_2$ + i     & \textsc{pl part} \\
      \textbf{kat} t \textbf{o} ite   & $x_1$ + t + $x_2$ + ite   & \textsc{pl part} \\
      \textbf{kat} t \textbf{o} se    & $x_1$ + t + $x_2$ + se    & \textsc{sg ill} \\
      \textbf{kat} t \textbf{o} ise   & $x_1$ + t + $x_2$ + ise   & \textsc{pl ill} \\
      \textbf{kat} t \textbf{o} z     & $x_1$ + t + $x_2$ + z     & \textsc{sg ine} \\
      \textbf{kat} t \textbf{o} iz    & $x_1$ + t + $x_2$ + iz    & \textsc{pl ine} \\
      \textbf{kat}   \textbf{o} sse   & $x_1$ + $x_2$ + sse       & \textsc{sg ela} \\
      \textbf{kat} t \textbf{o} isse  & $x_1$ + t + $x_2$ + isse  & \textsc{pl ela} \\
      \textbf{kat}   \textbf{o} lle   & $x_1$ + $x_2$ + lle       & \textsc{sg all} \\
      \textbf{kat} t \textbf{o} ille  & $x_1$ + t + $x_2$ + ille  & \textsc{pl all} \\
      \textbf{kat}   \textbf{o} l     & $x_1$ + $x_2$ + l         & \textsc{sg ade} \\
      \textbf{kat} t \textbf{o} il    & $x_1$ + t + $x_2$ + il    & \textsc{pl ade} \\
      \textbf{kat}   \textbf{o} lte   & $x_1$ + $x_2$ + lte       & \textsc{sg abl} \\
      \textbf{kat} t \textbf{o} ilte  & $x_1$ + t + $x_2$ + ilte  & \textsc{pl abl} \\
      \textbf{kat}   \textbf{o} ssi   & $x_1$ + $x_2$ + ssi       & \textsc{sg tran} \\
      \textbf{kat} t \textbf{o} issi  & $x_1$ + t + $x_2$ + issi  & \textsc{pl tran} \\
      \textbf{kat} t \textbf{o} ssaa  & $x_1$ + t + $x_2$ + ssaa  & \textsc{sg term} \\
      \textbf{kat} t \textbf{o} issaa & $x_1$ + t + $x_2$ + issaa & \textsc{pl term} \\
      \textbf{kat}   \textbf{o} ka    & $x_1$ + $x_2$ + ka        & \textsc{sg com} \\
      \textbf{kat} t \textbf{o} ika   & $x_1$ + t + $x_2$ + ika   & \textsc{pl com} \\
    \end{tabular}
    \caption{Väljundi tüüpsõnamall (kus\-juures $x_1 = $ \textit{kat} ja $x_2 = $ \textit{o} vastab leitud ühisosajadale).}
    \label{tab:väljundtabel_katto}
  \end{minipage}
\end{table}
\spacing{1.5}








\newpage
\section{Vadja morfoloogiliste tüüpsõnade analüüs}
\label{sec:analüüs}

See osa kirjeldab ekstraktmorfoloogiaga leitud vadja keele morfoloogilisi tüüpsõnu ja analüüsib nende vastavust vadja keele grammatikatega ja ajaloolise morfoloogiaga.




\subsection{Põhivormid ja analoogiavormid}

Selles osas selgitatakse välja vadja keele tüüpsõnade põhi- ja analoogiavormid sõnaliigiti.

\cite{erelt_eesti_2007} järgi ``[p]õhivormid on need vormid, mida pole võimalik teiste vormide alusel tuletada ning mille moodustamiseks tuleb iga sõnatüübi korral anda vastavad reeglid.'' ja ``[a]naloogiavormid on vormid, mida saab moodustada mingi põhivormi analoogial.''




\subsubsection{Käändsõnad}
% Eesti keele käändsõna põhivormid on ainsuse nimetav, ainsuse omastav, ainsuse osastav, mitmuse omastav ja mitmuse osastav. Põhivormiks tuleb tingimisi lugeda ka ainsuse lühikest sisseütlevat.

\subsubsection{Tegusõnad}




\newpage
\section{Programmkoodi tuletamine}
\label{sec:programmkoodi-tuletamine}

Programmkoodi tuletamise all peetakse siin töös silmas mistahes protsessi, mille käigus tuletatakse mingi üldisema kirjelduse põhjal programmkoodi ühe või mitme konkreetse programmeerimiskeskkona jaoks.

Üldine kirjeldus (või teisisõnu ontoloogia) kirjeldab faktuaalselt \textit{mida} ning tuletatud programmkood kirjeldab konkreetselt \textit{kuidas} seda teadmist rakendada.

Töös kasutatakse keskseks kirjelduseks leksikaalset ressurssi, mille peamine osa koosneb ekstraktmorfoloogiaga leitud tüüpsõnade mallidest.

Keskse kirjelduse leksikaalset ressurssi hoitakse rahvusvahelise standardi vormingus \textit{Lexical Markup Framework} (\cite{iso/tc_37/sc_4_language_2007}).

Programmkoodi tuletavad nn generaatorid. Töös esitatakse kaht generaatorit, üks programmeerimiskeele Grammatical Framework jaoks ning teine keeletehnoloogilise taristu Giellatekno integreerimise jaoks.




\subsection{Keskne kirjeldus Lexical Markup Framework vormingus}

% sissejuhatav tekst
Sissejuhatav tekst, mis on e-sõnastike ja leksikaalsete andmebaaside rahvusvaheline standard Lexical Markup Framework (\cite{iso/tc_37/sc_4_language_2007}) ja milleks seda kasutatakse. (märksõnu: semantika eeldefineeritud märgenduskeel; koostöövõime)

% laiendimoodulid
Standard koosneb mitmest laiendimoodulist (vt nt \cite{francopoulo_lmf_2013}). Siinne töö kasutab kahte: morfoloogia moodul (\textit{LMF Morphology Extension}) ja morfoloogiliste paradigmade moodul (\textit{LMF Morphological Pattern Extension}).

% morfoloogiamoodul eesmärk
Morfoloogiamooduli eesmärgiks on kirjeldada morfoloogiat mahu kaudu, s.o kirjeldada lekseemi loendades kõiki selle muutvorme.

% paradigmamooduli eesmärk
Morfoloogiliste paradigmade mooduli eesmärgiks on seevastu kirjeldada sisu kaudu, s.o kirjeldada neid kriteeriume ja reegleid, millega saab moodustada kõik ühe lekseemi muutvormid. Selles töös kirjeldatakse ekstraktmorfoloogia tüüpsõnamalle antud mooduliga.

% topeltkirjeldus ju liigne?
Sama nähtuse kirjeldamine nii mahus kui ka sisus võib tunduda liigsena, ent niiviisi võimaldatakse rohkem informatsiooni hoidmist.

% aga ei ole -- kuigi kõlab rohkem nagu diskussiooni alla kuuluvat?
Näiteks võib iga lekseemi muutvormi kohta hoida informatsiooni nende reaalsetest korpusesinemustest. Niiviisi on võimalik klassifitseerida tüüpsõnade teoretiseeringutaset, kui ühe ja sama tüüpsõna alla kuuluvate lekseemide korpusleiud kinnitavad igat selle muutvormi, ei ole see teoretiseeritud.

% @TODO: mis veel häid külgi topeltkirjeldamisega on?

% mis seal veel hoitakse?
Peale morfoloogilise informatsiooni


\subsection{Grammatical Framework morfoloogiakomponent}

Mis on programmeerimiskeel Grammatical Framework ja milleks seda kasutatakse.




\subsection{Giellatekno taristuga integreerimine}

Mis on keeletehnoloogiline taristu Giellatekno ja milleks seda kasutatakse. Kes seda kasutavad.








\newpage
\section{Kokkuvõte}

Magistritöö on kirjeldanud süsteemi, millega on ühelt poolt defineeritud vadja keele normatiivne morfoloogia ja mille põhjal teisalt tuletatakse automaatselt morfoloogiline keeletehnoloogia.

Morfoloogilise normatiivi vajadust ajendab Heinike Heinsoo läbiviidud kursused keelekümbluskoolis Ämmesse Vunukassaa ja normatiiv on hõlpsasti muudetav-parendatav ilma programmeerimisoskusteta.

Saadud morfoloogilist tüübistikku on analüüsitud vadja keele grammatikatega ja põhjendatud ajaloolise morfoloogiaga.

---

% töö tuumik on tüüpsõnakirjeldused
Töö keskseks osaks on ekstraktmorfoloogiameetodiga saadud tüüpsõnakirjeldused.
% kirjeldused LMFi ja neist genereeritakse kood
Kirjeldused kodeeritakse koos sõnastikuga ümber standardsesse vormingusse ja saadud leksikaalse ressursi järgi tuletatakse automaatselt programmkoodi kahe keeletehnoloogilise platvormi jaoks, ja tagatakse seega vadja keele tugi nendes platvormides.

% seetõttu töötab ekstraktmorfoloogia liidesena
Niivisii kasutatakse ekstraktmorfoloogia meetodit kasutaja\-liidesena, mille abil koostatakse arvutimorfoloogia ainult tüüpsõnade muutvormitabeleid sedastades -- mitte programmeerides.

% kirjeldus kesksel kohal, parandused õigesse kohta
Magistritöös esitatud töövoog paneb leksikaalse ressursi kesksele kohale ja tuletatud tehnoloogia sellest teiseseks. Uue sõnavara ja vigade parandused tehakse ressursis, mitte mitmes tehnoloogias eraldi.

% standardi kasutamine tagab pikaajalise loetavus
Kuna nii tüüpsõnade kirjeldused, kui ka ülejäänud sõnastik kodeeritakse rahvusvahelise standardi Lexical Markup Framework vormingusse, tagatakse võimaluse ressursi pikaajaliseks arhiveerimiseks. Leksikaalne ressurss on loetav ja arusaadav palju kauem, kui seda on programmeerimiskood.

% panus dokumentaalsele lingvistikale
Viimase tõttu püüab magistritöö ühendada arvutuslingvistika ja dokumenteeriva lingvistika valdkondi.







\newpage
\section{Põhimõisted ja lühendid}
Siin loetletakse töös kasutatud mõisted ja lühendid koos nende tähendustega.

\spacing{1}
\glsaddall
% \makeglossaries
\printglossary[title={},toctitle={}]
\spacing{1.5}








\newpage
\section{Kirjandus}
\label{sec:kirjandus}
% @TODO: kuidas määrata biblatex-i keele eesti keelele?
\spacing{1}
{
  %\renewcommand*{\bibfont}{\small}
  \printbibliography[heading=none]
}
\spacing{1.5}







\newpage
\section{The use of Extract Morphology for Automatic Derivation of Language Technology for Votic}

An English language summary of this work.








\end{document}

% Local Variables:
% TeX-engine: xelatex
% End:
