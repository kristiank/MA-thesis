\paragraph*{\vadja{\underline{jalgõ}z}}
\vadja{\underline{jalgõ}sõ}, \vadja{\underline{jalgõ}ssõ}, \vadja{\underline{jalgõ}ssõ}, \vadja{\underline{jalgõ}sõss}, \vadja{\underline{jalgõ}sõd}, \vadja{\underline{jalgõ}ssijõ}, \vadja{\underline{jalgõ}ssiit}, \vadja{\underline{jalgõ}ssiisõ}, \vadja{\underline{jalgõ}ssiiss}
 \\
Tüüpsõna hõlmab vormisõnastiku lekseeme \vadja{jalgõz, kagluz, kavaluz, lad̕d̕uz, l̕innõz, porotuz, raskõuz, sõrmuz, varõz} ja \vadja{aluz}.
