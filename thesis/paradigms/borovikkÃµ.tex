\paragraph*{\vadja{\underline{borovik}kõ}}
\vadja{\underline{borovik}a}, \vadja{\underline{borovik}ka}, \vadja{\underline{borovik}kasõ}, \vadja{\underline{borovik}õss}, \vadja{\underline{borovik}õd}, \vadja{\underline{borovik}kojõ}, \vadja{\underline{borovik}koit}, \vadja{\underline{borovik}koisõ}, \vadja{\underline{borovik}koiss}
 \\
Sõnatüüp hõlmab vormisõnastiku lekseeme: \vadja{borovikkõ, domovikkõ, durakkõ, fartukkõ, fiizikkõ, fookusnikkõ, frištikkõ, gribanikkõ, harakkõ, itikkõ, joožikkõ, kaamenšikkõ, kabakkõ, kamal̕ikkõ, katol̕ikkõ, kelkkõ, koomikkõ, kopekkõ, latikkõ, luzikkõ, luukkõ, markkõ, muuzikkõ, mõiznikkõ, noorikkõ, nuužnikkõ, obakkõ, paikkõ, palkkõ, pinžakkõ, podarkkõ, poštimarkkõ, rankkõ, rohosirkkõ, tarkkõ, tikkõ, tubakkõ, urokkõ, vakkõ, vunukkõ, bašmukkõ}.
