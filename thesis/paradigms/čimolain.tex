\paragraph*{\vadja{\underline{čimolai}n}}
\vadja{\underline{čimolai}zõ}, \vadja{\underline{čimolai}ssõ}, \vadja{\underline{čimolai}zõsõ}, \vadja{\underline{čimolai}zõss}, \vadja{\underline{čimolai}zõd}, \vadja{\underline{čimolai}sijõ}, \vadja{\underline{čimolai}siit}, \vadja{\underline{čimolai}siisõ}, \vadja{\underline{čimolai}siiss}
 \\
Tüüpsõna hõlmab vormisõnastiku 37 lekseemi: \vadja{čimolain, greekklain, hatukkõin, iirikkõin, il̕l̕õkkõin, iloin, jõkain, kehnokkõin, keskolin, kõikõllain, kõrvõlin, leivekkõin, luin, lättilain, magnettiin, main, mokomõin, mustõlain, nain, partõin, perennain, prikukkõin, puin, roottsilain, ruskolain, saunlain, soomõlain, sopuin, sukulain, talviin, tarttulain, tõin, ukrainalain, virolain, õhtõgoin, ühellain} ja \vadja{audžikkõin}.
