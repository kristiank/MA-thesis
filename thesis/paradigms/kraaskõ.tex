
\vspace{1.8em}
\begin{minipage}{\textwidth}
\stepcounter{mallinumber}
\textbf{Tüüpsõnamall \arabic{mallinumber}\,\vadja{kraaskõ}}\\

\begin{sideways}
\begin{tabular}{l l}
muutvormimall & tunnused \\
\hline
\underline{kraa}\,+\,skõ & \textsc{ sg nom } \\
\underline{kraa}\,+\,zga & \textsc{ sg gen } \\
\underline{kraa}\,+\,ska & \textsc{ sg par } \\
\underline{kraa}\,+\,skasõ & \textsc{ sg ill } \\
\underline{kraa}\,+\,zgõz & \textsc{ sg ine } \\
\underline{kraa}\,+\,zgõss & \textsc{ sg ela } \\
\underline{kraa}\,+\,zgõllõ & \textsc{ sg all } \\
\underline{kraa}\,+\,zgõll & \textsc{ sg ade } \\
\underline{kraa}\,+\,zgõlt & \textsc{ sg abl } \\
\underline{kraa}\,+\,zgõssi & \textsc{ sg tra } \\
\underline{kraa}\,+\,zgõssaa & \textsc{ sg ter } \\
\underline{kraa}\,+\,zgõka & \textsc{ sg com } \\
\underline{kraa}\,+\,zgõd & \textsc{ pl nom } \\
\underline{kraa}\,+\,skojõ & \textsc{ pl gen } \\
\underline{kraa}\,+\,skoit & \textsc{ pl par } \\
\underline{kraa}\,+\,skoisõ & \textsc{ pl ill } \\
\underline{kraa}\,+\,skoiz & \textsc{ pl ine } \\
\underline{kraa}\,+\,skoiss & \textsc{ pl ela } \\
\underline{kraa}\,+\,skoillõ & \textsc{ pl all } \\
\underline{kraa}\,+\,skoill & \textsc{ pl ade } \\
\underline{kraa}\,+\,skoilt & \textsc{ pl abl } \\
\underline{kraa}\,+\,skoissi & \textsc{ pl tra } \\
\underline{kraa}\,+\,skoissaa & \textsc{ pl ter } \\
\underline{kraa}\,+\,skoika & \textsc{ pl com } \\
\end{tabular}
\end{sideways}
\captionof{table}{Tüüpsõna \arabic{mallinumber}\,\textit{kraaskõ} ekstraheeritud muutvormimallid.}
\label{tab:tüüpsõnamall-kraaskõ}

\end{minipage}

 
\vspace{1em}
\noindent Tüüpsõna hõlmab vormisõnastiku 6 lekseemi: \vadja{\underline{kraa}skõ, \underline{lai}skõ, \underline{nagriskaa}skõ, \underline{ni}skõ, \underline{pa}skõ} ja \vadja{\underline{kaa}skõ}.
