
\vspace{1.8em}
\begin{minipage}{\textwidth}
\stepcounter{mallinumber}
\textbf{Tüüpsõnamall \arabic{mallinumber}\,\vadja{biblioteekkõ}}\\

\begin{sideways}
\begin{tabular}{l l}
muutvormimall & tunnused \\
\hline
\underline{biblioteek}\,+\,kõ & \textsc{ sg nom } \\
\underline{biblioteek}\,+\,a & \textsc{ sg gen } \\
\underline{biblioteek}\,+\,ka & \textsc{ sg par } \\
\underline{biblioteek}\,+\,kasõ & \textsc{ sg ill } \\
\underline{biblioteek}\,+\,kõz & \textsc{ sg ine } \\
\underline{biblioteek}\,+\,õss & \textsc{ sg ela } \\
\underline{biblioteek}\,+\,õllõ & \textsc{ sg all } \\
\underline{biblioteek}\,+\,õll & \textsc{ sg ade } \\
\underline{biblioteek}\,+\,õlt & \textsc{ sg abl } \\
\underline{biblioteek}\,+\,õssi & \textsc{ sg tra } \\
\underline{biblioteek}\,+\,kõssaa & \textsc{ sg ter } \\
\underline{biblioteek}\,+\,õka & \textsc{ sg com } \\
\underline{biblioteek}\,+\,õd & \textsc{ pl nom } \\
\underline{biblioteek}\,+\,kijõ & \textsc{ pl gen } \\
\underline{biblioteek}\,+\,kiit & \textsc{ pl par } \\
\underline{biblioteek}\,+\,kiisõ & \textsc{ pl ill } \\
\underline{biblioteek}\,+\,kiiz & \textsc{ pl ine } \\
\underline{biblioteek}\,+\,kiiss & \textsc{ pl ela } \\
\underline{biblioteek}\,+\,kiillõ & \textsc{ pl all } \\
\underline{biblioteek}\,+\,kiill & \textsc{ pl ade } \\
\underline{biblioteek}\,+\,kiilt & \textsc{ pl abl } \\
\underline{biblioteek}\,+\,kiissi & \textsc{ pl tra } \\
\underline{biblioteek}\,+\,kiissaa & \textsc{ pl ter } \\
\underline{biblioteek}\,+\,kijka & \textsc{ pl com } \\
\end{tabular}
\end{sideways}
\captionof{table}{Tüüpsõna \arabic{mallinumber}\,\textit{biblioteekkõ} ekstraheeritud muutvormimallid.}
\label{tab:tüüpsõnamall-biblioteekkõ}

\end{minipage}

 
\vspace{1em}
\noindent Tüüpsõna hõlmab vormisõnastiku 10 lekseemi: \vadja{\underline{biblioteek}kõ, \underline{hoik}kõ, \underline{ikolook}kõ, \underline{jaanikuk}kõ, \underline{kolk}kõ, \underline{kon̕jõk}kõ, \underline{kuk}kõ, \underline{rok}kõ, \underline{suk}kõ} ja \vadja{\underline{bambuk}kõ}.
