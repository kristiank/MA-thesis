

\vspace{3.5em}
\noindent \begin{minipage}{\textwidth}
\stepcounter{mallinumber}
\noindent \textbf{Tüüpsõnamall \arabic{mallinumber}\,\vadja{kanka}}\\

\begin{sideways}
\begin{tabular}{l l}
muutvormimall & tunnused \\
\hline
\underline{kank}\,$\oplus$\,a & \textsc{ sg nom } \\
\underline{kank}\,$\oplus$\,a & \textsc{ sg gen } \\
\underline{kank}\,$\oplus$\,atõ & \textsc{ sg par } \\
\underline{kank}\,$\oplus$\,asõ & \textsc{ sg ill } \\
\underline{kank}\,$\oplus$\,az & \textsc{ sg ine } \\
\underline{kank}\,$\oplus$\,ass & \textsc{ sg ela } \\
\underline{kank}\,$\oplus$\,allõ & \textsc{ sg all } \\
\underline{kank}\,$\oplus$\,all & \textsc{ sg ade } \\
\underline{kank}\,$\oplus$\,alt & \textsc{ sg abl } \\
\underline{kank}\,$\oplus$\,assi & \textsc{ sg tra } \\
\underline{kank}\,$\oplus$\,assaa & \textsc{ sg ter } \\
\underline{kank}\,$\oplus$\,aka & \textsc{ sg com } \\
\underline{kank}\,$\oplus$\,ad & \textsc{ pl nom } \\
\underline{kank}\,$\oplus$\,õjõ & \textsc{ pl gen } \\
\underline{kank}\,$\oplus$\,õit & \textsc{ pl par } \\
\underline{kank}\,$\oplus$\,õisõ & \textsc{ pl ill } \\
\underline{kank}\,$\oplus$\,õiz & \textsc{ pl ine } \\
\underline{kank}\,$\oplus$\,õiss & \textsc{ pl ela } \\
\underline{kank}\,$\oplus$\,õillõ & \textsc{ pl all } \\
\underline{kank}\,$\oplus$\,õill & \textsc{ pl ade } \\
\underline{kank}\,$\oplus$\,õilt & \textsc{ pl abl } \\
\underline{kank}\,$\oplus$\,õissi & \textsc{ pl tra } \\
\underline{kank}\,$\oplus$\,õissaa & \textsc{ pl ter } \\
\underline{kank}\,$\oplus$\,õika & \textsc{ pl com } \\
\end{tabular}
\end{sideways}
\captionof{table}{Tüüpsõna \arabic{mallinumber}\,\textit{kanka} ekstraheeritud muutvormimallid.}
\label{tab:tüüpsõnamall-kanka}

\end{minipage}

 
\vspace{1em}
\noindent Tüüpsõna hõlmab vormisõnastikus 8 lekseemi: \vadja{\underline{kank}a, \underline{kõrk}a, \underline{maikk}a, \underline{makk}a, \underline{rusk}a, \underline{valk}a, \underline{õik}a} ja \vadja{\underline{harm}a}.
