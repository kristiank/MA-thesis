\paragraph*{\vadja{\underline{flak}k\underline{u}}}
\vadja{\underline{flak}\underline{u}}, \vadja{\underline{flak}k\underline{u}a}, \vadja{\underline{flak}k\underline{u}sõ}, \vadja{\underline{flak}\underline{u}ss}, \vadja{\underline{flak}\underline{u}d}, \vadja{\underline{flak}k\underline{u}jõ}, \vadja{\underline{flak}k\underline{u}it}, \vadja{\underline{flak}k\underline{u}isõ}, \vadja{\underline{flak}k\underline{u}iss}
 \\
Tüüpsõna hõlmab vormisõnastiku lekseeme \vadja{flakku, herkku, jõkilikko, kakko, kakku, kiikku, kolkku, kukko, kurkku, kuuzikko, lepikko, liivikko, luikko, lukku, lõõkku, majakko, musikko, mäčizikko, naizikko, oomnikko, pettelikko, rehtelkakku, seukko, võrkko, õzrikko, čerikko}.
