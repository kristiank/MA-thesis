\paragraph*{\vadja{\underline{biblioteek}kõ}}
\vadja{\underline{biblioteek}a}, \vadja{\underline{biblioteek}ka}, \vadja{\underline{biblioteek}kasõ}, \vadja{\underline{biblioteek}õss}, \vadja{\underline{biblioteek}õd}, \vadja{\underline{biblioteek}kijõ}, \vadja{\underline{biblioteek}kiit}, \vadja{\underline{biblioteek}kiisõ}, \vadja{\underline{biblioteek}kiiss}
 \\
Tüüpsõna hõlmab vormisõnastiku lekseeme \vadja{biblioteekkõ, hoikkõ, ikolookkõ, jaanikukkõ, kolkkõ, kon̕jõkkõ, kukkõ, rokkõ, sukkõ, bambukkõ}.
