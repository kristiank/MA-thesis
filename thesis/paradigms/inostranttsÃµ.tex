\paragraph*{\vadja{\underline{inostrant}t\underline{s}õ}}
\vadja{\underline{inostrant}\underline{s}a}, \vadja{\underline{inostrant}t\underline{s}a}, \vadja{\underline{inostrant}t\underline{s}asõ}, \vadja{\underline{inostrant}\underline{s}õss}, \vadja{\underline{inostrant}\underline{s}õd}, \vadja{\underline{inostrant}t\underline{s}ojõ}, \vadja{\underline{inostrant}t\underline{s}oit}, \vadja{\underline{inostrant}t\underline{s}oisõ}, \vadja{\underline{inostrant}t\underline{s}oiss}
 \\
Tüüpsõna hõlmab vormisõnastiku lekseeme: \vadja{inostranttsõ, liittsõ, tablittsõ, vattsõ, õttsõ, bol̕nittsõ}.
