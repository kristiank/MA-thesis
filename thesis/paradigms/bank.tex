\paragraph*{\vadja{\underline{bank}}}
\vadja{\underline{bank}a}, \vadja{\underline{bank}a}, \vadja{\underline{bank}asõ}, \vadja{\underline{bank}õss}, \vadja{\underline{bank}õd}, \vadja{\underline{bank}ojõ}, \vadja{\underline{bank}oit}, \vadja{\underline{bank}oisõ}, \vadja{\underline{bank}oiss}
 \\
Tüüpsõna hõlmab vormisõnastiku lekseeme: \vadja{bank, bl̕uud, bl̕uudõčk, boran, fartõl, fialk, figur, fortočk, frikad̕el̕k, golod, greettsin, gupk, invaliid, kaban, kamal, kamin, kanal, kipun, kluub, kohin, l̕ihoratk, mašin, mašinist, muudõr, omõn, pagan, pen̕sioner, sammõl, zanavesk, žurnalist, tarelk, vaahtõr, viks, ahvõn}.
