\paragraph*{\vadja{\underline{koivui}n}}
\vadja{\underline{koivui}zõ}, \vadja{\underline{koivui}ssõ}, \vadja{\underline{koivui}zõsõ}, \vadja{\underline{koivui}zõss}, \vadja{\underline{koivui}zõd}, \vadja{\underline{koivui}zijõ}, \vadja{\underline{koivui}ziit}, \vadja{\underline{koivui}ziisõ}, \vadja{\underline{koivui}ziiss}
 \\
Tüüpsõna hõlmab vormisõnastiku lekseeme: \vadja{koivuin, kultõin, kõltõin, pakkõin, rohoin, uutin, voosin, kalttõin}.
