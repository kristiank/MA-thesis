

\vspace{3.5em}
\noindent \begin{minipage}{\textwidth}
\stepcounter{mallinumber}
\noindent \textbf{Tüüpsõnamall \arabic{mallinumber}\,\vadja{kraaskõ}}\\

\begin{sideways}
\begin{tabular}{l l}
muutvormimall & tunnused \\
\hline
\underline{kraa}\,$\oplus$\,skõ & \textsc{ sg nom } \\
\underline{kraa}\,$\oplus$\,zga & \textsc{ sg gen } \\
\underline{kraa}\,$\oplus$\,ska & \textsc{ sg par } \\
\underline{kraa}\,$\oplus$\,skasõ & \textsc{ sg ill } \\
\underline{kraa}\,$\oplus$\,zgõz & \textsc{ sg ine } \\
\underline{kraa}\,$\oplus$\,zgõss & \textsc{ sg ela } \\
\underline{kraa}\,$\oplus$\,zgõllõ & \textsc{ sg all } \\
\underline{kraa}\,$\oplus$\,zgõll & \textsc{ sg ade } \\
\underline{kraa}\,$\oplus$\,zgõlt & \textsc{ sg abl } \\
\underline{kraa}\,$\oplus$\,zgõssi & \textsc{ sg tra } \\
\underline{kraa}\,$\oplus$\,zgõssaa & \textsc{ sg ter } \\
\underline{kraa}\,$\oplus$\,zgõka & \textsc{ sg com } \\
\underline{kraa}\,$\oplus$\,zgõd & \textsc{ pl nom } \\
\underline{kraa}\,$\oplus$\,skojõ & \textsc{ pl gen } \\
\underline{kraa}\,$\oplus$\,skoit & \textsc{ pl par } \\
\underline{kraa}\,$\oplus$\,skoisõ & \textsc{ pl ill } \\
\underline{kraa}\,$\oplus$\,skoiz & \textsc{ pl ine } \\
\underline{kraa}\,$\oplus$\,skoiss & \textsc{ pl ela } \\
\underline{kraa}\,$\oplus$\,skoillõ & \textsc{ pl all } \\
\underline{kraa}\,$\oplus$\,skoill & \textsc{ pl ade } \\
\underline{kraa}\,$\oplus$\,skoilt & \textsc{ pl abl } \\
\underline{kraa}\,$\oplus$\,skoissi & \textsc{ pl tra } \\
\underline{kraa}\,$\oplus$\,skoissaa & \textsc{ pl ter } \\
\underline{kraa}\,$\oplus$\,skoika & \textsc{ pl com } \\
\end{tabular}
\end{sideways}
\captionof{table}{Tüüpsõna \arabic{mallinumber}\,\textit{kraaskõ} ekstraheeritud muutvormimallid.}
\label{tab:tüüpsõnamall-kraaskõ}

\end{minipage}

 
\vspace{1em}
\noindent Tüüpsõna hõlmab vormisõnastikus 6 lekseemi: \vadja{\underline{kraa}skõ, \underline{lai}skõ, \underline{nagriskaa}skõ, \underline{ni}skõ, \underline{pa}skõ} ja \vadja{\underline{kaa}skõ}.
