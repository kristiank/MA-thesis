\paragraph*{\vadja{\underline{karjuš}š\underline{i}}}
\vadja{\underline{karjuš}\underline{i}}, \vadja{\underline{karjuš}š\underline{i}a}, \vadja{\underline{karjuš}š\underline{i}sõ}, \vadja{\underline{karjuš}\underline{i}ss}, \vadja{\underline{karjuš}\underline{i}d}, \vadja{\underline{karjuš}š\underline{i}jõ}, \vadja{\underline{karjuš}š\underline{i}it}, \vadja{\underline{karjuš}š\underline{i}isõ}, \vadja{\underline{karjuš}š\underline{i}iss}
 \\
Tüüpsõna hõlmab vormisõnastiku 4 lekseemi: \vadja{karjušši, latõšši, potašši} ja \vadja{fal̕šši}.
