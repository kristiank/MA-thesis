\paragraph*{\vadja{\underline{az̕z̕}õ}}
\vadja{\underline{az̕z̕}a}, \vadja{\underline{az̕z̕}a}, \vadja{\underline{az̕z̕}asõ}, \vadja{\underline{az̕z̕}õss}, \vadja{\underline{az̕z̕}õd}, \vadja{\underline{az̕z̕}ojõ}, \vadja{\underline{az̕z̕}oit}, \vadja{\underline{az̕z̕}oisõ}, \vadja{\underline{az̕z̕}oiss}
 \\
sõnatüüp hõlmab lekseeme \vadja{az̕z̕õ, bad̕d̕õ, bahvõlõ, bl̕aahõ, bobrõ, borkkanõ, bruudõ, čirjavõ, čirjõ, d̕eelõ, dobrõ, filmõ, glaizõ, grammõ, gribavihmõ, iivõ, jumalõ, jurmõ, kabjõ, kaglõ, kagrõ, kajagõ, kambõlõ, kanavõ, karjõ, kassõ, katagõ, kavalõ, kvartirõ, lad̕d̕õ, ladvõ, lahjõ, lahnõ, lainõ, laivõ, liivõ, linnõ, l̕istõ, maailmõ, maamõ, mahlõ, mannõ, marjõ, matalõ, metlõ, muragõ, mussõmarjõ, naglõ, n̕egrõ, niglõ, ostanofkõ, paglõ, progonõ, pudrõ, pupuškõ, rauhõ, saappõgõ, sarjõ, saunõ, siglõ, sisavõ, sl̕ifkõ, summõ, surmõ, suukkurliivõ, sõbrõ, šuubõ, ženihõ, taičinõ, trubõ, vihmõ, vikahtõ, villõ, õravõ, õzrõ, akkunõ}
