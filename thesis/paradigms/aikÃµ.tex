

\vspace{3.5em}
\noindent \begin{minipage}{\textwidth}
\stepcounter{mallinumber}
\noindent \textbf{Tüüpsõnamall \arabic{mallinumber}\,\vadja{aikõ}}\\

\begin{sideways}
\begin{tabular}{l l}
muutvormimall & tunnused \\
\hline
\underline{ai}\,$\oplus$\,kõ & \textsc{ sg nom } \\
\underline{ai}\,$\oplus$\,ga & \textsc{ sg gen } \\
\underline{ai}\,$\oplus$\,ka & \textsc{ sg par } \\
\underline{ai}\,$\oplus$\,kasõ & \textsc{ sg ill } \\
\underline{ai}\,$\oplus$\,gõz & \textsc{ sg ine } \\
\underline{ai}\,$\oplus$\,gõss & \textsc{ sg ela } \\
\underline{ai}\,$\oplus$\,gõllõ & \textsc{ sg all } \\
\underline{ai}\,$\oplus$\,gõll & \textsc{ sg ade } \\
\underline{ai}\,$\oplus$\,gõlt & \textsc{ sg abl } \\
\underline{ai}\,$\oplus$\,gõssi & \textsc{ sg tra } \\
\underline{ai}\,$\oplus$\,gõssaa & \textsc{ sg ter } \\
\underline{ai}\,$\oplus$\,gõka & \textsc{ sg com } \\
\underline{ai}\,$\oplus$\,gõd & \textsc{ pl nom } \\
\underline{ai}\,$\oplus$\,kojõ & \textsc{ pl gen } \\
\underline{ai}\,$\oplus$\,koit & \textsc{ pl par } \\
\underline{ai}\,$\oplus$\,koisõ & \textsc{ pl ill } \\
\underline{ai}\,$\oplus$\,koiz & \textsc{ pl ine } \\
\underline{ai}\,$\oplus$\,koiss & \textsc{ pl ela } \\
\underline{ai}\,$\oplus$\,koillõ & \textsc{ pl all } \\
\underline{ai}\,$\oplus$\,koill & \textsc{ pl ade } \\
\underline{ai}\,$\oplus$\,koilt & \textsc{ pl abl } \\
\underline{ai}\,$\oplus$\,koissi & \textsc{ pl tra } \\
\underline{ai}\,$\oplus$\,koissaa & \textsc{ pl ter } \\
\underline{ai}\,$\oplus$\,koika & \textsc{ pl com } \\
\end{tabular}
\end{sideways}
\captionof{table}{Tüüpsõnamall \arabic{mallinumber}\,\vadja{aikõ} ekstraheeritud muutvormimallid.}
\label{tab:tüüpsõnamall-aikõ}

\end{minipage}

 
\vspace{1em}
\noindent Tüüpsõnamall \vadja{aikõ} hõlmab vormisõnastikus 10 lekseemi: \vadja{\underline{ai}kõ, \underline{jal}kõ, \underline{lii}kõ, \underline{lõn}kõ, \underline{nah}kõ, \underline{rah}kõ, \underline{vil}kõ, \underline{vin}kõ, \underline{võl}kõ} ja \vadja{\underline{aastai}kõ}.
