\paragraph*{\vadja{\underline{baldõhin}a}}
\vadja{\underline{baldõhin}a}, \vadja{\underline{baldõhin}a}, \vadja{\underline{baldõhin}asõ}, \vadja{\underline{baldõhin}ass}, \vadja{\underline{baldõhin}ad}, \vadja{\underline{baldõhin}ojõ}, \vadja{\underline{baldõhin}oit}, \vadja{\underline{baldõhin}oisõ}, \vadja{\underline{baldõhin}oiss}
 \\
Tüüpsõna hõlmab vormisõnastiku lekseeme: \vadja{baldõhina, barabana, fotokartočka, grana, griba, kala, kana, liha, lina, litra, maja, raha, suma, sõna, tara, telefona, televizora, tila, vana, astia}.
