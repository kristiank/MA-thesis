

\vspace{3.5em}
\noindent \begin{minipage}{\textwidth}
\stepcounter{mallinumber}
\noindent \textbf{Tüüpsõnamall \arabic{mallinumber}\,\vadja{õhsõ}}\\

\begin{sideways}
\begin{tabular}{l l}
muutvormimall & tunnused \\
\hline
\underline{õh}\,$\oplus$\,sõ & \textsc{ sg nom } \\
\underline{õh}\,$\oplus$\,za & \textsc{ sg gen } \\
\underline{õh}\,$\oplus$\,sa & \textsc{ sg par } \\
\underline{õh}\,$\oplus$\,sasõ & \textsc{ sg ill } \\
\underline{õh}\,$\oplus$\,zõz & \textsc{ sg ine } \\
\underline{õh}\,$\oplus$\,zõss & \textsc{ sg ela } \\
\underline{õh}\,$\oplus$\,zõllõ & \textsc{ sg all } \\
\underline{õh}\,$\oplus$\,zõll & \textsc{ sg ade } \\
\underline{õh}\,$\oplus$\,zõlt & \textsc{ sg abl } \\
\underline{õh}\,$\oplus$\,zõssi & \textsc{ sg tra } \\
\underline{õh}\,$\oplus$\,sõssaa & \textsc{ sg ter } \\
\underline{õh}\,$\oplus$\,zõka & \textsc{ sg com } \\
\underline{õh}\,$\oplus$\,zõd & \textsc{ pl nom } \\
\underline{õh}\,$\oplus$\,sijõ & \textsc{ pl gen } \\
\underline{õh}\,$\oplus$\,siit & \textsc{ pl par } \\
\underline{õh}\,$\oplus$\,siisõ & \textsc{ pl ill } \\
\underline{õh}\,$\oplus$\,siiz & \textsc{ pl ine } \\
\underline{õh}\,$\oplus$\,siiss & \textsc{ pl ela } \\
\underline{õh}\,$\oplus$\,siillõ & \textsc{ pl all } \\
\underline{õh}\,$\oplus$\,siill & \textsc{ pl ade } \\
\underline{õh}\,$\oplus$\,siilt & \textsc{ pl abl } \\
\underline{õh}\,$\oplus$\,siissi & \textsc{ pl tra } \\
\underline{õh}\,$\oplus$\,siissaa & \textsc{ pl ter } \\
\underline{õh}\,$\oplus$\,sijka & \textsc{ pl com } \\
\end{tabular}
\end{sideways}
\captionof{table}{Tüüpsõna \arabic{mallinumber}\,\textit{õhsõ} ekstraheeritud muutvormimallid.}
\label{tab:tüüpsõnamall-õhsõ}

\end{minipage}

 
\vspace{1em}
\noindent Tüüpsõna hõlmab vormisõnastikus 2 lekseemi: \vadja{\underline{õh}sõ} ja \vadja{\underline{sor}sõ}.
