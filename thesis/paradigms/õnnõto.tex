\paragraph*{\vadja{\underline{õnnõt}\underline{o}}}
\vadja{\underline{õnnõt}t\underline{o}ma}, \vadja{\underline{õnnõt}\underline{o}ta}, \vadja{\underline{õnnõt}t\underline{o}masõ}, \vadja{\underline{õnnõt}t\underline{o}mass}, \vadja{\underline{õnnõt}t\underline{o}mad}, \vadja{\underline{õnnõt}t\underline{o}mijõ}, \vadja{\underline{õnnõt}t\underline{o}miit}, \vadja{\underline{õnnõt}t\underline{o}miisõ}, \vadja{\underline{õnnõt}t\underline{o}miiss}
 \\
Tüüpsõna hõlmab vormisõnastiku 2 lekseemi: \vadja{õnnõto} ja \vadja{hoolito}.
